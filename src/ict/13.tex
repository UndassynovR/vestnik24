\newpage
{\bfseries ҒТАМР 81.93.29}
\hfill {\bfseries \href{https://doi.org/10.58805/kazutb.v.3.24-500}{https://doi.org/10.58805/kazutb.v.3.24-500}}

\sectionwithauthors{Б. Рысбекқызы, Д.Т. Кожанова, Г.К. Кабиева,Ә.Ж. Изат, Л.Б. Кадырова, А.М. Леонтьева}{SQL-ИНЪЕКЦИЯЛАРЫНАН ВЕБ-ҚОСЫМШАЛАР ДЕРЕКТЕР ҚОРЫНЫҢ ОСАЛДЫҒЫН
ЖӘНЕ ҚОРҒАНЫСЫН ЗЕРТТЕУ}
\begin{center}

{\bfseries Б. Рысбекқызы\envelope, Д.Т. Кожанова, Г.К. Кабиева,
Ә.Ж. Изат, Л.Б. Кадырова, 
\\А.М. Леонтьева}

Әбілқас Сағынов атындағы Қарағанды техникалық университеті, Қарағанды,
Қазақстан
\end{center}
\envelope Корреспондент-автор: Bakhytgulz@mail.ru\vspace{0.5cm}

Қазіргі уақытта веб-технологиялар интернет-дүкендерде, банктерде,
кәсіпорындардың веб-парақ-шаларында барлық жерде қолданылады. Есесіне,
деректер қоры көбінесе веб-қосымшаларды жазу үшін қолданылады. Деректер
қоры ақпараттық қауіпсіздікке төнетін қауіптермен сипатталады.
Дерек-терге қол жеткізу үшін аутентификация процесінде шабуылдаушы SQL
Injection типті ақпараттық шабуылды қолдана алады. Бұл шабуылдың мәні
веб-технологиялар мен SQL қиылысындағы қатені пайдалану болып табылады.
Бұл пайдаланушы деректерін өңдеуге арналған көптеген веб-парақшалар
зиянды кодты енгізуге әкелуі мүмкін деректер қорына арнайы SQL сұранысын
қалыптастыратынды-ғына байланысты.

Мақалада SQL инъекцияларынан веб-қосымшалар деректер қорының негізгі
қауіптері мен осалды-ғы қарастырылады. Жұмыс SQL инъекцияларының негізгі
түрлерін, осалдықтарды анықтау және алдын алу әдістерін және
веб-қосымшалардың осалдығын тексеру тәсілдерін талдауды қамтиды.
Деректер базасын SQL инъекцияларынан қорғау әдістеріне және тәжірибені
дамытуға баса назар аударылады.

Зерттеу тақырыбы өзекті, өйткені заманауи әлемде интернет біздің
өміріміздің ажырамас бөлігіне айналды. Қосымшаларды пайдаланудың өсуімен
құпия деректерді қорғау үшін қауіпсіздік шаралары-на деген қажеттілік
артты. Веб-қосымшалардың осалдығы қауіпсіздіктің бұзылуына әкелуі
мүмкін, бұл пайдаланушылардың жеке және қаржылық ақпаратын жоғалтуға,
ұйымдардың беделіне нұқсан келтіруге және заңды салдарға әкелуі мүмкін.

{\bfseries Түйін сөздер:} SQL-инъекция, веб-қосымша, деректер қоры,
деректерді қорғау, шабуыл, ақпарат-тық қауіпсіздік.


\sectionheading{ИССЛЕДОВАНИЕ УЯЗВИМОСТИ И ЗАЩИТЫ БАЗ ДАННЫХ ВЕБ ПРИЛОЖЕНИЙ
ОТ~SQL-ИНЪЕКЦИЙ}
\begin{center}
{\bfseries Б. Рысбекқызы\envelope, Д.Т. Кожанова Г.К. Кабиева,
А.Ж. Изат, Л.Б. Кадырова,}

{\bfseries А.М. Леонтьева}

Карагандинский технический университет имени Абылкаса Сагинова,
Караганда, Казахстан,

e-mail: Bakhytgulz@mail.ru
\end{center}

В настоящее время веб-технологии используются повсеместно в
интернет-магазинах, банках, веб-страницах предприятий. При этом базы
данных часто используются для написания веб-приложений. Для базы данных
характерны угрозы информационной безопасности. В процессе аутентификации
для получения доступа к данным злоумышленник может использовать
информационную атаку типа SQL Injection. Суть данной атаки заключается в
использовании ошибки на стыке веб-технологий и SQL. Это обусловлено тем,
что многие web-страницы для обработки пользовательских данных, формируют
специальный SQL запрос к базам данных, что может привести к внедрению
вредоносного кода.

В статье рассматриваются основные угрозы и уязвимость баз данных
веб-приложений от SQL-инъекций. Работа включает в себя анализ основных
видов SQL-инъекций, методов обнаружения и предотвращения уязвимостей, а
также подходы к тестированию уязвимостей веб-приложений. Основное
внимание уделено методам защиты баз данных от SQL-инъекций и разработке
практичес-ких рекомендаций по обеспечению безопасности веб-приложений.

Тематика исследования актуальна так как в современном мире, так как
интернет стал неотъемле-мой частью нашей жизни. С ростом использования
приложений также возросла потребность в мерах безопасности для защиты
конфиденциальных данных. Уязвимости веб-приложений могут привести к
нарушениям безопасности, что может привести к потере личной и финансовой
информации пользо-вателей, ущербу для репутации организаций и юридическим
последствиям.

{\bfseries Ключевые слова:} SQL-инъекция; веб-приложение; базы данных;
защита данных; атака; информа-ционная безопасность.


\sectionheading{RESEARCH OF VULNERABILITY AND PROTECTION OF WEB APPLICATION
DATABASES FROM SQL INJECTIONS}
\begin{center}
{\bfseries B. Rysbekyzy\envelope, D.T. Kozhanova G.K. Kabieva,
A.Zh. Izat, L.B.Kadyrova,}

{\bfseries A.M.Leontyeva}

Abylkas Saginov Karaganda Technical University, Karaganda, Kazakhstan,

e-mail: Bakhytgulz@mail.ru
\end{center}

Currently, web technologies are used everywhere in online stores, banks,
and enterprise web pages. At the same time, databases are often used to
write web applications. The database is characterized by information
security threats. During the authentication process, an attacker can use
an information attack such as SQL Injection to gain access to data. The
essence of this attack is to exploit an error at the intersection of web
technologies and SQL. This is due to the fact that many web pages form a
special SQL query to databases to process user data, which can lead to
the introduction of malicious code.

The article discusses the main threats and vulnerability of web
application databases from SQL injections. The work includes an analysis
of the main types of SQL injections, methods for detecting and
preventing vulnerabilities, as well as approaches to testing web
application vulnerabilities. The main focus is on methods for protecting
databases from SQL injections and developing practical recommendations
for ensuring the security of web applications.

The topic of the research is relevant in the modern world, since the
Internet has become an integral part of our lives. As the use of apps
has increased, the need for security measures to protect sensitive data
has also increased. Web application vulnerabilities can lead to security
breaches that can result in the loss of users\textquotesingle{} personal
and financial information, damage to organizations\textquotesingle{}
reputations, and legal consequences.

{\bfseries Keywords:} SQL injection; web application; databases; data
protection; attack; information security.
\begin{multicols}{2}

{\bfseries Кіріспе.} Веб-қосымшадағы шабуылдардың ең танымал түрлерінің
бірі SQL инъекциясы болып табылады. SQL-инъекция - бұл зиянды
пайдаланушыға өзінің жеке параметрлерін қолданыстағы SQL-сұраныс
жасауына немесе SQL-сұраныс жасауды болдырмауға және өзінің жеке сұраныс
жасауын ұсынуға мүмкіндік беретін SQL дерекқорына арнайы бағытталған
инъекциялық шабуыл түрі.

SQL-инъекция ең кең таралған, бірақ жалғыз түрі емес. Инъекциялық
шабуылдар екі негізгі компоненттен тұрады: интерпретатор және
пайдаланушыдан түсетін пайдалы жүктеме, ол қандай да бір интерпретаторға
оқылады {[}1-2{]}. Бұл инъекциялық шабуылдар FFMPEG (бейнекомпрессор)
сияқты командалық жолдың утилитасына, сондай-ақ, деректер қорына
(дәстүрлі жағдайда SQL-инъекцияға) қарсы болуы мүмкін дегенді білдіреді.

{\bfseries Материалдар мен әдістер.} SQL-инъекция ең классикалық аталатын
инъекция түрі болып табылады. SQL жолы HTTP пайдалы жүктемесінде
экрандалады, бұл соңғы пайдаланушының атынан пайдаланушы SQL
сұраныстарына әкеледі.

Дәстүрлі түрде көптеген OSS пакеттері PHP және SQL (жиі MySQL)
комбинациясын пайдалана отырып жасалған. Әзірлеу тарихында-ғы
SQL-инъекциялардың көптеген жиі айтылған осалдықтары PHP-нің түсініктер,
логика және деректер коды арасындағы интерполяцияға неғұрлым жұмсақ
тәсілді ұстануы нәтижесінде туындады. Ескі мектептің PHP әзірлеушілері
өздерінің PHP файлдарында SQL, HTML және PHP комбинацияларын - PHP
қолдайтын ұйымдастыру моделін қолданар еді, ол дұрыс қолданылмай, бұл
осал PHP кодының көп болуына әкелетін еді.

Пайдаланушыға жүйеге кіруге мүмкіндік беретін форумның ескі
бағдарламалық жасақтамасына арналған PHP код блогының мысалын
қарастырайық:
\end{multicols}


\begin{verbatim}
  <?php if ($_SERVER['REQUEST_METHOD'] != 'POST') { 
    echo'
    <div class="row">
    <div class="small-12 columns">
    <form method="post" action="">
    <fieldset class="panel">
    <center>
    <h1>Sign In</h1><br>
    </center>
    <label>
    <Input type="text" id="username" name="username" 
    placeholder="Username">
    </label>
    <label>
    <input type="password" id="password" name="password" 
    placeholder="Password">
    </label>
    <center>
    <input type="submit" class="button" value="Sign In">
    </center>
    </fieldset>
    </form>
    </div>
    </div>';
    } else {
    // the user has already filled out the login form.
    // pull in database info from config.php
    $Servername = getenv('IP');
    $Username = $mysqlUsername;
    $Password = $mysqlPassword;
    $Database = $mysqlDB;
    $Dbport = $mysqlPort;
    $database = new mysqli($servername, $username, $password,
     $database,$dbport);
    if ($database->connect_error) {
    echo "ERROR: Failed to connect to MySQL";
    die;
    }
    $sql = "SELECT userId, username, admin, moderator FROM users
     WHERE username = '".$_POST['username']."' AND 
     password = '".sha1($_POST['password'])."';";
    $result = mysqli_query($database, $sql);
    }
\end{verbatim}
\begin{multicols}{2}

Көріп отырғаныңыздай, берілген кіру кодында PHP, SQL және HTML
араласқан. Сонымен қатар, SQL-сұранысы сұраныс жолағын жасамас бұрын
ешқандай тазартусыз сұрау параметрлерін біріктіру негізінде жасалады
{[}3{]}.

HTML, PHP және SQL кодтарының тоғысуы PHP негізіндегі веб-қосымшалар
үшін SQL инъекциясының енуіне едәуір жеңілдетті. Тіпті WordPress сияқты
кейбір ірі OSS PHP қосымшалары бұған дейін осындай жағдайдың құрбаны
болған.

PHP-ден алынған қауіпсіздік сабақтары басқа тілдерде көрініс тапты және
қазіргі заманғы веб-қосымшаларда SQL-ді енгізуге байланысты осалдықтарды
табу әлдеқайда қиын. Дегенмен, бұл әлі де мүмкін және қауіпсіз
бағдарламалаудың озық әдістерін қолданбайтын қосымшаларда жиі кездеседі.

{\bfseries Нәтижелер мен талқылау. \emph{Инъекцияға қарсы қорғаныстарды
зерттеу.}} Әдетте әзірлеушінің CLI және пайдаланушы жіберген деректері
бар автоматтандырудың кез келген түрін жазуына дұрыс назар аудармауының
нәтижесінде бұл шабуылдар әлі де жиі кездеседі.

Инъекциялық шабуылдар сондай-ақ, жазықтық-тың кең ауданын қамтиды.
Инъекцияны CLI-ге немесе серверде жұмыс істейтін кез келген басқа
оқшауланған интерпретаторға қарсы пайдалануға болады (ол ОЖ деңгейіне
жеткенде, оның орнына ол команда инъекциясына айналады) {[}4-5{]}.
Нәтижесінде, инъекция стиліндегі шабуылдардан қалай қорғанатынымыз-ды
қарастырғанда, мұндай қорғау шараларын бірнеше санатқа бөлу оңай.

Ең алдымен, біз SQL-инъекция шабуылдары-нан қорғанысты бағалауымыз керек
- инъекцияның ең кең тараған және белгілі түрі. SQL инъекцияларынан
қорғану үшін не істей алатынымызды зерттегеннен кейін, бұл
қорғаныстардың қайсысы инъекциялық шабуылдардың басқа түрлеріне
қолданылатынын көруге болады. Ақыр соңында, біз инъекциядан қорғаудың
бірнеше жалпы әдістерін бағалай аламыз, олар инъекцияның негізіндегі
шабуылдардың қандай да бір жиынына тән емес.

\emph{SSQL-инъекциясының салдарын жұмсарту{\bfseries . }}SQL инъекциясы --
инъекциялық шабуылдың ең көп таралған түрі және сол сияқты қорғанудың ең
оңай түрлерінің бірі. Ол өте кең таралғандықтан, әрбір дерлік күрделі
веб-қосымшаға әсер етуі мүмкін (SQL дерекқорларының таралуына
байланысты) SQL инъекциясына қарсы көптеген жұмсарту және қарсы шаралар
әзірленді.

Сонымен қатар, SQL инъекциялық шабуылдар SQL интерпретаторында орын
алатындықтан, мұндай осалдықтарды анықтау өте қарапайым болуы мүмкін
{[}6{]}. Тиісті анықтау және жұмсарту стратегиялары бар болса,
веб-қосымшаның SQL инъекциялық шабуылына ұшырау мүмкіндігі өте төмен.

\emph{SQL инъекциясын анықтау.} Кодтық базаңызды SQL инъекциялық
шабуылдардан қорғауға дайындау үшін алдымен SQL инъекциясы қабылдайтын
форманы және сіздің кодтық базаңыздағы ең осал болуы мүмкін орындарды
қарап шығуыңыз керек {[}7{]}.

Көптеген заманауи веб-қосымшаларда SQL операциялары сервер жағындағы
маршрутизация деңгейінен кейін орындалады. Бұл клиент туралы ештеңе
қызықтырмайтынын білдіреді.

Мысалы, веб-қосымшаның код репозиторийінің файлдық құрылымы бар, ол
келесідей:

/api

/routes

/utils

/analytics

/routes

/client

/pages

/scripts

/media

Біз клиентті іздеуді жіберіп алатынымызды білеміз, бірақ біз талдау
бағытын қарастыруымыз керек, өйткені ол OSS-те салынған болса да, ол
аналитика деректерін сақтау үшін қандай да бір дерекқорды пайдалануы
мүмкін. Егер деректер құрылғылар мен сессиялар арасында сақталса, олар
серверлік жадта, дискіде (журналдарда) немесе дерекқорда сақталады.

Серверде көптеген қолданбалар бірден көп дерекқорды пайдаланатынын
білуіміз керек. Бұл қолданба, мысалы, SQL Server-ді және MySQL-ді
пайдаланады дегенді білдіруі мүмкін. Сондықтан сервердегі іздеу кезінде
SQL сұраныстарын бірнеше SQL тілінің іске асырылуынан табу үшін ортақ
сұраныстарды пайдалану қажет.

Бұдан басқа, кейбір серверлік бағдарламалар доменге тән тілді (DSL)
пайдаланады, ол біздің атымыздан SQL шақыруларын орындауы мүмкін, бірақ
бұл шақырулар өңделмеген SQL шақыруына ұқсас құрылымданбайды.

Қолданыстағы кодтық базаны SQL-инъекция-ның әлеуетті тәуекелдеріне дұрыс
талдау жасау үшін, біз барлық алдыңғы DSL және SQL түрлерінің тізімін
жасап, оны бір жерде сақтауымыз керек.

Егер біздің қолданба Node.js қолданбасы болып табылса және ол мыналарды
қамтыса:

\begin{itemize}
  \setlength{\itemindent}{1cm} 
\item
  SQL сервері -- NodeMSSQL адаптері (npm) арқылы
\item
  MySQL -- mysql адаптері (npm) арқылы
\end{itemize}

онда екі SQL іске асыруынан SQL сұраныстарын таба алатын код базасында
іздеулерді құрылымдауды қарастыруымыз керек.

Бақытымызға орай, Node.js-пен жеткізілетін модульдерді импорттау жүйесі
бұл тапсырманы JavaScript тілінің қолданылу аймағымен бірге жеңілдетеді
{[}8{]}. Егер SQL кітапханасы пермодуль негізінде импортталса,
сұрауларды табу импортты іздеу сияқты оңай болады:
\end{multicols}

\begin{verbatim}

  const sql = require('mssql')
  // OR
  const mysql = require('mysql');
  
\end{verbatim}

\begin{multicols}{2}

Екінші жағынан, егер бұл кітапханалар жаһандық деңгейде жарияланса
немесе ата-аналық класстан мұраға қалса, іздеу жұмыстары біршама
күрделене түседі.

Node.js үшін жоғарыда аталған екі SQL адаптерінің екеуі де .query (x)
шақыруымен аяқталатын синтаксисті қолданады, бірақ кейбір адаптерлер
әріптік синтаксисті қолданады:
\end{multicols}
\begin{verbatim}

  const sql = require('sql');
  const getUserByUsername = function(username) {
  const q = new sql();
  q.select('*');
  q.from('users');
  q.where(`username = ${username}`);
  q.then((res) => {
  return `username is : ${res}`;
  });
  };
  
\end{verbatim}
\begin{multicols}{2}

\emph{Дайындалған нұсқаулықтар.} Жоғарыда айтылғандай, SQL-сұраныстар
бұрын кең таралған және өте қауіпті. Бірақ көптеген жағдайларда олардан
қорғану онша қиын емес.

SQL-ді іске асырудың көпшілігі қолдай бастаған әзірлемелердің бірі
дайындалған операторлар болып табылады. Дайындалған нұсқаулықтар
SQL-сұраныстарда пайдаланушы-лар ұсынатын деректерді пайдалану кезінде
тәуекелді едәуір төмендетеді. Сонымен қатар, дайындалған нұсқаулықтар
өте оңай зерттеледі және SQL-сұраныстарды түзетуді айтарлықтай
жеңілдетеді {[}9{]}.

Дайындалған нұсқаулықтар көбінесе инъекциядан қорғаудың «бірінші желісі»
болып саналады. Дайындалған нұсқаулықтарды іске асыру оңай, Интернетте
жақсы құжатталған және инъекциялық шабуылдарды тоқтату үшін өте тиімді.

Дайындалған нұсқаулықтар айнымалылар үшін толтырғыш мәндері бар
сұранысты алдын ала құрастыру арқылы жұмыс істейді. Олар байланыстырушы
айнымалылар ретінде белгілі, бірақ көбінесе жай толтырғыш айнымалылар
деп аталады. Сұраныс құрастырылғаннан кейін толтырғыштар әзірлеуші
ұсынған мәндермен ауыстырылады. Осы екі сатылы процестің нәтижесінде
сұраныс мақсаты пайдаланушы жіберген кез келген деректер қаралмай тұрып
белгіленеді.

Дәстүрлі SQL сұрауында пайдаланушы жіберген деректер (айнымалылар) және
сұраныстың өзі дерекқорға жол ретінде бірге жіберіледі. Бұл дегеніміз,
егер пайдаланушы деректері манипуляцияланса, бұл сұраныстың ниетін
өзгерте алады.

Дайындалған нұсқаулықты пайдаланған кезде, сұраныстың мақсаты
пайдаланушы жіберген деректер SQL интерпретаторына жіберілмес бұрын
нақты анықталғандықтан, сұраныстың өзін өзгерту мүмкін емес. Бұл
пайдаланушыларға арналған SELECT операциясын кез келген жолмен
экрандауға және ЖОЮ операциясына түрлендіруге болмайтынын білдіреді.
Егер пайдаланушы бастапқы сұранысты орындамаса және жаңасын бастаса,
SELECT операциясынан кейін қосымша сұранысты орындау мүмкін емес.
Дайындалған нұсқаулар SQL енгізу тәуекелдерінің көпшілігін жояды және
барлық дерлік негізгі SQL дерекқорларымен қамтамасыз етіледі: MySQL,
Oracle, PostgreSQL, Microsoft SQL Server және т.б.

Дәстүрлі SQL-сұраныстар мен дайындалған нұсқаулықтар арасындағы жалғыз
маңызды ымыраласу өнімділік болып табылады {[}10-11{]}. Дерекқорға бір
өтініштің орнына дерекқор дайындалған нұсқаулықты ұсынады, одан кейін
компиляциядан кейін және сұранысты орындау кезінде енгізу үшін
айнымалылар болады. Көптеген қосымшаларда бұл өнімділікті жоғалту
минималды болады.

Синтаксистік тұрғыдан дайындалған нұсқау-лықтар дерекқордан дерекқорға
және адаптерден адаптерге ерекшеленеді.

MySQL-де дайындалған нұсқаулықтар өте қарапайым келеді:
\end{multicols}
\begin{verbatim}

  PREPARE q FROM 'SELECT name, barCode 
                 FROM products 
                 WHERE price <= ?';
SET @price = 12;
EXECUTE q USING @price;
DEALLOCATE PREPARE q;
  
\end{verbatim}
\begin{multicols}{2}

Осы дайындалған нұсқаулықта біз бағасы төмен өнімдер үшін MySQL
дерекқорынан сұраймыз (біз аты мен штрих кодын қайтарғымыз келеді)?.

Біріншіден, сұранысымызды q атауымен сақтау үшін PREPARE операторын
қолданамыз. Бұл сұраныс құрылады және пайдалануға дайын болады. Әрі
қарай, @price айнымалы мәнін 12 мәніне меншіктейміз. Бұл, мысалы,
электрондық коммерция сайтына қарсы сүзгіден өткізетін пайдаланушылар
орнатуы үшін жақсы айнымалы болар еді. Содан кейін біз ? толтыру үшін
@price беретін сұрауды орындаймыз. Соңында, біз оны жадтан жою үшін q
бойынша DEALLOCATE қолданамыз, осылайша оның атаулар кеңістігі басқа
нәрселер үшін пайдаланылуы мүмкін.

Бұл қарапайым дайындалған нұсқаулықта q @price көмегімен орындалмас
бұрын құрастырылады. Егер @price 5-ке тең болса да; UPDATE users WHERE
id = 123 SET balance =10000, қосымша сұрау деректермен
құрастырылмағандықтан іске қосылмайды.

Бұл сұраудың әлдеқайда қауіпсіз нұсқасы:
\end{multicols}
\begin{verbatim}

  'SELECT name, barcode from products WHERE price <= ' + price + ';'

\end{verbatim}
\begin{multicols}{2}

Сіз анық көріп отырғаныңыздай, дайындалған нұсқаулықтарды алдын ала
құрастыру SQL инъекцияларын жеңілдетудегі маңызды бірінші қадам болып
табылады және оны веб-қосымшаңызда мүмкін болған жерде пайдалану керек
{[}12{]}.

\emph{Деректер қорына тән қорғаныстар.} Кеңінен қабылданған дайындалған
нұсқаулықтарға қосымша әрбір SQL негізгі дерекқоры қауіпсіздікті арттыру
үшін өзінің жеке функцияларын ұсынады. Oracle, MySQL, MS SQL және SOQL
SQL сұраныстарында пайдалану үшін қауіпті деп саналатын таңбалар мен
таңбалар жиынтығын автоматты түрде экрандау әдістерін ұсынады. Осы
санациялар туралы шешім қабылданатын әдіс нақты пайдаланылатын дерекқор
мен механизмге байланысты болады.

Oracle (Java) келесі синтаксиспен шақырылуы мүмкін кодтаушыны ұсынады:
\end{multicols}
\begin{verbatim}

  ESAPI.encoder().enodeForSQL(new OracleCodec(), str);

\end{verbatim}
\begin{multicols}{2}

Осыған ұқсас, MySQL баламалы функционал-дылықты ұсынады. MySQL
бағдарламасында дұрыс экрандалмаған жолдарды пайдалануды болдырмау үшін
мыналарды пайдалануға болады:
\end{multicols}

\begin{verbatim}
  SELECT QUOTE('test''case');
\end{verbatim}
\begin{multicols}{2}

MySQL жүйесіндегі QUOTE функциясы кері қиғаш сызықтарды, жалғыз
тырнақшаларды немесе NULL мәндерін болдырмайды және бір тырнақшалы жолды
дұрыс қайтарады.

MySQL mysql\_real\_escape\_string () ұсынады. Бұл функция барлық алдыңғы
қисық сызықтарды және жалғыз тырнақшаларды болдырмайды, сонымен қатар,
қос тырнақшаларды ,\textbackslash n және\textbackslash r (linebreak)
болдырмайды.

Қатерлі таңбалар жиынтығын айналып өту үшін дерекқорға тәуелді жолдарды
тазалау құралдарын пайдалану жолдан айырмашылығы SQL-литералды жазуды
қиындата отырып, SQL-инъекция тәуекелін төмендетеді. Егер параметрленуі
мүмкін емес сұраныс орындалса, олар әрқашан пайдаланылуы тиіс - бірақ
оларды жан-жақты қорғау ретінде емес, жұмсарту ретінде қарау керек.

\emph{Жалпы инъекциялық қорғаныс.} SQL инъекцияларынан қорғану
мүмкіндігіне қоса, сіз сондай-ақ, сіздің қосымшаңыздың инъекцияның басқа
аз таралған түрлерінен қорғалғанына көз жеткізуіңіз керек {[}13{]}.
Жоғарыда айтылғандай, инъекциялық шабуылдар командалық жолдың немесе
интерпретатордың утилитасының кез келген түріне қарсы болуы мүмкін.

Біз SQL инъекцияларынан басқа мақсаттарды іздеуге және инъекцияға
күтпеген осалдықтың пайда болу қаупін азайту үшін барлық қолданбаның
логикасында әдепкі кодтау әдістерін қолдануға тиіспіз.

\emph{Инъекцияларға арналған ықтимал мақсаттар.} Біз бейне немесе
суреттерді сығудың CLI интерфейстері әлеуетті енгізу мақсаты ретінде
пайдаланылуы мүмкін сценарийді қарастырған болатынбыз. Бірақ енгізу
FFMPEG сияқты командалық жолдың утилиттерімен шектелмейді. Ол мәтіндік
енгізуді қабылдайтын және мәтінді қандай да бір интерпретатордың
көмегімен түсіндіретін немесе мәтінді қандай да бір командалар тізімімен
салыстыратын скрипттің кез келген түріне қолданылады.

Әдетте, жүйеге енгізуді іздеу кезінде тәуекел дәрежесі жоғары мынадай
объектілер пайдаланылады:

- Міндеттерді жоспарлаушылар

- Қысу/оңтайландыру кітапханалары

- Қашықтағы сақтық көшірме скрипттері

- Дерекқорлар

- Тіркеушілер

- Хосттың кез келген операциялық жүйесін шақыру

- Кез келген интерпретатор немесе компилятор

Сіздің веб-қосымшаңыздың компоненттерін енгізудің әлеуетті тәуекелі
тұрғысынан бірінші рет ранжирлеу кезінде оларды жоғарыда келтірілген
жоғары тәуекел объектілерінің тізімімен салыстырыңыз. Бұл сіздің
зерттеуіңізге арналған бастапқы нүктелеріңіз болып табылады.

Тәуелділіктер де тәуекел болуы мүмкін, себебі көптеген тәуелділіктер
өздерінің (суб) тәуелділіктерін әкеледі, олар көбінесе осы санаттардың
біріне түсуі мүмкін.

\emph{Ең аз өкілеттілік принципі. Ең аз өкілеттілік принципі} (көбінесе
ең аз артықшылықтар принципі деп аталады) абстракцияның маңызды ережесі
болып табылады, оны қауіпсіз веб-қосымшаларды құруға тырысу кезінде
әрқашан пайдалану керек. Бұл принцип кез келген жүйеде жүйенің әрбір
мүшесінің өз жұмысын орындау үшін қажетті ақпарат пен ресурстарға ғана
қол жеткізуі керек екенін айтады.

Бағдарламалық қамтамасыз ету әлемiнде мынадай принцип қолданылуы мүмкiн:
«бағдарламалық жүйедегi әрбiр модуль осы модульдiң дұрыс жұмыс iстеуi
үшiн қажеттi деректер мен функцияларға ғана қол жеткiзуге тиiс».

Теорияда бұл қарапайым болып көрінеді, бірақ қажет болған жағдайда
масштабты веб-қосымшаларда сирек қолданылады. Бұл принцип шын мәнінде
маңыздырақ болады, өйткені қолданба күрделілігі бойынша масштабталады,
өйткені күрделі қолданбадағы модульдер арасындағы өзара әрекеттесу
күтпеген жанама әсерлер әкелуі мүмкін.

Веб-қосымшаңызбен біріктірілген және пайдаланушы профильдерінің
фотосуреттерінің сақтық көшірмесін автоматты түрде жасайтын пәрмен жолы
интерфейсін жасап жатқаныңызды елестетіп көріңіз {[}14{]}. Бұл CLI
терминалдан (қолмен сақтық көшірме жасау) немесе веб-қосымшаңыз
орнатылған бағдарламалау тілінде жазылған адаптер арқылы шақырылады.
Егер бұл CLI ең аз өкілеттік қағидаты бойынша құрылған болса, онда CLI
бұзылған болса да, қосымшаның қалған бөлігі бұзылмас еді. Екінші
жағынан, әкімші ретінде орындалатын CLI алаяқтық инъекциялық шабуыл
анықталған және жұмыс істеген жағдайда бүкіл қолданба серверін аша
алады.

Ең аз өкілеттілік принципі әзірлеушілер үшін кедергі болып көрінуі
мүмкін - қосымша жазбаларды, бірнеше кілттерді басқару және т.б. - бірақ
осы қағидатты дұрыс іске асыру сіздің қосымшаңыз бұзылған жағдайда
ұшырайтын тәуекелді шектейді.

\emph{Ақ тізімге қосу пәрмендері.} Инъекция үшін ең үлкен қауіп - бұл
клиент (пайдаланушы) орындау үшін серверге командалар жіберетін
веб-қосымшадағы функционалдылық. Бұл нашар архитектуралық тәжірибе және
оны кез келген жағдайда болдырмау керек {[}15{]}.

Пайдаланушы таңдаған пәрмендер серверде кез келген контексте орындалуы
керек болғанда, оларға әлеуетті нұқсан келтіруге немесе қолданбаның
күйін өзгертуге (дұрыс пайдаланбаған жағдайда) мүмкіндік беретін қосымша
қадамдар қажет болады. Пайдаланушы пәрмендерінің сервермен тікелей
түсіндірілуіне мүмкіндік берудің орнына, пайдаланушыға қолжетімді
пәрмендердің нақты анықталған ақ тізімін жасау керек. Бұл команда
синтаксисіне (тәртібі, жиілігі, параметрлері) қосымша ретінде барлығын
қара тізім форматында емес, ақ тізім форматында сақтай отырып, бірге
пайдаланылуы тиіс. Келесі мысалды қарастырайық:

Келесі мысалды қарастырайық:
\end{multicols}
\begin{verbatim}
  <div class="options">
  <h2>Commands</h2>
  <input type="text" id="command-list"/>
  <button type="button" onclick="sendCommands()">Серверге пәрмендерді жіберу </button>
  </div>
  const cli = require('../util/cli');
  /*
  * Клиенттен пәрмендерді қабылдайды, оларды CLI-ге қарсы іске қосады.
  */
  const postCommands = function(req, res) {
  cli.run(req.body.commands);
  };
  
\end{verbatim}
\begin{multicols}{2}

Бұл жағдайда клиент CLI кітапханасы қолдайтын кез-келген командаларды
серверге қарсы орындай алады. Бұл дегеніміз, CLI жұмыс уақыты мен толық
көлемі Соңғы пайдаланушыға CLI қолдайтын командаларды беру арқылы қол
жетімді, тіпті егер олар әзірлеуші қолдануға арналмаған болса да.

Неғұрлым түсініксіз жағдайда, барлық пәрмендерге әзірлеуші рұқсат берген
болуы мүмкін, бірақ синтаксис, тәртіп және жиілік серверде CLI-ге қарсы
күтпеген функционалдық (инъекция) жасау үшін біріктірілуі мүмкін. Тез
және лас жұмсарту - бұл тек бірнеше командалардың ақ тізімі:
\end{multicols}

\begin{verbatim}
  const cli = require('../util/cli');
  const commands = [
  'print',
  'cut',
  'copy',
  'paste',
  'refresh'
  ];
  /*
  /*
  * Клиенттен пәрмендерді қабылдайды, егер олар ақ тізім массивінде 
  *пайда болса, оларды ТЕК CLI-ге қарсы іске қосады.
  */
  const postCommands = function(req, res) {
  const userCommands = req.body.commands;
  userCommands.forEach((c) => {
  if (!commands.includes(c)) { return res.sendStatus(400); }
  });
  cli.run(req.body.commands);
  };
  

\end{verbatim}
\begin{multicols}{2}

Бұл жылдам және лас жұмсарту командалардың ретіне немесе жиілігіне
қатысты мәселелерді шеше алмауы мүмкін, бірақ бұл клиент немесе соңғы
пайдаланушы пайдалануға арналмаған командаларды шақыруға жол бермейді
{[}16{]}. Қара тізім қолданылмайды, өйткені қосымшалар уақыт өте келе
дамиды. Қара тізімдер пайдаланушыға қажетсіз функционалдылық деңгейлерін
беретін жаңа пәрмен қосылған жағдайда қауіпсіздік қаупі ретінде
қарастырылады.

Пайдаланушы енгізуі қабылданып, CLI-ге жіберілген кезде, әрқашан қара
тізім тәсілін емес, ақ тізім тәсілін таңдаңыз.

{\bfseries Қорытынды.} Инъекциялық шабуылдар классикалық түрде деректер
қорына, атап айтқанда SQL деректер қорына жатады. Бірақ дерекқорлар
дұрыс жазылған кодсыз және конфигурациясыз инъекциялық шабуылдарға осал
болса да, API соңғы нүктесі (немесе тәуелділік) өзара әрекеттесетін кез
келген CLI инъекцияның құрбаны болуы мүмкін.

Негізгі SQL дерекқорлары SQL инъекциясының алдын алу шараларын ұсынады,
бірақ SQL инъекциясы қолданбаның архитектурасымен және қате жазылған
клиент-сервер кодымен әлі де мүмкін. Кодтық базаға ең аз өкілеттік
принципін енгізу сіздің ұйымыңызға және веб-қосымшаңыздың
инфрақұрылымына келтірілген зиянды азайту арқылы бұзылған жағдайда
веб-қосымшаңызға көмектеседі. Бастапқыда қауіпсіздік принциптеріне
негізделген қосымша ешқашан клиентке (пайдаланушыға) серверде
орындалатын сұраныс немесе пәрмен беруге мүмкіндік бермейді.

Егер пайдаланушы енгізген деректерді сервер жағындағы операцияларға
түрлендіру қажет болса, бұл операциялар жалпы функционалдылықтың бір
бөлігі ғана қолжетімді болуы үшін және қауіпсіздікті тексерудің жауапты
тобы қауіпсіз ретінде тексерген функционалдылық қана ақ тізімге
енгізілуі тиіс.

Осы басқару элементтерін пайдалана отырып, веб-қосымша инъекция
стиліндегі осалдыққа ие болу ықтималдығы әлдеқайда төмен болады.

SQL-инъекция шабуылдары кеңінен танымал және оларға дайын болғанымен,
инъекция шабуылдары сервер API сұранысына жауап ретінде пайдаланатын кез
келген CLI утилитасына қарсы болуы мүмкін.

SQL дерекқоры (жиі) инъекциядан жақсы қорғалған. Автоматтандыру белгілі
SQL-инъекция шабуылдарын тестілеу үшін өте ыңғайлы, өйткені шабуыл әдісі
өте жақсы құжатталған. SQL-инъекция істен шыққан жағдайда, кескін
компрессорларын, резервтік көшіру утилиттерін және командалық жолдың
басқа интерфейстерін әлеуетті мақсатты нысандар ретінде қарастырыңыз.

\end{multicols}


\begin{center}
  {\bfseries Әдебиеттер}
\end{center}

\begin{noparindent}

1.Ge, X., Paige, R.F., Polack, F.A., Chivers, H. and Brooke, P.J. Agile
Development of Secure Web Applications. Proceedings of the 6th
International Conference on Web Engineering. Palo Alto. 2006,
-P:305-312. DOI 10.1145/1145581.1145641

2. Norwawi, N.M. and Selamat, M.H. Secure E-Commerce Web Development
Framework// Infor-mation Technology Journal. 2011, Vol.10 (4) -
P.769-778. DOI 10.3923/itj.2011.769.778

3.McGraw, G. and Viega, J. Building Secure Software. In RTO/NATO
Real-Time Intrusion Detection Symp. 2002.

4.Mouratidis, H., Jürjens, J. and Fox, J. Towards a Comprehensive
Framework for Secure Systems \\Development //Advanced Information Systems
Engineering. Springer, Berlin Heidelberg.- 2006.- P. 48-62. DOI
10.1007/11767138\_5

5.Гафнер, В. В. Информационная безопасность.Учебное пособие. В 2 ч. Ч.1
/ В. В. Гафнер.\\ -- Екатеринбург:Урал.гос.пед.ун-т,2009.- 155 с. ISBN
978-5-7186-0414-6

6.Булыгина О.В. Методические указания по выполнению расчетно-графической
работы по дисципли-не «Безопасность веб-приложений» {[}Электронный
ресурс{]}: электронные методические указания для студентов, обучающихся
по направлению 10.04.01 «Информационная безопасность» / Булыгина О.В. --
Электрон. дан. -- Смоленск: РИО филиала ФГБОУ ВО «НИУ «МЭИ» в г.
Смоленске, 2019.- Дата обращения: 26.10.2023.

7. Афанасьев А.А., Веденьев Л.Т., Воронцов А.А., Газизова Э.Р.
Аутентификация. Теория и практика обеспечения безопасного доступа к
информационным ресурсам.-Учебное пособие.-Москва: Горячая линия-Телеком.
-2012. -550 с. - ISBN 978-59912-0257-2

8.Прохорова, О.В. Информационная безопасность и защита информации
{[}Электронный ресурс{]}: учебник/ Прохорова О.В.--- Электрон. текстовые
данные.--- Самара: Самарский государственный архитектурно-строительный
университет, ЭБС АСВ, 2014-113 c.--- Режим доступа:
\\http://www.iprbookshop.ru/43183.-ЭБС «IPRbooks»- Дата обращения:
26.10.2023.

9.Буренин, С.Н. Web-программирование и базы данных {[}Электронный
ресурс{]} : учеб. практикум / С. Н. Буренин. - Москва: Моск. гуманит.
ун-т. 2014. -120 с. https://edu.rsreu.ru/res/specialities/disc\\\_edu\_programs/80240-edu\_program-file-Web.-
Дата обращения: 10.11.2023.

10.Бьюли, А. Изучаем SQL. Учебное пособие {[}Текст{]} / А. Бьюли --
Символ-Плюс. 2007. -312 с.: ил. -- ISBN 978-5-93286-051-9.

11.Jan vom Brocke, Jan Mendling. Business Process Management Cases:
Digital Innovation and Business Transformation in Practice - Springer,
2018.- 610 p. ISBN 978-3-319-86372-6

12. Jacobson, A. Cakula, S. Automated Learning Support System to Provide
Sustainable Cooperation between Adult Education Institutions and
Enterprises // Procedia Computer Science.- 2015.- Vol. 43.- P. 127-133.
DOI 10.1016/j.procs.2014.12.017

13.Huoing, M. Truong. Integrating learning styles and adaptive
e-learning system: Current developments, problems and opportunities
//Computers in Human Behavior.- 2016.- Vol.55, Part B. -P:1185-1193. DOI
10.1016/j.chb.2015.02.014

14.Marcella La Rosa, Pnina Soffer. Business Process Management Workshops
-- Springer. 2012.-P.840. ISBN-13 978-3642362866

15.Philipp J. Pratt, Mari Z. Last A Guide to SQL - Course Technolog.//
Cengage Learning.- 320 p. 2014. ISBN-13 978-1111527273

16.PostgreSQL vs MySQL // habr {[}Электронный ресурс{]}. URL:
https://habr.com/company/mailru/blog/\\248845/ -Д ата обращения:
21.04.2023.

\end{noparindent}

\begin{center}
  {\bfseries References}
\end{center}

\begin{noparindent}

1.Ge, X., Paige, R.F., Polack, F.A., Chivers, H. and Brooke, P.J. Agile
Development of Secure Web Applications. Proceedings of the 6th
International Conference on Web Engineering. Palo Alto. 2006,
-P:305-312. DOI 10.1145/1145581.1145641

2. Norwawi, N.M. and Selamat, M.H. Secure E-Commerce Web Development
Framework// Infor-mation Technology Journal. 2011, Vol.10 (4) -
P.769-778. DOI 10.3923/itj.2011.769.778

3.McGraw, G. and Viega, J. Building Secure Software. In RTO/NATO
Real-Time Intrusion Detection Symp. 2002.

4.Mouratidis, H., Jürjens, J. and Fox, J. Towards a Comprehensive
Framework for Secure Systems\\Development //Advanced Information Systems
Engineering. Springer, Berlin Heidelberg.- 2006.- P. 48-62. DOI
10.1007/11767138\_5

5.Gafner, V. V. Informacionnaja bezopasnost\textquotesingle.Uchebnoe
posobie. V 2 ch. Ch.1 / V. V. Gafner.\\ --
Ekaterinburg:Ural.gos.ped.un-t,2009.- 155 s. ISBN 978-5-7186-0414-6

6.Bulygina O.V. Metodicheskie ukazanija po vypolneniju
raschetno-graficheskoj raboty po discipline
«Bezopasnost\textquotesingle{} veb-prilozhenij» {[}Jelektronnyj
resurs{]}: jelektronnye metodicheskie ukazanija dlja studentov,
obuchajushhihsja po napravleniju 10.04.01 «Informacionnaja
bezopasnost\textquotesingle» / Bulygina O.V. -- Jelektron. dan. --
Smolensk: RIO filiala FGBOU VO «NIU «MJeI» v g. Smolenske, 2019.- Data
obrashhenija: 26.10.2023.

7. Afanas\textquotesingle ev A.A., Veden\textquotesingle ev L.T.,
Voroncov A.A., Gazizova Je.R. Autentifikacija. Teorija i praktika
obespechenija bezopasnogo dostupa k informacionnym resursam.-Uchebnoe
posobie.-Moskva: Gorjachaja linija-Telekom. -2012. -550 s. - ISBN
978-59912-0257-2

8.Prohorova, O.V. Informacionnaja bezopasnost\textquotesingle{} i
zashhita informacii {[}Jelektronnyj resurs{]}: uchebnik/ Prohorova
O.V.--- Jelektron. tekstovye dannye.--- Samara: Samarskij
gosudarstvennyj arhitekturno-stroitel\textquotesingle \\ nyj universitet,
JeBS ASV, 2014-113 c.--- Rezhim dostupa:
http://www.iprbookshop.ru/43183.-JeBS\\ «IPRbooks»- Data obrashhenija:
26.10.2023.

9.Burenin, S.N. Web-programmirovanie i bazy dannyh {[}Jelektronnyj
resurs{]} : ucheb. praktikum / S. N. Burenin. - Moskva: Mosk. gumanit.
un-t. 2014. -120 s.https://edu.rsreu.ru/res/specialities/disc\_edu\\\_programs/80240-edu\_program-file-Web.-
Data obrashhenija: 10.11.2023.

10.B\textquotesingle juli, A. Izuchaem SQL. Uchebnoe posobie {[}Tekst{]}
/ A. B\textquotesingle juli -- Simvol-Pljus. 2007. -312 s.: il. -- ISBN
978-5-93286-051-9.

11.Jan vom Brocke, Jan Mendling. Business Process Management Cases:
Digital Innovation and Business Transformation in Practice - Springer,
2018.- 610 p. ISBN 978-3-319-86372-6

12. Jacobson, A. Cakula, S. Automated Learning Support System to Provide
Sustainable Cooperation between Adult Education Institutions and
Enterprises // Procedia Computer Science.- 2015.- Vol. 43.- P. 127-133.
DOI 10.1016/j.procs.2014.12.017

13.Huoing, M. Truong. Integrating learning styles and adaptive
e-learning system: Current developments, problems and opportunities
//Computers in Human Behavior.- 2016.- Vol.55, Part B. -P:1185-1193. DOI
10.1016/j.chb.2015.02.014

14.Marcella La Rosa, Pnina Soffer. Business Process Management Workshops
-- Springer. 2012.-P.840. ISBN-13 978-3642362866

15.Philipp J. Pratt, Mari Z. Last A Guide to SQL - Course Technolog.//
Cengage Learning.- 320 p. 2014. ISBN-13 978-1111527273

16.PostgreSQL vs MySQL//habr {[}Jelektronnyj resurs{]}. URL:
https://habr.com/company/mailru/blog/\\248845/ -Data obrashhenija:
21.04.2023.

\end{noparindent}


\emph{{\bfseries Авторлар туралы мәліметтер}}

\begin{noparindent}

Рысбекқызы Б. - PhD докторы, аға оқытушысы, Әбілқас Сағынов атындағы
Қарағанды техникалық университеті, Қарағанды, Қазақстан, e-mail:
Bakhytgulz@mail.ru;

Кожанова Д.Т. - оқытушысы, Әбілқас Сағынов атындағы Қарағанды техникалық
университеті,\\Қарағанды, Қазақстан, e-mail: kadirova\_dinara@mail.ru;

Кабиева Г.К. - оқытушысы, Әбілқас Сағынов атындағы Қарағанды техникалық
университеті,\\Қарағанды, Қазақстан, e-mail: gulnarakabieva@mail.ru;

Изат А.Ж. - КазМИРР инженері, Әбілқас Сағынов атындағы Қарағанды
техникалық университеті, Қарағанды, Қазақстан, e-mail:
celovek.dobra7@gmail.com;

Кадырова Л.Б. - аға оқытушысы, Әбілқас Сағынов атындағы Қарағанды
техникалық университеті, Қарағанды, Қазақстан, e-mail:
lyakaeldos@mail.ru;

Леонтьева А.М. - аға оқытушысы, Әбілқас Сағынов атындағы Қарағанды
техникалық университеті, Қарағанды, Қазақстан, e-mail: ternast@mail.ru.

\end{noparindent}

\emph{{\bfseries Information about authors}}
\begin{noparindent}

Rysbekkyzy B. - PhD, senior lecturer, Abylkas Saginov Karaganda
Technical University, Karaganda, Kazakhstan, e-mail: Bakhytgulz@mail.ru;

Kozhanova D.T. - lecturer, Abylkas Saginov Karaganda Technical
University, Karaganda, Kazakhstan, e-mail: kadirova\_dinara@mail.ru;

Kabieva G.K. - lecturer, Abylkas Saginov Karaganda Technical University,
Karaganda, Kazakhstan, e-mail: gulnarakabieva@mail.ru;

Izat A.Zh. - engineer of KazMIRR, Abylkas Saginov Karaganda Technical
University, Karaganda, \\Kazakhstan e-mail: celovek.dobra7@gmail.com;

Kadyrova L.B. - senior lecturer, Abylkas Saginov Karaganda Technical
University, Karaganda, Kazakhstan, e-mail: lyakaeldos@mail.ru;

Leontyeva A.M. - senior lecturer, Abylkas Saginov Karaganda Technical
University, Karaganda, \\Kazakhstan, e-mail: ternast@mail.ru.


\end{noparindent}


