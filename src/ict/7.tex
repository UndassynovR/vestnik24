
\newpage
{\bfseries IRSTI 44.01.77}

\sectionwithauthors{Y. Mardenov, Zh. Iztaev, Hu Wen-Tsen, D. Mardenova,D. Baumuratova}{ANALYSIS OF REAL-WORLD AND SIMULATION MODELS AND ALGORITHMS FOR
DETECTING ATTACKS IN WIRELESS SENSOR NETWORKS}

\begin{center}

{\bfseries \textsuperscript{1,2}Y. Mardenov\textsuperscript{🖂} ,
\textsuperscript{3}Zh. Iztaev, \textsuperscript{3}Hu Wen-Tsen,
\textsuperscript{1,2}D. Mardenova, \textsuperscript{1,2}D. Baumuratova}

\textsuperscript{1}International Science Complex "Astana", Astana,
Kazakhstan,

\textsuperscript{2}Astana International University, Astana, Kazakhstan,

\textsuperscript{3}M.Auezov South Kazakhstan State University, Shymkent,
Kazakhstan
\end{center}
Корреспондент-автор: \emph{emardenov@gmail.com}\vspace{0.5cm}

Presented are real-world and simulation models, methods, and tools for
simulating attacks on Wireless Sensor Networks (WSNs) intended for use
in network vulnerability research. A comparative analysis was conducted
to identify their advantages and disadvantages. The research
demonstrated that integrating real-world and simulation approaches
contributes to increased accuracy and reliability in attack detection.
Recommendations are proposed for developing flexible and scalable
simulation models, improving the efficiency of attack detection
algorithms, and regularly updating models in accordance with changing
WSN operating conditions and emerging threats.

{\bfseries Key words:} WSN, Real-world models, Simulation models, WSN
attack detection, Attack detection algorithms, Comparative analysis of
detection tools, Integration of detection methods, Network security.



\sectionheading{АНАЛИЗ НАТУРНЫХ И ИМИТАЦИОННЫХ МОДЕЛЕЙ И АЛГОРИТМОВ ВЫЯВЛЕНИЯ АТАК БСС}
\begin{center}
{\bfseries \textsuperscript{1,2}Е. Марденов\textsuperscript{🖂},
\textsuperscript{3}Ж. Изтаев, \textsuperscript{3}Ху Вен-Цен,
\textsuperscript{1,2}Д. Марденова, \textsuperscript{1,2}Д. Баумуратова}

\textsuperscript{1}Международный научный комплекс «Астана», Астана,
Казахстан,

\textsuperscript{2}Международный университет «Астана», Астана,
Казахстан,

\textsuperscript{3} Южно-Казахстанский государственный университет имени
М. Ауэзова, Шымкент, Казахстан,

е-mail: emardenov@gmail.com
\end{center}

В данной статье представлен анализ натурных и имитационных моделей и
алгоритмов выявления атак на беспроводные сенсорные сети (БСС). Описаны
разработанные натурные и имитационные модели, методы и инструменты для
имитации атак, а также результаты экспериментальных исследова-ний.
Сравнительный анализ выявляет преимущества и недостатки каждого подхода,
подчеркивая необходимость интеграции натурных и имитационных методов для
достижения наибольшей точности и надежности в обнаружении атак. В статье
предложены рекомендации по развитию гибких и масшта-бируемых имитационных
моделей, улучшению алгоритмов обнаружения атак и регулярному обновле-нию
моделей в соответствии с изменяющимися условиями и угрозами. Результаты
исследования подчеркивают важность комбинированного использования
натурных и имитационных подходов для повышения уровня безопасности БСС.

{\bfseries Ключевые слова:} Беспроводные сенсорные сети (БСС), Натурные
модели, Имитационные моде-ли, Выявление атак, Алгоритмы обнаружения,
Сравнительный анализ, Интеграция методов, Безопас-ность сетей


\sectionheading{СЫМСЫЗ СЕНСОРЛЫҚ ЖЕЛІЛЕРДІҢ ТАБИҒИ ЖӘНЕ СИМУЛЯЦИЯЛЫҚ МОДЕЛЬДЕР
МЕН ШАРУАЛДАРДЫ АНЫҚТАУ АЛГОРИТМДЕРІН ТАЛДАУ}
\begin{center}
{\bfseries \textsuperscript{1,2}Е. Марденов\textsuperscript{🖂},
\textsuperscript{3}Ж. Изтаев, \textsuperscript{3}Ху Вэн-Цен,
\textsuperscript{1,2}Д. Марденова, \textsuperscript{1,2}Д. Баумуратова}

\textsuperscript{1} «Астана» халықаралық ғылыми кешені, Астана,
Қазақстан,

\textsuperscript{2}Астана халықаралық университеті, Астана, Қазақстан,

\textsuperscript{3} М.Әуезов атындағы Оңтүстік Қазақстан мемлекеттік
университеті, Шымкент, Қазақстан,

е-mail: emardenov@gmail.com
\end{center}

Бұл мақалада сымсыз сенсорлық желілерге шабуылдарды анықтаудың табиғи
және имитациялық модельдері мен алгоритмдерін талдау ұсынылған.
Әзірленген табиғи және имитациялық модельдер, шабуылдарды модельдеу
әдістері мен құралдары және эксперименттік зерттеулердің нәтижелері
сипатталған. Салыстырмалы талдау шабуылдарды анықтауда ең жоғары дәлдік
пен сенімділікке қол жеткізу үшін табиғи және имитациялық әдістерді
біріктіру қажеттілігін көрсете отырып, әрбір тәсіл-дің артықшылықтары мен
кемшіліктерін анықтайды. Мақалада икемді және масштабталатын модель-деу
модельдерін дамыту, шабуылдарды анықтау алгоритмдерін жақсарту және
өзгеретін жағдайлар мен қауіптерге сәйкес модельдерді үнемі жаңарту
бойынша ұсыныстар берілген. Зерттеу нәтижелері қауіпсіздік деңгейін
жақсарту үшін табиғи және имитациялық тәсілдерді біріктіріп қолданудың
ма-ңыздылығын көрсетеді сымсыз сенсорлық желілер.

{\bfseries Түйін сөздер:} Сымсыз сенсорлық желілер, табиғи модельдер,
модельдеу модельдері, шабуылдар-ды анықтау, анықтау алгоритмдері,
салыстырмалы талдау, әдістерді біріктіру, желі қауіпсіздігі
\begin{multicols}{2}

{\bfseries Introduction.} Wireless Sensor Networks (WSNs) are widely and
comprehensively utilized, playing a crucial role in addressing various
practical tasks in military, industrial, and domestic spheres. WSNs
represent a multifunctional communication foundation of cyber-physical
systems with artificial intelligence elements, providing connectivity
between various sensor devices and systems that can collect, process,
and transmit environmental data in real-time. This foundation enables
effective automatic monitoring and control of various processes and
objects over extensive and hard-to-reach areas {[}1, 2{]}.

However, the use of WSNs is associated with certain risks due to their
security vulnerabilities. Attacks by malicious actors on such networks
can lead to serious consequences, including data interception,
tampering, and disruption of the functionality of technical equipment,
particularly sensor devices that are fundamental elements of WSNs.
Consequently, there is increasing importance in developing effective
measures to counter potential threats within WSN security frameworks.

A critical measure to combat these threats is the development of
efficient methods for detecting, recogni-zing, and preventing network
attacks. Specifically, analytical and simulation models of attacker
actions on WSNs allow for the study of processes within WSNs induced by
these attacks. Such models are based on mathematical and statistical
descriptions of attacker and defender behaviors using real network data,
parameters, and characteristics. They enable the creation of virtual
environments for comprehensive simula-tion of WSNs, including the
operation of sensor nodes, communication processes between nodes over
radio channels, data reception and transmission, and routing {[}1, 3,
4{]}.

This study aims to conduct testing and comparative analysis of natural
and simulation models of attacks on WSNs, attack detection algorithms,
and to develop recommendations for their effective implementation.

{\bfseries Materials and methods.} \emph{Analytical review of literary
sources on the research issue.}

There are several different types of attacks that can occur in WSNs.

Threats to confidentiality involve interception and observation, where
attackers intercept data or analyze traffic to obtain confidential
information. Threats to integrity are associated with data modification,
source impersonation, message replay, and message denial, which lead to
data distortion and incorrect network operation. Attackers can alter
messages, spoof sources, replay intercepted data, or deny
sending/receiving messages. Availability threats aim to disrupt message
delivery, causing network service denial. These attacks include Denial
of Service (DoS), node capture, and resource depletion attacks, leading
to node overload or network disconnection. Such attacks can severely
disrupt network operation, highlighting the importance of protecting
against them. {[}2, 4, 5{]}

The paper presents {[}6{]} a new intrusion detection model for WSNs
using fuzzy neural networks and feedforward neural networks.
Experimental results show that the proposed model achieves detection
rates averaging 97.8\% with maximum detection accuracy of 98.8\%.
Evaluations were compared against benchmark models based on support
vector machines (SVM), decision trees (DT), and random forest (RF)
models.

Authors {[}7{]} introduced detection of multiple attacks in wireless
sensor networks using artificial neural networks. The dataset is split
into training and testing using a multi-layer perceptron artificial
neural network to detect ten classes of attacks, including DoS attacks.
Research using benchmark datasets UNSW-NB, WSN-DS, NSL-KDD, and
CICIDS2018 showed that the proposed system achieves an average detection

In article {[}8{]}, the use of spatial information for detecting and
localizing multiple attacks across single and multiple nodes is
presented. A scalable and energy-efficient anomaly detection mechanism
based on clusters (SEECAD) is described for detecting DoS attacks
without key management schemes to enhance network lifespan. Detection
speed, false alarm rate, packet delivery ratio, overhead costs, energy
consumption, and average packet delay are various performance metrics
used to evaluate network performance.

In {[}9{]}, an enhanced high-performance secure routing protocol based
on clustering is proposed. A key feature of this protocol is its
consideration of aspects such as energy consumption, packet reduction,
congestion management, encrypted data transmission, and monitoring of
malicious nodes to improve data management quality. To demonstrate the
feasibility of the proposed method, performance metrics such as
ransomware attack detection level, ergodic residual energy per round,
early clone attack detection, throughput maximization, delay, maximum
throughput, and network lifespan maximization were used.

The works {[}10{]} conducted modeling to demonstrate that the proposed
EdDSA-XOR functionality reduces time and energy costs by 0.13\% and
0.07\% respectively, compared to other methods. Node authentication in
the network was tested against "man-in-the-middle" attacks.

The paper proposes {[}11{]} an effective method for detecting black hole
and Sybil attacks using the Adaptive Taylor Sail (Adaptive Taylor-SFO)
algorithm. The BSS nodes are modeled in the network, followed by routing
using Adaptive Taylor-SFO. The router was developed by integrating the
Adaptive concept with the Taylor series and the Sail Fish optimizer
(SFO) to select the optimal route considering adaptability metrics such
as delay, energy, and distance. Black hole and Sybil attack detection is
performed by the Deep stacked automatic encoder. Thus, the proposed
system effectively classifies normal, black hole, and Sybil attacks. The
analysis of the reviewed works made it possible to determine the most
common types of attacks on WSNs (Table 1), the mechanisms of their
impact, possible consequences and methods of mitigating the
consequences.
\end{multicols}

\newpage
{\bfseries Table 1 - Most common attacks on WSN}
\begin{longtable}[]{|@{}
  >{\raggedright\arraybackslash}p{(\columnwidth - 8\tabcolsep) * \real{0.0723}}|
  >{\raggedright\arraybackslash}p{(\columnwidth - 8\tabcolsep) * \real{0.1910}}|
  >{\raggedright\arraybackslash}p{(\columnwidth - 8\tabcolsep) * \real{0.3046}}|
  >{\raggedright\arraybackslash}p{(\columnwidth - 8\tabcolsep) * \real{0.2203}}|
  >{\raggedright\arraybackslash}p{(\columnwidth - 8\tabcolsep) * \real{0.2117}}@{}|}
\hline
№ & Attack name & Mechanism of action & Consequences & Mitigation Strategies \\
\hline
\endfirsthead
\hline
№ & Attack name & Mechanism of action & Consequences & Mitigation Strategies \\
\hline
\endhead
\hline
\endfoot
\hline
\endlastfoot
1 & Routing attack {[}6{]} & Routing attacks involve manipulating
routing mechanisms to redirect or block data flows. Attackers exploit
vulnerabilities across various layers of the network protocol stack.
Examples include black hole attacks, where malicious nodes discard
received data, and wormhole attacks, where attackers create shortcuts
between remote nodes. & Data loss. Network segmentation. Resource
exhaustion. & Secure routing protocols. Anomaly detection. Cooperative
verification. Hop count verification. Location verification. \\
\hline
& The man in the middle attack {[}10{]} & Man-in-the-middle attack
involves an attacker secretly intercepting and relaying messages between
two communicating nodes without their knowledge. The attacker can
manipulate the contents of the messages or simply eavesdrop on them.
Man-in-the-middle attacks exploit the absence of secure communication
channels and can occur at various protocol levels, including
application, transport, and network layers. & Data falsification:
Unauthorized access leading to breach of confidentiality. & Encryption.
Public Key Infrastructure (PKI). Certificate revocation. Timestamps and
one-time passwords. Intrusion detection systems. \\
\hline
& Sibyl {[}12{]} & Sybil attack involves creating a network of malicious
nodes that impersonate legitimate nodes. The attacker’s
goal is to inject false information or disrupt network communication.
These attacks can undermine data accuracy, routing efficiency, and
overall network functionality. & Data integrity

Routing manipulation

Resource exhaustion & Behavioral analysis

Trust-based systems

Physical layer measurements

Reputation mechanisms

Cryptographic methods \\
\hline
& Eavesdropping {[}13{]} & In an eavesdropping attack, an attacker is
placed within the range of two or more sensor nodes using the
transmission of unencrypted or weakly encrypted data. The attacker
passively intercepts data packets without changing the functionality of
the network. Eavesdropping can occur at various levels of the
communications stack, from the physical layer to the application layer. & - Data confidentiality

- Data integrity

- Network mapping & - Encryption

- Secure key exchange

- Frequency hopping

- Intrusion detection

- Secure protocols \\
\hline
& Denial of Service (DoS)

{[}14,15{]} & A DoS attack exploits vulnerabilities in WSNs to reduce
their performance or even disable them. Attackers employ various methods
such as flooding the network with excessive traffic or exploiting
protocol vulnerabilities. In the context of WSNs, attacks can target
nodes, communication channels, or the sink node responsible for
aggregating data. & Data loss

Resource depletion

Network partitioning

Delayed responses & Intrusion detection

Rate limiting

Traffic filtering

Energy consumption management

Collaborative defense \\
\hline
\end{longtable}


\emph{Empirical models and attack detection algorithms.}

\emph{Description of the empirical models used for attack detection}
\begin{multicols}{2}

Natural models for detecting attacks in WSNs involve physically
implemented networks where nodes and sensors are deployed in real
operational conditions. These models utilize real devices such as
microcontrollers, radio modules, and sensors that interact within
realistic environmental settings. The primary advantage of natural
models lies in their ability to accurately reproduce real network
operation scenarios, including potential external interferences and
physical attacks.

Research on natural models for attack detection in wireless sensor
networks includes functional and quantitative characteristics. Attack
detection methods are categorized into signature-based, anomaly-based,
and hybrid approaches, covering attacks on availability,
confidentiality, integrity, and authentication. Hardware and network
characteristics of sensors and nodes, data processing algorithms, and
monitoring systems play a crucial role. Quantitative metrics include
detection accuracy, detection time, energy consumption, throughput,
delay, and scalability.

For instance, platforms like TinyOS and Contiki are used to test
intrusion detection systems, achieving 95\% accuracy with low false
positive rates. Machine learning-based systems such as K-means and SVM
can achieve classification accuracies up to 98\%. Distributed detection
methods include autonomous algorithms that depend on node density and
algorithm complexity {[}15, 16{]}. Table 2 presents empirical
experiments on attack detection and their outcomes.
\end{multicols}

{\bfseries Table 2 - Full-scale experiments to detect attacks and their
results}
\begin{longtable}[]{|@{}
  >{\raggedright\arraybackslash}p{(\columnwidth - 6\tabcolsep) * \real{0.0923}}|
  >{\raggedright\arraybackslash}p{(\columnwidth - 6\tabcolsep) * \real{0.1915}}|
  >{\raggedright\arraybackslash}p{(\columnwidth - 6\tabcolsep) * \real{0.3709}}|
  >{\raggedright\arraybackslash}p{(\columnwidth - 6\tabcolsep) * \real{0.3453}}@{}|}
\hline
№ & Name & Description & Results \\
\hline
\endfirsthead
\hline
№ & Name & Description & Results \\
\hline
\endhead
\hline
\endfoot
\hline
\endlastfoot
1. & jamming attack & Nodes of the network were deployed in an open
space with various obstacles. A jamming attack was initiated using a
powerful radio transmitter, creating interference within a specific
frequency range. & It was found that the jamming attack significantly
reduces signal strength and increases packet loss frequency. The
detection system was able to identify the attack based on signal
strength and packet loss analysis, achieving a detection accuracy of
92\%. \\
\hline
2. & resource exhaustion attack & Nodes in the experiment were
programmed to perform energy-intensive tasks. The attacker sent a large
number of false requests to the nodes to accelerate their battery
discharge. & Nodes with depleted resources ceased normal operation. The
detection model based on energy consumption monitoring successfully
identified the attack with 87\% accuracy, enabling timely network
protection measures to be implemented. \\
\hline
3. & replay attack & The experiment involved nodes equipped with
built-in authentication mechanisms. The attacker retransmitted
previously intercepted legitimate messages. & The detection system based
on timestamps and authentication algorithms successfully identified
repeated messages with 95\% accuracy, preventing the execution of false
commands. \\
\hline
\end{longtable}




{\bfseries Table 3 - Evaluation of the effectiveness of full-scale models and algorithms}
\begin{longtable}{|p{7cm}|p{8cm}|}
  \hline
  \textbf{accuracy} & \textbf{detection time} \\ \hline
  The ability of the model to correctly identify attacks and minimize false positives was evaluated in the conducted experiments. Accuracy ranged from 87\% to 95\%, depending on the type of attack and the algorithm applied. & 
  The time required to identify an attack after its onset. Natural models demonstrated the ability to detect attacks in real time, which is critical for preventing damage. \\ \hline
  \textbf{resource consumption} & \textbf{adaptability} \\ \hline
  The volume of computational and energy resources required for algorithm operation. Efficient algorithms minimize resource consumption, which is particularly crucial for sensor nodes with limited batteries. & 
  The ability of the model and algorithms to adapt to changes in the environment and new types of attacks. Natural models have demonstrated good adaptability when new nodes are added or when the network topology changes. \\ \hline
  \end{longtable}
  

\begin{multicols}{2}

The effectiveness of full-scale models and attack detection algorithms is assessed based on several key parameters (Table 3) 

Thus, natural models and intrusion detection algorithms in WLANs are
effective tools for studying and protecting networks, ensuring high
accuracy and timely detection of attacks in real operational conditions.

\emph{Imitative models and intrusion detection algorithms}

Description of Developed Simulation Models. Simulation models are
software tools designed to replicate the operations of Wireless Sensor
Networks (WSNs) and simulate various attack scenarios in a controlled
environment. Within the scope of the conducted research, simulation
models were tested that accurately reproduce the behavior of sensor
nodes, communication protocols, and interactions with the external
environment. These models are based on the following principles:

1. Multi-layered architecture of the model: The simulation model
includes physical, data link, network, and application layers, enabling
detailed reproduction of all aspects of Wireless Sensor Network (WSN)
operation.

2. Network topology modeling: Supports various topologies such as mesh,
star, and tree, allowing exploration of how topology affects resilience
to attacks.

3. Parameter flexibility: The model allows configuration of node
parameters such as transmitter power, data transmission rate, and energy
consumption, crucial for investigating different attack scenarios.

All simulation models are built using diverse mathematical and
computational methods. These models facilitate testing and analyzing
network behavior under attack, evaluating detection accuracy, and
justifying the realism of simulated conditions (Table-4).
\end{multicols}




{\bfseries Table 4 - Composition and structure of models}
\begin{longtable}[]{|@{}
  >{\raggedright\arraybackslash}p{(\columnwidth - 6\tabcolsep) * \real{0.0752}}|
  >{\raggedright\arraybackslash}p{(\columnwidth - 6\tabcolsep) * \real{0.1812}}|
  >{\raggedright\arraybackslash}p{(\columnwidth - 6\tabcolsep) * \real{0.3248}}|
  >{\raggedright\arraybackslash}p{(\columnwidth - 6\tabcolsep) * \real{0.4188}}@{}|}
\hline
№ & Model & Composition and structure of models & Mathematical description \\
\hline
\endfirsthead
\hline
№ & Model & Composition and structure of models & Mathematical description \\
\hline
\endhead
\hline
\endfoot
\hline
\endlastfoot
1 & Network layer & \emph{Graph model: Sensors and nodes are represented
as a graph} 𝐺 ( 𝑉 , 𝐸 ) G(V,E), where 𝑉 V - a set of vertices (nodes),
and 𝐸 E - many edges (communication channels).

\emph{Topology:} The parameters of the network topology are defined,
including the distance between nodes, node density, and network type
(e.g., star, tree, mesh network). & \emph{G}(\emph{V},\emph{E})=\{(\emph{v\textsubscript{i}}\textsubscript{\hspace{0pt}},\emph{v\textsubscript{j}}\hspace{0pt})
∣ \emph{v\textsubscript{i}}\hspace{0pt},\emph{v\textsubscript{j}}\hspace{0pt}∈\emph{V}, \emph{e\textsubscript{ij}}\hspace{0pt}∈\emph{E}\} (1)

Где

𝑉=\{𝑣\textsubscript{1},𝑣\textsubscript{2},\ldots,𝑣\textsubscript{𝑛}\} - many nodes,

𝐸=\{𝑒\textsubscript{𝑖𝑗}\} E=\{e\textsubscript{ij}\} - many communication channels. \\
\hline
2 & Traffic model & \emph{Data flow:} The distribution of traffic between nodes is described. This is achieved using probabilistic models such as Poisson distribution or Markov models. & \emph{λij}\hspace{0pt}=Rate(\emph{vi}\hspace{0pt}→\emph{vj}\hspace{0pt}) (2)

where

𝜆\textsubscript{𝑖𝑗} - traffic intensity between nodes 𝑣\textsubscript{𝑖}\hspace{0pt} and 𝑣\textsubscript{𝑗}\hspace{0pt}, which may follow a Poisson distribution: (3) \\
\hline
3 & Attack model & \emph{Types of attacks:} Models are defined for various types of attacks, such as DoS attacks, data interception attacks, and data integrity attacks.

\emph{Attacker behavior:} The strategy of the attacker is determined, including the frequency and intensity of attacks. & (4)

where

𝑑 - distance to target,

𝜃 - interception angle. \\
\hline
4 & Detection model & \emph{Methods:} Implementation includes detection algorithms such as signature-based, anomaly-based, and hybrid methods.

\emph{Machine learning algorithms:} Utilized for traffic classification, such as K-means, SVM, neural networks. & Anomalous method: (5)

where

𝑥\textsubscript{𝑖} x\textsubscript{i} \hspace{0pt} - measured value,

μ\textsubscript{i} \hspace{0pt} - average value,

𝑤\textsubscript{𝑖} \hspace{0pt} - weight coefficient.

Machine learning algorithms: K-means: (5)

Гдеде

𝐽 - loss function,

𝑘 - number of clusters,

𝑥 \textsubscript{𝑗}\textsuperscript{( 𝑖 )} \hspace{0pt} - data points,

𝜇\textsubscript{𝑖} \hspace{0pt} - cluster centroids \\
\hline
\end{longtable}

\begin{multicols}{2}

Model adequacy assessment involves comparing simulation results with
real-world data, including topology parameters, traffic intensity, and
attack frequency. A model is considered adequate if its behavior does
not statistically differ from real data, often verified using tests like
the Kolmogorov-Smirnov test. An example application of such models could
include testing intrusion detection systems on the TinyOS platform,
achieving a detection accuracy of 95\% with a false positive rate of
less than 2\%, utilizing a hybrid approach to enhance accuracy and
minimize energy consumption.

\emph{Methods and tools for simulating attacks on wireless sensor
networks}

For implementing simulation models, a number of modern tools and methods
were utilized to ensure high accuracy and scalability of the research.
The key tools include:

1. NS-3 (Network Simulator 3): A powerful tool for network simulation
that allows reproduction of a wide range of protocols and attack
scenarios on WLANs. NS-3 provides detailed modeling of node behavior and
interactions between nodes.

2. MATLAB/Simulink: Used for mathematical modeling and analysis of
attack detection algorithms. MATLAB facilitates the development and
testing of complex algorithms, as well as the analysis of data obtained
from simulations.

3. Omnet++: A tool for modeling and simulating networks, offering high
flexibility in network parameter configuration and attack scenarios.
Omnet++ supports extensibility, enabling integration of custom models
and algorithms.

These tools collectively support comprehensive modeling, simulation, and
analysis of wireless sensor networks (WSNs), enabling researchers to
evaluate the performance and effectiveness of various security
mechanisms against different types of attacks.

{\bfseries Results and discussion.} Results of simulation experiments and
their analysis. Within the

framework of conducted simulation experiments, various types of attacks
on WLANs were simulated, including jamming attacks, resource exhaustion
attacks, replay attacks, and spoofing attacks. The results of the
experiments enabled a detailed analysis of the effectiveness of the
proposed models and attack detection algorithms (Table 5).
\end{multicols}

{\bfseries Table 5 - Composition and structure of models}
\begin{longtable}[]{|@{}
  >{\raggedright\arraybackslash}p{(\columnwidth - 6\tabcolsep) * \real{0.0475}}|
  >{\raggedright\arraybackslash}p{(\columnwidth - 6\tabcolsep) * \real{0.1443}}|
  >{\raggedright\arraybackslash}p{(\columnwidth - 6\tabcolsep) * \real{0.5248}}|
  >{\raggedright\arraybackslash}p{(\columnwidth - 6\tabcolsep) * \real{0.2835}}@{}|}
\hline
\toprule
№ & Attack & Description of the experiment & Results \\
\hline
\endfirsthead
\hline
\toprule
№ & Attack & Description of the experiment & Results \\
\hline
\endhead
\hline
\bottomrule
\endfoot
\bottomrule
\endlastfoot
1. & jamming attack & An experiment to simulate a jamming attack was
conducted using a powerful transmitter in Omnet++, a platform for
modeling network systems. The experiment involved creating a wireless
sensor network of 100 nodes with a random topology and a transmission
radius of 50 meters. The experiment consisted of three stages: network
initialization without attack, introduction of the jamming transmitter,
and data collection. The collected data included received signal
strength indicator (RSSI), packet loss, and transmission delay,
amounting to approximately 10,000 records. Data processing was performed
using statistical methods and machine learning algorithms such as
K-means and SVM. The data processing methodology included data
filtering, analysis of signal strength levels, and anomaly
classification {[}17, 18{]}. & The results showed that the jamming
attack significantly reduces communication quality and increases
latency. Simulation algorithms were able to detect the attack with 94\%
accuracy by analyzing signal strength and packet loss rates. The
experiment demonstrated the possibility of using this technique in real
wireless sensor networks. \\
\hline
2. & resource exhaustion attack & A simulation experiment for a resource
exhaustion attack was conducted using the NS-3 network simulator. The
experimental setup included a wireless sensor network comprising 50
nodes, each equipped with a limited battery. The experiment was planned
by creating the network, setting battery parameters, and launching a
series of attacks involving sending a large number of false requests to
the nodes. During the experiment, data on energy consumption, node
response time, and failure rate were collected. Approximately 5000 data
records were gathered, covering all stages of the attack. Data
processing was carried out using an energy consumption monitoring
methodology developed and described in {[}19{]}. This methodology
included data filtering, analysis of energy consumption time series, and
detection of deviations from normal behavior. & Algorithms based on this
methodology were able to identify abnormal behavior with an accuracy of
89\% and an average attack detection time of 2.3 seconds. The
effectiveness of the methodology was confirmed by its high accuracy and
rapid detection of attacks. The experiment demonstrated that the
proposed methodology is effective for application in real-world
conditions of wireless sensor networks. \\
\hline
3. & replay attack & Experimental studies aimed at examining replay
attacks were conducted using a model created in the Simulink
environment. In this experiment, the model simulated the repeated
transmission of intercepted messages, mimicking a scenario where an
attacker could re-execute previously executed commands. The experimental
setup consisted of a network model including several nodes and data
transmission mechanisms configured to implement replay attacks. The
experiment planning involved configuring model parameters, defining
attack characteristics, and selecting detection methods.

During the experiment, data related to message timestamps, as well as
parameters of authentication and integrity verification methods, were
collected. The total amount of gathered data was about 2,000 records,
covering various attack scenarios and the system\textquotesingle s
responses to them. The data were processed using algorithms based on
analyzing message timestamps and authentication methods. The data
processing methodology included filtering out repeated messages and
verifying their compliance with expected time intervals {[}19{]}. & The
experimental results showed that algorithms based on timestamps and
authentication methods successfully detected replayed messages with an
accuracy of 97\%, significantly reducing the risk of executing false
commands and enhancing the effectiveness of the replay attack defense
system. \\
\hline
4. & spoofing & Experimental studies focused on spoofing attacks were
conducted using Simulink software. The experiment was designed by
creating scenarios in which an attacker sent false messages, pretending
to be legitimate network nodes, with the aim of infiltrating the system.
Parameters of the transmitted messages, such as node identifier, message
content, and timestamp, were recorded for data collection. The total
amount of data collected was approximately 3,000 records, covering
various attack scenarios and network responses. For data analysis,
algorithms for node identity verification and behavior analysis
developed by the experiment\textquotesingle s authors were applied. The
data processing methodology included the identification of anomalous
nodes, comparison of their behavior with samples of normal functioning,
and detection of deviations {[}20{]}. & As a result of the experiment,
the model was able to effectively detect spoofing attacks with 92\%
accuracy. The system\textquotesingle s response time to detect the
attack was 1.8 seconds, demonstrating the high reactivity and efficiency
of the developed algorithms. The obtained results confirm the
effectiveness of the proposed spoofing attack protection methodology and
its readiness for practical application in real network systems. \\
\hline
\end{longtable}

\begin{multicols}{2}

Analysis of the results showed that the developed simulation models and
algorithms are highly effective in detecting attacks on WSNs. The
experiments conducted allowed for a detailed study of network behavior
under various types of attacks and proposed algorithms that demonstrate
high accuracy and promptness in detection. The results confirm the
feasibility of using simulation models in the research and development
of protection systems for wireless sensor networks.

\emph{Comparative analysis of full-scale and simulation approaches.}

\emph{Methodology for Comparing Physical and Simulation Models.}

To conduct a comparative analysis of physical and simulation models, the
following methodological steps were developed and applied:

Selection of Representative Attack Scenarios: Typical attack scenarios
were chosen, such as jamming, resource exhaustion attacks, replay
attacks, and spoofing. These scenarios cover a wide range of threats to
wireless sensor networks (WSNs).

Construction of Physical Models: The implementation of physical models
involved deploying sensor nodes in real operational conditions. Various
network topologies, such as mesh and star, were used to ensure a
diversity of conditions. Real devices were subjected to attack impacts
to collect data on network behavior.

Creation of Simulation Models: Simulation models were developed using
tools like NS-3 and Omnet++, allowing for accurate reproduction of the
conditions and behavior of sensor nodes, as well as attack impacts.
These models were configured to match the conditions of the physical
experiments.

Comparison of Physical and Simulation Models: The comparison was
conducted based on several criteria, including attack detection
accuracy, response time, resource consumption, and adaptability to
changes in network conditions (Table 6).
\end{multicols}

\newpage
{\bfseries Table 6 - Composition and structure of models}
\begin{longtable}[]{|@{}
  >{\raggedright\arraybackslash}p{(\columnwidth - 2\tabcolsep) * \real{0.5009}}|
  >{\raggedright\arraybackslash}p{(\columnwidth - 2\tabcolsep) * \real{0.4991}}@{}|}
\hline
\toprule
{\bfseries accuracy} & {\bfseries detection time} \\
\hline
\endfirsthead
\hline
\toprule
{\bfseries accuracy} & {\bfseries detection time} \\
\hline
\endhead
\hline
\bottomrule
\endfoot
\bottomrule
\endlastfoot
The model\textquotesingle s ability to correctly identify attacks and
minimize false positives. Accuracy was measured as the ratio of
correctly detected attacks to the total number of attacks. & The time
required to identify an attack after it has begun. Fast detection is
critical to minimizing the damage from attacks. \\
\hline
{\bfseries resource consumption} & {\bfseries adaptability} \\
\hline
The amount of computing and energy resources required to run detection
algorithms. This criterion is especially important for WSN nodes with
limited batteries and computing power. & The ability of the model and
algorithms to adapt to changes in network conditions and new types of
attacks. This includes the model\textquotesingle s ability to work
across different network topologies and load changes. \\
\hline
\end{longtable}


{\bfseries Table 7 - Results of comparative analysis, identified advantages
and disadvantages of models}

\begin{longtable}[]{|@{}
  >{\raggedright\arraybackslash}p{(\columnwidth - 4\tabcolsep) * \real{0.2220}}|
  >{\raggedright\arraybackslash}p{(\columnwidth - 4\tabcolsep) * \real{0.3678}}|
  >{\raggedright\arraybackslash}p{(\columnwidth - 4\tabcolsep) * \real{0.4102}}@{}|}
\hline
\toprule
& \textbf{Advantages:} & \textbf{Flaws:} \\
\hline
\endfirsthead
\hline
\toprule
& \textbf{Advantages:} & \textbf{Flaws:} \\
\hline
\endhead
\hline
\bottomrule
\endfoot
\bottomrule
\endlastfoot
\textbf{Full-scale models} & - Highly realistic: Full-scale models accurately reflect actual operating conditions, including physical disturbances and unforeseen factors. 

- Relevance of data: Data collected in field experiments are direct results of the operation of real devices and protocols. & - High costs: Deploying and maintaining full-scale models requires significant financial and time resources.

- Limited scalability: It is difficult and expensive to scale up field experiments to large networks or different scenarios. \\
\hline
\textbf{Simulation models} & - Flexibility and scalability: Simulation models are easily customized and scalable for different scenarios and network topologies.

- Low costs: Simulation experiments are carried out in a software environment, which significantly reduces costs compared to full-scale experiments. & - Limited realism: Simulation models may not fully account for all real-world physical and environmental factors, which may lead to variations in results.

- Dependence on model accuracy: The effectiveness of simulation models is highly dependent on the accuracy of reproducing real-world conditions and network behavior. \\
\hline
\end{longtable}

\begin{multicols}{2}

In general, both approaches have their strengths and weaknesses, but
their combination can provide the most complete and reliable analysis of
attacks on WSNs. Full-scale models provide high accuracy and data
relevance, while simulation models offer flexibility and
cost-effectiveness. The optimal solution is to use full-scale
experiments to verify and calibrate simulation models, which allows you
to combine the advantages of both approaches.

{\bfseries Conclusions.} Analysis of full-scale and simulation models and
algorithms for identifying attacks on WSNs shows that both approaches
have their own unique advantages and disadvantages. Full-scale models
provide highly accurate and up-to-date data because they reproduce
real-life network operating conditions. However, their use is associated
with high costs and limited scalability. Simulation models, in contrast,
offer flexibility and cost-effectiveness, allowing easy adjustment of
parameters and scale-up of experiments, but may not fully account for
all real-world physical factors.

Based on the presented data, we can conclude that the combined use of
full-scale and simulation approaches is optimal in the context of
ensuring the security of wireless sensor networks. The integration of
natural and simulation methods makes it possible to jointly use their
advantages, ensuring high accuracy and reliability of attack detection
algorithms. Using field data to calibrate and verify simulation models
plays an important role in achieving high accuracy in network
vulnerability analysis. Recommendations for improving models and
algorithms, including developing flexible and scalable simulation
models, improving attack detection algorithms, and regularly updating
models, are aimed at increasing the effectiveness of the attack
detection system. This combined use of methods and the development of
infrastructure for field experiments seem to be the most effective ways
to improve the security of wireless sensor networks in the face of
rapidly changing threats. This work by the staff of the International
Scientific Complex "Astana" is carried out with the financial support of
the Science Committee of the Ministry of Science and Higher Education of
the Republic of Kazakhstan (Grant No. AP19680345).
\end{multicols}

\begin{center}
  {\bfseries References}
  \end{center}

\begin{noparindent}

1. Lei Zou, Zidong Wang, Bo Shen, Hongli Dong, Guoping Lu, Encrypted
Finite-Horizon Energy-to-Peak State Estimation for Time-Varying Systems
Under Eavesdropping Attacks: Tackling Secrecy Capacity, IEEE/CAA Journal
of Automatica Sinica. -2023. Vol. 10(4). -P. 985-996. DOI 10.1109/JAS.2023.123393.

2. A. Adamova, T. Zhukabayeva and Y. Mardenov Machine Learning in
Action: An Analysis of its Applica-tion for Fault Detection in Wireless
Sensor Networks // 2023 IEEE International Conference on Smart
Information Systems and Technologies (SIST), Astana, Kazakhstan, 2023,
P.506-511. DOI 10.1109/SIST\\58284.2023.10223548.

3.G.P.S. Kumar, J. R. R. Kumar and S. R. T, Design of Secure
Communication Methodologies for WSN Assisted IoT Applications, 2022 2nd
Asian Conference on Innovation in Technology (ASIANCON), Ravet, India.//
-2022. -P.1-5. DOI 10.1109/ASIANCON55314.2022.9908931.

4. S.A.H. Antar et al. Classification of Energy Saving Techniques for
IoT-based Heterogeneous.Wireless Nodes //Procedia Comput. Sci. -2020.-
Vol.171. - P. 2590-2599

5. Kalaivanan Karunanithy et al. Cluster-tree based energy efficient
data gathering protocol for industrial automation using WSNs and IoT//
J. Indust. In format. Integrat.// -2020.- Vol.19. DOI 10.1016/j.jii.\\2020.100156

6. Ezhilarasi, M., Gnanaprasanambikai, L., Kousalya, A. et al. A novel
implementation of routing attack detection scheme by using fuzzy and
feed-forward neural networks //Soft Comput. -2023. Vol. 27.- P.
4157-4168. DOI 10.1007/s00500-022-06915-1

7. J. Panda, and S. Indu Localization and Detection of Multiple Attacks
in Wireless Sensor Networks Using Artificial Neural Network// Wireless
Communications and Mobile Computing.--2023.-Vol. 7. - P.1-29. DOI 10.1155/2023/2744706

8. Premkumar, M., Ashokkumar, S.R., Jeevanantham, V. et al. Scalable and
Energy Efficient Cluster Based Anomaly Detection Against Denial of
Service Attacks in Wireless Sensor Networks.// Wireless Pers Commun.
-2023.- Vol.129.- P.2669-2691.DOI 10.1007/s11277-023-10252-3

9. Roberts M.K., Ramasamy P. An improved high performance clustering
based routing protocol for wireless sensor networks in IoT// Telecommun
Syst. -2023.- Vol. 82.- P. 45-59. DOI 10.1007/s11235-022-00968-1

10. Yuvaraj, N., Raja, R.A., Karthikeyan, T. et al. Improved
Authentication in Secured Multicast Wireless Sensor Network (MWSN) Using
Opposition Frog Leaping Algorithm to Resist Man-in-Middle Attack//
Wireless Pers Commun. -2022.- Vol. 123. - P. 71715-1731
DOI 10.//1007/s11277-021-09209-1

11. M. Kumar and J. Ali, Adaptive Taylor-Sail Fish Optimization based
deep Learning for Detection of Black Hole and Sybil Attack in Wireless
Sensor Network// International Conference on Sustainable Computing and
Data Communication Systems (ICSCDS), Erode, India. -2023. - P.
1237-1244. DOI 10.1109/\\ICSCDS56580.2023.10104946.

12. Orman, A., Üstün, Y. \& Dener, M. Detailed analysis of sybil attack
in wireless sensor networks // International Journal of Sustainable
Engineering and Technology.-2023.-Vol.7(1)-P.41-54.
\\https://dergipark.org.tr/en/pub/usmtd/issue/78577/1305047

13. Y. Liu, X. Ma, L. Shu, G. P. Hancke, and A. M. Abu-Mahfouz, From
Industry 4.0 to Agriculture 4.0: current status, enabling technologies,
and research challenges// IEEE Transactions on Industrial Informatics.
-2021.-Vol. 17(6)- P. 4322-4334. DOI 10.1109/TII.2020.3003910

14. A. Williams, P. Suler,J. Vrbka Business process optimization,
cognitive decision-making algorithms, and artificial intelligence
data-driven internet of things systems in sustainable smart
manufacturing//Journal of Self-Governance and Management Economics.
-2020. Vol.8(4).-P. 39-48. DOI 10.22381/JSME8420204

15. Wendi Rabiner Heinzelman, Anantha Chandrakasan,Hari Balakrishnan.
Energy-efficient communication protocol for wireless microsensor
networks//In Proceedings of the 33rd annual Hawaii international
\\conference on system sciences. -2000. - P. 10.

16. Ibrahim Alrashdi, Ali Alqazzaz, Raed Alharthi, Esam Aloufi, Mohamed
A Zohdy Hua Ming. Fbad: Fog-based attack detection for iot healthcare in
smart cities //In 2019 IEEE 10th Annual Ubiquitous Computing,
Electronics \& Mobile Communication Conference (UEMCON). -2019. --P.
515-522. DOI 10.1109/UEMCON47517.2019.8992963

17. Chen M., Liu W., Zhang, N., Li J., Ren Y., Yi M., Liu A. GPDS: A
Multi-Agent Deep Reinforcement Learning Game for Anti-Jamming Secure
Computing in MEC Network// Expert Syst. Appl. -2022.- Vol.210. DOI
10.1016/j.eswa.2022.118394

18. Abdullah, Manal et al. Energy Efficient Ensemble K-means and SVM for
Wireless Sensor Network. International //Inter.J. of Computers and
Technology. -2013.- Vol.11(9).- P. 3034-3042. DOI10.24297/ijct.\\v11i9.3409

19. Desnitsky, V.; Kotenko, I.; Zakoldaev, D. Evaluation of Resource
Exhaustion Attacks against Wireless Mobile Devices// Electronics. -2019.
Vol. 8(5). DOI 10.3390/electronics8050500

20. Chhimwal, Mrs \& Rawat, Deepesh. (2021). Comparison between
Different Wireless Sensor Simulation Tools// IOSR Journal of Electronics
and Communication Engineering. -2021. Vol.5(2).-P.54-60.
\end{noparindent}

\emph{{\bfseries Information about the authors}}
\begin{noparindent}

E. Mardenov - Director of the Department of Information Technology at
Astana International University, Research Fellow at Astana International
Scientific Complex, Astana, Kazakhstan, e-mail: emardenov\\@gmail.com

Zh. Iztaev - Candidate of Pedagogical Sciences, Associate Professor at
M. Auezov South Kazakhstan State University, Shymkent, Kazakhstan,
e-mail: Zhalgasbek71@mail.ru;

Hu Wen-Cen - Professor, M. Auezov South Kazakhstan State University,
Leading Researcher at Astana International Scientific Complex, Astana,
Kazakhstan,e-mail: qbcba@bk.ru;

D. Mardenova - Lecturer at Astana International University, Junior
Research Fellow at Astana International Scientific Complex, Astana,
Kazakhstan. e-mail: mardenovadana@gmail.com;

D. Baumuratova - PhD, Senior Lecturer at Astana International
University, Junior Research Fellow at Astana International Scientific
Complex, Astana, Kazakhstan. e-mail: dilaram\_baumuratova@aiu.edu.kz
\end{noparindent}

\emph{{\bfseries Сведения об авторах}}
\begin{noparindent}

Е.Марденов - директор департмаента информационных технологий
Международного университета Астана, научный сотрудник Международного
научного комплекса Астана, Астана, Казахстан, e-mail:
emardenov@gmail.com;

Ж. Изтаев - Кандидат педагогических наук, доцент Южно-Казахстанский
государственный универси-тет им. М. Ауезова, Шымкент, Казахстан, e-mail:
Zhalgasbek71@mail.ru;

Ху Вен-Цен - профессор · Южно-Казахстанский государственный университет
им. М. Ауезова, веду-щий научный сотрудник Международного научного
комплекса Астана, Астана, Казахстан, e-mail: qbcba@bk.ru;

Д. Марденова - преподаватель Международного университета Астана, младший
научный сотрудник Международного научного комплекса Астана, Астана,
Казахстан, e-mail: mardenovadana@gmail.com;

Д. Баумуратова - PhD, старший преподаватель Международного университета
Астана, младший \\научный сотрудник Международного научного комплекса
Астана, Астана, Казахстан, e-mail:\\ dilaram\_baumuratova@aiu.edu.kz

\end{noparindent}
