\newpage
{\bfseries МРНТИ 20.20.20}

\sectionwithauthors{А.Т. Мазакова, Ш.А.Джомартова, Т.Ж. Мазаков, Тойкенов Г.Ч., М.С. Алиаскар}{ПАРАМЕТРИЧЕСКАЯ УСТОЙЧИВОСТЬ БЕСПИЛОТНОГО ЛЕТАТЕЛЬНОГО АППАРАТА}

\begin{center}
{\bfseries \textsuperscript{1}А.Т. Мазакова,
\textsuperscript{1}Ш.А.Джомартова, \textsuperscript{1,2}Т.Ж.
Мазаков\textsuperscript{🖂}, \textsuperscript{3}Тойкенов Г.Ч.,\textsuperscript{2}М.С. Алиаскар}

\textsuperscript{1}Казахский национальный университет имени аль-Фараби,
Алматы, Казахстан,

\textsuperscript{2}Международный инженерно-технологический университет,
Алматы, Казахстан,

\textsuperscript{3}Казахский национальный женский педагогический
университет, Алматы, Казахстан
\end{center}
Корреспондент-автор: \emph{tmazakov@mail.ru} \vspace{0.5cm}

Цель данной работы заключается в создании алгоритмов и программного
обеспечения для автомати-зированной проверки условий устойчивости
динамики беспилотного летательного аппарата, чья математическая модель
представлена системой обыкновенных дифференциальных уравнений. С помощью
системы аналитических вычислений было разработано приложение, которое
позволяет автоматически линеаризовать систему нелинейных уравнений и
строить характеристический полином линейной системы обыкновенных
дифференциальных уравнений. Предложенные методы и модели имеют высокую
практическую ценность для анализа устойчивости различных экономических и
техни-ческих систем.

{\bfseries Ключевые слова}: БПЛА, динамика, математическая модель,
управляемость, устойчивость.


\sectionheading{ҰШҚЫШСЫЗ ҰШУ АППАРАТЫНЫҢ ПАРАМЕТРЛІК ТҰРАҚТЫЛЫҒЫ}
\begin{center}
{\bfseries \textsuperscript{1}А.Т. Мазақова,
\textsuperscript{1}Ш.А.Джомартова, \textsuperscript{1,2}Т.Ж.
Мазаков\textsuperscript{🖂}, \textsuperscript{3}Тойкенов Г.Ч.,\textsuperscript{2}М.С. Әлиаскар}

\textsuperscript{1}Әл-Фараби атындағы Қазақ ұлттық университеті, Алматы,
Қазақстан,

\textsuperscript{2}Халықаралық инженерлік-технологиялық университеті,
Алматы, Қазақстан,

\textsuperscript{3}Қазақ ұлттық қыздар педагогикалық университеті,
Алматы, Қазақстан,

e-mail: tmazakov@mail.ru
\end{center}

Бұл жұмыстың мақсаты математикалық моделі қарапайым дифференциалдық
теңдеулер жүйесімен ұсынылған ұшқышсыз ұшу аппараты динамикасының
тұрақтылық шарттарын автоматтандырылған тексеру үшін алгоритмдер мен
бағдарламалық қамтамасыз етуді құру болып табылады. Аналитика-лық есептеу
жүйесінің көмегімен сызықтық емес теңдеулер жүйесін автоматты түрде
сызықтандыруға және қарапайым дифференциалдық теңдеулердің сызықтық
жүйесінің сипаттамалық көпмүшесін құруға мүмкіндік беретін қосымша
жасалды. Ұсынылған әдістер мен модельдер әртүрлі экономика-лық және
техникалық жүйелердің тұрақтылығын талдау үшін жоғары практикалық мәнге
ие.

{\bfseries Түйінді сөздер:} ҰҰА, динамика, математикалық модель,
басқарылғыштық, тұрақтылық.



\sectionheading{PARAMETRIC STABILITY OF AN UNMANNED AERIAL VEHICLE}
\begin{center}
{\bfseries \textsuperscript{1}A.T. Mazakova,
\textsuperscript{1}S.A.Dzhomartova, \textsuperscript{1,2}T.J.
Mazakov\textsuperscript{🖂}, \textsuperscript{3}Toikenov G.Ch., \textsuperscript{2}M.S. Aliaskar}

\textsuperscript{1}Al-Farabi Kazakh National University, Almaty,
Kazakhstan,

\textsuperscript{2}International engineering and Technology University,
Almaty, Kazakhstan,

\textsuperscript{3}Kazakh National Women\textquotesingle s teacher
training University, Almaty, Kazakhstan

e-mail: tmazakov@mail.ru
\end{center}

The aim of this research is to design algorithms and develop software
for the automated assessment of stability criteria in the dynamic
behavior of unmanned aerial vehicles (UAVs), modeled through systems of
ordinary differential equations (ODEs). By employing advanced symbolic
computation techniques, an application has been created that facilitates
the automatic linearization of nonlinear equation systems and the
derivation of the characteristic polynomial for the resulting linearized
ODE systems. The introduced methodologies and models provide substantial
practical significance in the evaluation of stability across a broad
range of both economic and technical systems.

{\bfseries Keywords:} UAV, dynamics, mathematical model, controllability,
stability.

\begin{multicols}{2}

{\bfseries Введение.} Беспилотная авиация стремительно развивается по всему
миру, что обусловлено спросом на лёгкие и относительно недорогие
летательные аппараты, обладающие высокой манёвренностью и способные
решать широкий спектр задач. Беспилотные летательные аппараты (БПЛА)
активно используются в военных операциях на глобальном уровне, а также
успешно выполняют гражданские задачи, включая линеаризацию {[}1-5{]}.

Исследование различных авиационных систем часто сводится к созданию
математических моделей, описываемых нелинейными обыкновен-ными
дифференциальными уравнениями. Для нелинейных систем до сих пор не
разработаны универсальные методы. Изучение подобных моделей требует
обязательного учета характера нелинейностей.

В общем виде нелинейная модель может быть представлена как система
обыкновенных дифференциальных уравнений:


\begin{equation}
  \frac{dq}{dt} = f( q,\Pr,t) + B(t)u\
\end{equation}


Где:

Pr -- вектор параметров размерности \emph{l;}

\(q\)\emph{(t)} -- вектор переменных модели размерности \emph{n};

\emph{u(t)} -- входы модели, задающие способы управления;

время \(t \in \lbrack 0,T\rbrack\). \(T\) -- задано.

Предполагается, что вектор-функция \(f(q,\theta,t)\) определена и
непрерывна вместе со своими частными производными по \(q\) .

К системе уравнений (1) добавляются начальные условия:


\begin{equation}
q(0) = q_{0}.
\end{equation}



На управление даются ограничения

\end{multicols}
\begin{equation}
(u(t) \in U = \left\{ u(t):\ u_{i}(t) \in C\lbrack\lbrack 0,T\rbrack; - L_{i} \leq u_{i}(t) \leq L_{i},\ i = \overline{1,m},t \in \lbrack 0,T\rbrack \right\}.
\end{equation}



{\bfseries Материалы и методы.} Нелинейной модели (1) соответствует
линеаризованная система дифферен-циальных уравнений:


\begin{equation}
\dot{q} = A\left( \Pr,t \right)q + B(t)u,
\end{equation} 

\begin{equation}
q(0) = q_{0},
\end{equation}

где \(A(\Pr\),t) - \emph{n*n --} матрица элементы которой зависят от
вектора параметров и времени \(t \in \lbrack 0,T\rbrack\).

Матрица \(A\left( \Pr,t \right)\) определяется из (1) следующим образом:

\begin{equation}
A\left( \Pr,t \right) = \ \frac{\partial f\left( q^{s},\Pr,t \right)}{\partial q}.
\end{equation} 
\begin{multicols}{2}

В (6) вектор-функция \(q^{s}\)\emph{(t)} (размерности \emph{n})
\(t \in \lbrack 0,T\rbrack\), предполагается заданной исходя из
требований к поставленной задаче.

Для задачи управляемости \(q^{s}\)\emph{(t)} может быть задана следующим
образом:

\begin{equation}
q^{s}(t) = const = q_{T\ }\, t \in \lbrack 0,T\rbrack.
\end{equation} 

где \(q_{T\ }\) представляет собой желаемое конечное состояние системы
(1).

Задача управляемости заключается в следующем: существует ли такое
управление \emph{u(t)}, которое удовлетворяет условию (3) и переводит
систему (4) из начального состояния (5) в заданное конечное состояние
(7) за определённое время \(T.\)

Так как $\lambda_k$ - корни алгебраического
характеристического уравнения, выведенного из уравнения 
\\ \boldmath$\det(A-\lambda_kE)= 0$ (\emph{Е} -- единичная матрица), определяют устойчивость системы, задача
устойчивости сводится к алгебраической проблеме: при каких условиях
корни этого уравнения будут иметь отрицательные вещественные части и
только такие корни. А. Гурвиц в 1885 году нашёл решение этой задачи,
предложив косвенный критерий устойчивости малых колебаний.

Современная теория устойчивости базируется на определении, введённом
Ляпуновым, которое является наиболее общим и определило не только объём
и содержание вопросов, рассматриваемых в современной теории
устойчивости, но и развитие качественных методов исследования
дифференциальных уравнений для решения этих задач.

Определение устойчивости по Ляпунову формулируется следующим образом.

Определение (устойчивости по Ляпунову). Для системы
\(\frac{\mathbf{dy}}{\mathbf{dt}}\mathbf{= Y}\left( \mathbf{y,t} \right)\)
движение \(\mathbf{y = f(t)}\) называется устойчивым, если для любого
{\varepsilon > 0} существует {\delta > 0} такое,
что из неравенства
\(\left\| \mathbf{y(}\mathbf{t}_{\mathbf{0}}\mathbf{)}\mathbf{-}\mathbf{f(}\mathbf{t}_{\mathbf{0}}\mathbf{)} \right\|{< \delta}\)
следует неравенство
\(\left\| \mathbf{y(t)}\mathbf{-}\mathbf{f(t)} \right\|\mathbf< \pmb{\varepsilon}\)
при \(\mathbf{t}\mathbf{\geq}\mathbf{t}_{\mathbf{0}}\).



\end{multicols}


Рассмотрим следующую математическую модель динамики БПЛА


\begin{equation}
\begin{matrix}
\dot{V} = g(n_{xa} - sin\Theta) \\
\dot{\Theta} = g(n_{ya}cos\gamma - cos\Theta)/V \\
\begin{matrix}
\dot{\Psi} = - gn_{ya}sin\gamma/(Vcos\Theta) \\
\dot{x} = Vcos\Theta\cos\Psi \\
\begin{matrix}
\dot{y} = Vsin\Theta \\
\dot{z} = - Vcos\Theta\sin\Psi
\end{matrix}
\end{matrix}
\end{matrix} 
\end{equation} 


\begin{equation}
n_{xa} = \frac{Pcos\alpha - X_{a}}{mg},\ n_{ya} = \frac{Psin\alpha + Y_{a}}{mg}
\end{equation} 

\begin{multicols}{2}

Здесь:

\emph{x, y, z} -- координаты центра масс самолета в нормальной земной
системе координат;

\emph{V} -- скорость полета;

\emph{ϴ} - угол наклона траектории, \emph{Ψ} -- угол курса, \emph{α} --
угол атаки, \emph{γ} -- угол крена;

\emph{P} -- тяга двигателя;

\(X_{a}\) -- аэродинамическое сопротивление;

\(Y_{a}\) -- аэродинамическая подъемная сила;

\emph{m} -- масса самолета;

\emph{g} -- ускорение свободного падения;

\(n_{xa}\) -- продольная перегрузка;

\(n_{ya}\) -- поперечная перегрузка (в поточных осях координат)
{[}6-7{]}.

В качестве управляющих переменных в (8) принимается перегрузки
\(n_{xa}\), \(n_{ya}\) и угол крена \emph{γ}.

\end{multicols}

Введем обозначения:
\begin{longtable}[]{@{}
  >{\raggedright\arraybackslash}p{(\columnwidth - 2\tabcolsep) * \real{0.9342}}
  >{\raggedright\arraybackslash}p{(\columnwidth - 2\tabcolsep) * \real{0.0658}}@{}}
\begin{minipage}[b]{\linewidth}\raggedright
\[q = \begin{bmatrix}
V \\
\Theta \\
\begin{matrix}
\Psi \\
x \\
\begin{matrix}
y \\
z
\end{matrix}
\end{matrix}
\end{bmatrix},\ q_{0} = \begin{bmatrix}
V_{0} \\
\Theta_{0} \\
\begin{matrix}
\Psi_{0} \\
x_{0} \\
\begin{matrix}
y_{0} \\
z_{0}
\end{matrix}
\end{matrix}
\end{bmatrix},\ q_{1} = \begin{bmatrix}
V_{1} \\
\Theta_{1} \\
\begin{matrix}
\Psi_{1} \\
x_{1} \\
\begin{matrix}
y_{1} \\
z_{1}
\end{matrix}
\end{matrix}
\end{bmatrix}\]
\end{minipage} & \begin{minipage}[b]{\linewidth}\raggedright
(10)
\end{minipage} \\
\end{longtable}


Далее исследуется проблема устойчивости линеаризованной модели вида:


\begin{equation}
  \frac{\mathbf{dq}}{\mathbf{dt}} = \mathbf{A}(\mathbf{\Pr})\mathbf{q}
  \tag{11}
\end{equation}




где коэффициенты матрицы \emph{А(Pr)} зависят от параметров \emph{Pr},
характеризующих механические пара-метры (такие как вес, метрические
характеристики, инерционность и т.п.).

\emph{Определение.} Систему (11) с матрицей \emph{А}, элементы которой
зависят от параметров \emph{Pr}, назовем параметрически асимптотически
устойчивой по Ляпунову, если для некоторого \emph{Pr} существует реше-ние
\(\mathbf{q(}\mathbf{Pr,}\mathbf{t}\mathbf{),}\text{     }\mathbf{t}\mathbf{\in}\mathbf{\lbrack 0,}\mathbf{\infty}\mathbf{)}\),
такое что справедливо утверждение:
\begin{enumerate}
  \def\labelenumi{\arabic{enumi})}
  \setlength{\itemindent}{1cm}
  \item
    для любых \(\varepsilon\) \textgreater 0 и
    \(\mathbf{t}_{0} \in [0, \infty)\)
    существует
    \(\delta = \delta(\varepsilon, t_{0})\)
    такое, что для всех решений
    \(\mathbf{q} = \mathbf{q}(\mathbf{Pr}, t)\), удовлетворяющих
    условию
    \(\left\| \mathbf{q}(t_{0}) \right\| < \delta\),
    справедливо неравенство
    \(\left\| \mathbf{q}(\mathbf{Pr}, t) \right\| < \varepsilon\),
    при
    \(t \in [t_{0}, \infty)\);
  \item
    для любого
    \(t_{0} \in [0, \infty)\)
    существует
    \(\lambda = \lambda(t_{0})\)
    такое, что все решения
    \(\mathbf{q} = \mathbf{q}(\mathbf{Pr}, t)\), удовлетворя-ющие
    условию
    \(\left\| \mathbf{q}(t_{0}) \right\| < \lambda\),
    обладают свойством:
  \end{enumerate}
  
\begin{equation*}
  \lim_{t \to \infty} \|q(Pr,t)\| = 0
\end{equation*}
    

Как известно, для определения устойчивости системы (11) анализируются
свойства собственных значений матрицы. Аналогично, для определения
параметрической устойчивости матрицы строится характеристический полином
с коэффициентами, зависящими от параметров \emph{Pr}:



\begin{equation}
  \phi_{A}(\lambda) = \det(\lambda E - A(\text{Pr})) = p_n \lambda^n + p_{n-1} \lambda^{n-1} + \dots + p_0
  \tag{12}
\end{equation}





где
\(\mathbf{p}_{\mathbf{i}}\mathbf{,i =}\overline{\mathbf{0,n}}\mathbf{-}\)зависят
от параметров \emph{Pr}.


\emph{Необходимое условие устойчивости:} все коэффициенты
характеристического полинома (12) для фиксированного значения \emph{Pr}
должны находиться в положительной.

Составим матрицу Гурвица

\begin{center}
\(\mathbf{M =}\begin{bmatrix} 
\mathbf{p}_{\mathbf{1}} & \mathbf{p}_{\mathbf{0}} & \mathbf{0} & \mathbf{0} & \mathbf{...} & \mathbf{0} \\
\mathbf{p}_{\mathbf{3}} & \mathbf{p}_{\mathbf{2}} & \mathbf{p}_{\mathbf{1}} & \mathbf{p}_{\mathbf{0}} & \mathbf{...} & \mathbf{0} \\
\mathbf{...} & \mathbf{...} & \mathbf{...} & \mathbf{...} & \mathbf{...} & \mathbf{...} \\
\mathbf{p}_{\mathbf{2}\mathbf{n}\mathbf{-}\mathbf{1}} & \mathbf{p}_{\mathbf{2}\mathbf{n}\mathbf{-}\mathbf{2}} & \mathbf{p}_{\mathbf{2}\mathbf{n}\mathbf{-}\mathbf{3}} & \mathbf{p}_{\mathbf{2}\mathbf{n}\mathbf{-}\mathbf{4}} & \mathbf{...} & \mathbf{p}_{\mathbf{n}}
\end{bmatrix}\),
\end{center}
где принято \(\mathbf{p}_{\mathbf{j}}\mathbf{= 0}\) при
\(\mathbf{j < 0}\) и \(\mathbf{j > n}\).

Обозначим через
\(\Delta_{1}, \Delta_{2}, \ldots, \Delta_{n}\)
главные диагональные миноры матрицы \(M\):

\[
\begin{array}{rl}
\Delta_{1} & = p_{1}, \\
\Delta_{2} & = \left| \begin{matrix}
p_{1} & p_{0} \\
p_{3} & p_{2}
\end{matrix} \right|, \\
\Delta_{n} & = \left| M \right| = p_{n} \Delta_{n-1}
\end{array}
\]


которые в свою очередь являются функциями от параметров \emph{Pr}.

\emph{Критерий параметрической устойчивости Гурвица:} для того чтобы
некоторого значения параметра \emph{Pr} собственные значения матрицы
A(Pr) были
\(\mathbf{Re}{\lambda}_{\mathbf{j}}\left( \mathbf{A(}\mathbf{\Pr}\mathbf{)} \right)\mathbf{< 0,j =}\overline{\mathbf{1,n}}\)
необходимо и достаточно, чтобы главные диагональные миноры
\({\Delta}_{{1}}{,}{\Delta}_{{2}}{,...,}{\Delta}_{{n}}\)
матрицы \(\mathbf{M}\) находились в правой полуплоскости, т.е.
\({\Delta}_{{j}}{> 0,j =}\overline{{1,n}}\).


Для автоматизированного построения характеристического полинома,
зависящего от параметров \emph{Pr}, на языке MatLab был реализован
алгоритм Леверье-Фадеева, использующий методы компьютерной алгебры. На
основе полученного характеристического полинома формируется матрица
Гурвица, также зависящая от параметров \emph{Pr}. Далее для этой
параметрической матрицы Гурвица вычисляются главные миноры и их
определители.

Используя программу, разработанную на MatLab и представленную ниже,
можно получить следую-щие результаты линеаризации системы (8).
\begin{lstlisting}
  syms q1 q2 q3 q4 q5 q6 q A B
q0 = [2.2 1.5 1.4 1.3 1.1 1.0];
q = [q1 q2 q3 q4 q5 q6];
t = [0 10];   
y0 = q0;
g = 9.8; 
P = 2000; 
m = 3; 
alfa = 30; 
gamma = 45; 
Xa = 0.32; 
Ya = 0.4;
nxa = (P*cos(alfa) - Xa)/(m*g);  
nya = (P*sin(alfa) + Ya)/(m*g);
F = [g*(nxa-sin(q2)); g*(nya*cos(gamma)-cos(q2))/q1; -g*nya*sin(gamma)/(q1*cos(q2)); q1*cos(q2)*cos(q3); q1*sin(q2); -q1*cos(q2)*sin(q3)];
for i = 1:length(q0)
  for j = 1:length(q0)
    A(i,j) = diff(F(i),q(j));
  end
end
B = subs(A,q,q0);
disp(vpa(A,3));
disp(vpa(B,3));

\end{lstlisting}




Представим результат выполнения программы в следующем виде:

\begin{longtable}{@{}
  >{\raggedright\arraybackslash}p{(\columnwidth - 2\tabcolsep) * \real{0.9342}}
  >{\raggedright\arraybackslash}p{(\columnwidth - 2\tabcolsep) * \real{0.0658}}@{}}
\begin{minipage}[b]{\linewidth}\raggedright
\[A(q) = \frac{\partial f(q,t)}{\partial q} = \begin{pmatrix}
\begin{matrix}
a_{11} & a_{12} \\
a_{21} & a_{22}
\end{matrix} & \begin{matrix}
\ldots & a_{16} \\
\ldots & a_{26}
\end{matrix} \\
\begin{matrix}
\ldots & \ldots \\
a_{61} & a_{62}
\end{matrix} & \begin{matrix}
\ldots & \ldots \\
\ldots & a_{66}
\end{matrix}
\end{pmatrix}\]
\end{minipage} & \begin{minipage}[b]{\linewidth}\raggedright
(13)
\end{minipage} \\

\end{longtable}


Выпишем только ненулевые элементы матрицы \(A(q)\):

\(a_{12} = - 9.80*cos(q2)\),
\(a_{21} = - 1.*( - 346. - 9.80*cos(q2))/q1\hat{}2\),

\(a_{22} = \ 9.80*sin(q2)/q1\) , \(a_{31} = - 560./q1\hat{}2/cos(q2)\),

\(a_{32} = 60./q1/cos(q2)\hat{}2*sin(q2)\),
\(a_{41} = \ cos(q2)*cos(q3)\),

\(a_{42} = - \ x1*sin(x2)*cos(x3)\), \(a_{43} = - q1*cos(q2)*sin(q3)\),

\(a_{51} = \ sin(q2)\), \(a_{52} = q1*cos(q2)\),
\(a_{61} = - cos(q2)*sin(q3)\),

\(a_{62} = \ q1*sin(q2)*sin(q3)\), \(a_{63} = \  - q1*cos(q2)*cos(q3)\).

Вычислим значение матрицы \(A(q)\) в точке
\(q_{0} = \ \lbrack 2.2\ 1.5\ 1.4\ 1.3\ 1.1\ 1.0\rbrack.\)
\begin{longtable}{@{}
  >{\raggedright\arraybackslash}p{(\columnwidth - 2\tabcolsep) * \real{0.9342}}
  >{\raggedright\arraybackslash}p{(\columnwidth - 2\tabcolsep) * \real{0.0658}}@{}}
\begin{minipage}[b]{\linewidth}\raggedright
\(A\left( q_{0} \right) = \begin{pmatrix}
\begin{matrix}
0 \\
\begin{matrix}
0.014 \\
\begin{matrix}
0.0. \\
\begin{matrix}
0.012 \\
\begin{matrix}
 - 0.959 \\
 - 0.069
\end{matrix}
\end{matrix}
\end{matrix}
\end{matrix}
\end{matrix} & \begin{matrix}
 - 0.707 \\
\begin{matrix}
0.453 \\
\begin{matrix}
0.0 \\
\begin{matrix}
 - 0.37 \\
\begin{matrix}
0.156 \\
2.162
\end{matrix}
\end{matrix}
\end{matrix}
\end{matrix}
\end{matrix} & \begin{matrix}
\begin{matrix}
0 \\
\begin{matrix}
0 \\
\begin{matrix}
0 \\
\begin{matrix}
0.153 \\
\begin{matrix}
0 \\
0.026
\end{matrix}
\end{matrix}
\end{matrix}
\end{matrix}
\end{matrix} & \begin{matrix}
\begin{matrix}
0 \\
\begin{matrix}
0 \\
\begin{matrix}
0 \\
\begin{matrix}
0 \\
\begin{matrix}
0 \\
0
\end{matrix}
\end{matrix}
\end{matrix}
\end{matrix}
\end{matrix} & \begin{matrix}
\begin{matrix}
0 \\
\begin{matrix}
0 \\
\begin{matrix}
0 \\
\begin{matrix}
0 \\
\begin{matrix}
0 \\
0
\end{matrix}
\end{matrix}
\end{matrix}
\end{matrix}
\end{matrix} & \begin{matrix}
0 \\
\begin{matrix}
0 \\
\begin{matrix}
0 \\
\begin{matrix}
0 \\
\begin{matrix}
0 \\
0
\end{matrix}
\end{matrix}
\end{matrix}
\end{matrix}
\end{matrix}
\end{matrix}
\end{matrix}
\end{matrix}
\end{pmatrix}\).
\end{minipage} & \begin{minipage}[b]{\linewidth}\raggedright
(14)
\end{minipage} \\

\end{longtable}


В результате работы программы был получен аналитический вид матрицы
\emph{А} и её значение при
\(q_{0} = \ \lbrack 2.2\ 1.5\ 1.4\ 1.3\ 1.1\ 1.0\rbrack\).

В дальнейшем эта матрица \emph{А}, представленная в виде (3.10), была
использована для анализа устойчи-вости и управляемости модели БПЛА в
заданной точке.
\begin{multicols}{2}

{\bfseries Результаты и обсуждение}. Процесс линеаризации модели (6) для
исходной нелинейной системы (1) оказывается весьма трудоёмким, особенно
при увеличении размерности 𝑛, и становится практически неосуществимым
при n \textgreater{} 3. Более того, при проведении линеаризации часто
проявляется «человеческий фактор», что снижает точность вычислений и не
гарантирует корректности результатов. Это подчеркивает актуальность
задачи автоматизации процесса линеаризации нелинейных моделей {[}8{]}.

Для решения этой проблемы предлагается использовать системы компьютерной
алгебры (СКА), такие как MatLab {[}9-10{]}. Эти системы предоставляют
широкий спектр возможностей для работы с алгебраическими выражениями: от
базовых операций, таких как вычисление и дифференцирование, до более
сложных процедур, включая разложения в ряды и интегрирование. СКА
находят широкое применение в таких областях, как аэрокосмическая
промышленность.

Критерии параметрической устойчивости, изложенные выше, также были
реализованы в приложении, разработанном авторами {[}11{]}.

Результаты численных расчётов полностью согласуются с экспериментальными
данными. Кроме того, результаты сохраняются в текстовые файлы, что
позволяет визуализировать одномерные графики динамики БПЛА с помощью
MatLab, для которого была написана специальная программа.

В качестве перспективного направления можно рассматривать использование
интерваль-ной математики для анализа условий устойчивости БПЛА
{[}12-14{]}.

{\bfseries Выводы.} Статья посвящена исследованию управления и устойчивости
беспилотных летательных аппаратов (БПЛА) с использованием математических
моделей. Основное внимание уделяется линеаризации нелинейных систем
дифференциальных уравнений, описывающих динамику БПЛА, и анализу
устойчивости с применением критерия Гурвица.

Выводы по статье можно сформулировать следующим образом:

\begin{itemize}
  \setlength{\itemindent}{1cm}
\item
  \emph{Значимость беспилотной авиации.} Беспилотные летательные
  аппараты играют важную роль как в военной, так и в гражданской сферах,
  что обусловливает необходимость развития математических моделей для их
  эффективного управления.
\item
  \emph{Нелинейные модели}. Для моделирования динамики БПЛА часто
  используются системы нелинейных дифференциальных уравнений.
  Линеаризация таких моделей является сложной задачей, особенно для
  систем высокой размерности.
\item
  \emph{Автоматизация процесса линеаризации.} Для повышения точности и
  эффективности расчётов предлагается автоматизация процесса
  линеаризации с использованием систем компьютерной алгебры, таких как
  MatLab.
\item
  \emph{Анализ устойчивости.} Устойчивость БПЛА исследуется с
  использованием матрицы Гурвица и критерия параметрической
  устойчивости. Применение метода позволяет выявить условия, при которых
  система будет асимптотически устойчива.
\item
  \emph{Практическая значимость.} Разработанная программа на MatLab
  позволяет автоматизиро-вать процесс построения характеристических
  полиномов и анализа устойчивости, что подтверждено экспериментальными
  результата-ми.
\item
  \emph{Перспективы.} Одним из направлений дальнейших исследований может
  стать использование интервальной математики для более точного анализа
  устойчивости БПЛА в условиях неопределённости.
\end{itemize}

Таким образом, статья подчеркивает важность применения современных
методов математического моделирования и автоматизации для анализа
устойчивости и управляемости БПЛА.

\emph{{\bfseries Финансирование}. Работа выполнена за счет средств НИИ
математики и механики при КазНУ имени аль-Фараби и грантового
финансирования научных исследований на 2023--2025 годы по проекту
AP19678157.}
\end{multicols}


\begin{center}
  {\bfseries Литература}
  \end{center}


\begin{noparindent}
\begin{enumerate}
\def\labelenumi{\arabic{enumi}.}
\item
  Логинов А.А. Актуальность использования беспилотных летательных
  аппаратов // Актуальные проб-лемы авиации и космонавтики.- 2015.-Т. 1.-
  С. 704-705
\item
  Хуснутдинов Т.Д., Щербакова А.В., Комарова А.П., Рублевская Е.В.,
  Решетников А.Ю. Перспективы использования беспилотных летательных
  аппаратов в инновационных проектах// Актуальные проб-лемы авиации и
  космонавтики.- 2017.- Т. 3.- С. 139-141
\end{enumerate}

3.Лю Ш., Ли., Тан Ц., Ву Ш., Годье Ж.-Л. Разработка беспилотных
транспортных средств.- М.:ДМК Пресс.- 2022.-246 с. ISBN
978-5-97060-969-9

4.Августов Л.И и др. Навигация летательных аппаратов в околоземном
пространстве. -- М.: ООО «Научтехлитиздат», 2015. -592с. ISBN
978-5-93728-146-3.

5.Бернс В.А., Долгополов А.В. и др. Экспериментальный модальный анализ
летательных аппаратов. -- Новосибирск: НГТУ, 2023. -- 328 с. ISBN
978-5-7782-3209-9

6.Танг Тхань Лам Системный анализ и оптимизация режимов полета для
управления летательным аппаратом // Автореф. диссер. канд. техн. наук,
спец. 05.13.01, Москва, 2015. -- 155 с.

7. Mazakova A., Jomartova Sh.,Vfzakov T., Shormanov., Amirkhanov B.
Controllability of an unmanned aerial vehicle.// 2022 IEEE 7th
International Energy Conference (ENERGYCON) - C.1-5.
DOI~10.1109/\\ENERGYCON53164.2022.9830244

8.Mazakova A., Jomartova Sh., Wójcik W., Mazakov T., Ziyatbekova G.
Automated Linearization of a System of Nonlinear Ordinary Differential
Equations// Intl.Journal of electronics and
Telecommunications.-2023.-Vol.69(4).-\\ P.655-660. DOI:
10.24425/ijet.2023.147684

9.Смоленцев Н.К. MatLAb. Программирование на Visual C\#, Borland
JBuilder, VBA. -- М.: ДМК Пресс, 2009. -- 464с.

10.Дьяконов В., Абраменкова И. MATLAB. Обработка сигналов и изображений.
Специальный спра-вочник. -- СПб.: Питер, 2002. - 608 с. ISBN
5-318-00667-1.

11.А.с. №45318 от 2.05.2024 г. Мазақова Ә.Т., Мазаков Т.Ж., Джомартова
Ш.А. Определение устой-чивости БПЛА. Программа для ЭВМ.

12.Issimov N., Mazakov T., Mamyrbayev O.,Ziyatbekova G. Application of
fuzzy and interval analysis to the study of the prediction and control
model of the eridemiologic situation// Journal of Theoretical and
Applied information Technology.-Vol.96(14).- P.4358-4368

13. Dzhomartova Sh.A., Mazakov T.Zh, Karymsakova N.T.,Z haydarov A.M.
Comparison of Two Interval Arithmetic// Applied Mathematical
Sciences.-2014.-Vol.8(72).- P.-3593-3598. DOI.10.12988/ams.2014.\\44301

14.Mazakov T.,Wójcik W.,Jomartova Sh., Karymsakova N.,Ziyatbekova G.,
Tursynbai A. The Stability Interval of the Set of Linear
System//Intl.Journal of electronics and
Telecommunications.-2021.-Vol.67(2).- P.155-161. DOI:
10.24425/ijet.2021.135958
\end{noparindent}
 
\begin{center}
  {\bfseries References}
  \end{center}


\begin{noparindent}

1.Loginov A.A. Aktual\textquotesingle nost\textquotesingle{}
ispol\textquotesingle zovanija bespilotnyh letatel\textquotesingle nyh
apparatov // Aktual\textquotesingle nye problemy aviacii i
kosmonavtiki.- 2015.-T. 1.- S. 704-705.{[}in Russ.{]}

2.Husnutdinov T.D., Shherbakova A.V., Komarova A.P., Rublevskaja E.V.,
Reshetnikov A.Ju. Perspektivy ispol\textquotesingle zovanija bespilotnyh
letatel\textquotesingle nyh apparatov v innovacionnyh proektah//
Aktual\textquotesingle nye problemy aviacii i kosmonavtiki.- 2017.- T.
3.- S. 139-141.{[}in Russ.{]}

3.Lju Sh., Li., Tan C., Vu Sh., God\textquotesingle e Zh.-L. Razrabotka
bespilotnyh transportnyh sredstv.- M.:DMK Press.- 2022.-246 s. ISBN
978-5-97060-969-9.{[}in Russ.{]}

4.Avgustov L.I i dr. Navigacija letatel\textquotesingle nyh apparatov v
okolozemnom prostranstve. -- M.: OOO \\«Nauchtehlitizdat», 2015. -592s.
ISBN 978-5-93728-146-3. {[}in Russ.{]}

5.Berns V.A., Dolgopolov A.V. i dr. Jeksperimental\textquotesingle nyj
modal\textquotesingle nyj analiz letatel\textquotesingle nyh apparatov.
\\-- Novosibirsk: NGTU, 2023. -- 328 s. ISBN 978-5-7782-3209-9.{[}in
Russ.{]}

6.Tang Than\textquotesingle{} Lam Sistemnyj analiz i optimizacija
rezhimov poleta dlja upravlenija letatel\textquotesingle nym apparatom
// Avtoref. disser. kand. tehn. nauk, spec. 05.13.01, Moskva, 2015.-155
s. {[}in Russ.{]}

7. Mazakova A., Jomartova Sh.,Vfzakov T., Shormanov., Amirkhanov B.
Controllability of an unmanned aerial vehicle.// 2022 IEEE 7th
International Energy Conference (ENERGYCON) - C.1-5.
DOI~10.1109/\\ENERGYCON53164.2022.9830244

8.Mazakova A., Jomartova Sh., Wójcik W., Mazakov T., Ziyatbekova G.
Automated Linearization of a System of Nonlinear Ordinary Differential
Equations// Intl.Journal of electronics and
Telecommunications.-2023.-Vol.69(4).- P.655-660. \\DOI:
10.24425/ijet.2023.147684

9.Smolencev N.K. MatLAb. Programmirovanie na Visual C\#, Borland
JBuilder, VBA. -- M.: DMK Press, 2009. -- 464 s. {[}in Russ.{]}

10.D\textquotesingle jakonov V., Abramenkova I. MATLAB. Obrabotka
signalov i izobrazhenij. Special\textquotesingle nyj spravochnik. --
SPb.: Piter, 2002. - 608 s. ISBN 5-318-00667-1. {[}in Russ.{]}

11.A.s. №45318 ot 2.05.2024 g. Mazaқova Ә.T., Mazakov T.Zh., Dzhomartova
Sh.A. Opredelenie ustojchi-vosti BPLA. Programma dlja JeVM. {[}in
Russ.{]}

12.Issimov N., Mazakov T., Mamyrbayev O.,Ziyatbekova G. Application of
fuzzy and interval analysis to the study of the prediction and control
model of the eridemiologic situation// Journal of Theoretical and
Applied information Technology.-Vol.96(14).- P.4358-4368

13. Dzhomartova Sh.A., Mazakov T.Zh, Karymsakova N.T.,Z haydarov A.M.
Comparison of Two Interval Arithmetic// Applied Mathematical
Sciences.-2014.-Vol.8(72).- P.-3593-3598. DOI.10.12988/ams.\\2014.44301

14.Mazakov T.,Wójcik W.,Jomartova Sh., Karymsakova N.,Ziyatbekova G.,
Tursynbai A. The Stability Interval of the Set of Linear
System//Intl.Journal of electronics and
Telecommunications.-2021.-Vol.67(2).- P.155-161. DOI:
10.24425/ijet.2021.135958
\end{noparindent}


\emph{{\bfseries Сведение об авторах}}
\begin{noparindent}

Мазақова А.Т. -- докторант Казахского национального университета
им.аль-Фараби, Алматы,\\ Казахстан, e-mail: aigerym97@mail.ru;

Джомартова Ш.А. - доктор технических наук, доцент, Казахский
национальный университет им. аль-Фараби, Алматы, Казахстан, e-mail:
jomartova@mail.ru;

Мазаков Т.Ж. -- доктор физико-математических наук, профессор, Казахский
национальный универси-тет им. аль-Фараби, Алматы, Казахстан, e-mail:
tmazakov@mail.ru;

Тойкенов Г.Ч. - кандидат физико-математических наук, доцент, Казахский
национальный женский педагогический университет, Алматы, Казахстан,
e-mail: gumyrbektoike@mail.ru;

Әлиасқар М.С. - докторант Казахского национального университета
им.аль-Фараби, Алматы, \\Казахстан, e-mail: m.alyasqar@gmail.ru
\end{noparindent}
 
\emph{{\bfseries Information about the authors}}


\begin{noparindent}
Mazakova A.T. - PhD student of the Al-Farabi Kazakh National University,
Almaty, Kazakhstan, \\e-mail: aigerym97@mail.ru;

Jomartova Sh.A. - Doctor of Technical Sciences, Associate Professor,
Al-Farabi Kazakh National University, Almaty, Kazakhstan, e-mail:
jomartova@mail.ru;

Mazakov T.Zh. -- Doctor of Physical and mathematical sciences,
professor, Al-Farabi Kazakh National University, Almaty, Kazakhstan,
e-mail: tmazakov@mail.ru;

Tokenov G.Ch. - Candidate of Physical and Mathematical Sciences,
Associate Professor, Kazakh National Women\textquotesingle s Pedagogical
University, Almaty, Kazakhstan, e-mail: gumyrbektoike@mail.ru;

Aliaskar M.S. - Lecturer at the International University of Engineering
and Technology, Almaty, Kazakhstan,\\e-mail: m.alyasqar@gmail.ru
\end{noparindent}
