\newpage
{\bfseries МРНТИ 65.63.35}

{\bfseries РАЗРАБОТКА И ОПТИМИЗАЦИЯ ТЕХНОЛОГИИ СУБЛИМАЦИОННОЙ СУШКИ
КОБЫЛЬЕГО МОЛОКА: АНАЛИЗ СОСТАВА, МЕТОДОВ И ВЫХОДОВ ПРОДУКЦИИ}

{\bfseries \textsuperscript{1}А.Б. Рахматуллина, \textsuperscript{1,
2}Ф.Т.Диханбаева, \textsuperscript{1}Д.А.
Тлевлесова\textsuperscript{🖂},}

{\bfseries \textsuperscript{2}М.К. Изтилеуов, \textsuperscript{3}Б.К.
Калемшарив}

\textsuperscript{1}Институт механики и машиноведения имени академика
У.А. Джолдасбекова, Алматы, Казахстан,

\textsuperscript{2}Алматинский технологический университет, Алматы,
Казахстан,

\textsuperscript{3}Казахский агротехнический исследовательский
университет им. С. Сейфуллина, Астана, Казахстан

{\bfseries \textsuperscript{🖂}}Корреспондент-автор: tlevlessova@gmail.com

В статье представлено исследование по разработке и оптимизации
технологии производства сухих молочных продуктов из кобыльего молока с
использованием метода сублимационной сушки. Основной целью исследования
являлось определение оптимальных условий сублимационной сушки для
максимального сохранения питательных и биологически активных компонентов
молока. В ходе экспериментов был проведен детальный анализ
физико-химических свойств кобыльего молока, включая содержание белка,
жира, лактозы и минеральных солей. Исследовались различные температурные
режимы сублимационной сушки, чтобы определить их влияние на выход и
качество сухого молока.

Результаты показали, что оптимальная температура сушки составляет около
35°C, при которой достигается максимальный выход сухого молока с
минимальными потерями питательных веществ. Выход сухого молока составил
67.14 грамм на 600 грамм жидкого молока. Полученные данные были
сопоставлены с существующими литературными данными, что подтвердило
эффективность выбранного метода.

В статье также обсуждаются технологические параметры сублимационной
сушки и их влияние на качество конечного продукта. На основе полученных
результатов разработаны рекомендации по оптимизации процесса сушки
кобыльего молока. Дальнейшие исследования будут направлены на улучшение
технологических процессов и увеличение выхода готовой продукции.

{\bfseries Ключевые слова:} кобылье молоко, сублимационная сушка,
оптимизация технологии, питательные вещества, физико-химический анализ.

{\bfseries БИЕ СҮТІН МҰЗДАТЫП КЕПТІРУ ТЕХНОЛОГИЯСЫН ӘЗІРЛЕУ ЖӘНЕ
ОҢТАМАЛАНДЫРУ: ҚҰРАМЫН, ӘДІСТЕРІН ЖӘНЕ ӨНІМ ШЫҒЫМЫН ТАЛДАУ}

{\bfseries \textsuperscript{1}А.Б. Рахматуллина,
\textsuperscript{1,2}Ф.Т.Диханбаева, \textsuperscript{1}Д.А.
Тлевлесова\textsuperscript{🖂},}

{\bfseries \textsuperscript{2}М.Қ. Ізтілеуов, \textsuperscript{3}Б.Қ.
Қалемшарив}

\textsuperscript{1}Академик Ө.Ә. Жолдасбеков атындағы Механика және
инженерия институты, Алматы, Қазақстан,

\textsuperscript{2}Алматы технологиялық университеті, Алматы, Қазақстан,

\textsuperscript{3}С.Сейфуллина атындағы Қазақ агротехникалық зерттеу
университеті. Астана, Қазақстан,

e-mail: tlevlessova@gmail.com

Мақалада мұздатып кептіру әдісімен бие сүтінен құрғақ сүт өнімдерін
өндіру технологиясын жасау және оңтайландыру бойынша зерттеу берілген.
Зерттеудің негізгі мақсаты сүттің тағамдық және биологиялық белсенді
компоненттерін барынша сақтау үшін мұздату әдісімен кептірудің оңтайлы
шарттарын анықтау болды. Тәжірибе барысында бие сүтінің құрамындағы
ақуыз, май, лактоза және минералды тұздарды қамтитын физика-химиялық
қасиеттеріне егжей-тегжейлі талдау жасалды. Құрғақ сүттің шығымы мен
сапасына әсерін анықтау үшін мұздатып кептірудің әртүрлі температуралық
жағдайлары зерттелді.

Нәтижелер кептірудің оңтайлы температурасы 35°C шамасында екенін
көрсетті, бұл қоректік заттардың аз шығынымен сүт ұнтағының максималды
шығымына қол жеткізеді. Құрақ сүттің шығымы 600 грамм сұйық сүттен 67,14
грамм болды. Алынған мәліметтер таңдалған әдістің тиімділігін растайтын
бар әдебиет деректерімен салыстырылды.

Сондай-ақ мақалада мұздатып кептірудің технологиялық параметрлері және
олардың соңғы өнім сапасына әсері қарастырылған. Алынған нәтижелер
бойынша бие сүтін кептіру процесін оңтайландыру бойынша ұсыныстар
әзірленді. Әрі қарайғы зерттеулер технологиялық процестерді жетілдіруге
және дайын өнімнің шығымдылығын арттыруға бағытталатын болады.

{\bfseries Түйін сөздер:} бие сүті, мұздатып кептіру, технологияны
оңтайландыру, қоректік заттар, физика-химиялық талдау.

{\bfseries DEVELOPMENT AND OPTIMIZATION OF FREEZE-DRYING TECHNOLOGY FOR
MARE\textquotesingle S MILK: COMPOSITION ANALYSIS, METHODS, AND PRODUCT
YIELD}

{\bfseries \textsuperscript{1} A.B. Rakhmatulina, \textsuperscript{1, 2}
F.T. Dikhanbayeva, \textsuperscript{1} D.A. Tlevlessova*,}

{\bfseries \textsuperscript{2} M.K. Iztileuov, \textsuperscript{3} B.K.
Kalemshariv}

\textsuperscript{1}Institute of Mechanics and Engineering named after
Academician U.A. Zholdasbekov, Almaty, Kazakhstan,

\textsuperscript{2}Almaty Technological University, Almaty, Kazakhstan,

\textsuperscript{3}Kazakh Agrotechnical Research University named after
S. Seifullin, Astana, Kazakhstan,

e-mail: tlevlessova@gmail.com

The article presents research on the development and optimization of
technology for producing powdered dairy products from
mare\textquotesingle s milk using the freeze-drying method. The main
objective of the study was to determine the optimal freeze-drying
conditions to preserve the nutritional and biologically active
components of the milk. The experiments included a detailed analysis of
the physicochemical properties of mare\textquotesingle s milk, including
protein, fat, lactose, and mineral salt content. Various freeze-drying
temperature regimes were studied to determine their impact on the yield
and quality of the powdered milk.

The results showed that the optimal drying temperature is approximately
35°C, at which maximum powdered milk yield is achieved with minimal
nutrient loss. The powdered milk yield was 67.14 grams from 600 grams of
liquid milk. The obtained data were compared with existing literature,
confirming the effectiveness of the selected method.

The article also discusses the technological parameters of freeze-drying
and their impact on the quality of the final product. Based on the
results, recommendations for optimizing the process of freeze-drying
mare\textquotesingle s milk were developed. Further research will focus
on improving technological processes and increasing the yield of the
finished product.

{\bfseries Key words:} mare\textquotesingle s milk, freeze-drying,
technology optimization, nutrients, physicochemical analysis.

{\bfseries Введение.} Цель данного исследования заключается в разработке
технологии производства сухих молочных продуктов из кобыльего молока. В
ходе работы были проведены анализы кобыльего молока и эксперименты по
сублимационной сушке для определения оптимальных условий производства.
Также был проведен анализ существующих научных исследований в области
переработки кобыльего молока и сублимационной сушки молочных продуктов.

Разработка технологии сухих молочных продуктов из кобыльего молока
требует глубокого понимания его состава и свойств, а также анализа
существующих методов переработки и сушки. В данном разделе приведен
обзор научных статей, посвященных различным аспектам кобыльего молока и
его переработки.

Кобылье молоко имеет уникальный состав, который включает в себя высокое
содержание лактозы и низкое содержание жира по сравнению с коровьим
молоком. Оно богато витаминами (особенно витаминами группы B и витамином
C) и минералами, такими как кальций, магний и фосфор. Эти свойства
делают кобылье молоко ценным продуктом для детского и диетического
питания. Например, в работе {[}1{]} указано, что кобылье молоко обладает
высокой биологической ценностью, что подтверждается результатами наших
анализов, показавшими содержание белка 4.64\% и жира 3.67\% .

В исследовании {[}2{]} обсуждается стабильность цвета ферментированного
кобыльего молока и его адаптация к составу коровьего молока. Важность
выбора технологий, которые увеличивают срок хранения и сохраняют
питательные свойства молока, подчеркивается в их работе .

В работе {[}3{]} рассматривается производство кобыльего молока в
маргинальных зонах и его потенциал как пищевого продукта. Авторы
отмечают, что технологическая обработка, направленная на продление срока
хранения молока, имеет важное значение для его использования в качестве
коммерческого продукта.

Сублимационная сушка является предпочтительным методом для сохранения
биологически активных компонентов молока. Авторы работы {[}4{]} в своем
обзоре подчеркивают, что сублимационная сушка позволяет сохранить
структуру белков и витаминов, что особенно важно для кобыльего молока. В
исследовании {[}5{]} обсуждаются функциональные свойства сублимационно
высушенного кобыльего молока, включая его пенистые свойства, которые
могут быть полезны для создания новых продуктов .

Кондыбаев А. и др. {[}6{]} в своем исследовании описывают производство
ферментированных продуктов из кобыльего молока, таких как кумыс. Авторы
подчеркивают важность ферментации для увеличения объема продукта и
повышения содержания кислоты и этанола, что делает кумыс ценным
диетическим продуктом .

Авторы {[}7{]} разработали ферментированный молочный продукт на основе
кобыльего молока и молочнокислых микроорганизмов. Их исследование
подчеркивает значение правильного выбора микроорганизмов для улучшения
вкусовых и питательных свойств конечного продукта .

Авторы {[}8{]} изучали трансформацию традиционной индустрии кобыльего
молока в Казахстане в креативную индустрию. Авторы обсуждают внедрение
технологии вакуумной сублимации, которая позволяет производить
высококачественные сухие молочные продукты из кобыльего молока, тем
самым способствуя развитию местной экономики и улучшению качества жизни
населения .

Сухое кобылье молоко используется в производстве различных продуктов,
включая детские смеси, диетические добавки и косметические средства.
Применение сухого кобыльего молока позволяет расширить спектр
использования этого продукта и повысить его стабильность и срок хранения
{[}9{]}.

Анализ существующих научных статей подтверждает, что кобылье молоко
обладает высокой питательной ценностью и уникальными свойствами, которые
делают его ценным продуктом для различных применений. Сублимационная
сушка является оптимальным методом для сохранения биологически активных
компонентов молока, а ферментация позволяет создавать новые ценные
продукты. Дальнейшие исследования направлены на оптимизацию процессов
переработки и сушки кобыльего молока для повышения выхода и качества
конечного продукта.

\emph{Цель исследования:}

Разработка и оптимизация технологии производства сухих молочных
продуктов из кобыльего молока с использованием метода сублимационной
сушки для сохранения питательных и биологически активных компонентов.

\emph{Задачи исследования:}

\begin{itemize}
\item
  провести детальный анализ физико-химических свойств кобыльего молока,
  включая содержание белка, жира, лактозы, минеральных солей и других
  компонентов.
\item
  изучить влияние различных температурных режимов на выход и качество
  сухого кобыльего молока.
\item
  Определить оптимальную температуру полок и давления в камере для
  максимального сохранения питательных веществ.
\item
  провести экспериментальные исследования для определения выхода сухого
  молока при различных температурных режимах сублимационной сушки.
\item
  Анализировать влияние температуры на эффективность сушки и выход
  конечного продукта.
\item
  сравнить результаты экспериментов с существующими данными из научной
  литературы по сублимационной сушке молочных продуктов.
\item
  Оценить преимущества и недостатки предложенной технологии в сравнении
  с аналогичными методами.
\item
  на основе полученных данных разработать рекомендации по оптимизации
  процесса сублимационной сушки кобыльего молока.
\item
  Предложить возможные направления для дальнейших исследований и
  улучшения технологии.
\end{itemize}

{\bfseries Материалы и методы.}

\emph{Материалы}

1. Кобылье молоко: 3 кг, жирность: 3.67\%, белок: 4.64\%, сухое
вещество:16.13\%, СОМО: 12.46\%, Минеральные соли: 1.03\%, Плотность:
1.045 г/см³, Точка замерзания: -0.0529°C, Общий белок: 4.57\%,
температура: 23.6°C, лактоза: 6.85\%, Калорийность: 81.24 ккал, pH:
6.98, кислотность: 6°Т, содержание спирта: 0\%

2. Оборудование:

\begin{itemize}
\item
  Сублимационная сушилка (Freeze Dryer), производство КНР,
\item
  Аналитические весы,
\item
  Термометры,
\item
  pH-метр,
\item
  Анализатор качества молока «Лактан 1-4» исполнение 220
\item
  Вискозиметрический анализатор молока "Соматос-Мини"
\item
  Лабораторные контейнеры и пробирки,
\end{itemize}

\emph{Методы}

1. Кобылье молоко было собрано из фермы в Майкудуке и доставлено в
лабораторию в стерильных условиях. Молоко было тщательно перемешано и
разделено на части по 600 г для дальнейших экспериментов.

2. Анализ состава молока выполнялся на анализаторах молока.

- Минеральные соли: Определены методом озоления.

- Точка замерзания: Определена с помощью криоскопа.

- Лактоза: Определена ферментативным методом.

- pH: Измерен с помощью калиброванного pH-метра.

- Кислотность: Определена титриметрическим методом.

3. Сублимационная сушка:

- Образцы молока по 600 г подвергались сублимационной сушке при
различных температурных режимах (25°C, 30°C, 35°C, 40°C, 45°C).

- Температура полок и давление в камере контролировались и записывались
каждые 4 минуты в течение эксперимента.

- Температура десублиматора поддерживалась в пределах -23.9°C до
-26.8°C.

4. Определение выхода сухого молока:

- По окончании сушки каждое высушенное молоко взвешивалось для
определения массы сухого молока.

- Выход сухого молока рассчитывался как отношение массы сухого молока к
исходной массе жидкого молока.

5. Анализ и обработка данных:

- Все измерения проводились в трехкратной повторности для обеспечения
точности.

- Данные обрабатывались с использованием статистических методов для
определения средней величины и стандартного отклонения.

- Результаты экспериментов сравнивались с литературными данными для
оценки эффективности и качества полученного продукта.

Эти материалы и методы были выбраны на основе предварительных
исследований и анализа существующей литературы. Применение данных
методов позволило достичь высокой точности в измерении состава молока и
эффективности сублимационной сушки, что подтверждено в ряде научных
исследований

{\bfseries Результаты и обсуждение.} Анализы кобыльего молока были
проведены 30.05.2024 г. Результаты представлены в таблице 1:

{\bfseries Таблица 1-Результаты анализов свежего кобыльего молока
(Майкудук)}

\begin{longtable}[]{@{}
  >{\raggedright\arraybackslash}p{(\columnwidth - 4\tabcolsep) * \real{0.0661}}
  >{\raggedright\arraybackslash}p{(\columnwidth - 4\tabcolsep) * \real{0.4836}}
  >{\raggedright\arraybackslash}p{(\columnwidth - 4\tabcolsep) * \real{0.4502}}@{}}
\toprule\noalign{}
\begin{minipage}[b]{\linewidth}\raggedright
№
\end{minipage} & \begin{minipage}[b]{\linewidth}\raggedright
Показатели
\end{minipage} & \begin{minipage}[b]{\linewidth}\raggedright
Количество
\end{minipage} \\
\midrule\noalign{}
\endhead
\bottomrule\noalign{}
\endlastfoot
1 & Жирность, \% & 3.67 \\
2 & Белок, \% & 4.64 \\
3 & Сухое вещество, \% & 16.13 \\
4 & СОМО, \% & 12.46 \\
5 & Минеральные соли \% & 1.03 \\
6 & Плотность, г/см³ & 1.045 \\
7 & Точка замерзания, С & -0.0529 \\
8 & Общий белок, \% & 4.57 \\
9 & Температура С° & 23.6 \\
10 & Лактоза 6.85, \% & 6.85 \\
11 & Содержание воды, \% & 0 \\
12 & Калорийность, ккалл & 81.24 \\
13 & pH & 6.98 \\
14 & Кислотность Т° & 6 \\
15 & Содержание спирта, \% & - \\
\end{longtable}

Анализ кобыльего молока показывает, что оно обладает высоким содержанием
белка (4.64\%) и жира (3.67\%). Высокая калорийность (81.24 ккал) и
значительное содержание лактозы (6.85\%) подтверждают его питательную
ценность. Эти результаты согласуются с данными из научной литературы,
где указывается на высокую биологическую ценность кобыльего молока, что
делает его подходящим для использования в детском и диетическом питании.
Значение pH (6.98) и кислотность (6°Т) указывают на свежесть и хорошее
качество молока.

\emph{Эксперименты по сублимационной сушке}

В ходе экспериментов была определена выходная масса сухого молока при
различных температурных режимах сублимационной сушки. Исходные данные и
результаты представлены на рисунке 1:

Т

{\bfseries Рис. 1 -- Выход сухого молока в зависимости от температуры}

На представленной диаграмме (рис.1) изображена зависимость выхода сухого
кобыльего молока от времени или других экспериментальных условий (ось
X), где по оси Y обозначен выход продукта в граммах. График имеет форму
полинома третьей степени (кубическая кривая), уравнение которой
представлено как:

\(y = {0.926x}^{3} - 11.508x^{2} + 49.7x - 38.33\) (1)

где y -- выход сухого молока, x -- время или другие условия
эксперимента.

Значение коэффициента детерминации R\textsuperscript{2}=0.9875 указывает
на высокую степень соответствия модели экспериментальным данным.
Наблюдается увеличение выхода сухого молока по мере увеличения значения
оси X до определенной точки, после чего рост стабилизируется или
замедляется. График показывает, что на начальных стадиях эксперимента
прирост выхода продукта наиболее интенсивный, затем он становится более
плавным.

График содержит ошибки (погрешности) измерений, представленные в виде
горизонтальных и вертикальных отрезков. Вертикальные отрезки показывают
вариации в выходе сухого молока, что может быть связано с
экспериментальными неточностями или естественной вариативностью
образцов. Горизонтальные отрезки указывают на вариации значений оси X,
что также может отражать экспериментальные условия.

Из анализа графика следует, что существует оптимальная область значений
оси X, при которых выход сухого молока максимален и стабилен. Это
подтверждается стабилизацией кривой после определенного значения.
Дальнейшее увеличение значения X приводит к снижению прироста выхода,
что может свидетельствовать о достижении предела эффективности данного
метода сублимационной сушки. Присутствие погрешностей указывает на
необходимость учёта возможных отклонений в экспериментальных условиях и
повторяемости результатов. Это важно для будущих исследований и
масштабирования процесса. На основе полученных данных, рекомендуется
проводить дальнейшие эксперименты в пределах оптимальной области
значений оси X, чтобы максимизировать выход сухого молока и
минимизировать затраты.

Необходимо также учитывать и минимизировать экспериментальные
погрешности для повышения точности и повторяемости результатов.

Таким образом, проведенный анализ демонстрирует успешность выбранного
метода и указывает на возможности дальнейшей оптимизации процесса
сублимационной сушки для повышения выхода сухого кобыльего молока.

\emph{Параметры при сушке кобыльего молока на сублимационной установке
(Майкудук)}

Процесс сушки был проведен при различных параметрах температуры полок и
давления в камере. В таблице 2 приведены основные параметры:

{\bfseries Таблица 2- Параметры при сушке кобыльего молока на
сублимационной}

{\bfseries сушке (Майкудук)}

\begin{longtable}[]{@{}
  >{\raggedright\arraybackslash}p{(\columnwidth - 14\tabcolsep) * \real{0.1457}}
  >{\raggedright\arraybackslash}p{(\columnwidth - 14\tabcolsep) * \real{0.2341}}
  >{\raggedright\arraybackslash}p{(\columnwidth - 14\tabcolsep) * \real{0.1698}}
  >{\raggedright\arraybackslash}p{(\columnwidth - 14\tabcolsep) * \real{0.0902}}
  >{\raggedright\arraybackslash}p{(\columnwidth - 14\tabcolsep) * \real{0.0901}}
  >{\raggedright\arraybackslash}p{(\columnwidth - 14\tabcolsep) * \real{0.0901}}
  >{\raggedright\arraybackslash}p{(\columnwidth - 14\tabcolsep) * \real{0.0901}}
  >{\raggedright\arraybackslash}p{(\columnwidth - 14\tabcolsep) * \real{0.0901}}@{}}
\toprule\noalign{}
\begin{minipage}[b]{\linewidth}\raggedright
Время
\end{minipage} & \begin{minipage}[b]{\linewidth}\raggedright
Показатели
\end{minipage} & \begin{minipage}[b]{\linewidth}\raggedright
Ед. измерения
\end{minipage} &
\multicolumn{5}{>{\raggedright\arraybackslash}p{(\columnwidth - 14\tabcolsep) * \real{0.4504} + 8\tabcolsep}@{}}{%
\begin{minipage}[b]{\linewidth}\raggedright
Значения
\end{minipage}} \\
\midrule\noalign{}
\endhead
\bottomrule\noalign{}
\endlastfoot
\multirow{3}{=}{09.30} & Температура полок & & 24.7 & 29.6 & 34.7 & 39.9
& 44.6 \\
& Давление в камере & Па & 57.8 & 57.8 & 57.8 & 57.8 & 57.8 \\
& Температура десублиматора & С° & -23.9 & -23.9 & -23.9 & -23.9 &
-23.9 \\
\multirow{3}{=}{09.34} & Температура полок & С° & 24.7 & 29.7 & 34.8 &
39.7 & 44.9 \\
& Давление в камере & Па & 65.1 & 65.1 & 65.1 & 65.1 & 65.1 \\
& Температура десублиматора & С° & -24.1 & -24.1 & -24.1 & -24.1 &
-24.1 \\
\multirow{3}{=}{09.38} & Температура полок & С° & 24.8 & 29.9 & 34.9 &
39.6 & 44.4 \\
& Давление в камере & Па & 70.0 & 70.0 & 70.0 & 70.0 & 70.0 \\
& Температура десублиматора & С° & -24.1 & -24.1 & -24.1 & -24.1 &
-24.1 \\
\multirow{3}{=}{09.44} & Температура полок & С° & 24.8 & 29.8 & 34.7 &
39.9 & 44.6 \\
& Давление в камере & Па & 80.2 & 80.2 & 80.2 & 80.2 & 80.2 \\
& Температура десублиматора & С° & -24.3 & -24.3 & -24.3 & -24.3 &
-24.3 \\
\multirow{3}{=}{09.51} & Температура полок & С° & 24.5 & 29.7 & 34.6 &
39.7 & 44.8 \\
& Давление в камере & Па & 90.2 & 90.2 & 90.2 & 90.2 & 90.2 \\
& Температура десублиматора & С° & -24.4 & -24.4 & -24.4 & -24.4 &
-24.4 \\
\multirow{3}{=}{09.57} & Температура полок & С° & 24.7 & 29.8 & 34.8 &
39.5 & 44.8 \\
& Давление в камере & Па & 99.2 & 99.2 & 99.2 & 99.2 & 99.2 \\
& Температура десублиматора & С° & -24.6 & -24.6 & -24.6 & -24.6 &
-24.6 \\
\multirow{3}{=}{09.58} & Температура полок & С° & 24.5 & 29.6 & 34.9 &
39.5 & 44.9 \\
& Давление в камере & Па & 100 & 100 & 100 & 100 & 100 \\
& Температура десублиматора & С° & -24.6 & -24.6 & -24.6 & -24.6 &
-24.6 \\
\multirow{3}{=}{10.00} & Температура полок & С° & 24.7 & 29.9 & 34.5 &
39.6 & 44.7 \\
& Давление в камере & Па & 49.3 & 49.3 & 49.3 & 49.3 & 49.3 \\
& Температура десублиматора & С° & -24.6 & -24.6 & -24.6 & -24.6 &
-24.6 \\
\multirow{3}{=}{10.02} & Температура полок & С° & 24.6 & 29.8 & 34.9 &
39.9 & 44.5 \\
& Давление в камере & Па & 50.1 & 50.1 & 50.1 & 50.1 & 50.1 \\
& Температура десублиматора & С° & -24.6 & -24.6 & -24.6 & -24.6 &
-24.6 \\
\multirow{3}{=}{10.8} & Температура полок & С° & 24.7 & 29.8 & 34.8 &
39.6 & 44.8 \\
& Давление в камере & Па & 60.0 & 60.0 & 60.0 & 60.0 & 60.0 \\
& Температура десублиматора & С° & -25.0 & -25.0 & -25.0 & -25.0 &
-25.0 \\
\multirow{3}{=}{09.29} & Температура полок & С° & 24.6 & 29.6 & 34.9 &
39.7 & 44.7 \\
& Давление в камере & Па & 79.4 & 79.4 & 79.4 & 79.4 & 79.4 \\
& Температура десублиматора & С° & -26.8 & -26.8 & -26.8 & -26.8 &
-26.8 \\
\end{longtable}

Для оценки значимости различий в температурах полок был проведен
дисперсионный анализ (ANOVA). Результаты анализа показали следующие
значения: F-значение= 38069.58, p-значение= 2.09e\textsuperscript{-86}.
Значение p-значения (2.09e-86) значительно меньше уровня значимости
0.05, что указывает на высокую статистическую значимость различий между
температурами полок. Высокое F-значение (38069.58) подтверждает наличие
существенных различий в температурах полок в разные временные интервалы.
Соответственно, значительные различия в температурах полок указывают на
то, что различные температурные режимы оказывают значительное влияние на
процесс сублимационной сушки.

{\bfseries Выводы.} Результаты подтверждают важность тщательного контроля
температурного режима полок и давления в камере для обеспечения
максимального выхода и качества сухого кобыльего молока. Это согласуется
с выводами из статей 2024 года, таких как работы в {[}10{]} и {[}11{]},
которые подчеркивают значимость оптимальных условий для сохранения
биологически активных компонентов молока.

Оптимальные параметры сублимационной сушки:

Оптимальные температуры полок (34.7°C - 39.9°C) и стабильное давление
(57.8 Па - 100 Па) обеспечивают максимальный выход сухого молока и
сохранение его качественных характеристик. Эти данные подтверждаются
исследованиями, представленными в специальных выпусках и статьях{[}10,
12{]}.

Практическое применение результатов:

Результаты исследования могут быть использованы для оптимизации
промышленного процесса производства сухого кобыльего молока. Это
подтверждается работами, опубликованными в 2024 году, которые
подчеркивают значимость правильного выбора технологических параметров
для улучшения качества конечного продукта.

Дальнейшие исследования:

Для уточнения оптимальных параметров процесса сублимационной сушки и
изучения их влияния на сохранение питательных и биологически активных
компонентов кобыльего молока необходимы дальнейшие исследования.

Таким образом выводы и рекомендации согласуются с существующей научной
литературой, подтверждая эффективность и важность выбранных методов для
производства качественного сухого кобыльего молока

{\bfseries Рекомендации}

\begin{itemize}
\item
  для обеспечения максимального выхода сухого молока рекомендуется
  поддерживать температуру полок в пределах 34.7°C - 39.9°C.
\item
  необходимо поддерживать давление в камере в пределах 57.8 Па - 100 Па
  для обеспечения стабильного процесса сушки.
\item
  температура десублиматора должна оставаться в диапазоне от -23.9°C до
  -26.8°C для сохранения качественных характеристик молока.
\item
  рекомендуется проведение дальнейших исследований для уточнения
  оптимальных параметров процесса сублимационной сушки и изучения их
  влияния на сохранение питательных и биологически активных компонентов
  кобыльего молока.
\end{itemize}

Таким образом, представленные данные и их анализ дают четкое понимание о
ключевых параметрах процесса сублимационной сушки кобыльего молока, что
способствует улучшению его эффективности и качества.

\emph{{\bfseries Финансирование.} Данное исследование
финансировалось/финансируется Комитетом по науке Министерства науки и
высшего образования Республики Казахстан (грант № BR21881957 Разработка
технологии глубокой переработки и оборудования вакуум-сублимационной
сушки кобыльего и верблюжьего молока.}

{\bfseries Литература}

1. Salimei E., Fantuz F. Equid milk for human consumption //
International Dairy Journal. -2012. -Vol. 24(2), -P. 130-142. DOI
10.1016/J.IDAIRYJ.2011.11.008

2. Teichert J., Cais-Sokolińska D., Danków R., Pikul J. Color stability
of fermented mare\textquotesingle s milk and a fermented beverage from
cow\textquotesingle s milk adapted to mare\textquotesingle s milk
composition. Foods. mdpi.com. -2020. --Vol. 9(2). DOI
10.3390/foods9020217

3. Miraglia, N., Salimei, E., \& Fantuz, F. Equine milk production and
valorization of marginal areas-A review. Animals. mdpi.com. -2020.
--Vol. 10(2). DOI 10.3390/ani10020353

4. Ratti, C. Hot air and freeze-drying of high-value foods: A review //
Journal of Food Engineering, -2001. --Vol. 49(4). --P. 311-319. DOI
10.1016/S0260-8774(00)00228-4

5. Cais-Sokolińska, D., Teichert, J., \& Gawałek, J. Foaming and other
functional properties of freeze-dried mare\textquotesingle s milk.
Foods. mdpi.com. -2023. --Vol. 12(11). DOI 10.3390/foods12112274

6. Kondybayev, A., Loiseau, G., Achir, N., Mestres, C., \& Shaikh, A.
Fermented mare milk product (Qymyz, Koumiss) // International Dairy
Journal. Elsevier. -2021. --Vol. 119. DOI 10.1016/J.IDAIRYJ.2021.105065

7. Simonenko E.S., Begunova A.V. Development of fermented milk product
based on mare milk and lactic microorganisms association. Vopr Pitan.
voprosy-pitaniya.ru. -2021. --Vol. 90(5). --P. 115-125 DOI
10.33029/0042-8833-2021-90-5-115-125

8. Baibokonov, D., Yang, Y., \& Tang, Y. Understanding the traditional
mares\textquotesingle{} milk industry\textquotesingle s transformation
into a creative industry: Empirical evidence from Kazakhstan // Growth
and Change. -2021. --Vol. 52(1). DOI10.1111/grow.12478

9. Hinz, K., O'Connor, P. M., Huppertz, T., Ross, R. P., \& Kelly, A. L.
Comparison of the principal proteins in bovine, caprine, buffalo, equine
and camel milk // Journal of Dairy Research. -2012. --Vol. 79(2). --P.
185-191. DOI10.1017/s0022029912000015

10. Cais-Sokolińska D., Teichert J., Gawałek J. Foaming and Other
Functional Properties of Freeze-Dried Mare's Milk // Foods. -2024.
--Vol. 12(11). https://doi.org/10.3390/foods12112274

11. Milkify\textquotesingle s Freeze-Drying Benefits // Milkify. -2024.
URL: https://www.milkify.me

12. Bhatta S., Stevanovic Janezic T., Ratti, C. Freeze-Drying of
Plant-Based Foods // Foods. -2024. --Vol. 9. DOI10.3390/foods9010087.

\emph{Сведения об авторах}

Рахматулина А.Б.- PhD, доцент, Отдел машиноведения и робототехники
Институт механики и машиноведения имени академика У. А. Джолдасбекова,
Алматы, Казахстан, e-mail: kazrah@mail.ru;

Диханбаева Ф.Т. -- доктор технических наук, профессор кафедры
«Технология продуктов питания», Алматинский технологический университет,
Институт механики и машиноведения имени академика У. А. Джолдасбекова,
Алматы, Казахстан, е-mail: fatima6363@mail.ru;

Тлевлесова Д.А.-- PhD, ассоциированный профессор, ТОО «КазНИИППП»,
Институт механики и машиноведения имени академика У. А. Джолдасбекова,
Алматы, Казахстан, e-mail: tlevlessova@gmail.com;

Изтилеуов М.К.-магистр., Алматинский технологический университет,
Алматы, Казахстан, еmail: m.iztileuov@mail.ru;

Калемшарив Б. -- докторант, Казахский агротехнический исследовательский
университет им. С. Сейфуллина, Астана, Казахстан, е-mail:
begjan.ae@gmail.com.

\emph{{\bfseries Information about the authors}}

Rakhmatulina A.B. -- PhD, Associate Professor, Institute of Mechanics
and Engineering named after Academician U.A. Zholdasbekov, Almaty,
Kazakhstan, e-mail: kazrah@mail.ru;

Dikhanbaeva F.T. -- Doctor of Technical Sciences, Professor, Department
of "Food Technology," Almaty Technological University, Institute of
Mechanics and Engineering named after Academician U.A. Zholdasbekov,
Almaty, Kazakhstan, e-mail: fatima6363@mail.ru;

Tlevlessova D.A. -- PhD, Associate Professor, LLP "KazNII PPP,"
Institute of Mechanics and Engineering named after Academician U.A.
Zholdasbekov, Almaty, Kazakhstan, e-mail: tlevlessova@gmail.com;

Iztileuov M.K. -- Master's Degree, Almaty Technological University,
Almaty, Kazakhstan, e-mail: m.iztileuov@mail.ru;

Kalemshariv B. -- Doctoral Candidate, Kazakh Agrotechnical Research
University named after S. Seifullin, Astana, Kazakhstan, e-mail:
begjan.ae@gmail.com.\newpage
{\bfseries МРНТИ 65.33.29}

{\bfseries Производство ОБОГАЩЕННЫХ МАКАРОННЫХ ИЗДЕЛИЙ комплексом
нанокарбоксилатов и тонкодисперсным порошком зерновых культур}

{\bfseries \textsuperscript{1}Д.А. Шаймерденова\textsuperscript{🖂},
\textsuperscript{2}А.М. Омаралиева, \textsuperscript{3}Б.К. Тарабаев,
\textsuperscript{1}Л.Т. Сарбасова,}

{\bfseries \textsuperscript{1}С.С. Ануарбекова, \textsuperscript{1}Д.Б.
Искакова, \textsuperscript{1}А.А. Шаймерденов , \textsuperscript{1}Д.А.
Тастанов}

\textsuperscript{1}ТОО «Научно-производственное предприятие «Инноватор»,
Астана, Казахстан,

\textsuperscript{2} Казахский университет технологии и бизнеса
им.К.Кулажанова, Астана, Казахстан,

\textsuperscript{3} Казахский агротехнический исследовательский
университет им. С. Сейфуллина, Астана, Казахстан

\textsuperscript{🖂}Коррреспондент-автор: e-mail: darigash@mail.ru

Объект исследований -- макаронные изделия, обогащённые тонкодисперсной
мукой из гречихи и чечевицы, и комплексом нанонокарбоксилатов.

По данным аналитиков, потребление макаронных изделий в Казахстане ---
одно из самых высоких в мире, Казахстан уступает в этом только Италии.
За январь -- декабрь 2022 года производство макарон составило 166,1 тыс.
тонн, что на 4\% больше, чем в 2021 году {[}1{]}.

Учитывая, что макаронные изделия популярны и потребляются в большом
количестве, имеется возможность реально и эффективно проводить
профилактику различных видов заболеваний с помощью выпуска изделий с
использованием обогащающих комплексов микроэлементов и витаминов. Тем
более, что основным сырьем для производства макарон является
«обедненная» пшеничная мука высшего сорта. В то же время, продукты
питания серьёзно дорожают, что неизбежно сказывается на
продовольственной безопасности республики {[}1{]}.

В этой связи необходима разработка отечественных технологий производства
обогащённых макаронных изделий, что позволит повысить их
конкурентоспособность за счет снижения себестоимости.

В рамках исследований получены макаронные изделия, обогащенные наиболее
ценной по содержанию белка и клетчатки чечевичным и гречневым
тонкодисперсным порошком в количестве 10\%. Для обогащения макаронных
изделий микро- и макроэлементами произведен подбор и расчет рецептур
нанокарбоксилатов с установлением их количества, исходя из норм
потребления человека. По химическому и микроэлементному составу
определены наиболее полноценные добавки.

Результаты исследований позволили установить, что потребительские
свойства макаронных изделий, обогащенных тонкодисперсным порошком из
гречихи, соответствуют нормативным показателям. Комплекс
нанокарбоксилатов не повлиял на потребительские свойства макарон, но
значительно увеличил содержание магия и цинка.

{\bfseries Ключевые слова:} макароны, тонкодисперсный порошок,
нанокарбоксилаты, микроэлементы, чечевичная мука, гречневая мука.

{\bfseries НАНОКАРБОКСИЛАТТАР КЕШЕНІМЕН ЖӘНЕ ДӘНДІ ДАҚЫЛДАРДЫҢ ЖҰҚА
ДИСПЕРСТІ ҰНТАҒЫМЕН БАЙЫТЫЛҒАН МАКАРОН ӨНІМДЕРІН ӨНДІРУІ}

{\bfseries \textsuperscript{1}Д.А. Шаймерденова\textsuperscript{🖂},
\textsuperscript{2}А.М. Омаралиева, \textsuperscript{3}Б.К. Тарабаев,
\textsuperscript{1}Л.Т.Сарбасова,}

{\bfseries \textsuperscript{1}С.С. Ануарбекова, \textsuperscript{1}Д.Б.
Искакова, \textsuperscript{1}А.А. Шаймерденов , \textsuperscript{1}Д.А.
Тастанов}

\textsuperscript{1}«Инноватор» Ғылыми-өндірістік кәсіпорны» ЖШС, Астана,
Қазақстан,

\textsuperscript{2} Қ.Құлажанов атындағы Қазақ технология және бизнес
университеті, Астана, Қазақстан,

\textsuperscript{3}С.Сейфуллин атындағы Қазақ агротехникалық зерттеу
университеті, Астана, Қазақстан,

e-mail: darigash@mail.ru

Зерттеу нысаны-қарақұмық пен жасымық ұнымен және нанокарбоксилаттар
кешенімен байытылған макарон өнімдері.

Сарапшылардың пікірінше, Қазақстанда макарон өнімдерін тұтыну әлемдегі
ең жоғары көрсеткіштердің бірі болып табылады, Қазақстан бұл жағынан
Италиядан кейін екінші орында. 2022 жылдың қаңтар -- желтоқсан айларында
макарон өндірісі 166,1 мың тоннаны құрады, бұл 2021 жылмен салыстырғанда
4\% - ға өсті {[}1{]}

Макарон өнімдері танымал және көп мөлшерде тұтынылатындығын ескере
отырып, микроэлементтер мен дәрумендерді байытатын кешендерді қолдана
отырып, өнімдерді шығару арқылы әртүрлі аурулардың алдын-алуды нақты
және тиімді жүргізуге болады. Сонымен қатар, макарон өндірісінің негізгі
шикізаты-жоғары сортты "сарқылған" бидай ұны. Сонымен қатар, Азық-түлік
қатты бағаланады, бұл сөзсіз республиканың азық-түлік қауіпсіздігіне
әсер етеді {[}1{]}.

Осыған байланысты байытылған макарон өнімдерін өндірудің отандық
технологияларын әзірлеу қажет, бұл өзіндік құнын төмендету есебінен
олардың бәсекеге қабілеттілігін арттыруға мүмкіндік береді.

Зерттеулер аясында 10\% мөлшерінде жасымық және қарақұмық жұқа дисперсті
ұнтағымен ақуыз және талшық құрамы бойынша ең құнды байытылған макарон
өнімдері алынды. Макарон өнімдерін микро және макроэлементтермен байыту
үшін нанокарбоксилаттардың рецептураларын таңдау және есептеу адамның
тұтыну нормаларына сүйене отырып, олардың санын белгілеу арқылы жүзеге
асырылады. Химиялық және микроэлементтік құрамы бойынша ең толыққанды
қоспалар анықталды.

Зерттеу нәтижелері қарақұмық ұнтағымен байытылған макарон өнімдерінің
тұтынушылық қасиеттері нормативтік көрсеткіштерге сәйкес келетіндігін
анықтады. Нанокарбоксилат кешені макаронның тұтынушылық қасиеттеріне
әсер етпеді, бірақ оның құрамын едәуір арттырды сиқыр және мырыш.

{\bfseries Түйін сөздер:} макарон, жұқа дисперсті ұнтак,
нанокарбоксилаттар, микроэлементтер, жасымық ұны, қарақұмық ұны.

{\bfseries PRODUCTION OF ENRICHED PASTA PRODUCTS WITH A COMPLEX OF
NANOCARBOXYLATES AND FINE-DISPERSED CEREAL FLOUR}

{\bfseries \textsuperscript{1}D.A. Shaimerdenova\textsuperscript{🖂},
\textsuperscript{2}A.M. Omaralieva, \textsuperscript{3}B.K. Tarabaev,
\textsuperscript{1}L.T. Sarbasova,\\
\textsuperscript{1}S.S. Anuarbekova, \textsuperscript{1}D.B. Iskakova,
\textsuperscript{1}A.A. Shaimerdenov, \textsuperscript{1}D.A. Tastanov}

\textsuperscript{1}«Research and Production Enterprise «Innovator» LLP,
Astana,

\textsuperscript{2}K.Kulazhanov Kazakh University of Technology and
Business, Astana, Kazakhstan,

\textsuperscript{3}Kazakh Agrotechnical Research University named after
S. Seifullin, Astana, Kazakhstan,

e-mail : darigash@mail.ru

The object of research is pasta enriched with fine flour from buckwheat
and lentils, and a complex of nanocarboxylates.

According to analysts, the consumption of pasta in Kazakhstan is one of
the highest in the world, Kazakhstan is second only to Italy in this. In
January -- December 2022, pasta production amounted to 166.1 thousand
tons, which is 4\% more than in 2021 {[}1{]}

Considering that pasta is popular and consumed in large quantities, it
is possible to really and effectively prevent various types of diseases
by producing products using enriching complexes of trace elements and
vitamins. Moreover, the main raw material for the production of pasta is
"depleted" wheat flour of the highest grade. At the same time, food is
seriously becoming more expensive, which inevitably affects the food
security of the republic {[}1{]}.

In this regard, it is necessary to develop domestic technologies for the
production of enriched pasta, which will increase their competitiveness
by reducing the cost.

As part of the research, the most valuable fortified pasta in terms of
protein and fiber content was obtained with a fine-dispersed powder of
lentils and buckwheat in an amount of 10\%. The selection and
calculation of the formulations of nanocarboxylates for the enrichment
of pasta with micro and Macroelements is carried out by establishing
their number, based on the norms of human consumption. The most complete
additives in terms of chemical and microelement composition were
identified.

The results of the study found that the consumer properties of pasta
enriched with buckwheat powder correspond to regulatory indicators. The
nanocarboxylate complex did not affect the consumer properties of pasta,
but significantly increased the content of maggot and zinc.

{\bfseries Key words:} pasta, fine powder, nanocarboxylates, trace
elements, lentil flour, buckwheat flour.

{\bfseries Введение.} Макаронные изделия --- популярный во всем мире
продукт питания, известный простотой приготовления, хорошей
стабильностью при хранении, низкой стоимостью, простотой приготовления и
низким гликемическим индексом (ГИ). Макаронные изделия состоят в
основном из углеводов (70--76\%), белков (\textasciitilde10--14\%),
липидов (\textasciitilde1,8\%), пищевых волокон (\textasciitilde2,9\%) и
небольшого количества минералов и витаминов, Углеводы же в пищевых
продуктах являются важным источником энергии для человека {[}2{]}.
Однако, в макаронах мало пищевых волокон, витаминов, незаменимых
аминокислот и минералов {[}2{]}, т.к. при помоле для приготовления
макаронной муки происходит потеря этих компонентов. Макаронные изделия
можно считать хорошим средством для включения биологически активных
ингредиентов (белков, фитохимических веществ, минералов, витаминов и т.
д.), как это признано Всемирной организацией здравоохранения и
Управлением по санитарному надзору за качеством пищевых продуктов и
медикаментов США, поскольку в некоторых ситуациях до 10--15\% не-
традиционных ингредиентов могут быть добавлены без существенной потери
качества макаронных изделий в зависимости от используемого ингредиента и
технологии обработки макаронных изделий {[}3,4{]}.

Однако, польза от добавленного ингредиента, которую предполагается
обеспечить, может быть ограничена таким уровнем включения. При
разработке пищевых продуктов с биологически активными соединениями
получаемые продукты часто имеют технологические недостатки,
нежелательный внешний вид и органолептические свойства, что делает их
менее привлекательными для потребителей или просто нерентабельными в
производстве.

Основным сырьем, применяемым в макаронном производстве, является
пшеничная мука. Действующий в Казахстане нормативный документ {[}5{]}
предусматривает использование в качестве основного сырья макаронного
производства пшеничной муки высшего или I сортов. Лучшим сырьем для
макарон является специальная макаронная мука из твердой или
высокостекловидной мягкой пшеницы. Из такой муки получаются изделия
лучшего качества, имеющие янтарно-желтый или соломенно-желтый цвет.

Мука, используемая в макаронном производстве, не должна содержать в
значительных количествах свободные аминокислоты, редуцирующие сахара и
активную полифенолоксидазу (тирозиназу), вызывающую потемнение теста и
ухудшение качества готовых изделий. Вода является составной частью
макаронного теста. Она обусловливает биохимические и физико-химические
свойства теста. Используют водопроводную питьевую воду, которая должна
быть умеренно жесткой и отвечать требованиям нормативных документов на
питьевую воду {[}6{]}.

За последние два десятилетия было проведено много исследований по
повышению питательной ценности макаронных изделий за счет включения
нетрадиционных ингредиентов из-за спроса потребителей, заботящихся о
своем здоровье, на функциональные продукты {[}7,8{]}. Эти ингредиенты
могут влиять на технологические свойства макаронных изделий, но их
воздействие на здоровье не всегда измеряется, а скорее предполагается
{[}9{]}.

Так, в последнее время макаронные фабрики выпускают макаронные изделия,
при производстве которых могут применяться такие добавки, как
томат-пасты, шпинат, щавель, морковный сок. Не менее ценными являются
макаронные изделия, обогащенные минеральными веществами и с повышенным
содержанием пищевых волокон.

Однако, ассортиментный ряд макаронных изделий функциональной
направленности в настоящее время не так обширен. В виду плохой
экологической обстановки в мире, а также малоподвижного образа жизни
люди страдают от нехватки разного рода витаминов и полезных веществ.
Именно поэтому возник острый вопрос в обогащении продуктов повседневного
питания человека витаминами, минеральными и прочими полезными веществами
{[}10{]}.

В последнее время широко применяются для обогащения макаронных изделий
порошкообразные растительные источники, неоспоримым преимуществом
которых является высокая концентрация биологически активных веществ,
т.к. их масса меньше массы исходного сырья в 6-8 раз, имеется
возможность использования при производстве мучных изделий с низкой
влажностью, ввиду этого - длительный срок хранения и хорошая
транспортабельность {[}11{]}.

Ввиду этого, применение тонкодисперсных растительных порошков из
зерновых культур в качестве обогатителей и улучшителей макаронных
изделий представляет значительный интерес.

Тонкодисперсный порошок из зерна зерновых и бобовых культур --- это
цельнозерновой продукт, получаемый в результате технологической
обработки зерна, т.е. измельчении.

Главным вопросом, требующим решения в настоящее время, является изучение
возможности расширения видов муки из зерновых культур и нормы внесения,
которые позволят как повысить пищевую ценность, так и сохранить высокие
потребительские качества макаронных изделий.

Анализ литературы показывает, что среднесуточное потребление с пищей
каждого микронутриента, необходимого для поддержания нормальной
физиологической деятельности человека, измеряется в миллиграммах или в
меньших количествах и отличаются от макроэлементов (углеводов, жиров и
белков) и макроминералов (кальция, магния и фосфора).

Диетическая потребность человека в любом микронутриенте определяется
многими факторами, включая его биодоступность, количество, необходимое
для поддержания его нормальных физиологических функций и прочих факторов
{[}12{]}.

В то же время, по мнению экспертов, недостаток макро- и микроэлементов
приводит к значительным проблемам со здоровьем. В связи с этим возникает
жизненно-необходимая потребность в искусственном обогащении рациона ими
для обеспечения современного человека необходимым набором минеральных
веществ.

Так, например, на сегодня большинство жителей США, европейских стран,
Японии и все увеличивающаяся часть населения менее развитых стран
вынуждены регулярно употреблять дополнительные количества питательных
веществ. По данным американских ученых-диетологов, среднестатистический
рацион современного американца обеспечивает лишь 50--60 \%
рекомендованной суточной потребности в магнии (дефицит магния отмечен у
75--85 \% обследованных жителей США), лишь на 50 \% -- меди, селена,
кальция, около 70 и 90 \% человек недополучают с пищевыми продуктами
цинка и хрома {[}13{]}.

Дефицит микроэлементов приводит к негативным последствиям для здоровья.
Так, анализ научных данных о некоторых наиболее важных микро- и
макроэлементах и их воздействия на организм человека, показало
следующее.

Недостаток цинка вызывает заболевания центральной нервной,
желудочно-кишечной, иммунной, эпидермальной, репродуктивной и костной
систем, может повысить восприимчивость к болезням и инфекциям, увеличить
время восстановления или, в некоторых случаях, ухудшить восстановление,
снизить умственную работоспособность и увеличить последствия осложнений.
Распространенность дефицита цинка в странах Африки и Южной Азии
варьируется от 15 до 50\% {[}14{]}. По данным других исследователей,
дефицит цинка затрагивает более половины населения мира {[}15{]}.
Казахстан, относясь к странам с невысоким уровнем дохода населения,
также подвержен риску дефицита цинка в рационе питания населения.

Дефицит селена связан с сердечно-сосудистыми заболеваниями, бесплодием,
миодегенеративными заболеваниями и снижением когнитивных функций. В
настоящее время изучается роль селена в лечении рака. По данным
китайских ученых, результаты 8-летнего наблюдения показали снижение
заболеваемости первичным раком печени на 35,1\% у пациентов с
добавлением селенизированной поваренной соли по сравнению с населением,
не получавшим такой добавки {[}16-17{]}. Дефицит селена затрагивает от
500 млн. до 1 млрд. человек во всем мире из-за недостаточного его
потребления {[}18{]}.

Магний является четвертым наиболее распространенным катионом в организме
и вторым наиболее распространенным внутриклеточным катионом после калия
{[}19{]}, участвует в синтезе белков, нуклеиновых кислот, обладает
стабилизирующим действием для мембран, необходим для поддержания
гомеостаза кальция, калия и натрия. Недостаток магния приводит к
гипомагниемии, повышению риска развития гипертонии, болезней сердца.
Магний является важным минералом для минерализации костей, мышечной
релаксации и ряда других клеточных функций {[}20{]}. Дефицит магния
распространен во всем мире. Так, по данным Costello R.B. et al. {[}21{]}
приблизительно 50\% американцев потребляют меньше, чем расчетная средняя
потребность в магнии, а некоторые возрастные группы потребляют
значительно меньше.

При этом наиболее важным вопросом при обогащении микроэлементами
макаронных изделий является форма их внесения, обуславливающая их
биодоступность.

Так, анализ литературных данных показывает, что, в основном, применяемые
способы обогащения направлены на внесение одного микроэлемента или его
комплекса с витаминами. Значительное количество исследований направлено
на разработку витаминно-минеральных комплексов с железом {[}22{]}.
Однако, исследования показывают, что основной проблемой является низкая
усвояемость вносимых микро- и макроэлементов.

В последнее время достижения нанотехнологий позволяют синтезировать
такие химические соединения, получение которых с помощью классических
химических реакций или вообще невозможно, либо очень проблематично.
Создано приоритетное направление в нанотехнологии, с помощью которого
получены чрезвычайно химически чистые карбоксилаты основных пищевых
кислот биогенных металлов (цинка, магния, марганца, железа, меди,
кобальта, молибдена и др.). Поскольку при получении указанных
карбоксилатов были непосредственно применены нанотехнологии, они были
названы «нанокарбоксилатами». {[}23{]}.

Ввиду этого, применение нанокарбоксилатов макро-и микроэлементов
значительно повысят пищевую ценность макаронных изделий.

Таким образом, обогащение таких продуктов повседневного спроса, как
макаронные изделия, комплексом микроэлементов и тонкодисперсной мукой
зерновых культур является востребованной технологией.

В данной работе представлены результаты обогащения макаронных изделий
комплексом нанокарбоксилатов и тонкодисперсной мукой гречихи и чечевицы.

{\bfseries Материалы и методы.} При выполнении работы использовали
стандартные, общепринятые физико-химические методы исследований.
Перечень использованных в исследованиях материалов и нормативных
документов, которым они соответствовали:

- пшеница -- ГОСТ 9353 - 2016; чечевица -- ГОСТ 7066 - 2019; гречиха --
ГОСТ 19092 - 2021; пшеничная мука - ГОСТ 26574 - 2017; вода питьевая -
СТ РК ГОСТ Р 51232 - 2003; смесь карбоксилатов магния, цинка, селена по
ТУ У 15.8-35291116-014:2011.

Показатели качества определяли в соответствии с методиками, изложенными
в следующих нормативных документах: определение содержания: белка - по
ГОСТ 10846-91; жира - по ГОСТ 29033-91; клетчатки - по ГОСТ 13496.2-91;
углеводов -- по ГОСТ 25832-89; показатели микробиологической
безопасности (дрожжи, плесени) - по ГОСТ 10444.12-2013; определение
магния и цинка -- по ГОСТ 32343-2013; селена -ГОСТ 31707-2012.

Мучная смесь включала пшеничную макаронную муку высшего сорта,
тонкодисперсный порошок из гречихи и чечевицы, и отдельно комплекс
микроэлементов из магния, селена, цинка. По рекомендации производителей
и для лучшего эффекта повышения микроэлементного состава макарон
комплексы микроэлементов вводят в макароны в составе мучной смеси в
сухом виде.

В качестве комплексных микроэлементов были использованы нанокарбоксилаты
полученных с помощью нанотехнологий учеными украинского НИИ
нанобиотехнологий. Нанокарбоксилаты были получены по реакции
взаимодействия наночастиц металлов, наночастиц оксидов металлов и
наночастиц гидроксидов металлов непосредственно с карбоновой кислотой
(Патент Украины на полезную модель 39392, МПК С07С 51/41, C07F 5/00,
C07F 15/00, С07С 53/126 (2008.01), С0.7С 53/10 (2008.01), A23L 1/00,
В82В 3/00. Опубл. 25.02.2009, Вюл.Ш 4, 2009 р.).

Количество комплексных микроэлементов были рассчитанны в соответствии с
рецептурами, в зависимости от среднесуточной потребности в
микроэлементах человека (таблица 1).

{\bfseries Таблица 1 - Расчет рецептуры комплексных микроэлементов}

\begin{longtable}[]{@{}
  >{\raggedright\arraybackslash}p{(\columnwidth - 2\tabcolsep) * \real{0.0511}}
  >{\raggedright\arraybackslash}p{(\columnwidth - 2\tabcolsep) * \real{0.9489}}@{}}
\toprule\noalign{}
\begin{minipage}[b]{\linewidth}\raggedright
№
\end{minipage} & \begin{minipage}[b]{\linewidth}\raggedright
Расчет рецептуры комплекса нанокарбоксилатов
\end{minipage} \\
\midrule\noalign{}
\endhead
\bottomrule\noalign{}
\endlastfoot
1 & При среднемесячном потреблении макарон в количестве 1,5 кг их
среднее суточное потребление будет составлять: 1,5 кг: 30 дней = 0,05 кг
= 50 г. \\
2 & Средние рекомендуемые величины суточного потребления человеком
микроэлементов составляют: магния -- 375 мг; цинка -- 10 мг; селена --
55 мкг {[}24{]} \\
3 & Продукты питания считаются обогащенными микроэлементами, если в 100
г продукта добавлено не менее 15\% рекомендуемой величины суточного
потребления человеком микроэлементов, то есть:

магния -- 375 мг х 0,15 =56,25 мг

цинка -- 10 мг х 0,15 = 1,5 мг

селена -- 55 мкг х 0,15 =8,25 мкг \\
4 & Соответственно, в 50 г макарон должно содержаться:

магния -- 56,25 мг х 0,50 =28,12 мг

цинка -- 1,5 мг х 0,50 = 0,75 мг

селена -- 8,25 мкг х 0,50 =4,13 мкг \\
5 & В нанокарбоксилатах микроэлементов, полученных с помощью
нанотехнологий, содержится следующее количество собственно
микроэлементов: магния -- 10,5 \%; цинка -- 29,2 \%; селена -- 11,2
\% \\
6 & Соответственно к 50 г макаронной муки необходимо добавить следующее
количество нанокарбоксилатов:

магния --28,12 мг : 10,5 \% х 100 \% = 267 мг

цинка --0,75 мг : 29,2 \% х 100 \% = 2,57 мг

селена -- 4,13 мкг : 11,2 \% х 100 \% = 36,9 мкг \\
7 & Соответственно, к 1000 г макаронной муки следует добавить:

магния --267 мг х 20 =5340 мг \textasciitilde{} 5 г 340 мг

цинка -- 2,57 мг х 20 =514 мг = 0,514 г

селена --36,9 мкг х 20 = 739 мкг \textasciitilde{} 0,000739 г \\
8 & Соответственно, необходимо будет приготовить следующее количество
нанокарбоксилатов микроэлементов для добавки к 1000 г макаронной муки:

Рецептура Mg + Zn + Se:

5 г 340 мг + 0,514 г + 0,000739 г =5,855 г. \\
\end{longtable}

По данным анализа источников, количество вносимых порошков
тонкодисперсной муки взяты в количестве 10\%, как наиболее оптимальное.
Разработана рецептура обогащённых макарон (табл.2).

{\bfseries Таблица 2 - Рецептура макарон, обогащенных тонкодисперсными
порошками гречневой и чечевичной муки}

\begin{longtable}[]{@{}
  >{\raggedright\arraybackslash}p{(\columnwidth - 4\tabcolsep) * \real{0.4027}}
  >{\raggedright\arraybackslash}p{(\columnwidth - 4\tabcolsep) * \real{0.1642}}
  >{\raggedright\arraybackslash}p{(\columnwidth - 4\tabcolsep) * \real{0.4331}}@{}}
\toprule\noalign{}
\multirow{2}{=}{\begin{minipage}[b]{\linewidth}\raggedright
Наименование сырья
\end{minipage}} &
\multicolumn{2}{>{\raggedright\arraybackslash}p{(\columnwidth - 4\tabcolsep) * \real{0.5973} + 2\tabcolsep}@{}}{%
\begin{minipage}[b]{\linewidth}\raggedright
Расход сырья, \%
\end{minipage}} \\
& \begin{minipage}[b]{\linewidth}\raggedright
Контроль
\end{minipage} & \begin{minipage}[b]{\linewidth}\raggedright
Опытные образцы обогащенных макаронных изделий
\end{minipage} \\
\midrule\noalign{}
\endhead
\bottomrule\noalign{}
\endlastfoot
Мука пшеничная макаронная высшего сорта & 100 & 90 \\
Тонкодисперсные порошки из гречневой муки & 0 & 10 \\
Тонкодисперсные порошки из чечевичной муки & 0 & 10 \\
Вода питьевая & По расчету & По расчету \\
\end{longtable}

Математическая обработка результатов проводилась с использованием
стандартных компьютерных программ MS Offiсe Exсel 2010 по общепринятым
методикам. Результаты экспериментальных исследований представлены
среднеарифметическими значениями, определенными из трех параллельных
измерений. при помощи высушивания.

{\bfseries Результаты и обсуждение.} В целях изучения возможности получения
специальных добавок из отечественного сырья проведен химический и
микробиологический анализ тонкодисперсных порошков из зерновых и бобовых
культур, по анализу литературы определенные, как наиболее полноценные по
химическому составу (таблица 3).

Полученные данные позволили определить наиболее ценные тонкодисперсные
порошки из зерновых и зернобобовых культур.

{\bfseries Таблица 3 - Химический и микробиологический анализ
тонкодисперсных порошков из зерновых и зернобобовых культур}

\begin{longtable}[]{@{}
  >{\raggedright\arraybackslash}p{(\columnwidth - 12\tabcolsep) * \real{0.2457}}
  >{\raggedright\arraybackslash}p{(\columnwidth - 12\tabcolsep) * \real{0.0908}}
  >{\raggedright\arraybackslash}p{(\columnwidth - 12\tabcolsep) * \real{0.0899}}
  >{\raggedright\arraybackslash}p{(\columnwidth - 12\tabcolsep) * \real{0.1396}}
  >{\raggedright\arraybackslash}p{(\columnwidth - 12\tabcolsep) * \real{0.1356}}
  >{\raggedright\arraybackslash}p{(\columnwidth - 12\tabcolsep) * \real{0.1370}}
  >{\raggedright\arraybackslash}p{(\columnwidth - 12\tabcolsep) * \real{0.1613}}@{}}
\toprule\noalign{}
\multirow{3}{=}{\begin{minipage}[b]{\linewidth}\raggedright
Наименование
\end{minipage}} &
\multicolumn{4}{>{\raggedright\arraybackslash}p{(\columnwidth - 12\tabcolsep) * \real{0.4560} + 6\tabcolsep}}{%
\begin{minipage}[b]{\linewidth}\raggedright
Химические показатели
\end{minipage}} &
\multicolumn{2}{>{\raggedright\arraybackslash}p{(\columnwidth - 12\tabcolsep) * \real{0.2983} + 2\tabcolsep}@{}}{%
\multirow{2}{=}{\begin{minipage}[b]{\linewidth}\raggedright
Показатели микробиологической безопасности
\end{minipage}}} \\
&
\multicolumn{4}{>{\raggedright\arraybackslash}p{(\columnwidth - 12\tabcolsep) * \real{0.4560} + 6\tabcolsep}}{%
\begin{minipage}[b]{\linewidth}\raggedright
массовая доля, \%
\end{minipage}} \\
& \begin{minipage}[b]{\linewidth}\raggedright
белка
\end{minipage} & \begin{minipage}[b]{\linewidth}\raggedright
жира
\end{minipage} & \begin{minipage}[b]{\linewidth}\raggedright
клетчатки
\end{minipage} & \begin{minipage}[b]{\linewidth}\raggedright
крахмала
\end{minipage} & \begin{minipage}[b]{\linewidth}\raggedright
дрожжи, КОЕ/г
\end{minipage} & \begin{minipage}[b]{\linewidth}\raggedright
плесени, КОЕ/г
\end{minipage} \\
\midrule\noalign{}
\endhead
\bottomrule\noalign{}
\endlastfoot
Тонкодисперсный порошок из пшеницы & 14,92 & 1,90 & 11,84 & 56,36 &
9*10\textsuperscript{1} & 5*10\textsuperscript{1} \\
Тонкодисперсный порошок из овса & 13,03 & 4,29 & 12,59 & 66,58 &
7*10\textsuperscript{1} & 1*10\textsuperscript{1} \\
Тонкодисперсный порошок из гречихи & 15,22 & 2,01 & 13,13 & 63,69 &
8*10\textsuperscript{1} & 3*10\textsuperscript{1} \\
Тонкодисперсный порошок из кукурузы & 9,05 & 1,49 & 10,96 & 42,78 &
9*10\textsuperscript{1} & Не обнаружено \\
Тонкодисперсный порошок из чечевицы & 22,82 & 1,92 & 10,95 & 52,67 &
7*10\textsuperscript{1} & 2*10\textsuperscript{1} \\
\end{longtable}

Для отбора наиболее перспективных с точки зрения использования как
улучшителей макаронных изделий тонкодисперсных порошков внимание
уделялось содержанию белка и клетчатки, как ценных и востребованных
компонентов растительного сырья.

Результаты химического состава полученных образцов показали, что
зернобобовые культуры обладают высоким содержанием массовой доли белка
(табл. 3). Максимальное значение содержится в тонкодисперсном порошке из
чечевицы, что составляет 22,82\%. Полученные данные согласуются с
литературными данными. Так, по данным Annalisa Romano et al {[}25{]},
чечевица известна как мясо бедняка, поскольку она является дешевым
источником белков (21--31\%).

Из зерновых культур наибольшее количество белка обнаружено в
тонкодисперсном порошке из гречихи -15,22\%, наименьшее -- из кукурузы
(9,05\%).

Таким образом, для получения обогащенных макаронных изделий выбраны
тонкодисперсные порошки из гречихи и чечевицы.

В «Учебно-научном макаронном центре» АО АТУ были проведены исследования
по производству обогащенных макаронных изделий (рис.1).

а) б) в)

\emph{а) макаронное изделие с тонкодисперсным порошком из гречихи; б)
макаронное изделие с тонкодисперсным порошком из чечевицы; в) макаронное
изделие с нанокарбоксилатами.}

{\bfseries Рис. 1 -- Фото обогащенных макаронных изделий в виде лапши}

Были получены 3 вида обогащенных макаронных изделий в виде лапши в
соответствии с разработанной рецептурой:

- макаронное изделие с тонкодисперсным порошком из гречихи;

- макаронное изделие с тонкодисперсным порошком из чечевицы;

- макаронное изделие с нанокарбоксилатами.

Результаты химического анализа обогащенных макаронных изделий с
нанокарбоксилатами и тонкодисперсными порошками из гречневой и
чечевичной муки представлены в таблице 4.

{\bfseries Таблица 4 - Химический анализ обогащенных макаронных изделий с
нанокарбоксилатами и тонкодисперсными порошками из гречневой и
чечевичной муки}

\begin{longtable}[]{@{}
  >{\raggedright\arraybackslash}p{(\columnwidth - 12\tabcolsep) * \real{0.3013}}
  >{\raggedright\arraybackslash}p{(\columnwidth - 12\tabcolsep) * \real{0.1270}}
  >{\raggedright\arraybackslash}p{(\columnwidth - 12\tabcolsep) * \real{0.1111}}
  >{\raggedright\arraybackslash}p{(\columnwidth - 12\tabcolsep) * \real{0.1587}}
  >{\raggedright\arraybackslash}p{(\columnwidth - 12\tabcolsep) * \real{0.1112}}
  >{\raggedright\arraybackslash}p{(\columnwidth - 12\tabcolsep) * \real{0.0793}}
  >{\raggedright\arraybackslash}p{(\columnwidth - 12\tabcolsep) * \real{0.1112}}@{}}
\toprule\noalign{}
\multirow{3}{=}{\begin{minipage}[b]{\linewidth}\raggedright
Наименование макаронных изделий
\end{minipage}} &
\multicolumn{6}{>{\raggedright\arraybackslash}p{(\columnwidth - 12\tabcolsep) * \real{0.6987} + 10\tabcolsep}@{}}{%
\begin{minipage}[b]{\linewidth}\raggedright
Химические показатели
\end{minipage}} \\
&
\multicolumn{3}{>{\raggedright\arraybackslash}p{(\columnwidth - 12\tabcolsep) * \real{0.3969} + 4\tabcolsep}}{%
\begin{minipage}[b]{\linewidth}\raggedright
массовая доля, \%
\end{minipage}} &
\multicolumn{3}{>{\raggedright\arraybackslash}p{(\columnwidth - 12\tabcolsep) * \real{0.3018} + 4\tabcolsep}@{}}{%
\begin{minipage}[b]{\linewidth}\raggedright
Микроэлементы
\end{minipage}} \\
& \begin{minipage}[b]{\linewidth}\raggedright
белка
\end{minipage} & \begin{minipage}[b]{\linewidth}\raggedright
клетчатки
\end{minipage} & \begin{minipage}[b]{\linewidth}\raggedright
углевода
\end{minipage} & \begin{minipage}[b]{\linewidth}\raggedright
Mg
\end{minipage} & \begin{minipage}[b]{\linewidth}\raggedright
Zn
\end{minipage} & \begin{minipage}[b]{\linewidth}\raggedright
Se
\end{minipage} \\
\midrule\noalign{}
\endhead
\bottomrule\noalign{}
\endlastfoot
С тонкодисперсным порошком из гречихи & 11,38 & 5,54 & 50,34 & - & - &
- \\
С тонкодисперсным порошком из чечевицы & 12,16 & 3,49 & 54,18 & - & - &
- \\
С нано-карбоксилатами & 9,33 & 2,17 & 53,75 & 36,84 & 0,78 & Не обн. \\
Контрольный образец & 9,31 & 2,16 & 52,4 & - & - & - \\
\end{longtable}

Результаты исследований показали, что обогащение макаронных изделий
тонкодисперсными порошками привели к значительному повышению пищевой
ценности. Так, добавление 10\% тонкодисперсного порошка из чечевицы к
мучной смеси для производства макарон привело к увеличению белка в
готовых изделиях, в сравнении с контрольным образцом, на 2,85\%,
клетчатки -- 1,33\%. Добавление 10\% гречневого тонкодисперсного порошка
увеличило массовую долю белка на 2,07\%, клетчатки -- 3,35\%.

Добавление нанокарбоксилатов магния, цинка и селена показало наличие в
готовых макаронных изделиях только магния и цинка. Отсутствие селена, по
данным исследований, объясняется тем, что и в прежних исследованиях
наблюдались потери селена в процессе помола муки и в технологиях пищевой
промышленности, что и привело к тому, что селен не был обнаружен в
мучной смеси {[}26{]}.

Характеристика полученных макаронных изделий:

- макаронное изделие с тонкодисперсным порошком из гречихи имеет светло
коричневый цвет, запах свойственный данному изделию, без постороннего
запаха, поверхность гладкая;

- макаронное изделие с тонкодисперсным порошком из чечевицы имеет темно
коричневый цвет, запах свойственный данному изделию, без постороннего
запаха, поверхность гладкая;

- макаронное изделие с нанокарбоксилатами имеет светло коричневый цвет,
запах свойственный данному изделию, без постороннего запаха, поверхность
гладкая.

Таким образом, все три варианта обогащения макаронных изделий можно
рекомендовать для производства.

{\bfseries Выводы.} Целями исследований было получение обогащённых
нанокарбоксилатами и тонкодисперсным порошком из зерновых культур
высокой питательной ценности, употребление которых позволило бы
максимально обеспечить суточные потребности в микронутриентах для
сохранения здоровья населения. Полученные в ходе исследований макаронные
изделия соответствовали требуемым потребительским достоинствам и могут
быть рекомендованы для производства.

\emph{{\bfseries Финансирование:} Представленные результаты получены в
рамках грантового финансирования № DP21681826 «Разработка технологии
производства макаронных изделий, обогащенных микроэлементами»}

{\bfseries Литература}

1. Любимая лапша: потребление макаронных изделий в Казахстане --- одно
из самых высоких в мире, РК уступает в этом разве что Италии. -URL:
https://finprom.kz/ru/article/lyubimaya-lapsha-potreblenie-makaronnyh-izdelij-v-kazahstane-odno-iz-samyh-vysokih-v-mire-rk-ustupaet-v-etom-razve-chto-italii
{[}Дата обращения 27.05.2024{]}

2. Sissons M. Pasta. In: Wrigley C., Corke H., Seetharaman K., Faubion
J., editors. Encyclopedia of Food Grains. 2nd ed. -- Oxford: Academic
Press, 2016. - P. 79--89. {[}Google Scholar{]}

3. Bustos M.C., Perez G.T., Leon A.E. Structure and quality of pasta
enriched with functional ingredients // RSC Adv. -- 2015. --Vol. 5.
-P.30780--30792. DOI: 10.1039/C4RA11857J.

4. Mercier S., Moresoli C., Mondor M., Villeneuve S., Marcos B. A
Meta-analysis of enriched pasta: What are the effects of enrichment and
process specifications on the quality attributes of pasta? Compr. Rev.
Food Sci. Food Saf. 2016;15:685--704. DOI: 10.1111/1541-4337.12207.

5. ГОСТ 31743---2017 «Изделия макаронные. Общие технические условия».
--Москва, Москва Стандартинформ, 2017. -9 с.6. ГОСТ 2874-82 «Вода
питьевая. Гигиенические требования и контроль за качеством». - ИПК
издательство стандартов Москва, 1997.

7. Wahanik A.L., Chang Y.K., Clerici P.S., Teresa M. How to make pastas
healthier? // Food Rev. Int. -- 2018. --Vol. (34). --P. 23--30. DOI:
10.1080/87559129.2016.1210634,

8. Li M., Zhu K.-X., Guo X.-N., Brijs K., Zhou H.-M. Natural additives
in wheat-based pasta and noodle products: Opportunities for enhanced
nutritional and functional properties //Compr. Rev. Food Sci. Food Saf.
-- 2014. --Vol.13(4). --P. 347--357. DOI: 10.1111/1541-4337.12066

9. Mike Sissons Development of novel pasta products with evidence based
impacts on health---a review. -2022. --Vol. 11(1). DOI:
10.3390/foods11010123

10. Dariusz Dziki. Current Trends in Enrichment of Wheat Pasta: Quality,
Nutritional Value and Antioxidant Properties. -- 2021. -Vol.9(8).
DOI:10.3390/pr9081280

11. Арсеньева, Л.Ю., Борисенко, О.В., Доценко, В.Ф. Теоретические и
практические аспекты использования тонкодиспергованых концентратов
пищевых волокон в технологи ржано-пшеничного хлеба // Научные работы
НУПТ. -2008. -№25. -- C. 115-119.

12. Anthony J. Hennessy J., Andrew R. Davies Disorders of Trace Elements
and Vitamins// Critical Care Nephrology (Second Edition). - 2009. --P.
540-545. DOI:10.1016/B978-1-4160-4252-5.50106-4

13. Скальный А. Микроэлементы: бодрость, здоровье, долголетие. Изд. 4-е,
дополненное, переработанное. --М.: Изд-во «Перо», 2019. -295 с. ISBN
978-5-00150-066-7

14. Micronutrient Deficiency. Hannah Ritchie, Max Roser. --URL:
https://ourworldindata.org/micronutrient-deficiency. {[}date of the
application 26.01.2023{]}

15. Hanife Akça and Süleyman Taban Biofortification: zinc enrichment
strategies in crops // Modern Concepts \& Developments in Agronomy.
Submission: -2021. --P. 778-782. DOI: 10.31031/MCDA.2021.08.000679

16. S Y Yu 1, Y J Zhu, W G Li Protective role of selenium against
hepatitis B virus and primary liver cancer in Qidong //Biological Trace
Element Research. 1997. --Vol. 56(1). --P. 117-124. DOI:
10.1007/BF02778987.

17. Kieliszek M, Błażejak S. Current Knowledge on the Importance of
Selenium in Food for Living Organisms: A Review // Molecules. -20167
--Vol. 21(5). DOI: 10.3390/molecules21050609

18. Aparna P. Shreenath; Muhammad Atif Ameer; Jennifer Dooley Selenium
Deficiency: StatPearls --URL:
https://www.ncbi.nlm.nih.gov/books/NBK482260/ {[}date of the application
03.06.2024{]}

19. Viering D. H. H. M., de Baaij J. H. F., Walsh S. B., Kleta R.,
Bockenhauer D. Genetic causes of hypomagnesemia, a clinical overview //
Pediatric Nephrology. -2017. --Vol. 32(7). --P. 1123--1135. DOI:
10.1007/s00467-016-3416-3

20. Abdullah M. Al Alawi, Sandawana William Majoni and Henrik Falhammar
Magnesium and Human Health: Perspectives and Research Directions //
International Journal of Endocrinology. -- 2018. DOI:
10.1155/2018/9041694

21. Costello RB, Elin RJ, Rosanoff A, et al.. Perspective: the case for
an evidence-based reference interval for serum magnesium: The time has
come // Advances in Nutrition: An International Review. -2016.
--Vol.7(6). --P. 977--993. DOI: 10.3945/an.116.012765

22. Philip G Crandall, Han-Seok Seo, Corliss A O\textquotesingle Bryan,
Jf C Meullenet. Physicochemical analysis of wheat flour fortified with
vitamin A and three types of iron source and sensory analysis of bread
using these flours// Journal of the Science of Food and Agriculture.
-2013. --Vol. 93(9). --P. 2299-2307. DOI:10.1002/jsfa.6043

23. .Shaimerdenova D.A., Chakanova Z.M., Sultanova M.Z., Shaimerdenova
P.R., Abdrakhmanov K.A. Instant cereals enriched with
carboxylatesInternational // Journal of Engineering and Technology(UAE).
-- 2018. --Vol. 7(2). --P. 140--144

24. Приказ Министра национальной экономики Республики Казахстан от 9
декабря 2016 года № 503 «Об утверждении научно обоснованных
физиологических норм потребления продуктов питания» // Министерство
юстиции Республики Казахстан. -2016.

25. Annalisa Romano, Veronica Gallo, Pasquale Ferranti. Paolo Masi
Lentil flour: nutritional and technological properties, in vitro
digestibility and perspectives for use in the food industry // Current
Opinion in Food Science. -2021. --Vol. 40. --P. 157-167. DOI:
10.1016/j.cofs.2021.04.003

26. Min Wang, Baoqiang Li, Shuang Li, Ziwei Song, Fanmei Kong, Xiaocun
Zhang Selenium in wheat from farming to food // Journal of Agricultural
and Food Chemistry. -- 2021. --Vol. 69. --P. 15458--15467. DOI:
10.1101/2021.07.17.452805.

{\bfseries References}

1. Lyubimaya lapsha: potreblenie makaronnykh izdelii v Kazakhstane ---
odno iz samykh vysokikh v mire, RK ustupaet v etom razve chto Italii.
-URL:
https://finprom.kz/ru/article/lyubimaya-lapsha-potreblenie-makaronnyh-izdelij-v-kazahstane-odno-iz-samyh-vysokih-v-mire-rk-ustupaet-v-etom-razve-chto-italii
(Data obrashcheniya 27.05.2024) {[}in Russian{]}

2. Sissons M. Pasta. In: Wrigley C., Corke H., Seetharaman K., Faubion
J., editors. Encyclopedia of Food Grains. 2nd ed. -- Oxford: Academic
Press, 2016. - P. 79--89. {[}Google Scholar{]}

3. Bustos M.C., Perez G.T., Leon A.E. Structure and quality of pasta
enriched with functional ingredients // RSC Adv. -- 2015. --Vol. 5.
-P.30780--30792. DOI: 10.1039/C4RA11857J.

4. Mercier S., Moresoli C., Mondor M., Villeneuve S., Marcos B. A
Meta-analysis of enriched pasta: What are the effects of enrichment and
process specifications on the quality attributes of pasta? Compr. Rev.
Food Sci. Food Saf. 2016;15:685--704. DOI: 10.1111/1541-4337.12207.

5. GOST 31743---2017 «Izdeliya makaronnye. Obshchie tekhnicheskie
usloviya». --Moskva, Moskva Standartinform, 2017. -9 s. 6. GOST 2874-82
«Voda pit\textquotesingle evaya. Gigienicheskie trebovaniya i
kontrol\textquotesingle{} za kachestvom». - IPK
izdatel\textquotesingle stvo standartov Moskva, 1997. {[}in Russian{]}

7. Wahanik A.L., Chang Y.K., Clerici P.S., Teresa M. How to make pastas
healthier? // Food Rev. Int. -- 2018. --Vol. (34). --P. 23--30. DOI:
10.1080/87559129.2016.1210634,

8. Li M., Zhu K.-X., Guo X.-N., Brijs K., Zhou H.-M. Natural additives
in wheat-based pasta and noodle products: Opportunities for enhanced
nutritional and functional properties //Compr. Rev. Food Sci. Food Saf.
-- 2014. --Vol.13(4). --P. 347--357. DOI: 10.1111/1541-4337.12066

9. Mike Sissons Development of novel pasta products with evidence based
impacts on health---a review. -2022. --Vol. 11(1). DOI:
10.3390/foods11010123

10. Dariusz Dziki. Current Trends in Enrichment of Wheat Pasta: Quality,
Nutritional Value and Antioxidant Properties. -- 2021. -Vol.9(8).
DOI:10.3390/pr9081280

11. Arsen\textquotesingle eva, L.Yu., Borisenko, O.V., Dotsenko, V.F.
Teoreticheskie i prakticheskie aspekty ispol\textquotesingle zovaniya
tonkodispergovanykh kontsentratov pishchevykh volokon v tekhnologi
rzhano-pshenichnogo khleba // Nauchnye raboty NUPT. -2008. -№25. -- C.
115-119. {[}in Russian{]}

12. Anthony J. Hennessy J., Andrew R. Davies Disorders of Trace Elements
and Vitamins// Critical Care Nephrology (Second Edition). - 2009. --P.
540-545. DOI:10.1016/B978-1-4160-4252-5.50106-4

13. Skal\textquotesingle nyi A. Mikroelementy: bodrost\textquotesingle,
zdorov\textquotesingle e, dolgoletie. Izd. 4-e, dopolnennoe,
pererabotannoe. --M.: Izd-vo «Pero», 2019. -295 s. ISBN
978-5-00150-066-7 {[}in Russian{]}

14. Micronutrient Deficiency. Hannah Ritchie, Max Roser. --URL:
https://ourworldindata.org/micronutrient-deficiency. {[}date of the
application 26.01.2023{]}

15. Hanife Akça and Süleyman Taban Biofortification: zinc enrichment
strategies in crops // Modern Concepts \& Developments in Agronomy.
Submission: -2021. --P. 778-782. DOI: 10.31031/MCDA.2021.08.000679

16. S Y Yu 1, Y J Zhu, W G Li Protective role of selenium against
hepatitis B virus and primary liver cancer in Qidong //Biological Trace
Element Research. 1997. --Vol. 56(1). --P. 117-124. DOI:
10.1007/BF02778987.

17. Kieliszek M, Błażejak S. Current Knowledge on the Importance of
Selenium in Food for Living Organisms: A Review // Molecules. -20167
--Vol. 21(5). DOI: 10.3390/molecules21050609

18. Aparna P. Shreenath; Muhammad Atif Ameer; Jennifer Dooley Selenium
Deficiency: StatPearls --URL:
https://www.ncbi.nlm.nih.gov/books/NBK482260/ {[}date of the application
03.06.2024{]}

19. Viering D. H. H. M., de Baaij J. H. F., Walsh S. B., Kleta R.,
Bockenhauer D. Genetic causes of hypomagnesemia, a clinical overview //
Pediatric Nephrology. -2017. --Vol. 32(7). --P. 1123--1135. DOI:
10.1007/s00467-016-3416-3

20. Abdullah M. Al Alawi, Sandawana William Majoni and Henrik Falhammar
Magnesium and Human Health: Perspectives and Research Directions //
International Journal of Endocrinology. -- 2018. DOI:
10.1155/2018/9041694

21. Costello RB, Elin RJ, Rosanoff A, et al.. Perspective: the case for
an evidence-based reference interval for serum magnesium: The time has
come // Advances in Nutrition: An International Review. -2016.
--Vol.7(6). --P. 977--993. DOI: 10.3945/an.116.012765

22. Philip G Crandall, Han-Seok Seo, Corliss A O\textquotesingle Bryan,
Jf C Meullenet. Physicochemical analysis of wheat flour fortified with
vitamin A and three types of iron source and sensory analysis of bread
using these flours// Journal of the Science of Food and Agriculture.
-2013. --Vol. 93(9). --P. 2299-2307. DOI:10.1002/jsfa.6043

23. .Shaimerdenova D.A., Chakanova Z.M., Sultanova M.Z., Shaimerdenova
P.R., Abdrakhmanov K.A. Instant cereals enriched with
carboxylatesInternational // Journal of Engineering and Technology(UAE).
-- 2018. --Vol. 7(2). --P. 140--144

24. Prikaz Ministra natsional\textquotesingle noi ekonomiki Respubliki
Kazakhstan ot 9 dekabrya 2016 goda № 503 «Ob utverzhdenii nauchno
obosnovannykh fiziologicheskikh norm potrebleniya produktov pitaniya» //
Ministerstvo yustitsii Respubliki Kazakhstan. -2016. {[}in Russian{]}

25. Annalisa Romano, Veronica Gallo, Pasquale Ferranti. Paolo Masi
Lentil flour: nutritional and technological properties, in vitro
digestibility and perspectives for use in the food industry // Current
Opinion in Food Science. -2021. --Vol. 40. --P. 157-167. DOI:
10.1016/j.cofs.2021.04.003

26. Min Wang, Baoqiang Li, Shuang Li, Ziwei Song, Fanmei Kong, Xiaocun
Zhang Selenium in wheat from farming to food // Journal of Agricultural
and Food Chemistry. -- 2021. --Vol. 69. --P. 15458--15467. DOI:
10.1101/2021.07.17.452805.

{\bfseries Сведения об авторах}

Шаймерденова Д.А. - доктор технических наук, ТОО
"Научно-производственное предприятие "Инноватор", Астана, Казахстан,
e-mail: darigash@mail.ru;

Омаралиева А.М. - кандидат технических наук, ассоциированный профессор
кафедры «Технология и стандартизация», АО «Казахский университет
технологии и бизнеса» им. К. Кулажанова» Астана, Казахстан, e-mail:
aigul-omar@mail.ru;

Тарабаев Б.К. - кандидат технических наук, «Казахский агротехнический
исследовательский университет. Сейфуллин", Астана, Казахстан,
e-mail:tarabaev50@mail.ru;

Сарбасова Л.Т. - кандидат технических наук, ТОО "Научно-производственное
предприятие "Инноватор", Астана, Казахстан, e-mail: sargt@mail.ru;

Ануарбекова С.С. -- кандидат медицинских наук, ТОО
"Научно-производственное предприятие "Инноватор", Астана, Казахстан,
e-mail: sanuarbekova@rambler.ru;

Искакова Д.Б. - ТОО "Научно-производственное предприятие "Инноватор ",
Астана, Казахстан, e-mail: damirais 61@mail.ru;

Шаймерденов А.А. - ТОО ``Научно-производственное предприятие ``Инноватор
``, Астана, Казахстан, e-mail: darigash@mail.ru;

Тастанов Д.А. - ТОО ``Научно-производственное предприятие ``Инноватор
``, Астана, Казахстан,e-mail: dias.tastanov@mail.ru

{\bfseries Information about the authors}

Shaimerdenova D. A. - Doctor of Technical Sciences, «Scientific and
production enterprise «Innovator» LLP, Astana, Kazakhstan, e-mail:
darigash@mail.ru;

Omaralieva A. M. - Candidate of Technical Sciences, Associate Professor
of the Department of Technology and Standardization of "K. Kulazhanov
Kazakh University of Technology and Business JSC" Astana, е-mail:
aigul-omar@mail.ru;

Tarabayev B.K. - Candidate of Technical Sciences, Kazakh Agrotechnical
Research University Seifullin, Astana, Kazakhstan, e-mail:
tarabaev50@mail.ru;

Sarbasova L.T. - Candidate of Technical Sciences, «Scientific and
production enterprise «Innovator» LLP, ,Astana, Kazakhstan, e-mail:
sargt@mail.ru;

Anuarbekova S.S. - Candidate of Medical Sciences, «Scientific and
production enterprise «Innovator» LLP, , Astana, Kazakhstan, e-mail:
sanuarbekova@rambler.ru;

Iskakova D.B. -«Scientific and production enterprise «Innovator» LLP,
Astana, Kazakhstan,e-mail: damirais 61@mail.ru;

Shaimerdenov A.A. - Scientific and Production Enterprise «Innovator»
LLP, Kazakhstan, Astana, LLP «Scientific and production enterprise
«Innovator», Astana, Kazakhstan, e-mail: darigash@mail.ru;

Tastanov D.A. - «Scientific and production enterprise «Innovator» LLP,
Astana, Kazakhstan, e-mail: dias.tastanov@mail.ru\newpage
{\bfseries IRSTI 65.51.03}

{\bfseries STUDY OF PHYSICAL AND CHEMICAL PROPERTIES OF SOFT DRINKS
OBTAINED WITH ADDITION OF GOOSEBERRY EXTRACT}

{\bfseries \textsuperscript{1}B.Khamitova}\textsuperscript{🖂}{\bfseries ,
\textsuperscript{2}F.Dikhanbayeva , \textsuperscript{1}G.Koshtayeva}

\textsuperscript{1}M.Auezov South-Kazakhstan University, Shymkent,
Kazakhstan,

\textsuperscript{2}Almaty Technological University, Almaty, Kazakhstan

\textsuperscript{🖂}Correspondent-author: barno-007@mail.ru

The manufacturing of juice-containing soft drinks is experiencing
significant expansion in both international and domestic markets,
including Kazakhstan. This tendency can be attributed to the aspiration
of the working-age population in developed countries to adopt a
health-conscious lifestyle. The manufacturing of beverages using natural
fruit and berry ingredients is highly intriguing in this context.
Enriching soft drinks with a diverse array of physiologically active
chemicals derived from plant materials is highly significant. This study
aims to investigate the physicochemical characteristics of soft drinks
and assess the potential application of gooseberry extract.~

In order to create a non-alcoholic beverage, the valuable wild
gooseberries were utilised. These gooseberries are rich in biologically
active elements, including vitamins, vitamin-like compounds, flavonoids,
minerals, and other chemicals.~

The acquired results are derived from a substantial volume of empirical
study and are founded upon an examination of literary data pertaining to
the chemical makeup of gooseberries. The study investigated the chemical
makeup of gooseberries using contemporary techniques of chemical
analysis. A recipe for a non-alcoholic beverage made using gooseberry
extract, free from any artificial ingredients, has been devised. The
organoleptic and physico-chemical features of it have been determined.~

{\bfseries Keywords:} soft drinks, gooseberriy, plant extracts, fruit and
berry raw materials, functional purpose, ultrasonic extraction method

{\bfseries ИССЛЕДОВАНИЕ ФИЗИКО-ХИМИЧЕСКИХ СВОЙСТВ БЕЗАЛКОГОЛЬНЫХ НАПИТКОВ,
ПОЛУЧЕННЫХ С ДОБАВЛЕНИЕМ ЭКСТРАКТА КРЫЖОВНИКА}

{\bfseries \textsuperscript{1} Б.М. Хамитова}\textsuperscript{🖂}{\bfseries ,
\textsuperscript{2} Ф.Т. Диханбаева, \textsuperscript{1}Г.Е. Коштаева}

\textsuperscript{1}Южно-Казахстанский университет им. М.Ауэзова,
Казахстан, Шымкент, Казахстан,

\textsuperscript{2}Алматинский технологический университет, Казахстан,
Алматы, Казахстан,

e-mail: barno-007@mail.ru

В последнее время наблюдается значительный рост производства
безалкогольных напитков, содержащих сок, как на внутреннем, так и на
международном рынках, включая Казахстан. Эту тенденцию можно связать с
желанием трудоспособного населения в промышленно развитых странах вести
здоровый образ жизни. В этом контексте производство напитков на основе
натуральных фруктово-ягодных компонентов вызывает большой интерес.
Актуальным является и добавление в безалкогольные напитки широкого
спектра биологически активных веществ, полученных из растительного
сырья. Цель данного исследования - изучение физико-химических свойств
безалкогольных напитков и оценка возможности применения экстракта
крыжовника.

Использование дикого крыжовника позволило получить безалкологольный
напиток, содержащий значительное количество биологически активных
компонентов, таких как витамины, витаминоподобные соединения,
флавоноиды, минералы и другие вещества.

Полученные результаты основаны на анализе литературы, посвященной
химическому составу крыжовника, а также на значительном объеме
проведенных экспериментальных исследований. В ходе исследования
использовались современные методы химического анализа для изучения
химического состава крыжовника. С использованием экстракта крыжовника
была разработана рецептура напитка, не содержащего искусственных
компонентов и спирта. Также помимо физико-химических свойств, были
выявлены и органолептические характеристики напитка.

{\bfseries Ключевые слова:} безалкогольные напитки, крыжовник, растительные
экстракты, фруктово-ягодное сырье, функционального назначения,
ультразвуковой способ экстрагирования

{\bfseries ҚАРЛЫҒАН СЫҒЫНДЫСЫ ҚОСЫЛҒАН АЛКОГОЛЬСІЗ СУСЫНДАРДЫҢ
ФИЗИКАЛЫҚ-ХИМИЯЛЫҚ ҚАСИЕТТЕРІН ЗЕРТТЕУ}

{\bfseries \textsuperscript{1}Б.М. Хамитова}\textsuperscript{🖂}{\bfseries ,
\textsuperscript{2} Ф.Т. Диханбаева, \textsuperscript{1}Г.Е. Коштаева}

\textsuperscript{1}М. Әузов атындағы Оңтүстік-Қазақстан университеті,
Шымкент, Қазақстан,

\textsuperscript{2}Алматы технологиялық университеті, Алматы, Қазақстан,

e-mail: barno-007@mail.ru

Құрамында шырыны бар алкогольсіз сусындар өндірісінің ассортименті
шетелде де, Қазақстанда да қарқынды өсуде. Бұл үрдіс дамыған елдердегі
тұрғындардың белсенді бөлігінің салауатты өмір салтына ұмтылысына
байланысты. Осыған орай, табиғи жеміс-жидек шикізаты негізіндегі
сусындар өндірісі үлкен қызығушылық тудыруда. Алкогольсіз сусындарды
өсімдік тектес биологиялық белсенді заттардың кең спектрімен байыту өте
маңызды. Бұл жұмыстың мақсаты алкогольсіз сусындардың физика-химиялық
қасиеттерін зерттеу, сонымен қатар қарлыған сығындысын пайдалану
мүмкіндігін зерттеу.

Алкогольсіз сусын алу үшін дәрумендер мен витаминге ұқсас қосылыстар,
флавоноидтар, минералдар және басқа заттар сияқты биологиялық белсенді
заттардың құнды көзі болып табылатын жабайы қарлыған пайдаланылды.

Алынған нәтижелер тәжірибелік зерттеулердің айтарлықтай көлеміне және
қарлығанның химиялық құрамы туралы әдеби деректерді талдауға
негізделген. Жұмыста қазіргі заманғы химиялық талдау әдістерін қолдану
арқылы қарлығанның химиялық құрамы зерттелді. Құрамында синтетикалық
компоненттері жоқ қарлыған сығындысы бар алкогольсіз сусынның рецепті
әзірленді. Оның органолептикалық және физика-химиялық көрсеткіштері
анықталды.

{\bfseries Түйін сөздер:} алкогольсіз сусындар, қарлыған, өсімдік
сығындылары, жеміс-жидек шикізаты, функционалдық мақсаты, ультрадыбыстық
экстрагирлеу тәсілі

{\bfseries Introduction.} The soft drink business is experiencing
significant growth and is one of the fastest-growing sub-industries in
the global food sector. Both the production and consumption of soft
drinks are exhibiting a consistent upward trend, both currently and in
the foreseeable future. Furthermore, alongside the proactive adoption of
novel packaging formats, a media-based advertising campaign is being
executed, effectively capturing the interest of a growing customer base
{[}1{]}.

The priority direction of that area is considered to be the
diversification of soft drinks, including low-calorie specialized drinks
with various functional orientations. Development of non-alcoholic
industry was to be carried out in two main directions: increasing the
production of drinks on fruit and berry and malt raw materials;
increasing the production of tonic and fortified drinks, as well as
"protection" drinks having a special purpose {[}2{]}.

At present, the challenges of the logical and efficient use of commonly
available plant materials as a valuable source of functional components
and the creation of healthy soft beverages are highly essential.

It is widely known that food has a significant impact on human health.
Antioxidants can help reduce environmental oxidative stress caused by
free radicals, which can damage the body\textquotesingle s cellular
system. Free radicals can be generated intracellularly due to the impact
of detrimental factors such radiation, UV radiation, and chemical
processes involving polycyclic aromatic hydrocarbons. Free radicals are
accountable for the partial or complete degradation of lipids and
proteins in the human body. The degradation mentioned causes cellular
and genetic mutations, as well as interactions with polyunsaturated
fatty acids, DNA, and proteins. These interactions ultimately contribute
to the development of various illnesses.

Conversely, antioxidants hinder the oxidation of lipids by interacting
with free radicals. Plant-based commodities, such as fruits and berries,
serve as the main sources of antioxidants. The reason for this is that
only plant-based products have the ability to generate bioflavonoids and
other polyphenolic compounds. The exploitation of indigenous plant
resources, which provide the greatest health benefits to those living in
the same region, is a particularly promising strategy {[}3,4{]}.

In recent years, scientists have focused their research on developing
innovative formulations and technologies for soft drinks that not only
quench thirst and provide refreshment, but also have physiological or
preventative effects. Considerable emphasis is placed on enhancing the
longevity of beverages throughout the storing process. There are novel
varieties of soft drinks that distinguish themselves from conventional
ones in terms of both ingredients and production methods, as well as in
terms of taste and their effects on the human body {[}5{]}.~

An efficient approach to addressing nutritional deficiencies caused by
vitamin deficiencies is the advancement of novel formulations and
technology for juice-based products with functional properties.
Therefore, it is necessary to develop novel plant-based products
utilising indigenous raw ingredients.~

The utilisation of plant-derived raw materials for the development of
novel food items offers several benefits owing to the elevated
bioactivity and bioavailability of the active food constituents present
in them. Fruits and berries have a limited duration before they spoil,
which necessitates the development of processing techniques to ensure a
continuous supply of these items to the population throughout the year.
The plants contain biologically active compounds that determine the
specific attributes of the resulting product and provide essential
technological characteristics. This eliminates the need for the addition
of flavours, colours, and preservatives. One method for maintaining the
advantageous qualities of fruits and berries, such as their antioxidant
capabilities, all year round is by creating fruit and berry extracts and
incorporating them into food {[}6{]}.~

The efficiency of the process of extracting biologically active
chemicals from plants is influenced by key technological elements such
as temperature, extraction time, degree of raw material grinding, type
of extractant, hydromodule, and others. Every variety of plant raw
material possesses certain parameters, modes, and conditions that have
been determined through experimental research {[}7{]}. To introduce
natural flavors, including essential oils, into drink recipes,
surfactants (surfactants) are needed to distribute them evenly
throughout the volume of the drink. Highly effective surfactants include
triterpene plant saponins, which have a wide range of pharmacological
effects (hypercholesterolemic, anticarcinogenic, hepatoprotective
effects; antioxidant, immunological effects, and so on) {[}8,9{]}.

Here are a few instances of biologically active supplements that have
been proposed: the ginseng biomass infusion, known as "BAD-GS," consists
of potassium, sodium ions, and 12 trace elements. The preparation called
"MIGI-K-LP" is derived from mussel meat and possesses radioprotective
and anti-inflammatory effects. The preparation called "Zosterin" is
obtained from seaweed and contains a substantial quantity of
polygalacturonic acid. In addition, the therapy process include
infusions of medicinal plants such as Chinese lemongrass, levzei
safflower, and eleuterococcus. This foundation has been employed in the
development of several beverages that possess both preventive and
therapeutic properties: {[}1,10{]}.

Drinks on flavors occupy a significant segment of the market, as they
are the most popular due to the presence of a large variety of flavoring
components, high organoleptic indicators and relatively low cost. For
flavouring beverages, artificial and identical natural flavors are
mainly used, and water of various degrees of carbonation is used as a
base {[}11,12{]}.

It is crucial to incorporate plant extracts in the formulation of
flavoured beverages to enhance the presence of their functional elements
and biologically active substances (BAS) with antioxidant properties.
This is due to the fact that contemporary clients possess tastes that
diverge from those held by prior generations. Plant raw materials
contain a substantial amount of phenolic compounds, alkaloids,
glycosides, polysaccharides, organic acids, essential oils, vitamins,
minerals, and other components. These molecules exert a favourable
influence on the physiological functioning of several systems inside the
human body, encompassing the digestive, urinary, cardiovascular,
immunological, and other systems {[}13{]}.~

Specialised beverages tailored for athletes are currently being
formulated, which include energy drinks infused with juices, extracts,
caffeine, ginseng preparations, and other natural adaptogens. A diverse
assortment of powdered drink combinations incorporating medicinal and
preventative characteristics derived from vegetable raw materials has
been created {[}14,15{]}.

For completion of losses of liquid during trainings and competitions use
specialized sports drinks generally on the basis of a carbohydrate
chloridno - sodium composition. But at the same time it is necessary
that sports drinks not only recovered losses of liquid, but also had
functional focus that is reached by enrichment of a compounding with
biologically active agents. A specific place is held by the substances
possessing adaptogenny action, in particular, extracts of plants, for
example, of an echinacea, a ginseng, ginger and a St.
John\textquotesingle s wort {[}16,17{]}.

Enriching soft drinks with polycomponent systems of plant extractives in
the form of concentrates and bases is a new approach to promoting
health, improving productivity, and supporting the
body\textquotesingle s natural healing processes.

{\bfseries Materials and methods}\emph{. Methods (methodology) of the
experiment}

Apple juice and gooseberry extracts were the primary components utilised
in the manufacturing of soft drinks.~

Due to the distinctive composition of gooseberry (Red Large variety),
which contains significant amounts of vitamins A and C, as well as
vitamins E, PP, B groups, and various minerals like potassium, calcium,
iron, zinc, and others, gooseberry was selected as the main
ingredient.~\\
The chemical makeup of gooseberries is influenced by various elements
such as the variety, age, soil conditions, and other environmental
factors. Consequently, the data regarding the chemical composition of
gooseberries from different sources are more prone to variation compared
to the data for other garden crops {[}1{]}.~

The health advantages of berries are attributed to a combination of
beneficial compounds and vital vitamins. The product\textquotesingle s
pulp is distinguished by the presence of pectins, minerals, and metals.
Table 1 shows the amount of beneficial components and vitamins found in
100 grammes of gooseberries.

{\bfseries Table 1 - Useful components and vitamins in composition of
gooseberries}

\begin{longtable}[]{@{}
  >{\raggedright\arraybackslash}p{(\columnwidth - 10\tabcolsep) * \real{0.1633}}
  >{\raggedright\arraybackslash}p{(\columnwidth - 10\tabcolsep) * \real{0.1668}}
  >{\raggedright\arraybackslash}p{(\columnwidth - 10\tabcolsep) * \real{0.1668}}
  >{\raggedright\arraybackslash}p{(\columnwidth - 10\tabcolsep) * \real{0.1669}}
  >{\raggedright\arraybackslash}p{(\columnwidth - 10\tabcolsep) * \real{0.1516}}
  >{\raggedright\arraybackslash}p{(\columnwidth - 10\tabcolsep) * \real{0.1845}}@{}}
\toprule\noalign{}
\begin{minipage}[b]{\linewidth}\raggedright
{\bfseries Vitamins}
\end{minipage} & \begin{minipage}[b]{\linewidth}\raggedright
{\bfseries Quantity, mg}
\end{minipage} & \begin{minipage}[b]{\linewidth}\raggedright
{\bfseries \% from 100 g norm}
\end{minipage} & \begin{minipage}[b]{\linewidth}\raggedright
{\bfseries Minerals}
\end{minipage} & \begin{minipage}[b]{\linewidth}\raggedright
{\bfseries Quantity, mg}
\end{minipage} & \begin{minipage}[b]{\linewidth}\raggedright
{\bfseries \% from 100 g norm}
\end{minipage} \\
\midrule\noalign{}
\endhead
\bottomrule\noalign{}
\endlastfoot
А & 0.033 & 3.6 & potassium & 260 & 10.4 \\
В\textsubscript{1} & 0.01 & 0.7 & calcium & 22 & 2.2 \\
В\textsubscript{2} & 0.02 & 1.1 & magnesium & 9.0 & 2.3 \\
В\textsubscript{4} & 42.1 & 5.11 & sodium & 23 & 1.8 \\
В\textsubscript{5} & 0.286 & 5.0 & sulfur & 18 & 1.8 \\
В\textsubscript{6} & 0.03 & 1.5 & phosphorus & 28 & 4.0 \\
В\textsubscript{9} & 5.0 mkg & 1.3 & chlorine & 1.0 & 3.5 \\
С & 30.0 & 33.3 & iron & 0.8 & 4.4 \\
Е & 0.5 & 3.3 & iodine & 1.0 & 0.7 \\
K & 7.8 mkg & 0.7 & manganese & 0.45 & 22.5 \\
РР & 0.4 & 2.0 & copper & 130 & 13 \\
Niacin B\textsubscript{3} & 0.3 & 3.1 & molybdenum & 12 & 17.1 \\
Antioxidants & 0.389±0.005 & 0.413±0.006 & fluorine & 12 & 0.3 \\
& & & chromous & 1.0 & 2.0 \\
& & & zink & 0.09 & 0.8 \\
\multicolumn{6}{@{}>{\raggedright\arraybackslash}p{(\columnwidth - 10\tabcolsep) * \real{1.0000} + 10\tabcolsep}@{}}{%
Note: compiled based on source {[}1{]}} \\
\end{longtable}

Gooseberries are popular in diets due to their low calorie level, high
liquid content, presence of fibre, and pectin content.~

Based on our analysis of the literature and existing recipes for
producing beverages, we have determined that incorporating gooseberries
into the formulation will enhance its composition, while also imparting
a more delicate hue and flavour to the drink.

\emph{Experimental part}

Extracts are highly concentrated juices that are free of pectin and can
be produced using sulfitated materials. Consequently, the extraction of
aromatic compounds does not occur throughout the manufacturing process.
The extracts are utilised in the production of carbonated beverages.~

Ultrasound is a highly promising technique for enhancing the extraction
of plant resources. Utilising the ultrasonic extraction method can
effectively decrease the time required for the procedure and result in a
more thorough extraction of compounds {[}18 in Russian{]}.~

The extraction of nutrients from a mixture depends not only on the
composition of the raw materials, but also on the specific type of
extractant used. In order to ascertain the most effective extractant and
the optimal percentage of raw materials, we generated numerous samples
of gooseberry extracts using the technique outlined below. To attain a
particle size of 1-2 millimetres, we measured and pulverised the
unprocessed components. Subsequently, we mixed the raw materials with
distilled water and an aqueous solution of ethyl alcohol, which had
concentrations of 10\%, 15\%, and 20\%. This was done using the standard
method and the ethyl alcohol solution had a volume of 40\% at room
temperature. The mixture was left for a duration of one hundred twenty
minutes {[}1{]}.~

The low-frequency ultrasonic device was utilised to perform ultrasonic
processing brand PLS-FSJ-300 made in China. The container containing a
sample of raw materials is positioned into an isothermal bath that has
been pre-heated to a temperature range of 38-40 degrees Celsius. The
reverse refrigerator initiates operation upon turning on the water pipe
valve. An electrically powered hoover pump is activated. Once the
residual pressure in the system has been measured and the length of
ultrasound treatment for the raw materials has been set to 15 minutes,
the low-frequency ultrasonic device is activated. After completing the
ultrasound processing of the raw materials, the vacuum pump is turned
off and the vacuum flow valve is opened to remove the container
containing the extract. Subsequently, the extract is strained using a
sieve, and the residual substance is then subjected to compression. The
obtained extract is forwarded for more investigation.~

{\bfseries Results and discussion.} The study employed physicochemical
research methodologies, adhering to the technical regulations and
standards specified for this particular product {[}19{]}. The sensory
parameters of the juice-containing beverage were determined using
established procedures {[}20{]}.~

Gooseberry extract was utilized at every stage of the soft drink
manufacturing process. The recipes for soft drinks containing gooseberry
extract are provided in table 2 {[}1{]}.

{\bfseries Table 2 - Formulations of soft carbonated drinks with goosberry
juice per 100 dal of finished product}

\begin{longtable}[]{@{}
  >{\raggedright\arraybackslash}p{(\columnwidth - 12\tabcolsep) * \real{0.1598}}
  >{\raggedright\arraybackslash}p{(\columnwidth - 12\tabcolsep) * \real{0.1407}}
  >{\raggedright\arraybackslash}p{(\columnwidth - 12\tabcolsep) * \real{0.1461}}
  >{\raggedright\arraybackslash}p{(\columnwidth - 12\tabcolsep) * \real{0.1369}}
  >{\raggedright\arraybackslash}p{(\columnwidth - 12\tabcolsep) * \real{0.1461}}
  >{\raggedright\arraybackslash}p{(\columnwidth - 12\tabcolsep) * \real{0.1339}}
  >{\raggedright\arraybackslash}p{(\columnwidth - 12\tabcolsep) * \real{0.1367}}@{}}
\toprule\noalign{}
\multirow{3}{=}{\begin{minipage}[b]{\linewidth}\raggedright
Raw material
\end{minipage}} &
\multicolumn{2}{>{\raggedright\arraybackslash}p{(\columnwidth - 12\tabcolsep) * \real{0.2868} + 2\tabcolsep}}{%
\begin{minipage}[b]{\linewidth}\raggedright
formulation 1
\end{minipage}} &
\multicolumn{2}{>{\raggedright\arraybackslash}p{(\columnwidth - 12\tabcolsep) * \real{0.2829} + 2\tabcolsep}}{%
\begin{minipage}[b]{\linewidth}\raggedright
formulation 2
\end{minipage}} &
\multicolumn{2}{>{\raggedright\arraybackslash}p{(\columnwidth - 12\tabcolsep) * \real{0.2705} + 2\tabcolsep}@{}}{%
\begin{minipage}[b]{\linewidth}\raggedright
formulation 3
\end{minipage}} \\
&
\multicolumn{6}{>{\raggedright\arraybackslash}p{(\columnwidth - 12\tabcolsep) * \real{0.8402} + 10\tabcolsep}@{}}{%
\begin{minipage}[b]{\linewidth}\raggedright
Content of raw material in juice
\end{minipage}} \\
& \begin{minipage}[b]{\linewidth}\raggedright
measuring unit
\end{minipage} & \begin{minipage}[b]{\linewidth}\raggedright
quantity
\end{minipage} & \begin{minipage}[b]{\linewidth}\raggedright
measuring unit
\end{minipage} & \begin{minipage}[b]{\linewidth}\raggedright
quantity
\end{minipage} & \begin{minipage}[b]{\linewidth}\raggedright
measuring unit
\end{minipage} & \begin{minipage}[b]{\linewidth}\raggedright
quantity
\end{minipage} \\
\midrule\noalign{}
\endhead
\bottomrule\noalign{}
\endlastfoot
sugar & kg & 75.16 & kg & 65.90 & kg & 29.26 \\
apple juice & \emph{l} & 95.5 & \emph{l} & 95.5 & \emph{l} & 95.5 \\
raspberry juice & \emph{l} & 23.46 & \emph{l} & 26.3 & \emph{l} &
24.7 \\
gooseberry extract & \emph{l} & 0.35 & \emph{l} & 1.43 & \emph{l} &
1.408 \\
citric acid & kg & 2.46 & kg & 2.32 & kg & 2.12 \\
essential oil & \emph{l} & 0.002 & \emph{l} & - & \emph{l} & 0.004 \\
color & kg & 0.35 & kg & - & kg & - \\
\multicolumn{7}{@{}>{\raggedright\arraybackslash}p{(\columnwidth - 12\tabcolsep) * \real{1.0000} + 12\tabcolsep}@{}}{%
Note: compiled based on source {[}1{]}} \\
\end{longtable}

Organoleptic indicators are assessed through visual observation and
taste evaluation to evaluate quality factors such as appearance, colour,
taste, scent, and transparency of the drink.~

The table 3 displays the sensory properties of the soft drink containing
gooseberry extract.~

{\bfseries Table 3 - Organoleptic characteristics of soft drink with
gooseberry extract}

\begin{longtable}[]{@{}
  >{\raggedright\arraybackslash}p{(\columnwidth - 6\tabcolsep) * \real{0.2500}}
  >{\raggedright\arraybackslash}p{(\columnwidth - 6\tabcolsep) * \real{0.2500}}
  >{\raggedright\arraybackslash}p{(\columnwidth - 6\tabcolsep) * \real{0.2500}}
  >{\raggedright\arraybackslash}p{(\columnwidth - 6\tabcolsep) * \real{0.2501}}@{}}
\toprule\noalign{}
\begin{minipage}[b]{\linewidth}\raggedright
Indicator
\end{minipage} & \begin{minipage}[b]{\linewidth}\raggedright
formulation 1
\end{minipage} & \begin{minipage}[b]{\linewidth}\raggedright
formulation 2
\end{minipage} & \begin{minipage}[b]{\linewidth}\raggedright
formulation 3
\end{minipage} \\
\midrule\noalign{}
\endhead
\bottomrule\noalign{}
\endlastfoot
Appearance & Nontransparent liquid, without seeds and impurities &
Nontransparent liquid, without seeds and impurities & Nontransparent
liquid, without seeds and impurities \\
Color & ruby & ruby & Saturated ruby \\
Taste, aroma & Taste is peculiar to gooseberry, pleasant aroma & Taste
is peculiar to gooseberry, pleasant aroma & Taste is peculiar to
gooseberry, pleasant aroma \\
\multicolumn{4}{@{}>{\raggedright\arraybackslash}p{(\columnwidth - 6\tabcolsep) * \real{1.0000} + 6\tabcolsep}@{}}{%
Note: compiled based on source {[}1{]}} \\
\end{longtable}

The table 4 displays the physical and chemical characteristics of the
soft drink containing gooseberry extract.

{\bfseries Table 4 - Physical and chemical parameters of soft drink with
gooseberry extract}

\begin{longtable}[]{@{}
  >{\raggedright\arraybackslash}p{(\columnwidth - 6\tabcolsep) * \real{0.2500}}
  >{\raggedright\arraybackslash}p{(\columnwidth - 6\tabcolsep) * \real{0.2500}}
  >{\raggedright\arraybackslash}p{(\columnwidth - 6\tabcolsep) * \real{0.2500}}
  >{\raggedright\arraybackslash}p{(\columnwidth - 6\tabcolsep) * \real{0.2501}}@{}}
\toprule\noalign{}
\begin{minipage}[b]{\linewidth}\raggedright
Indicator
\end{minipage} & \begin{minipage}[b]{\linewidth}\raggedright
formulation 1
\end{minipage} & \begin{minipage}[b]{\linewidth}\raggedright
formulation 2
\end{minipage} & \begin{minipage}[b]{\linewidth}\raggedright
formulation 3
\end{minipage} \\
\midrule\noalign{}
\endhead
\bottomrule\noalign{}
\endlastfoot
Mass share of dry substances, \% & 7.3 & 8.1 & 8.7 \\
Acidity, ml of 1 М solution NaОН for 100 ml of drink & 2.5 & 2.9 &
3.7 \\
Mass share of vitamin С, \%~ & 3.2 & 3.3 & 4.3 \\
Vitamin P, mg\% & 3.5 & 3.7 & 4.2 \\
Pectin substances, \% & 2.7-3.0 & 1.6-2.2 & 2.1-2.7 \\
рН & 4.4 & 4.6 & 4.4±0.2 \\
\multicolumn{4}{@{}>{\raggedright\arraybackslash}p{(\columnwidth - 6\tabcolsep) * \real{1.0000} + 6\tabcolsep}@{}}{%
Note: compiled based on source {[}1{]}} \\
\end{longtable}

Tables 3 and 4 demonstrate that the sensory, physical and chemical
characteristics of the soft drink align with the established criteria
for soft drinks. The preparation of the drink enables the following:
diversifying the range of options, enhancing the presence of
biologically active compounds, improving the sensory characteristics of
the product, and imparting functional properties to the drink {[}1{]}.~

The study findings reveal that the produced soft drink includes
significant amounts of biologically active components, including 3.2-4.3
mg\% of ascorbic acid and 3.5-4.2 mg\% of vitamin P.

{\bfseries Conclusions.} Based on the experimental results, it can be
concluded that gooseberry extract can be used as a supplement for soft
drinks. The utilisation of gooseberry extract in the manufacturing of
soft drinks serves as proof of its ability to enhance the sensory
characteristics of the beverage and enable the creation of a functional
beverage that offers both therapeutic and preventative benefits.

{\bfseries References}

1. B.M.Khamitova, B.T. Abdizhapparova, E.E. Tazhenov.~ Study of physical
and chemical properties of soft drinks obtained with addition of
gooseberry extract // Industrial Technologies and Engineering - 2019.
Vol. I. - P. 306-310.

2.Pomozova V.A. Proizvodstvo kvasa i bezalkogol\textquotesingle nyh
napitkov: Uchebnoe posobie. - SPb.: GIORD, 2006. - 192 s. ISBN
5-98879-029-1 {[}in Russian{]}

3. Donchenko G.V., Krichkovskaya L.V., CHernyshov S.I., Nikitchenko
YU.V. i dr. Prirodnye antioksidanty (biotekhnologicheskie,
biologicheskie i medicinskie aspekty): monografiya.
Har\textquotesingle kov: «Model\textquotesingle{} Vselennoj». 2011. -
376 s. {[}in Russian{]}

4. Belokurova E.V., Solohin S.A., Rodionov A.A. Razrabotka tehnologii
bulochnyh izdelij s vneseniem probioticheskogo bakkoncentrata
«Immunolakt»// Tekhnologii pishchevoj i pererabatyvayushchej
promyshlennosti APK -- produkty zdorovogo pitaniya -- 2016. -- № 3. --
S. 51-55. {[}in Russian{]}

5. SHumann, G. Bezalkogol\textquotesingle nye napitki:
syr\textquotesingle e, tekhnologii, normativy / G. SHumann ; perevod s
nemeckogo pod obshch. red. A. V. Oreshchenko i L. N. Benevolenskoj. -
Sankt-Peterburg : Professiya, 2004 (GP Tekhn. kn.). - 278 s. ISBN
5-93913-063-1

6. Filonova, G.L. Prjano-aromaticheskoe
syr\textquotesingle\textquotesingle e dlja sozdanija pozitivnoj
bezalkogol\textquotesingle\textquotesingle noj produkcii/G.L. Filonova,
I.L. Kovaleva, N.A. Komrakova, E.V. Nikiforova/ / Pivo i napitki. -- №5.
- 2015. - S. 58-61 {[}in Russian{]}

7. Skripnikov YU. G. Proizvodstvo plodovo-yagodnyh vin i sokov. -- M.:
Kolos, 1983. -- 256 s. {[}in Russian{]}

8. Gulcu-Ustundag, O. Saponins: properties, applications and processing.
/ O. Guclu-Ustundag, O. Mazza// Crit. Rev. Food Sci. Nutr. - 2007. -
Vol. 47. - P. 231-258 DOI: 10.1080/10408390600698197

9. Man S. Chemical study and medical application of saponins as
anti-cancer agents / S. Man, W. Gao, Y. Zhang // Fitoterapia. - 2010. -
Vol. 81. - P. 703-714 DOI: 10.1016/j.fitote.2010.06.004

10. SHlykova, A. P. Primenenie ekstrakta citronelly v tekhnologii
bezalkogol\textquotesingle nyh napitkov / A. P. SHlykova, E. O. Ivanova,
A. A. Kolobaeva, O. A. Kotik // Sovremennye naukoemkie tekhnologii. -- №
5. - 2014. - S. 192-196. {[}in Russian{]}

11. Sarafanova L. A. Primenenie pishchevyh dobavok v proizvodstve
napitkov / L.A. Sarafanova. - SPb.: Professiya, 2007. - 239 s. ISBN
5-93913-125-5 {[}in Russian{]}

12. Gazirovannye bezalkogol\textquotesingle nye napitki : receptury i
proizvodstvo / pod red. Djevida P. Stina i Filipa R. Jeshhersta ; per. s
angl. T. O. Zverevich. - Sankt-Peterburg : Professija, 2008. - 415 s.
ISBN 978-5-93913-160-5 (V per.) {[}in Russian{]}

13. Palagina, M. V. Resursy pishchevogo syr\textquotesingle ya
Dal\textquotesingle nevostochnogo regiona: ucheb. posobie / M. V.
Palagina, YA. V. Dubnyak, V. I. Golov. - Vladivostok: Izdat. dom
Dal\textquotesingle nevost. feder. un-ta, 2012. - 153 s. ISBN
978-5-7444-2728-3 {[}in Russian{]}

14. Gavrilova N.B., Petrova E.I.. Tekhnologiya produktov dlya
sportivnogo pitaniya // Molochnaya promyshlennost\textquotesingle. -
2013. - № 9. -S. 82-83 {[}in Russian{]}

15. Gavrilova N.B., SHCHetinin M.P., Moliboga E.A. Sovremennoe
sostoyanie i perspektivy razvitiya proizvodstva specializirovannyh
produktov dlya pitaniya sportsmenov // Voprosy pitaniya. 2017. - № 2. -
S. 100-106 {[}in Russian{]}

16. Atkins R. Biodobavki: prirodnaja al\textquotesingle ternativa
lekarstvam / per. s angl. G.I. Levitana. Minsk : Popurri, 2012. -800 s.
ISBN: 978-985-15-1243-6 {[}in Russian{]}

17. SHerman S.V., Kachak V.V., SHerman B.K. Nauchnye osnovy
formirovaniya sostava i potrebitel\textquotesingle skih harakteristik
gejnerov kak produktov intensivnogo sportivnogo pitaniya // Pishchevaya
promyshlennost\textquotesingle. 2012. - № 6. - S. 55-58. ISSN 0235-2486
{[}in Russian{]}

18. RodionovaN.S., Manukovskaya M.V., Nebol\textquotesingle sin A.E.,
Serchenya M.V. Primenenie metoda ul\textquotesingle trazvukovogo
jekstragirovanija v prigotovlenii napitka napravlennogo dejstvija iz
jagod chjornoj smorodiny// vestnik Voronezhskogo gosudarstvennogo
universiteta inzhenernyh tehnologij. -2016. -№(2). --S. 162-169№
DOI:10.20914/2310-1202-2016-2-162-169 {[}in Russian{]}

19. Aret V.A. Fiziko-himicheskie svojstva syr\textquotesingle ya i
gotovoj produkcii: uchebnoe posobie / V. A. Aret, B. L. Nikolaev, L. K.
Nikolaev. -- SPb. 2009. - 442 s. ISBN 978-5-98879-066-2 {[}in Russian{]}

20. GOST 6687.5-86. Produkciya bezalkogol\textquotesingle noj
promyshlennosti. Metody opredeleniya organolepticheskih pokazatelej i
ob"ema proizvodstva. --M, 1986. {[}in Russian{]}.

\emph{{\bfseries Information about the authors}}

Khamitova B.{\bfseries M}. -Candidate of Technical Sciences, Associate
Professor, M.Auezov South-Kazakhstan University, Shymkent, Kazakhstan,
e-mail: barno-007@mail.ru;

Dikhanbayeva F.- Doctor of Technology Professor Almaty Technological
University, Almaty, Kazakhstan, e-mail: fatima6363@mail.ru;

Koshtayeva G.E.- teacher M.Auezov South-Kazakhstan University, Shymkent,
Kazakhstan, e-mail: gulk1979@mail.ru

\emph{{\bfseries Сведения об авторах}}

Хамитова Б.М. - кандидат технических наук, ассоциированный профессор,
Южно-Казахстанский университет им. М.Ауэзова, Шымкент, Казахстан,
e-mail: barno-007@mail.ru;

Диханбаева Ф.Т. - доктор технических наук, профессор, Алматинский
технологический университет, Алматы, Казахстан, e-mail:
fatima6363@mail.ru;

Коштаева Г.Е. {\bfseries -} преподаватель Южно-Казахстанский университет
им. М.Ауэзова, Шымкент, Казахстан, e-mail: gulk1979@mail.ru