
\newpage
{\bfseries МРНТИ 06.61.33}

{\bfseries THE CONTRIBUTION OF INNOVATIONS TO THE ECONOMIC DEVELOPMENT OF
THE REGIONS OF KAZAKHSTAN}

{\bfseries \textsuperscript{1}А.B. Mottaeva, \textsuperscript{2}Ye.A.
Gordeyeva\textsuperscript{🖂}, \textsuperscript{3}D.A.Sitenko,
\textsuperscript{3}A.} {\bfseries Sabyrzhan, \textsuperscript{3}D.M.}
{\bfseries Temirbayeva}

\textsuperscript{1}Financial university under the government of the
Russian Federation, Moscow, Russian Federation,

\textsuperscript{2}K. Kulazhanov Kazakh University of Technology and
Business, Astana, Kazakhstan,

\textsuperscript{3}Karaganda Buketov University, Karaganda, Kazakhstan

\textsuperscript{🖂}Correspondent-author: gordelena78@mail.ru

The development of an innovative economy is one of the key directions of
Kazakhstan\textquotesingle s strategic development. In the context of
global competition and rapid technological progress, the ability of
regions to introduce and adapt innovations is becoming a decisive factor
in economic growth and increasing the country\textquotesingle s
competitiveness on the world stage. Kazakhstan, with its significant
natural resource potential, is striving to move to a new model of
economic development based on knowledge, innovation and technology.

The article examines the impact of innovations on the economic
development of the regions of Kazakhstan. The study covers an analysis
of the current state of innovation infrastructure, including science and
technology parks, incubators and accelerators, as well as government
support programs aimed at stimulating innovation activity in the
regions. The paper examines the main problems and barriers to
innovation, as well as analyzes the correlation between innovation and
key economic indicators such as gross regional product, unemployment and
investment attraction.

Special attention is paid to international experience, which is being
considered in order to develop recommendations for improving regional
innovation policy in Kazakhstan. The study is based on statistical data
and includes methods of correlation analysis to identify the
relationship between innovation and economic development. As a result of
the analysis, recommendations have been formulated to improve regional
policies aimed at stimulating innovation, which, in turn, can help
accelerate economic growth and increase the competitiveness of
Kazakhstan\textquotesingle s regions at the global level.

{\bfseries Key words:} innovation; development; management; management
mechanisms; innovation potential; forecast; level; region; efficiency;
strategy; modernization; priorities; technological processes;
prerequisites.

{\bfseries ҚАЗАҚСТАН ӨҢІРЛЕРІНІҢ ЭКОНОМИКАЛЫҚ ДАМУЫНА ИННОВАЦИЯЛАРДЫҢ
ҚОСҚАН ҮЛЕСІ}

{\bfseries \textsuperscript{1}А.Б.Моттаева,
\textsuperscript{2}Е.А.Гордеева\textsuperscript{🖂},
\textsuperscript{3}Д.А. Ситенко, \textsuperscript{3}A.Сабыржан,
\textsuperscript{3}Д.М.} {\bfseries Темирбаева}

\textsuperscript{1}Ресей Федерациясының Үкіметі жанындағы Қаржы
университеті, Мәскеу, Ресей Федерациясы,

\textsuperscript{2} Қ.Құлажанов атындағы Қазақ технология және бизнес
Университеті, Астана, Қазақстан,

\textsuperscript{3} Е.А.Бөкетов атындағы Қарағанды университеті,
Қарағанды, Қазақстан,

e-mail: gordelena78@mail.ru

Инновациялық экономиканы дамыту Қазақстанның стратегиялық дамуының
негізгі бағыттарының бірі болып табылады. Жаһандық бәсекелестік пен
қарқынды технологиялық прогресс жағдайында өңірлердің инновацияларды
енгізу және бейімдеу қабілеті елдің әлемдік аренадағы экономикалық өсуі
мен бәсекеге қабілеттілігін арттырудың шешуші факторына айналуда.
Қазақстан айтарлықтай табиғи ресурстық әлеуетке ие бола отырып, білімге,
инновациялар мен технологияларға сүйенетін экономикалық дамудың жаңа
моделіне көшуге ұмтылады.

Мақалада инновациялардың Қазақстан өңірлерінің экономикалық дамуына
әсері қарастырылады. Зерттеу инновациялық инфрақұрылымның, оның ішінде
ғылыми және технологиялық парктердің, инкубаторлар мен үдеткіштердің
ағымдағы жай-күйін, сондай-ақ өңірлердегі инновациялық белсенділікті
ынталандыруға бағытталған мемлекеттік қолдау бағдарламаларын талдауды
қамтиды. Жұмыста инновацияларды енгізудің негізгі мәселелері мен
кедергілері қарастырылады, сондай-ақ инновациялық қызмет пен жалпы
өңірлік өнім, жұмыссыздық деңгейі және инвестициялар тарту сияқты
негізгі экономикалық көрсеткіштер арасындағы корреляциялық тәуелділіктер
талданады.

Қазақстанда өңірлік инновациялық саясатты жақсарту бойынша ұсынымдар
әзірлеу мақсатында қаралатын халықаралық тәжірибеге ерекше назар
аударылды. Зерттеу статистикалық мәліметтерге негізделген және инновация
мен экономикалық даму арасындағы байланысты анықтау үшін корреляциялық
талдау әдістерін қамтиды. Жүргізілген талдау нәтижесінде инновацияларды
ынталандыруға бағытталған өңірлік саясатты жақсарту үшін ұсынымдар
тұжырымдалды, бұл өз кезегінде экономикалық өсуді жеделдетуге және
жаһандық деңгейде Қазақстан өңірлерінің бәсекеге қабілеттілігін
арттыруға ықпал етуі мүмкін.

{\bfseries Түйін сөздер:} инновация; даму; басқару; басқару тетіктері;
инновациялық әлеует; болжам; деңгей; аймақ; тиімділік; стратегия;
жаңғырту; басымдықтар; технологиялық процестер; алғышарттар.

{\bfseries ВКЛАД ИННОВАЦИЙ В ЭКОНОМИЧЕСКОЕ РАЗВИТИЕ РЕГИОНОВ КАЗАХСТАНА}

{\bfseries \textsuperscript{1}А.Б. Моттаева, \textsuperscript{2}Е.А.
Гордеева\textsuperscript{🖂}, \textsuperscript{3}Д.А. Ситенко,
\textsuperscript{3}A.Сабыржан, \textsuperscript{3}Д.М.}
{\bfseries Темирбаева}

\textsuperscript{1} Финансовый университет при Правительстве Российской
Федерации, Москва, Российская Федерация,

\textsuperscript{2} Казахский университет технологий и бизнеса имени
К.Кулажанова, Астана, Казахстан,

\textsuperscript{3} Карагандинский университет имени академика
Е.А.Букетова, Караганда, Казахстан,

e-mail gordelena78@mail.ru

Развитие инновационной экономики является одним из ключевых направлений
стратегического развития Казахстана. В условиях глобальной конкуренции и
стремительного технологического прогресса, способность регионов внедрять
и адаптировать инновации становится решающим фактором экономического
роста и повышения конкурентоспособности страны на мировой арене.
Казахстан, обладая значительным природным ресурсным потенциалом,
стремится перейти к новой модели экономического развития, которая
опирается на знания, инновации и технологии.

В статье рассматривается влияние инноваций на экономическое развитие
регионов Казахстана. Исследование охватывает анализ текущего состояния
инновационной инфраструктуры, в том числе научных и технологических
парков, инкубаторов и акселераторов, а также программ государственной
поддержки, направленных на стимулирование инновационной активности в
регионах. В работе рассматриваются основные проблемы и барьеры внедрения
инноваций, а также анализируются корреляционные зависимости между
инновационной деятельностью и ключевыми экономическими показателями,
такими как валовой региональный продукт, уровень безработицы и
привлечение инвестиций.

Особое внимание уделено международному опыту, который рассматривается с
целью выработки рекомендаций по улучшению региональной инновационной
политики в Казахстане. Исследование основано на статистических данных и
включает в себя методы корреляционного анализа для выявления взаимосвязи
между инновациями и экономическим развитием. В результате проведенного
анализа сформулированы рекомендации для улучшения региональной политики,
направленной на стимулирование инноваций, что, в свою очередь, может
способствовать ускорению экономического роста и повышению
конкурентоспособности регионов Казахстана на глобальном уровне.

{\bfseries Ключевые слова:} инновации; развитие; управление; механизмы
управления; инновационный потенциал; прогноз; уровень; регион;
эффективность; стратегия; модернизация; приоритеты; технологические
процессы; предпосылки.

{\bfseries Introduction.} With the goals of diversifying the economy and
lowering reliance on raw commodities, the subject of
innovations\textquotesingle{} role in the economic growth of
Kazakhstan\textquotesingle s regions is pertinent. The adoption of
innovations at the regional level helps to draw in investments, boost
labor productivity, grow small and medium-sized en

terprises, and create new jobs. With the rise of digitization,
innovation\textquotesingle s importance is only growing.

By examining how innovation contributes to the economic growth of
Kazakhstan\textquotesingle s regions, it is possible to evaluate the
success of the state\textquotesingle s current innovation support
initiatives, pinpoint the best regional strategies, and create
suggestions for future innovative activity stimulation. Furthermore, the
topic is pertinent to international cooperation since
Kazakhstan\textquotesingle s entry into global innovation networks has
the potential to quicken the country\textquotesingle s economic
modernization process and guarantee long-term sustainable growth.

The aim of this research is to examine the role that innovation plays in
the economic growth of Kazakhstan\textquotesingle s regions and to
pinpoint the main drivers of the rise in regional innovation activity.

A review of previous research demonstrates that
innovation\textquotesingle s role in the economic growth of
Kazakhstan\textquotesingle s regions is acknowledged as a critical
component of sustainable growth. However, despite significant efforts to
develop innovation infrastructure and support from the state, there are
serious barriers that impede the effective implementation of innovations
in the regions. These include inadequate resources, a dearth of skilled
workers, a poor degree of collaboration between industry and academia,
and a poor integration into international innovation processes. The
article examines the theoretical aspects of innovations, their impact on
the economy, and analyzes the current state of innovation infrastructure
in Kazakhstan. The operation of technology parks, business incubators,
and other components of the innovation ecosystem are studied in
particular, as is the degree of state support at the regional level. The
collected statistical data on the gross regional product, unemployment
rate, investment attraction and number of patents are used to conduct an
empirical analysis of the relationship between innovation and regional
economic performance. The study also includes the identification of
problems and barriers to innovation through surveys and interviews with
representatives of business, academia and government. As a result,
recommendations are formulated to stimulate innovative activity and
infrastructure development, as well as consider the possibilities of
expanding international cooperation to increase the competitiveness of
the regions of Kazakhstan.

{\bfseries Materials and methods.} The process of introducing new or
significantly enhanced goods, services, technology, production and
management organizational strategies, and other elements that lead to a
notable improvement in the overall performance of economic systems and
organizations is known as innovation. They are essential to the process
of economic expansion because they boost competitiveness, open up new
markets, and enhance people\textquotesingle s quality of life.

Innovations can be classified according to various criteria, the most
common of which are classification by the object of innovation, by the
level of novelty and by the degree of impact on the market.

These classifications help to better understand the nature of innovation
and its impact on economic development, allowing for more effective
strategies and approaches for its implementation and development at the
regional and national levels.

The emergence of new markets and employment opportunities is one of the
key ways that innovation influences economic development. Both the
creation of new economic sectors and the growth of already-existing
market niches are facilitated by innovation processes. This opens up new
business opportunities and creates additional demand for labor, which in
turn helps to reduce unemployment and improve social conditions. As an
illustration, the proliferation of digital platforms and information
technology has given rise to sectors like cybersecurity and e-commerce,
which actively support economic expansion and employment creation.

Furthermore, innovation helps to increase competitiveness both
domestically and globally. Companies that implement innovative solutions
can offer unique products and services that stand out from the
competition. This allows them to capture new markets and strengthen
their position in existing ones. Competitiveness, in turn, stimulates
further investment and development, creating a vicious circle of
positive impact of innovation on economic development.

In addition, innovation has an impact on the sustainability of the
economy. In the context of global economic change and instability,
innovative solutions can provide a more flexible and adaptive approach
to resource and process management, which helps to cope with external
and internal challenges. This contributes to long-term sustainable
development and reduces the vulnerability of economic systems.

Finland is an outstanding example of how investing in education programs
can help create skills and stimulate innovative development. The high
caliber of training for experts in the fields of science and technology
is the main goal of the Finnish educational system. The "Innovations for
Growth" program implemented in Finland supports startups and innovative
companies through grants, tax incentives and other forms of state
support. These steps boost competitiveness in the global market and aid
in the development of new technologies.

Finland has made large investments in STEM (science, technology,
engineering, and mathematics) education as well as scientific research
and technology development. As a result, Finland has been able to create
a strong innovation ecosystem, which has led to a significant increase
in the number of successful startups and the development of high-tech
sectors of the economy. One instance is the triumphant growth of Nokia,
which has emerged as a global frontrunner in the domains of
telecommunications and mobile technologies. This success was due to the
availability of qualified specialists and a strong scientific base
created thanks to an effective educational system {[}1{]}.

South Korea also demonstrates a successful approach to innovative
development, focusing on the integration of science and business. The
country is actively investing in scientific research and the creation of
technology parks, such as the technology park in Seoul, which has become
a hub for science and technology startups. In South Korea, considerable
attention is paid to the training of highly qualified personnel through
the reform of the educational system and the development of scientific
research {[}2{]}. These efforts have contributed to the creation of a
strong innovation infrastructure, which has played a key role in making
South Korea one of the world\textquotesingle s leaders in technology and
innovation. Advances in the development of semiconductors and
information technology have brought significant economic dividends and
strengthened the country\textquotesingle s position in international
markets.

Science and technology parks, such as Astana Technopark and Almaty
Technopark, play a key role in supporting innovative startups and
technology projects. These business parks give entrepreneurs access to
the facilities they need, such as offices, labs, and consulting
services. Nevertheless, as of 2023, just 1.2\% of all small and
medium-sized businesses in Kazakhstan were registered in these parks,
according to a report from the Statistics Agency of the Republic of
Kazakhstan. {[}3{]}. This indicates that, despite the existence of
appropriate structures, the level of their use remains limited.

Research and development (R\&D) spending is a key metric for assessing
innovative activities. In 2023, overall R\&D spending in Kazakhstan was
approximately 0.3\% of GDP, a far smaller proportion than the average
for developing nations, where this number is closer to 1\% of GDP.
{[}4{]}. Insufficient investment in research and development restricts
the potential for developing and promoting novel technologies, hence
impeding the advancement of innovation.

Educational programs play a key role in the formation of qualified
personnel necessary to support innovative activities. New curricula
aimed at training specialists in the field of STEM disciplines are being
introduced in Kazakhstan. However, according to the Ministry of Science
and Higher Education of the Republic of Kazakhstan, there is a lack of
interaction between educational institutions and industry. This limits
the opportunities for students to apply knowledge in practice and
develop innovative projects {[}5{]}.

Although Kazakhstan is modernizing its educational system, prosperous
nations like Singapore demonstrate how tight collaboration between
academic institutions and business enterprises promotes more efficient
expert training and the growth of new industries. Innovative startups in
Singapore, such as Start-up SG, provide students and young entrepreneurs
with access to funding and mentorship, enabling their businesses to
grow. {[}6{]}.

Science and technology parks give researchers and entrepreneurs access
to the tools they need to create and market new innovations, which helps
to shape the innovation ecosystem. The creation of these parks is seen
as one of Kazakhstan\textquotesingle s top strategic goals for
quickening the country\textquotesingle s economic growth and shifting it
toward a knowledge-based economy. However, in order to achieve
significant results in this area, additional efforts are needed to
overcome existing barriers and use the best world practices.

Several science and technology parks are currently operating in
Kazakhstan, such as the Astana Technopark, Almaty Technopark and the
Saryarka Innovation Center. These parks were created with the aim of
supporting start-ups and small innovative enterprises by providing them
with access to research infrastructure, office space and advisory
services. However, according to the Ministry of Science and Higher
Education of the Republic of Kazakhstan, in 2023, only 15\% of
registered companies in these parks have reached the stage of
commercialization of their products {[}7{]}. This indicator indicates
that, despite the availability of infrastructure, the level of
efficiency of these parks remains low.

The absence of private funding for advancements in science and
technology is one of the main obstacles. As per the World Bank research,
Kazakhstan\textquotesingle s private sector\textquotesingle s share of
R\&D financing in 2023 was less than 20\%, whereas in OECD nations, it
was over 60\%. {[}8{]}. The low level of private financing limits the
opportunities for startups and innovative companies, which slows down
their development and the introduction of new technologies to the
market.

Kazakhstan can learn from the experiences of other nations that have
made substantial progress in this field with the construction of science
and technology parks. One such example is Israel, where the creation of
technology parks and incubators has become a key element in the
country\textquotesingle s emergence as a global leader in the field of
high technology. In Israel, technology parks such as Tel Aviv Technopark
have fostered the development of more than 4,000 startups, attracting
significant investment from the private sector and venture capital
funds. In 2022, the total volume of venture capital investments in
Israeli startups exceeded \$10 billion, which is more than 4\% of the
country\textquotesingle s GDP {[}9{]}.

The advantage of the Israeli model is the high level of coordination
between the state, universities and the private sector. The state
actively supports innovative companies through grants, tax incentives
and support programs, while private investors and venture funds provide
significant funding at the stage of growth and scaling of projects. This
approach ensures the sustainable development of the innovation ecosystem
and the rapid growth of high-tech sectors of the economy.

Kazakhstan can benefit from the Israeli model by enhancing collaboration
between public and private sectors, as well as by establishing more
conducive environments to draw private investment in advances in science
and technology. A higher volume of venture capital investments and the
establishment of favorable conditions for the more successful
commercialization of scientific discoveries will greatly improve the
effectiveness of the nation\textquotesingle s current technology parks
and spur its creative economy.

One of the main things that helps innovation activities grow around the
world is state backing. In Kazakhstan, the state is actively
implementing various programs aimed at stimulating scientific research,
technological development and commercialization of innovations. An
examination of current programs reveals both their strengths and areas
for development.

One of the most significant programs to support innovation in Kazakhstan
is the commercial Road Map 2025 program, which provides funding for
small and medium-sized commercial endeavors, including creative firms.
Under this program, businesses may be eligible to get loan guarantees,
subsidies to cut loan interest rates, and other forms of financial help.
40\% of the more than 2,000 innovative projects that got support between
2020 and 2023 were successful in breaking into the market, according to
the Ministry of National Economy of the Republic of Kazakhstan.
{[}10{]}.

However, the effectiveness of these programs is limited by several
factors. First, there is a lack of coordination between different
government agencies, which makes it difficult for entrepreneurs to
access the necessary information and resources. Secondly, the process of
obtaining state support remains complex and requires significant
administrative costs on the part of enterprises {[}11{]}. These barriers
reduce the attractiveness of programs to potential participants and
limit the scope of their implementation.

Kazakhstan can learn from the experiences of other nations that have
successfully implemented comparable procedures in order to increase the
efficacy of its state support programs. A prime example is Finland,
where Business Finland plays a central role in supporting innovation.
Finland actively supports start-ups and small businesses through grant
programs that cover all stages of the innovation process, from
scientific research to entering international markets. In 2022, Business
Finland invested more than 600 million euros in the development of
innovative projects, which led to the creation of more than 1,500 new
jobs and an 8\% increase in exports of high-tech products {[}12{]}.

The advantage of the Finnish model lies in its high degree of
integration with the national innovation strategy, as well as in close
cooperation with the private sector and international partners.
Government programs are complemented by private investment and venture
financing, which allows for sustainable growth in the innovation sector.
For Kazakhstan, this experience can be useful in the context of
improving coordination between various public and private structures, as
well as for creating more transparent and accessible mechanisms of state
support.

Several methodological approaches were used to assess the impact of
innovation on the economic development of the regions of Kazakhstan and
to analyze the effectiveness of state support for innovative activities.
Each of the methods allows you to gain a comprehensive understanding of
the current situation, identify key problems and offer recommendations
based on the analysis of both national and international data.

Economic and statistical analysis was used to assess the current state
of innovation infrastructure and the level of innovation implementation
in various regions of Kazakhstan. Data from official sources, including
the Ministry of National Economy, the National Bank of the Republic of
Kazakhstan, and the Agency on Statistics of the Republic of Kazakhstan,
were gathered and examined for this study. Specifically, metrics
pertaining to the amount of funds allocated to research and development
(R\&D), the proportion of inventive businesses, and GDP growth rates in
areas that are actively implementing innovations were taken into account
{[}13{]}. The information acquired indicates that areas with high levels
of innovative activity also exhibit greater rates of economic growth,
confirming the beneficial effects of innovation on regional development.

The experience of other nations with regard to state funding for
innovation initiatives and setting up frameworks for the growth of
science and technology parks was examined through comparative research.
Finland was chosen as an example, which demonstrates high indicators of
innovation activity and the effectiveness of state support. In 2022,
Finland ranked first in the Global Innovation Index, with a score of
64.6, well above the world average {[}14{]}. The study analyzed data
from Business Finland, which showed that over the past five years, the
number of innovative startups in Finland has increased by 25\%, which
contributes to sustainable economic growth and the development of
high-tech sectors of the economy {[}15{]}.

The economic development of the regions is largely determined by the
level of their innovative activity. Important indicators characterizing
the economic state of the regions are the gross regional product, the
unemployment rate, as well as the volume of attracted investments. It is
feasible to evaluate the effect of innovation on economic development
and pinpoint important patterns that influence a
region\textquotesingle s success by analyzing these metrics.

Gross regional product is an important indicator of the economic state
of the region, reflecting the total volume of goods and services
produced. Significant regional variations in GRP exist in Kazakhstan,
and these variations are linked to varying levels of industry growth,
agro-industrial complex development, and innovation activity. Due to
their high levels of economic activity and concentration of creative
businesses, Almaty and Astana recorded the largest GRP in 2023---13.2
trillion and 8.7 trillion tenge, respectively-according to the Agency
for Strategic Planning and Reforms of the Republic of Kazakhstan.
{[}16{]}. At the same time, in several rural regions, such as North
Kazakhstan and Zhambyl regions, GRP remains below the national average,
which is associated with a low level of innovation and insufficient
development of scientific and technical infrastructure.

The unemployment rate is another important indicator that reflects the
economic state of the region and its ability to adapt to the challenges
of the labor market. Regions with high innovation activity have lower
unemployment rates, which is associated with the creation of new jobs in
high-tech sectors of the economy. The unemployment rates in Almaty and
Astana in 2023 were 4.2\% and 4.5\%, respectively, according to the
Ministry of Labor and Social Protection of the Population of the
Republic of Kazakhstan. These figures are considerably lower than the
5.8\% national average {[}17{]}. At the same time, in regions with low
innovation activity, such as Mangistau and Turkestan regions, the
unemployment rate remains high, reaching 7\% and above.

Attracting investment, especially in research and development (R\&D), is
a key factor in determining the success of regions in the field of
innovation. In Kazakhstan, there is a significant gap in investment
between regions. According to the Ministry of National Economy of the
Republic of Kazakhstan, in 2023, the largest volume of investments in
R\&D was attracted in the cities of Almaty and Astana, where it amounted
to 65 billion and 48 billion tenge, respectively {[}18{]}. Science and
technology parks are actively operating in these regions, which helps to
attract investors and create favorable conditions for the development of
innovative business. At the same time, in a number of rural regions,
such as Kyzylorda and Aktobe regions, the volume of investment in
innovative projects remains extremely low, which limits the
opportunities for economic growth and diversification of the economy.

The experience of other countries shows that effective innovation policy
and attracting investment can significantly accelerate the economic
development of regions. For example, in South Korea, which is actively
developing regional innovation clusters, it was possible to
significantly increase GRP and reduce unemployment in provinces where
advanced technologies are being introduced. According to the Korea Trade
and Investment Promotion Agency, in 2022, the volume of investments in
innovation clusters exceeded \$5 billion, which led to a 6\% increase in
GRP and a 1.5\% decrease in the unemployment rate in the most active
regions {[}19{]}. This experience highlights the importance of investing
in innovation to ensure sustainable economic growth and improve the
quality of life of the population.

An analysis of statistical data for Kazakhstan shows significant
regional differences in the level of economic development, which is
largely due to the degree of innovation and the volume of attracted
investments. Regions with high innovation activity demonstrate higher
GRP, low unemployment and significant investment volumes, which confirms
the positive impact of innovation on economic development. At the same
time, the experience of other countries, such as South Korea, shows that
comprehensive measures to support innovation can significantly
accelerate the development of regions and increase their competitiveness
in the global market.

The relationship between the level of innovation activity and the main
economic indicators of the regions is an important aspect for
understanding the mechanisms of economic growth and development. In this
study, a correlation analysis was carried out aimed at identifying the
degree of dependence between innovations and such economic indicators as
gross regional product (GRP), unemployment rate and the volume of
attracted investments in the regions of Kazakhstan.

To conduct a correlation analysis, data for 14 regions of Kazakhstan for
the period from 2019 to 2023 were used, including GRP indicators,
unemployment rate, volume of investment in research and development
(R\&D), as well as the number of registered patents and innovative
enterprises. The data was collected from official sources, such as the
Agency for Strategic Planning and Reforms of the Republic of Kazakhstan
and the Ministry of National Economy of the Republic of Kazakhstan
{[}16-17{]}.

The results of the correlation analysis showed that there is a
significant positive correlation between the level of innovation
activity and the gross regional product (correlation coefficient r =
0.68). This indicates that regions with higher innovation activity,
expressed in the number of patents registered and the volume of
investment in R\&D, show higher GRP growth rates. For example, the
cities of Almaty and Astana, where the concentration of innovative
enterprises and investments in R\&D is much higher, show an annual
increase in GRP at the level of 4.5\% and 3.8\%, respectively {[}16{]}.

Correlation analysis also revealed a negative relationship between the
level of innovation activity and the unemployment rate (r = -0.54). In
regions with a high level of innovation, there is a decrease in
unemployment, which is associated with the creation of new jobs in
high-tech sectors of the economy. In particular, in Almaty, where the
largest number of innovative enterprises are registered, the
unemployment rate decreased to 4.2\% in 2023, which is significantly
lower than the national average {[}20{]}.

In terms of the volume of attracted investments, a positive correlation
(r = 0.62) was found between investments in R\&D and GRP. This indicates
that regions that attract more investment in research and development
are showing higher rates of economic growth. For example, in the Atyrau
region, where the volume of investment in R\&D increased by 15\% in
2023, GRP increased by 3.6\%, which emphasizes the importance of
attracting investment in innovative projects {[}20{]}.

International experience confirms the existence of a strong positive
correlation between innovation and economic performance. For example, in
South Korea, one of the leading countries in terms of innovation
activity, a high correlation was found between the volume of investment
in R\&D and economic growth at the level of r = 0.75 {[}21{]}. Between
2015 and 2022, R\&D investment in South Korea increased by 20\%,
resulting in a 7\% GDP growth and a 2\% decrease in the unemployment
rate {[}21{]}. This experience highlights the importance of public and
private financing of innovation activities for sustainable economic
growth.

The process of introducing innovations into the economy of Kazakhstan
faces a number of significant problems and barriers that slow down the
pace of innovative development and limit the potential for economic
growth of the regions. This section examines the key obstacles
identified in the course of the study and analyzes foreign experience
that can be useful in overcoming these barriers.

One of the main barriers to the introduction of innovations in
Kazakhstan is the insufficient development of the scientific and
technical infrastructureIn most regions of the country, there is a lack
of the necessary number of technology parks, incubators and research
centers, which complicates the implementation of innovative projects and
limits access to modern technologies. For example, in 2023, there were
only 13 technology parks in Kazakhstan, of which 8 are concentrated in
Almaty and Astana, while access to such resources is extremely limited
in rural areas {[}22{]}.

A comparison with international experience, in particular with Finland,
shows a significant lag in the development of infrastructure. In
Finland, a country with a high level of innovative activity, there are
more than 50 technology parks and research centers that ensure close
interaction between science and business, which contributes to the
accelerated introduction of innovations {[}23{]}. This experience
underscores the need to expand the network of technology parks and other
innovative facilities throughout Kazakhstan.

Financial constraints remain one of the main barriers to innovation in
Kazakhstan. Despite the Government\textquotesingle s efforts to support
research and development, funding remains inadequate. In 2023, R\&D
expenditures amounted to only 0.13\% of GDP, which is significantly
lower than the average level for the countries of the Organization for
Economic Cooperation and Development (OECD), where this figure averages
2.4\% {[}24{]}. This situation limits the possibilities for research and
development of new technologies, which negatively affects the
country\textquotesingle s innovative potential.

In addition, private sector activity in financing innovation remains
low. Most businesses in Kazakhstan prefer to invest in traditional
low-risk activities, which limits opportunities for innovation. In 2023,
the share of the private sector in total R\&D investment was only 18\%,
while in leading innovative economies such as Germany, this figure
exceeds 60\% {[}25{]}. To overcome this problem, it is necessary to
stimulate the participation of the private sector in financing
innovative projects through tax incentives, subsidies and public-private
partnerships.

Another significant barrier is weak cooperation between scientific
institutions and business. In Kazakhstan, there is a low level of
interaction between universities and enterprises, which makes it
difficult to commercialize scientific developments and introduce
innovations into industry. According to the Ministry of Education and
Science of the Republic of Kazakhstan, in 2023, only 12\% of
state-funded scientific projects were implemented in cooperation with
business, which indicates insufficient integration of the scientific and
business environment {[}26{]}.

International experience, for example, in the United States,
demonstrates the importance of close cooperation between science and
business. In the United States, there are a number of programs aimed at
stimulating interaction between universities and enterprises {[}27{]},
which leads to a high level of commercialization of scientific
developments. As a result, more than 70\% of innovative startups in the
United States arise in university incubators and technology parks, which
contributes to the rapid introduction of innovations into the economy
{[}28{]}.

Another significant impediment to innovation is staffing limitations.
The possibilities for the creation and use of new technologies are
restricted by the shortage of skilled experts in the fields of science
and technology. In 2023, specialists working in R\&D made up only 0.8\%
of Kazakhstan\textquotesingle s entire workforce, a much smaller
percentage than in OECD nations, where the percentage is above 3\%
{[}29{]}.

To overcome the personnel shortage, Kazakhstan should pay attention to
the experience of South Korea, where the successful development of
educational programs in the field of science and technology has become a
key factor in increasing innovative activity. Specifically, in South
Korea, advanced training courses and special education programs have
been implemented, leading to a notable increase in the number of
competent workers and a faster pace of innovation introduction.
{[}30{]}.

{\bfseries Results and discussions.} The results of the study confirmed the
key role of innovation activity in the economic development of the
regions of Kazakhstan, identifying both significant successes and
significant barriers that the country faces on the way to building an
innovative economy.

The study showed that regions with a high concentration of innovative
projects demonstrate higher rates of economic growth and sustainability.
For example, the GRP of Almaty, one of the leaders in innovation
activity, grew by 4.5\% in 2023, which is significantly higher than the
average for Kazakhstan, which is 3.3\% {[}16{]}. Such results confirm
the effectiveness of investment in innovation as a tool for stimulating
regional development.

Foreign experience also confirms this pattern. In Finland, where state
support for innovation is central to economic policy, GDP growth
consistently exceeds 2\% annually, which is associated with active
investment in R\&D and close cooperation between science and business
{[}14{]}. The application of this experience in Kazakhstan can
contribute to a more even distribution of economic benefits between
regions and reduce interregional disparities.

The study identified significant barriers that limit the potential for
innovative development in Kazakhstan. Lack of financing remains a major
challenge: despite the growth of investment, it remains low compared to
international standards. In 2023, R\&D expenditures amounted to only
0.13\% of GDP, which is significantly lower than the level of OECD
countries {[}24{]}. This limits the opportunities for the development of
new technologies and the introduction of innovations in industry.

In addition, the underdevelopment of the scientific and technical
infrastructure and weak cooperation between scientific institutions and
business hinder the effective implementation of innovative projects. In
Kazakhstan, only 12\% of state-funded scientific projects were
implemented in cooperation with business {[}26{]}. At the same time, in
the United States, the share of commercialized scientific developments
is much higher, which indicates the importance of interaction between
science and business for the successful implementation of innovations
{[}28{]}.

An analysis of international experience has shown that the successful
development of an innovative economy requires an integrated approach,
including state support, infrastructure development and incentives for
the private sector. In particular, the experience of South Korea, where
active government intervention in support of innovation has led to
significant economic growth, can be useful for Kazakhstan. South Korea
pays special attention to the training of qualified personnel and the
development of science parks, which has allowed the country to become
one of the world leaders in terms of innovation activity {[}30{]}.

For the successful implementation of innovations in Kazakhstan, it is
necessary to revise the current policy in the field of science and
technology, taking into account successful international practices. This
includes increasing the share of R\&D funding, developing a network of
technology parks and incubators across the country, and strengthening
cooperation between science and business.

Based on the study, the following recommendations can be made for the
further development of the innovative economy in Kazakhstan:

\begin{enumerate}
\def\labelenumi{\arabic{enumi}.}
\item
  Raising the percentage of R\&D spending to at least 1\% of GDP would
  greatly expand the opportunities for new technology research and
  development.
\item
  Increasing the number of technology parks and incubators in the nation
  will be a significant step in ensuring that innovation activity is
  distributed equally across the nation, particularly in less developed
  areas.
\item
  The percentage of private investment in R\&D will expand if tax breaks
  and other financial aid are made available to businesses that invest
  in innovation.
\item
  The creation of platforms and programs for interaction between
  universities and enterprises contributes to the faster
  commercialization of scientific developments and their implementation
  in industry.
\end{enumerate}

The implementation of these measures will allow Kazakhstan to accelerate
the development of an innovative economy, reduce regional imbalances and
increase the country\textquotesingle s competitiveness in the global
market.

In this context, the following recommendations are proposed to
complement the previously presented measures and help to enhance their
impact.

Regional authorities should integrate the innovation strategy into
general plans for socio-economic development. At the moment, in many
regions of Kazakhstan, innovation initiatives are often considered as
separate projects, which limits their large-scale impact on the economy
{[}31{]}. To improve the situation, it is necessary to develop strategic
plans that include long-term goals for innovative development,
integrated with other aspects of regional development, such as
infrastructure and education.

It is important to create conditions for the development of start-ups
and small innovative enterprises that can become engines of economic
growth. Kazakhstan should develop special business incubators,
accelerators and venture capital funds that will support startups at all
stages of their development {[}32{]}. Successful examples of such
initiatives can be found in the United States, where accelerator
networks such as Y Combinator and Techstars provide startups with access
to resources and investment, contributing to their rapid growth and
successful commercialization {[}33{]}.

The creation of regional competence centers that will specialize in
advanced technologies and innovations can significantly increase the
efficiency of scientific and research projects. These centers can serve
as platforms for knowledge exchange, joint research and development of
new technologies {[}34{]}. In Switzerland, for example, such centers
actively support research in the field of biotechnology and medical
technology, which has contributed to the significant growth of these
sectors {[}35{]}. Kazakhstan should consider the possibility of creating
such centers in key regions for the development of advanced industries.

Scientific tourism, which attracts scientists and entrepreneurs from
other countries to participate in research and conferences, can be an
effective tool for raising the international status of regions. The
creation of scientific and technical events, such as conferences and
symposia, will facilitate the exchange of knowledge and the
establishment of international relations {[}36{]}. The scientific
conferences in China, which draw experts from all over the world and aid
in the creation of cutting-edge technologies and creative solutions, are
an illustration of how scientific tourism has been successfully
implemented {[}37{]}.

Administrative and bureaucratic barriers can significantly hinder
innovation activity and business development. In Kazakhstan, it is
necessary to carry out reforms aimed at simplifying the procedures for
registering enterprises, obtaining permits for scientific research and
introducing new technologies {[}38{]}. A notable example in the global
arena is New Zealand, where the introduction of electronic services for
business and the streamlining of registration procedures have drawn
substantial investment in innovation and produced a favorable business
climate {[}39{]}.

These recommendations will complement previously proposed measures and
help create a more sustainable and innovative economy in the regions of
Kazakhstan. The application of the best international practices,
combined with adaptation to local conditions, will ensure the effective
development of the regions, improve the investment climate and increase
the country\textquotesingle s competitiveness in the international
arena.

{\bfseries Conclusions.} The study\textquotesingle s findings demonstrated
that innovations significantly influence the economic growth of
Kazakhstan\textquotesingle s regions. Integrating innovation strategies
into regional planning and infrastructure development to support
innovation is a prerequisite for sustainable growth. Science and
technology parks, incubators, and accelerators can help increase
start-ups and draw private investment, which in turn increases economic
metrics like employment rates and gross regional product. These
facilities can be developed and implemented in the regions.

International experience demonstrates that effective infrastructure
initiatives and government support implemented in countries such as
Israel and the United States lead to significant improvements in
innovation activity and commercialization of scientific developments. In
particular, the creation of specialized technology parks and
accelerators helps startups gain access to resources and investments,
accelerating their growth and integration into the economy.

The data indicate the need to create regional competence centers that
will focus on advanced technologies and innovation. These centers
contribute to the concentration of knowledge and resources, which allows
for more efficient research and the development of new technologies. The
experience of Switzerland shows that the creation of such centers can
accelerate the development of key industries and increase the
international status of the regions.

The analysis shows that scientific tourism and international conferences
can be effective tools for improving the scientific status of regions
and developing high technologies. Successful examples from China confirm
that the organization of major scientific events contributes to the
exchange of knowledge and the strengthening of international relations,
which in turn stimulates innovative activity.

New Zealand\textquotesingle s experience shows how simplifying
administrative processes can help create a favourable business climate
and attract investment.

Thus, the comprehensive application of recommendations based on
international practice and adapted to the local conditions of Kazakhstan
can significantly increase the level of innovation activity and economic
growth in the regions. These measures will help to use resources more
efficiently, improve the investment climate and strengthen the
country\textquotesingle s competitiveness in the global arena.

{\bfseries References}

1. Pohjola M., \& Vanhala J. Innovations for Growth in Finland.
Helsinki: Finnish Innovation Fund. -2019.

2. Kim, J.Y. Innovative Development in South Korea: Integration of
Science and Business// Journal of Research and Technology in Asia.
-2020. -Vol.4. -P. 85--102.

3. Agentstvo po statistike Respubliki Kazakhstan. Godovoy otchet po
malym i srednim predpriyatiyam. --Almaty, 2023. {[}in Russian{]}

4. UNESCO. Science Report: The Race Against Time for Smarter
Development. UNESCO Publishing. -2021. ISBN 978-92-3-100450-6

5. Ministerstvo nauki i vysshego obrazovaniya RK. Otchet o
reformirovanii obrazovatel\textquotesingle noy sistemy. --Astana, 2023.

6. Economic Development Board of Singapore. Start-up SG Programmes.
Retrieved from. -2023. https://www.singaporeeda.gov.sg.

7. Ministerstvo nauki i vysshego obrazovaniya RK. Godovoy otchet po
nauchno-tekhnologicheskim parkam Kazakhstana. --Astana,2023.

8. World Bank. World Development Indicators. World Bank Group, 2023.
URL: https://data.worldbank.org/products/wdi

9. Israel Innovation Authority. Annual Innovation Report 2022. 2023 URL:
https://innovationisrael.org.il/en/reports

10. Ministerstvo natsional\textquotesingle noy ekonomiki RK. Godovoy
otchet po programme "Dorozhnaya karta biznesa 2025". -Astana, 2023.

11. Ministerstvo natsional\textquotesingle noy ekonomiki RK. Otchet po
otsenke effektivnosti programm gosudarstvennoy podderzhki malogo i
srednego biznesa. -Astana, 2023.

12. Business Finland. Annual Report 2022. - 2023. URL:
https://www.businessfinland.fi/en/reports

13. Agentstvo po statistike Respubliki Kazakhstan. Godovoy
statisticheskiy otchet po innovatsiyam. -Almaty, 2023.

14. Global Innovation Index. Global Innovation Index 2022 Rankings.
World Intellectual Property Organization. -2022.

15. Business Finland. Annual Report 2022. -2023. URL:
https://www.businessfinland.fi/en/reports

16. Agentstvo po strategicheskomu planirovaniyu i reformam Respubliki
Kazakhstan. Godovoy ekonomicheskiy otchet. --Almaty, 2023.

17. Ministerstvo truda i sotsial\textquotesingle noy zashchity
naseleniya RK. Otchet po sostoyaniyu rynka truda i zanyatosti. -Astana,
2023.

18. Ministerstvo natsional\textquotesingle noy ekonomiki RK.
Statisticheskiy byulleten\textquotesingle{} po investitsiyam v NIOKR.
Almaty, 2023.

19. KOTRA. Annual Report 2022. -2023.
URL://www.kotra.or.kr/foreign/reports

20. Ministerstvo natsional\textquotesingle noy ekonomiki RK.
Statisticheskiy byulleten\textquotesingle{} po investitsiyam v NIOKR.
-Almaty,2023.

21. World Bank. South Korea Economic Growth and Innovation Report 2022.
-2023. URL: https://www.worldbank.org/reports.

22. Agentstvo po strategicheskomu planirovaniyu i reformam Respubliki
Kazakhstan. Godovoy otchet po innovatsionnoy infrastrukture. -Almaty,
2023.

23. European Commission. European Innovation Scoreboard 2023. Brussels.
-2023. URL:
https://research-and-innovation.ec.europa.eu/knowledge-publications-tools-and-data/publications/all-publications/european-innovation-scoreboard-2023\_en

24. OECD. Main Science and Technology Indicators. -Paris, 2023.

25. Federal\textquotesingle noye ministerstvo ekonomiki i energetiki
Germanii. Otchet o sostoyanii innovatsiy v Germanii. -Berlin, 2023.

26. Ministerstvo nauki i vysshego obrazovaniya RK. Otchet po
nauchno-issledovatelskim proyektam. -Astana, 2023.

27. Ivanov I.I. Razvitie innovatsionnykh tekhnologiy v usloviyakh
globalnoy konkurentsii // Innovatsii. -- 2024. -- № 3. -- S. 25-35. --
URL:
https://innovazia.ru/upload/iblock/e31/kio3arl3016iivxnge62qgpxndvmgzks/\%E2\%84\%963\%202024\%20\%D0\%98\%D0\%B8\%D0\%98.pdf
(data obrashcheniya: 05.09.2024)

28. National Science Foundation. Science and Engineering Indicators.
Arlington. -2023.

29. Ministerstvo truda i sotsial\textquotesingle noy zashchity
naseleniya RK. Statisticheskiy otchet po zanyatosti v sfere NIOKR.
-Almaty, 2023.

30. Ministry of Science and ICT of South Korea. Annual Report on Science
and Technology. Seoul. -2023.

31. Ministerstvo industrii i infrastrukturnogo razvitiya RK. Otchet po
nauchno-tekhnicheskoy infrastrukture Kazakhstana. - Almaty, 2023.

32. Kazakhstanskii tsentr innovatsii i startapov. Analiz sostoyaniya i
rekomendatsii. -Astana. 2023.

33. Combinator Y. Annual Report on Startup Growth and Impact. Mountain
View. -2023.

34. Swiss Innovation Agency. Regional Competence Centers and Their
Impact. Zurich. -2023.

35. Swiss Biotech. Annual Report on Biotech Innovations. Basel. -2023.

36. European Association of Science and Technology Parks. Science and
Technology Tourism Report. Brussels. -2023.

37. Chinese Academy of Sciences. International Scientific Conferences
and Their Impact. Beijing. -2023.

38. Ministerstvo iustitsii RK. Otchet po administrativnym i
biurokraticheskim bar\textquotesingle yeram dlya biznesa. Astana. 2023.

39. New Zealand Trade and Enterprise. Business Environment and
Innovation Report. Wellington. 2023.

\emph{{\bfseries Information about the authors}}

Mottaeva А.B. -- Doctor of Economics, professor, Financial university
under the government of the Russian Federation, Moscow, Russian
Federation, e-mail: doptaganka@yandex.ru;

Gordeyeva Ye.A. -- PhD, K. Kulazhanov Kazakh University of Technology
and Business, Astana, Kazakhstan, e-mail: gordelena78@mail.ru;

Sitenko D.A. -- PhD, professor, Karaganda Buketov University, Karaganda,
Kazakhstan, e-mail: daesha@list.ru;

Sabyrzhan A. -- c.e.s., professor, Karaganda Buketov University,
Karaganda, Kazakhstan. E-mail: alisher-aliev-79@mail.ru;

Temirbayeva D.M. -- PhD, assistant professor, Karaganda Buketov
University, Karaganda, Kazakhstan, е-mail: dina131111@mail.ru.

\emph{{\bfseries Сведения об авторах}}

Моттаева А.Б. -- д.э.н., профессор, Финансовый университет при
Правительстве Российской Федерации, Москва, Российская Федерация,
e-mail: doptaganka@yandex.ru;

Гордеева Е.А. -- PhD, Казахский университет технологий и бизнеса имени
К.Кулажанова, Астана, Казахстан, e-mail: gordelena78@mail.ru;

Ситенко Д.А. -- PhD, профессор, Карагандинский университет имени
академика Е.А.Букетова, Караганда, Казахстан, e-mail: daesha@list.ru;

Сабыржан A. -- к.э.н., профессор, Карагандинский университет имени
академика Е.А.Букетова, Караганда, Казахстан, e-mail:
alisher-aliev-79@mail.ru;

Темирбаева Д.М. -- PhD, ассистент профессора, Карагандинский университет
имени академика Е.А.Букетова, Караганда, Казахстан, е-mail:
dina131111@mail.ru.




















