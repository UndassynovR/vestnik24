

\newpage
{\bfseries ҒТАМР 52.01.93}

{\bfseries ТАУ-КЕН КӘСІПОРЫНДАРЫНЫҢ ЖҰМЫСКЕРЛЕРІНДЕ}

{\bfseries КӘСІПТІК АУРУЛАРДЫҢ ДАМУЫНЫҢ ӨНДІРІСТІК ҚАУІП ФАКТОРЛАРЫ}

{\bfseries А.М. Құрманов, А.М. Рахметова}\textsuperscript{🖂}, {\bfseries Э.А.
Құлмағамбетова,}

{\bfseries Н.Б. Әбдрахманова, Н.Т. Сағындықова}

Қазақстан Республикасы Еңбек және халықты әлеуметтік қорғау
министрлігінің

Еңбекті қорғау жөніндегі республикалық ғылыми-зерттеу институты ШЖҚ РМК,

Астана,Қазақстан

\textsuperscript{🖂}Корреспондент - автор: ra\_anar@mail.ru

Мақалада кәсіптік аурулардың даму факторы ретінде тау-кен кәсіпорны
қызметкерлерінің денсаулығына зиянды және қауіпті өндірістік
факторлардың әсер ету проблемасының қазіргі жағдайы қарастырылған.

Салыстыру топтары ретінде өндіріс қызметкерлері мен Қосалқы персонал
зерттелді, авторлар өндірістік объектілерді аттестаттау, сауалнама
деректерін талдады.

Еңбек қызметі процесінде зиянды және қауіпті өндірістік-кәсіптік
факторлардың тұрақты және қарқынды әсеріне ұшыраған қызметкерлердің
еңбек жағдайлары кәсіптік аурулардың жоғары таралуына және даму қаупінің
жоғарылауына ықпал ететіні анықталды. Өндіріс қызметкерлерінде кәсіптік
аурулардың даму қаупін төмендетудің кешенді тәсілі үшін
санитарлық-гигиеналық еңбек жағдайларын жақсарту, зиянды және қауіпті
өндірістік факторлардың әсерінен ұжымдық және жеке қорғаныс құралдарын
жетілдіру, алдын алу шараларын әзірлеу қажет.

{\bfseries Түйін сөздер:} тау-кен кәсіпорны, зиянды, қауіпті өндірістік
факторлар, кәсіптік тәуекел, кәсіптік аурулар.

{\bfseries ПРОИЗВОДСТВЕННЫЕ ФАКТОРЫ РИСКА РАЗВИТИЯ ПРОФЕССИОНАЛЬНЫХ
ЗАБОЛЕВАНИЙ У РАБОТНИКОВ ГОРНОДОБЫВАЮЩИХ ПРЕДПРИЯТИЙ}

{\bfseries А.М. Курманов, А.М. Рахметова}\textsuperscript{🖂}, {\bfseries Э.А.
Кульмагамбетова,}

{\bfseries Н.Б. Абдрахманова, Н.Т. Сагиндикова}

РГП на ПХВ Республиканский научно-исследовательский институт по охране
труда

Министерства труда и социальной защиты населения Республики Казахстан,

Астана, Казахстан,

e-mail: ra\_anar@mail.ru

В статье рассмотрено современное состояние проблемы влияния вредных и
опасных производственных факторов на здоровье работников
горнодобывающего предприятия, как фактор развития профессиональных
заболеваний.

В качестве групп сравнения обследованы работники производства и
вспомогательный персонал, авторами проанализированы данные аттестации
производственных объектов, анкетирования.

Установлено, что условия труда работников, подвергающихся в процессе
трудовой деятельности постоянному и интенсивному воздействию вредных и
опасных производственно-профессиональных факторов, способствуют более
высокой распространённости и более высокому риску развития
профессиональных заболеваний. Для комплексного подхода по снижению риска
развития профессиональных заболеваний у работников производства,
необходимо улучшение санитарно-гигиенических условий труда,
совершенствование средств коллективной и индивидуальной защиты от
воздействия вредных и опасных производственных факторов, разработка
превентивных мер профилактики.

{\bfseries Ключевые слова}: горнодобывающее предприятие, вредные, опасные
производственные факторы, профессиональный риск, профессиональные
заболевания.

{\bfseries OCCUPATIONAL RISK FACTORS FOR THE DEVELOPMENT OF OCCUPATIONAL
DISEASES IN MINING WORKERS}

{\bfseries A.M. Kurmanov, A.M. Rakhmetova}\textsuperscript{🖂}, {\bfseries E.A.
Kulmagambetova,}

{\bfseries N.B. Abdrakhmanova, N.T. Sagindykova}

RSE at the National Research Institute for Occupational Safety of the
Ministry of Labor

and Social Protection of the Population of the Republic of Kazakhstan,

Astana, Kazakhstan,

e-mail: ra\_anar@mail.ru

The article considers the current state of the problem of the influence
of harmful and dangerous production factors on the health of mining
enterprise workers as a factor in the development of occupational
diseases.

Production workers and support staff were examined as comparison groups,
the authors analyzed the data of certification of production facilities
and questionnaires.

It has been established that the working conditions of workers who are
exposed to constant and intense exposure to harmful and dangerous
occupational factors in the course of their work contribute to a higher
prevalence and a higher risk of developing occupational diseases. For an
integrated approach to reduce the risk of developing occupational
diseases in production workers, it is necessary to improve sanitary and
hygienic working conditions, improve collective and personal protection
against the effects of harmful and dangerous industrial factors, and
develop preventive preventive measures.

{\bfseries Key words}: mining enterprise, harmful, dangerous production
factors, occupational risk, occupational diseases.

{\bfseries Кіріспе.} Қазақстанның еңбекке қабілетті халқының денсаулығын
сақтау елдің табысты әлеуметтік-экономикалық дамуын қамтамасыз ету үшін
аса маңызды міндет болып табылады және Қазақстан Республикасының
2024-2030 жылдарға арналған Қауіпсіз еңбек тұжырымдамасында көрініс
табады {[}1{]}.

Қолайсыз өндірістік факторлар кешенінің тұрақты әсеріне ұшырайтын
тәуекел тобы тау-кен өнеркәсібінің жұмысшылары болып табылады {[}2,
3{]}.

Тау-кен өнеркәсібі ел экономикасында жетекші орындардың бірін алады және
зиянды, ауыр және қауіпті еңбек жағдайлары бар сала болып қала береді.

Өндірістік кәсіпорындардың жұмысшыларының денсаулық жағдайының
критерийлерінің бірі ретінде еңбекке қабілеттілігін уақытша жоғалтумен
сырқаттанушылықты зерттеу оның деңгейі мен нақты өндірістік фактілер
арасындағы байланысты орнатуға, сырқаттанушылықтың салдарынан
кәсіпорындардың экономикалық залалын анықтауға және оны төмендету
жөніндегі іс-шараларды әзірлеуге мүмкіндік береді {[}4{]}.

Шетелдік және отандық ғалымдардың пайымдауынша, өндірістік - шартты
ауруларды дамытудың кәсіптік тәуекел деңгейін анықтауда гигиеналық
критерийлер бойынша жұмысшылардың еңбек жағдайларын бағалау априори,
алдын-ала және сол арқылы болжамды болып табылады және оны денсаулық
жағдайының көрсеткіштері арқылы жұмысшылардың ағзасына қолайсыз кәсіптік
факторлардың әсер ету қаупін постериори, түпкілікті бағалау арқылы
күшейту керек. Бұл көрсеткіштерге кәсіби және кәсіби анықталған ауру,
сондай-ақ олардың негізінде есептелген интегралды көрсеткіштер жатады.

Авторлардың пікірінше {[}5-7{]}, тау-кен жұмысшыларының кәсіптік
ауруларының дамуына әсер ететін зиянды және қауіпті өндірістік факторлар
негізінен: шу, инфрақызыл, ауа ультрадыбысы; діріл (жалпы және
жергілікті), негізінен фиброгендік әсер ететін аэрозольдер, химиялық
фактор, еңбек процесінің ауырлығы мен шиеленісі. Тау-кен кәсіпорнында
осы факторлардың төмендеуі қазіргі кездегі өзекті мәселе екені сөзсіз.

Кәсіптік аурулар - дамуында жұмыс ортасы мен еңбек процесінің зиянды
және/немесе қауіпті факторларының әсерімен тікелей себеп-салдарлық
байланысты байқайтын аурулар {[}8{]}.

Кәсіптік аурудың даму мерзімі зиянды және/немесе қауіпті өндірістік
факторлар мен жұмыстардың әсер ету деңгейі мен ұзақтығына байланысты.
Бұрын оларды диагностикалау ұзақ уақыт бойы еңбек қызметі кезінде
аурудың сирек және спецификалық емес белгілері пайда болуымен қиындайды
{[}3, 8{]}.

Ресми статистиканың деректері жұмысшылардың еңбек жағдайлары мен кәсіби
денсаулығының қолайсыз жағдайын көрсетеді. Соңғы жылдары ҚР-дағы еңбек
гигиенасы және кәсіптік аурулар ұлттық орталығының мәліметтері бойынша
тау-кен өндіру кәсіпорындарында жұмыс істейтіндердің кәсіптік
сырқаттанушылығының салыстырмалы деңгейі өсуде, 2021 жылы 49,8\%,
2022-52,8\%, 2023-68\%. 2023 жылға арналған этиологиялық принцип бойынша
бастапқы кәсіптік аурулар топтары бойынша бөлу: өнеркәсіптік
аэрозольдердің әсерінен туындаған аурулар -35,4\%; жеке органдар мен
жүйелердің функционалдық шамадан тыс жүктелуіне және шамадан тыс
жүктелуіне байланысты аурулар -- 49,5\%, физикалық факторлардың әсеріне
байланысты аурулар -- 12,55; химиялық факторлардың әсерінен туындаған
аурулар -- 2,1\%.

Негізінен тірек-қимыл жүйесі (36,2\%), тыныс алу жүйесі (28,9\%) және
жүйке жүйесі (22,5\%) аурулары басым болды. Кәсіптік аурулардың ең көп
таралған үш нозологиялық түрі радикулопатия (32,1\%), созылмалы бронхит
(27,7\%) және моно-полиневропатия (15,4\%) болды {[}9{]}.

Тау-кен кәсіпорындарының өндірістік жағдайлары жұмыс аймағының ауасында
болатын зиянды химиялық элементтер кешенінің әсерімен сипатталады, бұл
жұмысшыларда патологиялық жағдайлардың, атап айтқанда тыныс алу
органдарының ауруларының дамуына әкелуі мүмкін. Жер асты тау - кен
жұмысшыларының кәсіби патологиясы тыныс алу жүйесі
ауруларының-пневмокониоздардың, жедел және созылмалы шаң бронхиттерінің
үлкен таралуымен сипатталады. Бұрғылау және жару жұмыстарын жүргізу
кезінде ұсақтау жұмыс аймағының ауасына (құрамында кремний диоксиді бар
бейорганикалық шаң) көп мөлшерде шаңның бөлінуі байқалады. Жұмыс
аймағының ауасында кездесетін негізгі заттардың арасында канцерогенді
заттар бар: никель, қорғасын, формальдегид, кадмий, бензин(а)пирен.
Күкірт ангидридінің, никельдің, азот оксидтерінің, акролеиннің,
формальдегидтің, кадмийдің, тоқтатылған заттардың тыныс алу органдарына
бір бағытты әсері байқалады. Марганец, қорғасын, селен жүйке жүйесіне
теріс әсер етуі мүмкін. Қан жүйесіне никель, қорғасын, көміртегі оксиді,
жүрек-тамыр жүйесіне - көміртегі оксиді мен селен теріс әсер етеді.
Жұмысшылардың еңбек жағдайлары тыныс алу жүйесінің кәсіби аурулары мен
қатерлі ісіктердің даму қаупін тудыратын жұмыс аймағының ауасындағы
химиялық заттардың қарқынды әсерімен сипатталады {[}10{]}.

Көптеген жылдар бойы тау-кен жұмысшыларының кәсіби ауруы, әсіресе, осы
зерттеудің мақсаты болып табылатын Тау-кен өнеркәсібіндегі маңызды
медициналық, әлеуметтік және экономикалық проблема болып қала береді.

Зерттеудің мақсаты тау-кен кәсіпорнының зиянды және қауіпті өндірістік
факторларының жұмысшылардың денсаулығына, кәсіптік аурулардың даму
қаупіне әсер ету ерекшеліктерін зерттеу болды.

{\bfseries Материалдар мен әдістер.} Өндірістік объектілерді аттестаттау
деректері зерттелді, өндіріс қызметкерлеріне сауалнама жүргізілді және
талданды. Салыстырудың екі тобы ретінде тау-кен кәсіпорнының өндірістік
және қосалқы қызметкерлерінің қызметкерлері зерттелді.

74 жұмысшының еңбек жағдайларына бағалау жүргізілді, оның ішінде жерасты
жұмыстарымен айналысатындар 67,6 \% (50 бірлік), автокөлік кәсіпорнында
- 18,2 \% (14 бірлік), ашық аумақтағы жұмыстар - 8,8 \% (6 бірлік) және
биіктіктегі жұмыстар - 5,4 \% (4 бірлік). Еңбек жағдайларын бағалау оның
ауырлығы мен шиеленісін, жұмыс орындарының микроклиматының
параметрлерін, физикалық және химиялық факторлардың әсер ету сипатын
ескере отырып жүргізілді.

Кен орнының негізгі аймағын өнеркәсіптік игеру 2023 жылғы 1 қаңтарда
басталды. Құрамында мыс, күміс, молибден және селен сияқты элементтер
бар кенді өндіру және байыту бойынша.

{\bfseries Нәтижелер және талқылау.} Кәсіпорындағы кәсіптік тәуекелдің
құрылымы мен дәрежесін талдау көптеген кәсіптердегі еңбек жағдайларын
зиянды деп бағалауға мүмкіндік берді, ал өткізгіштерде, жол
жұмысшыларында, бұрғылаушыларда, бұрғылаушылар мен жарылғыштардың
көмекшісінде-қауіпті (тау жыныстарының құлауы, тастар, құлау, үйінділер,
электр тогының соғуы, жарылыс, шаң).

Профилактикалық іс-шаралар кешенін жоспарлау кезінде еңбек жағдайларына
байланысты, оның ішінде осы өндіріс үшін Нақты химиялық фактордың
болуына байланысты аурулар тобын анықтаған жөн. Априорлық кәсіптік
тәуекелді сипаттайтын зияндылық пен қауіптің жоғары дәрежесі тау-кен
кәсіпорны қызметкерлерінің денсаулығы үшін кәсіптік қауіптің жоғары
деңгейін болжауға мүмкіндік береді.

Ақпаратқа сүйене отырып, жұмыс орындарын аттестациялау деректерінің
талдауынан өндірістік орта жұмысшыларға теріс әсер ететін химиялық
факторлар кешенінің болуымен сипатталады (1-кесте).

{\bfseries 1-кесте. Жұмысшылардың денсаулығына әсер ететін химиялық
факторлар тізімі}

\begin{longtable}[]{@{}
  >{\raggedright\arraybackslash}p{(\columnwidth - 6\tabcolsep) * \real{0.0583}}
  >{\raggedright\arraybackslash}p{(\columnwidth - 6\tabcolsep) * \real{0.3237}}
  >{\raggedright\arraybackslash}p{(\columnwidth - 6\tabcolsep) * \real{0.1177}}
  >{\raggedright\arraybackslash}p{(\columnwidth - 6\tabcolsep) * \real{0.5002}}@{}}
\toprule\noalign{}
\begin{minipage}[b]{\linewidth}\raggedright
п/п
\end{minipage} & \begin{minipage}[b]{\linewidth}\raggedright
Заттың атауы
\end{minipage} & \begin{minipage}[b]{\linewidth}\raggedright
Қауіптілік класы
\end{minipage} & \begin{minipage}[b]{\linewidth}\raggedright
Жұмысшы денсаулығына теріс әсері
\end{minipage} \\
\midrule\noalign{}
\endhead
\bottomrule\noalign{}
\endlastfoot
1 & Темір оксиді (II, III) & 3 & Сидероз (пневмокониоз) \\
2 & Марганец және оның қосылыстары & 2 & Аллергенді. мутагендік әсер \\
3 & Фторлы газ тәрізді қосылыстар & 2 & Қаңқа сүйектерінің қалыптасуы
мен өсуінің бұзылуы \\
4 & Хром (VI) оксиді /хром бойынша есептегенде & 1 & Демікпе, дерматит,
аллергиялық реакциялар, мутациялар \\
5 & Қалайы оксиді & 3 & Пневмокониоз \\
6 & Қорғасын және оның қосылыстары & 1 & Анемия, гипертензия, бүйрек
жеткіліксіздігі, иммундық және репродуктивті жүйеге уытты әсер \\
7 & Азот (IV)диоксиді & 2 & Өкпе ісінуі, жүйке аурулары \\
9 & Фторидтер нашар еритін бейорганикалық & 2 & Қаңқа сүйектерінің
қалыптасуы мен өсуінің бұзылуы \\
10 & Бейорганикалық шаң (кремний диоксиді) & 3 & Силикоз \\
11 & Диметилбензол & 3 & Жүрек-қан тамырлар органдарының созылмалы
зақымдануы, жүйке жүйесінің жұмысындағы бұзылулар, тері реакциялары,
астма \\
12 & Көміртек оксиді & 4 & Иммунологиялық белсенділіктің төмендеуі,
жоғарылауы қандағы қант мөлшері, жүрекке оттегінің берілуін
әлсіретеді. \\
13 & Метилбензол & 3 & Асма, химиялық бронхит, пневмония, өкпе ісінуі \\
14 & Бутан-1-ол & 3 & Көз ауруы (конъюнктивит, кератит), ОЖЖ \\
15 & Этанол & 4 & Онкологиялық, жүрек-қан тамырлары ауруы, эндокриндік
аурулар \\
16 & 2- этоксиэтанол (этил эфирі) & - & Бронхит, өкпенің қабынуы,
бронх-өкпе жүйесінің созылмалы аурулары, нейропсихикалық аурулар,
энцелофапатия, уытты гепатит және т. б. \\
17 & Бутил ацетат & 3 & Орталық жүйке жүйесі жүрек-тамыр жүйесі, бауыр,
бүйректің уытты құбылыстары \\
18 & Пропан-2-он (ацетон) & 4 & Тұншығу, мидағы тыныс алу орталығының
сал ауруы, мидың токсикалық ісінуі, өкпе ісінуі \\
19 & Уайт-спирит & - & Дерматит, экзема \\
20 & Көмірсутектер С\textsubscript{12}-С\textsubscript{19} & 4 & Тыныс
алу орталықтарының сал ауруы, орталық жүйке жүйесі \\
21 & Кальций дигидроксиді & 3 & Жоғарғы тыныс жолдарының тітіркенуі,
көздің, терінің, асқазан-ішек жолдарының зақымдалуы \\
22 & Күкірт диоксиді & 3 & В\textsubscript{1}, В\textsubscript{12}
дәрумендеріне жойқын әсер етеді. \\
23 & Күйе & 3 & Мутагендік және канцерогендік әсер \\
24 & Керосин & - & Күйік, улану. \\
\end{longtable}

1-кестенің деректері негізінде 4-қауіптілік класына мыналар жатады:
көмірсутектер С\textsubscript{12}-С\textsubscript{19}, пропан-2-он
(ацетон), этанол, көміртегі оксиді және т. б., олар жұмысшылардың
ағзасына уытты әсерімен сипатталады, иммунологиялық белсенділіктің
төмендеуіне, қандағы қант мөлшерінің жоғарылауына әкеледі, жүрекке
оттегінің берілуін әлсіретеді, онкологиялық, жүрек-тамыр жүйесі,
эндокриндік аурулар. Ұзақ уақыт әсер еткенде тұншығу, мидағы тыныс алу
орталығының параличі, мидың токсикалық ісінуі, өкпе ісінуі пайда болады.

Кестедегі мәліметтерге сәйкес 3-қауіптілік класына - темір оксиді,
қалайы оксиді, бейорганикалық шаң, диметилбензол, метилбензол, бутан,
бутил ацетаты, кальций дигидроксиді, күкірт диоксиді, күйе жатады. Бұл
химиялық заттар пневмокониоз, силикоз, сондай-ақ жүрек-қан тамырлар
органдарының созылмалы зақымдануы, жүйке жүйесінің бұзылуы, тері
реакциялары, астма, созылмалы бронхит, өкпе ісінуі және т. б. сияқты
кәсіби ауруларға әкеледі.

Зерттеу нәтижелерінің деректеріне сәйкес 2-қауіптілік класы мыналарды
сипаттайды: марганец және оның қосылыстары, фторлы қосылыстар,
аллергенді және мутагендік әсері бар азот диоксиді, қаңқа сүйектерінің
қалыптасуы мен өсуінің бұзылуын тудырады және т. б.

Осылайша, кәсіпорынның өндірістік жағдайлары жұмыс аймағының ауасында
болатын зиянды химиялық элементтер кешенінің әсерімен сипатталады, бұл
қызметкерлерде патологиялық жағдайлардың, кәсіптік аурулардың дамуына
әкелуі мүмкін.

Сауалнама деректерін талдау кезінде барлық бөлімшелерде жұмыс
істейтіндердің жас-еңбек өтілінің құрамы нақты айырмашылықтар болмағаны
және 35 жыл, ал көмекші персонал үшін-40,5 жыл болғандығы анықталды.
Өндірістік персонал үшін орташа жұмыс өтілі - 8,6 жыл, ал қосалқы
персонал үшін - 11 жыл болды (2-кесте).

2 кестеде жұмыс істейтіндерді мамандық бойынша бөлу көрсетілген

{\bfseries 2-кесте. Жұмысшыларды мамаңдық бойынша бөлу}

\begin{longtable}[]{@{}
  >{\raggedright\arraybackslash}p{(\columnwidth - 10\tabcolsep) * \real{0.0452}}
  >{\raggedright\arraybackslash}p{(\columnwidth - 10\tabcolsep) * \real{0.5249}}
  >{\raggedright\arraybackslash}p{(\columnwidth - 10\tabcolsep) * \real{0.1081}}
  >{\raggedright\arraybackslash}p{(\columnwidth - 10\tabcolsep) * \real{0.1167}}
  >{\raggedright\arraybackslash}p{(\columnwidth - 10\tabcolsep) * \real{0.1026}}
  >{\raggedright\arraybackslash}p{(\columnwidth - 10\tabcolsep) * \real{0.1025}}@{}}
\toprule\noalign{}
\multicolumn{6}{@{}>{\raggedright\arraybackslash}p{(\columnwidth - 10\tabcolsep) * \real{1.0000} + 10\tabcolsep}@{}}{%
\begin{minipage}[b]{\linewidth}\raggedright
Кәсіпорынның функционалдық құрылымы

(ауысымда жұмыс істейтіндердің жалпы саны -74)
\end{minipage}} \\
\midrule\noalign{}
\endhead
\bottomrule\noalign{}
\endlastfoot
\multicolumn{4}{@{}>{\raggedright\arraybackslash}p{(\columnwidth - 10\tabcolsep) * \real{0.7949} + 6\tabcolsep}}{%
Өндірістік персонал

(n=68)} &
\multicolumn{2}{>{\raggedright\arraybackslash}p{(\columnwidth - 10\tabcolsep) * \real{0.2051} + 2\tabcolsep}@{}}{%
Көмекші персонал

(n=6)} \\
№ & Мамандықтар & абс. & \% & абс. & \% \\
1 & Жол-сапар жұмысшысы (4) & 4 & 5,88 & & \\
2 & Сорғы қондырғыларының машинисі (4) & 4 & 5,88 & & \\
3 & Үңгуші (10) & 10 & 14,73 & & \\
4 & Тиеу-жеткізу машинасының машинисі (14) & 14 & 20,59 & & \\
5 & Бұрғылаушы (5) & 5 & 7,35 & & \\
6 & Бұрғылаушының көмекшісі (5) & 5 & 7,35 & & \\
7 & Электр дәнекерлеуші & 1 & 1,47 & & \\
8 & Электромонтер (2) & 2 & 2,94 & & \\
9 & Электрослесарь-монтажшы (2) & 2 & 2,94 & & \\
10 & Өздігінен жүретін жерасты машиналарының машинисі (2) & 2 & 2,94 &
& \\
11 & Өздігінен жүретін жерасты машиналарының машинисі көтергіш (2) & 2 &
2,94 & & \\
12 & Жанар-жағармай материалдарын жеткізу машинисі (2) & 2 & 2,94 & & \\
13 & Тау-кен жұмысшысы & 1 & 1,47 & & \\
14 & Жарылғыш материалдарды таратушы (2) & 2 & 2,94 & & \\
15 & Жерасты тау-кен жұмысшысы (жарылғыш материалдарды жеткізу бойынша)
(2) & 2 & 2,94 & & \\
16 & Ұсатқыш (1) & 1 & 1,47 & & \\
17 & Радиометрист - зертханашы (1) & 1 & 1,47 & & \\
18 & Маркшейдерлік жұмыстардағы тау-кен жұмысшысы (1) & 1 & 1,47 & & \\
19 & Жерасты өздігінен жүретін машинасының жүргізушісі & 1 & 1,47 & & \\
20 & Бас маркшейдер (1) & 1 & 1,47 & & \\
21 & Жарғыш (5) & 5 & 7,35 & & \\
22 & Тау шебері (1) & & & 1 & 6 \\
23 & Бас энергетик & & & 1 & 6 \\
24 & Энергетик & & & 1 & 6 \\
25 & Механик & & & 1 & 6 \\
26 & Учаске шебері (1) & & & 1 & 6 \\
27 & Бас механик (1) & & & 1 & 6 \\
\end{longtable}

2-кестенің деректері негізінде кәсіпорынның функционалдық құрылымы
үрлегіштерден (14,73), тиеу-жеткізу машинасының машинисінен (20,59\%),
бұрғылаушы мен бұрғылаушының көмекшісінен (тиісінше 7,35\%), жол-сапар
жұмысшысынан, сорғы қондырғыларының машинисінен (тиісінше 5,88\%),
электромонтерден, электр слесарь-монтаждаушыдан, өздігінен жүретін
жерасты машинисінен тұрады машиналарды көтергіш, жерасты өздігінен
жүретін машинасының жүргізушісі, жарылғыш материалдарды таратушы,
жерасты тау-кен жұмысшысы (тиісінше 2,94\% - дан), сондай-ақ тау-кен
жұмысшылары, маркшейдерлік жұмыстардағы тау-кен жұмысшылары, ұсатқыштар,
радиометрист зертханашы, жерасты өздігінен жүретін машинасының
жүргізушілері, негізгі маркшейдерлер мен жарылғыштар (сәйкесінше
1,47\%).

Жұмыс орындарындағы еңбек жағдайларын бағалау кезінде көбінесе
өндірістік орта факторларының аралас және аралас әсері байқалады,
сондықтан өндірістік персоналдың жұмысшылары басым зиянды өндірістік
факторларды ескере отырып, 6 топқа бөлінді.

1-ші топқа 3.3--3.4 еңбек жағдайларының сыныптарына сәйкес келетін
гигиеналық критерийлер бойынша жоғары кәсіби тәуекелі бар адамдар кірді:
үңгуші, тиеу-жеткізу машиналарының машинистері, бұрғылау қондырғысының
машинистері, экскаватор машинистері, тау-кен массасын жою машинистері,
ұсатқыштар.

Осы топтағы жұмысшылардың еңбек жағдайларының ерекшелігі зиянды және
қауіпті факторлар кешенінің бірлескен әсері болып табылады: аралас
діріл; шу (тұрақты емес), спектрде төмен және орташа жиілікті
компоненттер басым; дененің мәжбүрлі жағдайы, жүйке-бұлшықет кернеуі;
шаңның жоғарылауы; газдың жоғарылауы; қолайсыз микроклиматтық
жағдайларда жұмыс істеу; еңбек процесінің ауырлығы мен шиеленісі де
гигиеналық тұрғыдан маңызды. Еңбек ауырлығы перфораторларда, бұрғылау
қондырғыларында, тиеу - жеткізу машиналарында жұмыс істеуге байланысты,
еңбек процесінің шиеленісі ауысымда 12 сағатты құрайды.

Шаң факторы 20-дан 70\% - ға дейінгі шаңда (оның ішінде мыс, күміс,
селен, молибден және т.б.) SiO\textsubscript{2} құрамындағы негізінен
фиброгендік әсер ететін аэрозольдерден тұратын күрделі химиялық
құрамдағы шаңмен ұсынылған. Негізгі көздері бөлінетін зиянды газдар
болып табылады жару жұмыстары, жұмыс істейтін автокөлік, процестер
тотығу және жану пайдалы, сондай-ақ зиянды газдардың бөлінуін
жыныстардан және межпластовых суларды. Ішкі жану қозғалтқыштарының
жұмысына байланысты пайдаланылған газдардың жекелеген, неғұрлым
гигиеналық маңызды компоненттері шекті рұқсат етілген концентрациядан
асатын мөлшерде анықталады.

2-ші топқа ауыр жүк көліктерінің жүргізушілері, бульдозеристер кірді.
Еңбек жағдайлары аралас әсермен сипатталады: Жалпы және жергілікті діріл
(3.1--3.2 класс), тұрақты емес тербелмелі шу (3.1 класс), сенсорлық және
эмоционалдық жүктемелер орын алады, құрамында кремний диоксиді (кен) бар
Бейорганикалық шаңның және дизель отынының жану өнімдерінің газдарының
концентрациясы шекті рұқсат етілген концентрациядан аспайды.

3-ші топқа шаң, шу және қолайсыз микроклимат сияқты факторлардың басым
әсеріне ұшыраған жұмысшылар кірді. Бұл топтың жұмысшыларының арасында
гигиеналық критерийлер бойынша еңбек жағдайларының 3.1--3.3 сыныптарына
сәйкес келетін өте жоғары (тау-кен кенжарлары) немесе жоғары (жарылғыш
заттар, бекіткіштер) кәсіби тәуекелі бар кәсіби топтар бар.

4-ші топтың жұмысшылары үшін (сорғы қондырғысының машинистері, тау--кен
жабдықтарын жөндеу және техникалық қызмет көрсету слесарлары) физикалық
жүктеме мен дірілмен үйлескен технологиялық жабдықтың кең жолақты шуының
басым әсері тән (3.2-3.3 класс).

5-ші топты электр-газбен дәнекерлеушілер құрады, олардың еңбек
жағдайлары өндірістік факторлар кешенінің әсерімен сипатталады:
дәнекерлеу аэрозолының және негізінен фиброгендік әсер ететін
аэрозольдің құрамына кіретін марганец, темір, көміртек, азот
қосылыстары; қолайсыз микроклимат, физикалық шамадан тыс кернеу, шу.
Жұмыс аймағының ауасындағы марганец қосылыстарының құрамы 0,2
мг/м\textsuperscript{3} шекті рұқсат етілген концентрациядан кезінде
0,25 мг/м\textsuperscript{3} дейін құрайды (3.1-3.2 класс).

6-топ жұмысшылары үшін (электр дәнекерлеушілер мен электр слесарлары)
кернеуі 42 В және одан жоғары электр қондырғыларына қызмет көрсету және
жөндеу кезінде электромагниттік өрістер мен сәулеленулер, биіктікте
жұмыс істеу тән өндірістік факторлар болып табылады. Сонымен қатар,
жұмыс жабдықтың жұмыс режиміне жауапкершілікпен байланысты еңбек
процесінің жүйке-эмоционалды күйзелісімен сипатталады.

Осылайша, тау-кен өндірісінің күрделілігі, бір-бірінен едәуір алыс
орналасқан жұмыс орындарының жерасты орналасуы, табиғи және техникалық
қауіп факторларының пайда болуы, жұмыс орындары мен еңбек жағдайларының
үнемі өзгеруі-бұл еңбек қызметі процесінде зиянды және қауіпті
өндірістік және қауіпті жағдайлардың тұрақты және қарқынды әсеріне
ұшырайтын жұмысшылардың еңбек жағдайлары мен қауіпсіздігіне әсер ететін
барлық факторлар. -кәсіби факторлардың жоғары таралуына және кәсіби
аурулардың даму қаупінің жоғарылауына ықпал етеді.

{\bfseries Қорытынды.} Жабық кеніштерде жоғары өнімді тау-кен жабдықтарын
пайдалану жағдайында жетекші кәсіптердің қызметкерлеріне ауысым кезінде
ауырлығы өзгеруі мүмкін және көбінесе рұқсат етілген шекті мөлшерден
асатын өндірістік факторлар кешені (шаң, газдану, шу, діріл, қолайсыз
микроклимат, еңбек процесінің ауырлығы мен шиеленісі) әсер етеді. Осы
өндірістің жұмыс орындарындағы еңбек жағдайларын жалпы бағалау кезінде
зияндылықтың әр түрлі дәрежесі бар және 4 класы бар қауіптіліктің 3
класы белгіленді.

Мәндері гигиеналық нормаларға сәйкес келмейтін анықталған зиянды
өндірістік факторлардың қатарына мыналар жатады: шу (тексерілгендердің
85,3\%), өндірістік діріл (89,7 \%), шаңдану (92,7\%), газдану (86,8
\%), электромагниттік сәулелену (85,3\%), жоғары және төмен температура
(тиісінше 5,9 және 7,4 \%), биіктікте жұмыс істеу (5,9 \%). (68
қызметкер), сондай-ақ жерасты жұмысшыларында, бұрғылаушыларда, сорғы
қондырғыларының жүргізушілері мен машинистерінде еңбек процесінің
ауырлығы (52,9\%)., бұл тірек-қимыл аппаратының, тыныс алу, жүрек-тамыр
жүйесінің ауруларына, ОЖЖ, бауыр, бүйрек, эндокриндік, онкологиялық және
мутагендік аурулардың уытты зақымдалуына әкелуі мүмкін.

Зерттеудің алынған нәтижелерін ескере отырып, зиянды және қауіпті
өндірістік факторлардың әсерін азайту және еңбек жағдайларын жақсарту
үшін мынадай ұсынымдарды көздеу ұсынылады: өндірістегі кәсіптік
тәуекелдерді төмендету жөніндегі алдын алу шараларын ұйымдастыру, еңбек
қауіпсіздігі нормаларының, қағидалары мен стандарттарының талаптарын
сақтауды қамтамасыз ету.

Мақалада «Қазақстан Республикасы Еңбек және халықты әлеуметтік қорғау
министрлігінің Еңбекті қорғау жөніндегі республикалық ғылыми-зерттеу
институты» ШЖҚ РМК-ның ғылыми-зерттеу жұмыстарын бағдарламалық-нысаналы
қаржыландыру шеңберінде «Еңбек жағдайлары және кәсіптік тәуекелдер:
«жасыл экономикаға» көшу шеңберіндегі жіктеу, санаттар және топтастыру
өлшемшарттары» (ЖТН BR22182667) тақырып бойынша ғылыми-техникалық
бағдарламаны іске асыру барысында алынған ғылыми зерттеулердің
нәтижелері ұсынылған.

{\bfseries Әдебиеттер}

1. Қазақстан Республикасы Үкіметінің 2023 жылғы 28 желтоқсандағы № 1182
Қаулысы Қазақстан Республикасының 2024 -- 2030 жылдарға арналған
Қауіпсіз еңбек тұжырымдамасын бекіту туралы. URL:
https://adilet.zan.kz/kaz/docs/P2300001182

2. Жеглова А\emph{.}В\emph{.} Системный подход к управлению
профессиональным риском нарушений здоровья работников горнорудной
промышленности. Автореф. дисс. докт. мед. наук-Москва.- 2009.- 44 с.

3. Куликов А.С. Оценка влияния вредных и опасных производственных
факторов на развитие профессиональных заболеваний на горнодобывающих
предприятиях /Научно-образовательный журнал для студентов и
преподавателей «StudNet» No6/2020.-С124-128

4. Д. С. Абитаев,Н. Ж. Ердесов, Б. С. Жумалиев, Т. Ф. Машина, Б. Серик,
М. Г. Калишев, Н. Шинтаева, С. Р. Жакенова Профессиональные риски и
состояние здоровья лиц, работающих в горнорудной промышленности
центрального Казахстана /Экология и гигиена. 2020. №2 -- С. 41-457. URI:
http://repoz.kgmu.kz/handle/123456789/496

5. Коршунов Г.И., Черкай З.Н., Мухина Н.В., Гридина Е.Б., Скударнов С.М.
Профессиональные болезни рабочих в горнодобывающей промышленности.
С-1-6.
https://cyberleninka.ru/article/n/professionalnye-bolezni-rabochih-v-gornodobyvayuschey-promyshlennosti/viewer

6. Siti N. E. I. Azizan R. Investigate the factors affecting safety
culture in the Malaysian mining industry Resources Policy Volume 85,
Part A,~August 2023, 103930
https://doi.org/10.1016/j.resourpol.2023.103930

7. Donoghue, A. M. (2004). Occupational health hazards in mining: an
overview. Occupational Medicine, 54(5), 283--289.
doi:10.1093/occmed/kqh072~

8. Профессиональная патология: национальное руководство / Под ред. Н.Ф.
Измерова. - М.: ГЭОТАР-Медиа, 2011.-С. 144-153. ISBN 13:
978-5-9704-1947-2

9. Горбанев С.А., Сюрин С.А., Фролова Н.М. Условия труда и
профессиональная патология горняков угольных шахт в Арктике. Медицина
труда и промышленная экология 2019 (8). -С. 452-457.
https://doi.org/10.31089/1026-9428-2019-59-8-452-457

10. Фадеев А.Г., Горяев Д.В., Шур П.З., Зайцева Н.В., Фокин В.А., Редько
С.В. Вредные вещества в воздухе рабочей зоны горнодобывающего сектора
металлургической промышленности как факторы риска для здоровья
работников (аналитический обзор)/ Анализ риска здоровью. -- 2024. -- №
2. -- С. 153-161. DOI: 10.21668/health.risk/2024.2.14

{\bfseries References}

1. Qazaqstan Respýblıkasy Úkimetiniń 2023 jylǵy 28 jeltoqsandaǵy № 1182
Qaýlysy Qazaqstan Respýblıkasynyń 2024 -- 2030 jyldarǵa arnalǵan
Qaýipsiz eńbek tujyrymdamasyn bekitý týraly. URL:
https://adilet.zan.kz/kaz/docs/P2300001182 {[}in Kazakh{]}

2. Zheglova A.V. Sistemnyĭ podhod k upravleniju
professional\textquotesingle nym riskom narusheniĭ
zdorov\textquotesingle ja rabotnikov gornorudnoĭ promyshlennosti.
Avtoref. diss. dokt. med. nauk-Moskva.- 2009.- 44 s. {[}in Russian{]}

3. Kulikov A.S. Ocenka vlijanija vrednyh i opasnyh proizvodstvennyh
faktorov na razvitie professional\textquotesingle nyh zabolevanij na
gornodobyvajushhih predprijatijah
/Nauchno-obrazovatel\textquotesingle nyĭ zhurnal dlja studentov i
prepodavateleĭ «StudNet» No6/2020.-S124-128 {[}in Russian{]}

4. D. S. Abitaev,N. Zh. Erdesov, B. S. Zhumaliev, T. F. Mashina, B.
Serik, M. G. Kalishev, N. Shintaeva, S. R. Zhakenova
Professional\textquotesingle nye riski i sostojanie
zdorov\textquotesingle ja lic, rabotajushhih v gornorudnoj
promyshlennosti central\textquotesingle nogo Kazahstana /Jekologija i
gigiena. 2020. №2 -- S. 41-457. URI:
http://repoz.kgmu.kz/handle/123456789/496 {[}in Russian{]}

5. Korshunov G.I., Cherkaĭ Z.N., Muhina N.V., Gridina E.B., Skudarnov
S.M. Professional\textquotesingle nye bolezni rabochih v
gornodobyvajushhej promyshlennosti. S-1-6.
https://cyberleninka.ru/article/n/professionalnye-bolezni-rabochih-v-gornodobyvayuschey-promyshlennosti/viewer
{[}in Russian{]}

6. Siti N. E. I. Azizan R. Investigate the factors affecting safety
culture in the Malaysian mining industry Resources Policy Volume 85,
Part A, August 2023, 103930
https://doi.org/10.1016/j.resourpol.2023.103930

7. Donoghue, A. M. (2004). Occupational health hazards in mining: an
overview. Occupational Medicine, 54(5), 283--289.
doi:10.1093/occmed/kqh072

8. Professional\textquotesingle naja patologija:
nacional\textquotesingle noe rukovodstvo / Pod red. N.F. Izmerova. - M.:
GJeOTAR-Media, 2011.-S. 144-153. ISBN 13: 978-5-9704-1947-2 {[}in
Russian{]}

9. Gorbanev S.A., Sjurin S.A., Frolova N.M. Uslovija truda i
professional\textquotesingle naja patologija gornjakov
ugol\textquotesingle nyh shaht v Arktike. Medicina truda i
promyshlennaja jekologija 2019 (8). -S. 452-457.
https://doi.org/10.31089/1026-9428-2019-59-8-452-457 {[}in Russian{]}

10. Fadeev A.G., Gorjaev D.V., Shur P.Z., Zajceva N.V., Fokin V.A.,
Red\textquotesingle ko S.V. Vrednye veshhestva v vozduhe rabochej zony
gornodobyvajushhego sektora metallurgicheskoj promyshlennosti kak
faktory riska dlja zdorov\textquotesingle ja rabotnikov (analiticheskij
obzor)/ Analiz riska zdorov\textquotesingle ju. -- 2024. -- № 2. -- S.
153-161. DOI: 10.21668/health.risk/2024.2.14 {[}in Russian{]}

\emph{{\bfseries Авторлар туралы мәліметтер}}

Құрманов А. М. - экономия ғылымдарының кандидаты, Қазақстан Республикасы
Еңбек және халықты әлеуметтік қорғау министрлігінің Еңбекті қорғау
жөніндегі республикалық ғылыми-зерттеу институты» ШЖҚ РМК (ҚР ЕХӘҚМ
РҒЗИ) бас директоры, Астана ., Қазақстан, e-mail: rniiot@rniiot.kz;

Рахметова А.М. -медицина ғылымдарының кандидаты, доцент, биомониторинг
және еңбек гигиенасы бөлімінің басшысы, «Қазақстан Республикасы Еңбек
және халықты әлеуметтік қорғау министрлігінің Еңбекті қорғау жөніндегі
республикалық ғылыми-зерттеу институты» ШЖҚ РМК, Астана, Қазақстан,
e-mail: ra\_anar@mail.ru;

Құлмағамбетова Э.А. - химия ғылымдарының кандидаты, биомониторинг және
еңбек гигиенасы бөлімінің жетекші ғылыми қызметкері, «Қазақстан
Республикасы Еңбек және халықты әлеуметтік қорғау министрлігінің Еңбекті
қорғау жөніндегі республикалық ғылыми-зерттеу институты» ШЖҚ РМК,
Астана, Қазақстан, e-mail: elya\_kulmagambet@mail.ru;

Әбдрахманова Н.Б. - тіршілік қауіпсіздігі және қоршаған ортаны қорғау
ғылымдарының магистрі, биомониторинг және еңбек гигиенасы бөлімінің аға
ғылыми қызметкері, «Қазақстан Республикасы Еңбек және халықты әлеуметтік
қорғау министрлігінің Еңбекті қорғау жөніндегі республикалық
ғылыми-зерттеу институты» ШЖҚ РМК, Астана, Қазақстан, e-mail:
nazgul122@mail.ru;

Сағиндикова Н.Т. - техника және технология ғылымдарының магистрі,
биомониторинг және еңбек гигиенасы бөлімінің аға ғылыми қызметкері,
«Қазақстан Республикасы Еңбек және халықты әлеуметтік қорғау
министрлігінің Еңбекті қорғау жөніндегі республикалық ғылыми-зерттеу
институты» ШЖҚ РМК, Астана, Қазақстан, e-mail: nursag79@mail.ru.

\emph{{\bfseries Information about authors}}

Kurmanov A.M. - Ph.D. in Economics, General Director of the RNIIOT of
the Ministry of Health of the Republic of Kazakhstan, Astana,
Kazakhstan, e-mail: rniiot@rniiot.kz;

Rakhmetova A.M. - Candidate of Medical Sciences, Associate Professor,
Head of the Department of Biomonitoring and Occupational Hygiene of the
RRIOSH of the Ministry of Health of the Republic of Kazakhstan, Astana,
Kazakhstan, e-mail: ra\_anar@mail.ru;

Kulmagambetova E. A.- Ph.D. - Senior Researcher at the Department of
Biomonitoring and Occupational Hygiene of the RRIOSH of the Ministry of
Health of the Republic of Kazakhstan, Astana, Kazakhstan, e-mail:
elya\_kulmagambet@mail.ru;

Abdrakhmanova N.B.- Master\textquotesingle s Degree in Life Safety and
Environmental Protection, Senior Researcher at the Department of
Biomonitoring and Occupational Hygiene of the Research Institute of the
Ministry of Health of the Republic of Kazakhstan, Astana, Kazakhstan,
e-mail: nazgul122@mail.ru;

Sagindikova N. T.- Master of Engineering and Technology, Senior
Researcher at the Department of Biomonitoring and Occupational Hygiene
of the Research Institute of the Ministry of Health of the Republic of
Kazakhstan, Astana, Kazakhstan, e-mail: nursag79@mail.ru.


