\newpage
IRSTI 53.01.91
\hfill {\bfseries \href{https://doi.org/10.58805/kazutb.v.3.24-529}{https://doi.org/10.58805/kazutb.v.3.24-529}}

\sectionwithauthors{N.I. Kopylov, Kh.B.Omarov}{PROCESSING OF SOLID ARSENIC -CONTAINING MATERIALS}

\begin{center}
(analytical review)

{\bfseries \textsuperscript{1}N.I. Kopylov,
\textsuperscript{2}Kh.B.Omarov\envelope}

\textsuperscript{1}Institute of Solid State Chemistry and
Mechanochemistry, Siberian Branch Russian Academy of Sciences,
Novosibirsk, Russia,

\textsuperscript{2} K.Kulazhanov named Kazakh University of Technology
and Business, Astana, Kazakhstan
\end{center}

\envelope Correspondent-author: homarov1963@mail.ru

This article provides an analytical review of scientific and technical
developments in the processing of solid arsenic-containing materials.
Hydro-, pyrometallurgical and combined methods for extracting arsenic
from dust, sublimates, and sludge from metallurgical production are
considered. The most acceptable technologies seem to be those in which
arsenic is removed from the process at the preliminary stage of
processing in the most stable, non-toxic form - arsenic sulfides, which
exist in nature as orpiment and realgar minerals. Studies conducted at
the Zh. Abishev Chemical and Metallurgical Institute (Karaganda,
Republic of Kazakhstan) and the Gidrotsvetmet Institute (Novosibirsk,
Russian Federation) on products from a number of arsenic-containing raw
material deposits for the removal of arsenic by oxidizing-sulfidizing
roasting in a shaft-type furnace have shown the promise of this
technology. The removal of arsenic from roasting products was 97-98.5\%,
simultaneously with the production of compact sulfide material with an
arsenic content of 60-70\% that does not require special disposal.

{\bfseries Key words:} metallurgical production, arsenic-containing
materials, leaching, oxidation-sulfiding roasting, acid, autoclave,
bacterial oxidation, arsenates, oxides and sulfides of arsenic.

\sectionheading{ҚҰРАМЫНДА МЫШЬЯК БАР ҚАТТЫ МАТЕРИАЛДАРДЫ ӨҢДЕУ}

\begin{center}
(аналитикалық шолу)

{\bfseries \textsuperscript{1}Н.И. Копылов,
\textsuperscript{2}Х.Б.Омаров\envelope}

\textsuperscript{1}Қатты денелер химиясы және механикохимия институты,
Сібір бөлімшесі Ресей ғылым академиясы, Новосибирск, Ресей,

\textsuperscript{2} Қ.Құлажанов атындағы Қазақ технология және бизнес
университеті, Астана, Қазақстан,

e-mail: homarov1963@mail.ru
\end{center}

Бұл мақалада мышьяк бар қатты материалдарды өңдеудегі ғылыми-техникалық
әзірлемелерге аналитикалық шолу берілген. Металлургиялық өндірістегі
шаңдардан, сублиматтардан және шламдардан мышьяк алудың гидро-,
пирометаллургиялық және аралас әдістері қарастырылған. Табиғатта
аурипигмент және реальгар минералдары түрінде болатын мышьяк сульфидтері
- мышьяк ең тұрақты, улы емес түрінде өңдеудің алдын ала сатысында
процестен шығарылатын технологиялар ең қолайлы болып табылады. Әбішев
атындағы Химия-металлургия институтында (Қарағанды, Қазақстан
Республикасы) және Гидроцветмет институтында (Новосибирск, Ресей
Федерациясы) мышьякты жою бойынша бірқатар кен орындарының өнімдеріне
жүргізілген зерттеулер. шахталық пеште тотықтырғыш-сульфидтендіргіш
күйдіру осы технологияның болашағын көрсетті. Қуырылған өнімдерден
мышьякты кетіру 97-98,5\% құрады, арнайы көмуді қажет етпейтін мышьяк
мөлшері 60-70\% болатын ықшам сульфидті материалды алумен бір мезгілде.

{\bfseries Түйін сөздер:} металлургиялық өндіріс, құрамында мышьяк бар
материалдар, сілтілеу, тотықтырғыш-сульфидтендіргіш күйдіру, қышқыл,
автоклав, бактериялық тотығу, арсенаттар, мышьяк оксидтері және
сульфидтер.

\sectionheading{ПЕРЕРАБОТКА ТВЕРДЫХ МЫШЬЯКСОДЕРЖАЩИХ МАТЕРИАЛОВ}

\begin{center}
(аналитический обзор)

{\bfseries \textsuperscript{1} Н.И. Копылов,
\textsuperscript{2}Х.Б.Омаров\envelope}

Институт химии твердого тела и механохимии Сибирского отделения

Российской Академии наук, Новосибирск, Россия,

Казахский университет технологии и бизнеса имени К.Кулажанова, Астана,
Казахстан,

e-mail: homarov1963@mail.ru
\end{center}

В данной статье приведен аналитический обзор научно-технических
раработок по переработке твердых мышьяксодержащих материалов.
Рассмотрены гидро-, пирометаллургические и комбинированные методы
извлечения мыщьяка из пылей, возгонов, шламов металлургического
производства. Наиболее приемлемыми представляются технологии, по которым
мышьяк на предварительной стадии передела выводится из процесса в
наиболее устойчивой, нетоксичной форме -- сульфидов мышьяка,
существующих в природе в виде минералов аурипигмента и реальгара.
Исследования, проведенные в Химико-металлургическом институте имени
Ж.Абишева (г. Караганда, Республика Казахстан) и институте
«Гидроцветмет» (г. Новосибирск, Российская Федерация) на продуктах
целого ряда месторождений мышьяксодержащего сырья по выводу мышьяка
окислительно-сульфидизирующим обжигом в печи шахтного типа, показали
перспективность данной технологии. Вывод мышьяка из продуктов обжига
составил 97-98,5 \%, одновременно с получением компактного, не
требующего специального захоронения сульфидного материала с содержанием
мышьяка 60-70\%.

{\bfseries Ключевые слова:} металлургическое производство, мышьяксодержащие
материалы, выщелачивание, окислительно-сульфидизирующий обжиг,
кислотное, автоклавное, бактериальное окисление, арсенаты, оксиды и
сульфиды мышьяка.

\begin{multicols}{2}
{\bfseries Introduction.} One of the problematic objects in the production
of heavy non-ferrous metals is arsenic, which is present in various
concentrations in ores and concentrates of non-ferrous and noble metals.
During their processing, arsenic leads to various technological
difficulties that worsen the quality of the base metal and other
by-products. The process is accompanied by environmental pollution with
arsenic, its accumulation in intermediate products, mandatory
neutralization and storage of arsenic-containing materials.

Metallurgical arsenic-containing industrial objects are conventionally
classified into four classes:

- natural raw materials, where arsenic phases are represented by
arsenopyrite, iron arsenides, thioarsenates and arsenides of non-ferrous
metals;

- products of metallurgical oxidation processes, where the form of
arsenic is arsenates, its oxides;

- products of reduction processes, where arsenic is elemental,
arsenides;

- sulfur and phosphorus (chalcogenides) of technical grades, where
arsenic is in the form of sulfides and monophosphide.

Products of arsenic purification of gases (dust, sublimes) and waste
water (cakes, sediments) in addition to arsenic contain significant
amounts of other valuable metals, as shown in Table 1.

The disposal of these materials is a complex technical problem and
includes either their comprehensive processing or safe disposal. The
degree of arsenic sublimation is determined by the composition and
temperature of the slag and matte, the temperature and composition of
the exhaust gases, as well as the partial pressure of oxygen in the gas
phase and a number of other factors {[}1, 2{]}. Most of the arsenic in
the dusts is in the form of arsenates (70\%), approximately 15\% of the
total content is oxides. The zinc and lead content reaches 9.7 and
36.8\%, respectively.

During the processing of copper raw materials and copper-containing
semi-finished products of lead production, lead, arsenic and other
metals are concentrated in converter dusts. Up to 90\% of lead, more
than 90\% of arsenic, up to 95\% of cadmium and more than 65\% of
rhenium and selenium from the amounts contained in the matte pass into
them. Fine converter dusts contain, \%: 42-50 lead, 0.3-0.7 cadmium,
15-28 arsenic, 0.3-0.5 bismuth, 0.2-1.0 selenium and 30-70 g / t
rhenium. The transfer of these dusts to lead production is one of the
main factors in the intra- and inter-plant circulation of arsenic.
\end{multicols}

\begin{table}[H]
\caption*{Table 1-Chemical composition of dust and sublimates from
metallurgical production, \%}
\centering
\begin{tabular}{|p{0.4\textwidth}|l|l|l|l|l|l|l|}
\hline
Arsenic containing materials                                    & As   & Sn   & Pb   & Zn   & Cu   & Fe  & S    \\ \hline
\begin{tabular}[c]{@{}l@{}}Fuming sublimates of tin production\\   1.\\   2.\\  3.\end{tabular} &
  \begin{tabular}[c]{@{}l@{}}7,4\\   1,6\\  2,4\end{tabular} &
  \begin{tabular}[c]{@{}l@{}}40,8\\   13,2\\  66,2\end{tabular} &
  \begin{tabular}[c]{@{}l@{}}13,8\\   11,7\\  0,4\end{tabular} &
  \begin{tabular}[c]{@{}l@{}}4,1\\   47,8\\  2,7\end{tabular} &
  \begin{tabular}[c]{@{}l@{}}0,05\\   0,17\\  0,03\end{tabular} &
  \begin{tabular}[c]{@{}l@{}}0,5\\   -\\  -\end{tabular} &
  \begin{tabular}[c]{@{}l@{}}3,7\\   4,5\\  0,8\end{tabular} \\ \hline
Fuming sublimates of lead-zinc production                       & 0,92 & -    & 12,2 & 58,6 & 0,75 & 0,6 & 1.4  \\ \hline
Dust from electric smelting of tin raw materials                & 1,1  & 27,6 & 1,2  & 28,4 & 0,03 & -   & 1,1  \\ \hline
Polymetallic raw material electric smelting dust                & 2,7  & 12,6 & 14,5 & 34,6 & 0,13 & -   & 5    \\ \hline
Electric smelting dust from tin refining                        & 1,5  & 34,2 & 0,2  & 0,8  & 0,16 & -   & 0,4  \\ \hline
Electric smelting dust from refining of Pb-Sn alloys            & 2,6  & 18,5 & 27,4 & 6,9  & 1,5  & 0,4 & 2,5  \\ \hline
Dust from agglomeration and shaft smelting of lead concentrates & 5,1  & -    & 54,8 & 11,8 & 0,22 & -   & -    \\ \hline
Dust from roasting zinc concentrates                            & 2,1  & -    & 31,7 & 14   & 0,74 & 2,1 & 13,6 \\ \hline
Dust from the smelting of copper sulphide concentrates          & 1    & -    & 8,3  & 7    & 13   & 5,4 & 28,5 \\ \hline
Copper production converter dust                                & 5,9  & -    & 32,2 & 18,9 & 3,3  & -   & -    \\ \hline
\end{tabular}
\end{table}

\begin{multicols}{2}
Copper-containing cakes are mainly added to the smelting furnace charge.
This is the simplest and least expensive method, but it seems the least
rational, since arsenic is returned to the copper smelting production,
accumulates in the electrolyte, thereby worsening the indicators and
ecology of the process.

The variety of methods for processing arsenic-containing dusts,
sublimates, and sludge from metallurgical production can be
conditionally divided into hydro-, pyrometallurgical, and combined
methods.

\emph{Hydrometallurgical methods.} A large set of reagents has been
proposed for extracting arsenic from dusts and sublimates: sulfuric,
hydrochloric and nitric acids, solutions of sodium and ammonium
carbonates, caustic soda, sodium sulfide and others. However, most of
them are not selective: heavy non-ferrous metals also pass into the
solution, which complicates their subsequent processing {[}1{]}.
Nevertheless, there are developments that meet the conditions of
selectivity.

For example, in work {[}2{]}, arsenic from arsenic-containing materials
was transferred into a solution with sodium hydroxide or sulfide.
Experiments have shown that under optimal conditions, the degree of
arsenic extraction into the solution in both cases reaches 95-98\%. In
the case of materials containing arsenic in the form of oxides, good
results are achieved using hot water leaching. Extraction from tin
production roasting furnace dust (44\%
As\textsubscript{2}O\textsubscript{3}) into solution is 93\% As.
Satisfactory selectivity of arsenic extraction (88-93\%) is achieved by
leaching converter dust from lead production with sodium sulfide
solutions (concentration 80-100 g/dm\textsuperscript{3}, 90-95
\textsuperscript{0}C, L:S = 6:1, 1 h). With three-stage leaching,
arsenic extraction reaches 99\%.

Selective extraction of arsenic into an alkaline solution can be
achieved by electrolytic leaching of dusts. Optimum process conditions:
С\textsubscript{NaOH} = 90-100 g/dm\textsuperscript{3}, L:S = (3-4):1.5,
50-75 \textsuperscript{0}C, D\textsubscript{k} = 1000-3000
A/cm\textsuperscript{2}. Lead is released at the cathode in the form of
a spongy sediment. To bind and precipitate lead and zinc from the
solution during leaching of arsenic with strong hot solutions of caustic
soda (100-250 g/dm\textsuperscript{3}), an additional 10-20
g/dm\textsuperscript{3} of sodium sulfide is added to the alkaline
solution {[}3{]}.

It is proposed to leach copper smelting converter dusts with 1.3-5.0\%
As and high contents of non-ferrous metals with sodium sulfide and
precipitate arsenic with various reagents, such as calcium oxide, copper
or iron sulfate, a mixture of phosphoric acid with calcium hydroxide
{[}4-6{]}. The scheme provides for the possibility of further processing
of sediments to obtain antiseptics for wood preservation or their safe
storage. When processing converter dusts with sodium sulfide solutions
(80 g / dm\textsuperscript{3} Na\textsubscript{2}S, 90
\textsuperscript{0}C, L:S = 1:6), 92.9\% of arsenic was extracted into
the solution in the form of thiosalt after 1 hour. With three-stage
leaching - up to 99\%.

For selective and deep (92-96\%) extraction of arsenic from tin
production fuming sublimates, alkaline leaching at a NaOH concentration
of 100-150 g/dm\textsuperscript{3}, L:S = 1:5-10, 80-100
\textsuperscript{0}C, duration of 1-2 h, stoichiometric
Na\textsubscript{2}S consumption for binding lead and zinc into sulfides
is proposed.

The most common method for extracting arsenic from copper-containing
cakes is the alkaline leaching scheme and subsequent precipitation of
calcium arsenate. The cake is treated with an alkali solution (70-90
g/dm\textsuperscript{3} NaOH) at 70-90 \textsuperscript{0}C, a
solid:liquid ratio of 5, and a time of 1-3 hours. Extraction into the
solution is 90-98\% As and no more than 1.0-1.5\% Cu. The
copper-containing residue can be returned to the batch preparation
department of the smelting shop. By introducing lime pulp (with a CaO:As
ratio \textgreater{} 1.5) into the solution at 80-90
\textsuperscript{0}C in 1.5-3 hours, it is possible to precipitate a
product containing 90-96\% As. This achieves alkali regeneration, and
the filtrate is returned to the alkaline leaching stage. As a result,
more than 90-92\% of arsenic passes into the burial product, and the
alkali consumption, taking into account regeneration, is 0.05 kg/kg As.

To extract copper and arsenic from copper electrolyte production
middlings, copper-arsenic cake is leached at a mass ratio of
H\textsubscript{2}SO\textsubscript{4}/Cu = (1.53-1.54):1. The leaching
solution is cooled to 20-80 \textsuperscript{0}C, the copper-containing
product is isolated, which is returned to the main production. The
mother liquor is evaporated to an As concentration of 600-800
g/dm\textsuperscript{3} and used to prepare antiseptic compositions.

Complex processing of lead production dust is complicated by its high
content of harmful impurities. The method of granulation with strong
sulfuric acid and subsequent heat treatment of the resulting granules in
a FB (fluidized bed) furnace allows removing up to 80-85\% of arsenic in
the "head" of the process and then extracting valuable components from
the resulting product {[}7{]}. Sublimates are captured in a wet gas
cleaning system, where the arsenic content in the solution reaches 20-30
g/dm\textsuperscript{3}, and in the sludge up to 40\%. The sludge is
leached and arsenic is precipitated from the combined solution with lime
milk in the form of calcium arsenate (10-12\% As).

There are developments of autoclave technologies for extracting arsenic
from dusts from processing copper and lead concentrates. Depending on
the composition of the dusts, solutions of sulfuric, nitric or
hydrochloric acids and a number of other reagents are used for their
leaching and removal of arsenic.

The process of arsenic dissolution in an alkaline solution can be
combined with the electrolytic extraction of heavy non-ferrous metals on
the cathode. In this case, alkali regeneration occurs, due to which the
equilibrium in the system shifts towards arsenic dissolution.
Alkaline-electrolytic treatment of converter dust ensures almost
complete extraction of arsenic (96-98\%) into the solution and lead
(91\%) into the cathode sponge. Zinc, cadmium, mercury, tellurium are
concentrated in an insoluble precipitate. In this case, there is a risk
of arsine formation. To prevent this during electrolysis, the original
dust is pre-treated with a 4\% NaOH solution. The alkaline solution
containing arsenic is fortified with alkali to 120
g/dm\textsuperscript{3}, cooled to 18-20 \textsuperscript{0}C and, after
separating the sodium arsenate crystals, sent for leaching of a new
portion of dust. Sodium arsenate is dissolved in water at 60-70
\textsuperscript{0}C and calcium arsenate is precipitated from the
resulting solution (50-60 g/dm\textsuperscript{3} As) with lime milk,
which is sent for disposal {[}3{]}.

\emph{Pyrometallurgical and combined methods.} At non-ferrous metallurgy
plants, up to 30\% of arsenic is concentrated in dust and sublimates
{[}8, 9{]}.

When processing sulfide copper concentrates with a low arsenic content
(up to 1\%) in order to obtain arsenic-free gases during smelting, it
was proposed {[}10{]} to roast the concentrate at 750
\textsuperscript{0}C in a mixture with calcium hydroxide in an amount
from 1.5 to 3 times greater than the stoichiometric amount required to
bind arsenic into calcium arsenate. During subsequent smelting, the
resulting calcium arsenate is converted into waste slag.

For relatively arsenic-rich materials (\textgreater5\% As) that do not
contain other subliming components, the main part of the arsenic
(80-90\%) can be distilled off in the form of sublimates containing up
to 90-98\% As\textsubscript{2}O\textsubscript{3} by firing at 550-700
\textsuperscript{0}C in multiple-hearth, muffle or other furnaces. In
order to prevent the formation of higher arsenic oxides and accelerate
the distillation, up to 10\% of fine coal and a sulfidizer (pyrite) are
added to the batch {[}11{]}. Metallurgical sublimates and dusts are
subjected to various types of smelting with concentration of valuable
metals (lead, zinc and others) in the melt, and arsenic - in sublimates
or in melts of rough alloy, mattes, slags. Sodium hydroxide, soda,
sodium sulfate, their mixtures are used as slag- and matte-forming
fluxing components. The process is carried out in a controlled
environment with the introduction of coke into the charge.

When smelting dusts using only sodium sulfate (22-30\% of the dust
mass), 75\% of the arsenic passes into the matte-slag melt and 23\% into
the crude lead. 93\% of the lead is concentrated in the lead melt. When
using a mixture of sodium sulfate and soda (40-45\% of the mixture in
the charge), the arsenic is concentrated in the matte-slag melt, and
with an excess of soda (about 60\%), 75-80\% of the arsenic passes into
the crude lead and only 20\% into the matte-slag melt. The extraction of
lead into the melt, containing 9.3\% As, was 93-94\% of the content in
the original dust. As a result, arsenic can be concentrated in either
the crude lead or the matte-slag melt during the smelting process, if
necessary. Black lead with a high arsenic content can be suitable for
special alloys for various purposes (high-capacity batteries, bearings,
etc.). Arsenic is leached from the stein-slag melt with water and then
precipitated with lime.

Systematic research and technological developments on the removal of
arsenic from metallurgical processes in the form of a non-toxic,
storable product -- sulfide -- were carried out at the Zh. Abishev
Chemical and Metallurgical Institute (Karaganda) {[}12-16{]}. A method
was developed for obtaining arsenic sulfide sublimates from sulfide
materials by roasting them at 650-700 \textsuperscript{0}C in a mixture
with pyrite in a ratio of arsenic to arsenic in the charge equal to
(1.8-2):1, followed by melting the powdered sublimates and obtaining a
compact alloy that is not oxidized in air and insoluble in water.
Methods have also been proposed for converting arsenic dust into sulfide
by sulfiding it with elemental sulfur at a temperature of 325-350
\textsuperscript{0}C, followed by obtaining sublimates by roasting at
700-800 \textsuperscript{0}C, as well as elemental sulfur at 325-350
\textsuperscript{0}C and then leaching with a solution of sodium sulfide
at a ratio of the latter to arsenic (3.4-3.5):1. The resulting arsenic
sulfide is precipitated from the solution with sulfuric acid at pH = 2.

Work on the removal of arsenic into a poorly soluble form during the
processing of refractory gold-arsenic ores and concentrates deserves
special attention. Until recently, about 75\% of gold in Russia was
mined from placers and 25\% from ore raw materials. At the same time, in
the near future, 75\% of the predicted resources and 53\% of the
reserves of Russian gold are concentrated in ore raw materials. At the
same time, gold mining in foreign countries has long been carried out
mainly at the expense of raw materials from primary deposits {[}17{]}.
It follows that the development strategy of this sub-sector will be
entirely determined by the processing of ore raw materials, a
significant part of which is refractory, difficult to process
arsenic-containing ores. The compositions of concentrates from some
deposits in Russia and the CIS are given in Table 2.
\end{multicols}

\begin{table}[H]
\caption*{Table 2. Chemical composition of some gold-arsenic concentrates
of Russia and the CIS, \%}
\centering
\begin{tabular}{|l|l|l|l|l|l|l|l|}
\hline
Deposit        & Au, g/t & Ag, g/t  & As      & Fe        & S         & C         & Sb        \\ \hline
Zodskoe        & 55,4    & 42,5     & 2,08    & 27,09     & 26,15     & Not found & Not found \\ \hline
Kokpatasskoe   & 32,4    & 7,4      & 9,96    & 26,6      & 24,1      & 4,3-11,0  & Not found \\ \hline
Bakyrchikskoe  & 10,8-34 & 15,4     & 5,7-9,6 & 10,0-13,8 & 10-17,8   & 7,6-19,2  & Not found \\ \hline
Nezhdaninskoe  & 21-150  & 120-1300 & 4,8-5,6 & 14,7-19,2 & 15-19,5   & 1,8-7,7   & 0,1       \\ \hline
Olimpiadinskoe & 49-63,1 & 4        & 3,73    & 21,98     & 14,5-20,7 & 0,4-5,6   & 0,1       \\ \hline
Majskoe        & 68,8    & 9,9      & 5,7     & 19,8      & 18,3      & 2,9       & 1,4       \\ \hline
Kjuchusskoe    & 36,5    & 9,8      & 4,6     & 10,1      & 5,56      & 1,38      & Not found \\ \hline
Zarmitan       & 35      & 239      & 16.4    & 30,17     & 29,47     & 0.15      & Not found \\ \hline
\end{tabular}
\end{table}

\begin{multicols}{2}
This type of raw material, in which some of the gold, as studies have
shown, is directly included in the crystal lattice of matrix minerals
(arsenopyrite, pyrite), cannot be processed by traditional methods.
Without preliminary opening of the minerals of the matrix base of
concentrates, gold extraction is possible only in the range from 10-15
\% to 60-70\%, depending on the specific composition of the raw material
{[}18{]}. At gold recovery plants, during the processing of such ore
using the technology of converting arsenopyrite into tailings, all the
bound gold goes into them, and when it is preserved in the concentrate,
arsenic remains in it and enters the technological process. It pollutes
the industrial products and must be removed from the process.

In order to open up the material for further gold extraction and to
remove arsenic from the process, oxidative roasting is widely used using
various types of furnace units and technologies: roasting in
multi-hearth, rotary drum furnaces, roasting in a fluidized bed, and
others.

Other methods of opening refractory ores and concentrates were also
tested and mastered, where oxidation of arsenopyrite and pyrite was
carried out by acid, autoclave, bacterial oxidation, chlorinating
roasting. The use of nitric acid as an oxidizer in acid opening of
refractory gold-arsenic raw materials allows removing arsenic and iron
in scorodite, which is poorly soluble in water. Carbon-containing
arsenopyrite concentrates, rich in silver and gold, are proposed to be
treated with sulfuric acid. In this case, arsenic and iron completely
pass into solution, and precious metals remain in the silicate residue
and are extracted as a free impurity (with a size of at least 50 μg) by
cyanidation.

To intensify the process of acid opening and more complete extraction of
precious metals from refractory sulfide-arsenide concentrates, it is
proposed to carry out the process of hydrothermal decomposition in
autoclaves (T = 120-225 \textsuperscript{0}C, P \textgreater{} 3.5 atm)
{[}19, 20{]} using oxygen as an additional oxidizer. Sulfuric, nitric,
hydrochloric acids, as well as chlorine-containing compounds, such as
CaCl\textsubscript{2}, are used as an acid agent.

Biological methods of decomposition and opening of sulphide and
sulphoarsenide concentrates of heavy non-ferrous and precious metals
belonging to the refractory group have been developed {[}3, 17{]}. This
was preceded by many years of research on obtaining the necessary
strains of bioculture and developing schemes and modes of the technology
of bacterial opening of raw materials. The technology of bacterial
opening has begun to be mastered in a number of gold mining countries:
South Africa, the USA, Canada, Australia, and in Russia by the Polyus
company at the Olimpiadinskoye and Nezhdaninskoye deposits.

A common disadvantage of all the above methods, as well as in
non-ferrous metallurgy in general, is the unsolved environmental
problem. During oxidative roasting, arsenic is distilled off in the form
of trioxide (hazard class 1) with its subsequent conversion to calcium
arsenate (hazard class 2). During acid, autoclave methods of opening and
bioleaching, arsenic is converted into the form of calcium-iron salts of
the scorodite type (hazard class 3). Thus, during bioleaching, the
resulting sediments intended for storage in dumps contain from 8 to 23
\% As in the form of complex salts related to the sulfoscorodite type.
However, the possibility of interaction of these compounds with the
environment and migration of arsenic in the hypergenesis zone with high
dispersion of the dump material and changes in external storage
conditions has been established. In order to prevent the environment
from being contaminated with arsenic from drainage solutions, as well as
from infiltration waters during floods and other natural disasters, it
is necessary to carry out burial in special storage facilities or burial
grounds, the construction of which requires significant expenditures and
the alienation of large areas of land {[}21-23{]}.

Therefore, work is being carried out {[}24{]} aimed at increasing the
stability of arsenic sulfide compounds. Arsenic sulfides obtained during
dearsenizing roasting are, in the main, a finely dispersed material with
a developed surface, which facilitates oxidation and leaching upon
contact with the external environment. Methods have been developed for
rendering harmless such sublimates by melting them into compact blocks
or pressing them into briquettes with subsequent coating with sulfur or
bitumen. These measures allow them to be stored in hazard class 3-4 in
ordinary warehouses and abandoned mines.

{\bfseries Conclusion.} Despite the large number of developments, none of
the alternative technologies to oxidative roasting has surpassed it in
terms of its level of use. For all these technologies, a number of key
problems remain unresolved: the complexity of the equipment, machinery
and process flow charts used; the toxicity of the resulting waste, the
need for expensive disposal and constant environmental monitoring of the
technical condition of temporary dumps and burial grounds and their
impact on the environment.

The most acceptable technologies seem to be those in which arsenic is
removed from the process at the preliminary stage of processing in the
most stable, non-toxic form - arsenic sulfides, which exist in nature as
orpiment and realgar minerals. Research conducted at the Zh. Abishev
Chemical and Metallurgical Institute (Karaganda, Republic of Kazakhstan)
and the Gidrotsvetmet Institute (Novosibirsk, Russian Federation) on
products from a number of arsenic-containing raw material deposits on
the removal of arsenic by oxidizing-sulfidizing roasting in a shaft-type
furnace showed the prospects of this technology. The removal of arsenic
from the roasting products was 97-98.5 \%, simultaneously with the
production of a compact sulfide material with an arsenic content of
60-70\% that does not require special disposal. In the cinder, which
after roasting can be processed either by the traditional method or by
smelting into a collector, the residual concentration of arsenic is less
than 0.15-0.20 \%.
\end{multicols}

\begin{center}
{\bfseries References}
\end{center}

\begin{noparindent}
1. Morales A., Cruells M., Roca A., Bergo R. Treatment of copper flash
smelter flue dusts for copper and zinc extraction and arsenic
stabilization. Hydrometallurgy. 2010. Vol. 105. P. 148-154.

DOI:10.1016/j.hydromet.2010.09.001\\
2. Okanigbe Do, Popoola Api, Adeleke A.A. Characterisation of copper
smelter dust for copper recovery. ProcediaManufact. 2017. Vol. 7. P.
121-126. DOI:10.1016/j.promfg.2016.12.032

3. Nabojchenko S.S., Mamjachikov S.V., Karelov S.V.
Mysh\textquotesingle jak v cvetnoj metallurgii. Ekaterinburg: UrO RAN,
2004. 240 s. ISBN 5-7691-1486-X {[}in Russian{]}

4. Mamjachenkov S.V., Sergeev V.A., Sergeeva Ju.F. Sovremennye sposoby
pererabotki pylej \\medeplavil\textquotesingle nyh predprijatij.
Butlerovskie soobshhenija. 2012. T. 30. № 5. S. 1-19. {[}in Russian{]}

5. Sergeeva Ju.F. Kompleksnaja pererabotka tonkih pylej
medeplavil\textquotesingle nogo proizvodstva OAO «SUMZ». Avtoref. dis.
kand. tehnicheskih nauk. 2013. Ekaterinburg. 23 s. {[}in Russian{]}

6. Omarov Kh.B., Absat Z.B., Aldabergenova S.K., Rahimzhanova N.Zh.,
Muzapparov A.A. Issledovanie processa osazhdenija
mysh\textquotesingle jaka iz mednogo jelektrolita psevdobrukitom.
Izvestija vuzov. Cvetnaja \\metallurgija. 2017. Vyp 6. S. 11-19. DOI:
dx.doi.org/10.17073/0021-3438-2017-6-11-19 {[}in Russian{]}

7. Mamyachenkov S.V., Khanzhin N.A., Anisimova O.S., Karimov K.A.
Extraction of non-ferrous metals and arsenic from fine dusts of copper
smelter production by combined technology.~Izvestiya. Non-Ferrous
Metallurgy. 2021;27(5):25-37. DOI:10.17073/0021-3438-2021-5-25-37

8. Moskalyk RR, Alfantazi AM. Review of copper pyrometallurgical
practice: today and tomorrow. Miner Eng. 2003; 16:893--919.
https://doi.org/10.1016/j.mineng.2003.08.002

9. Dosmukhamedov N, Kaplan V. Efficient removal of arsenic and antimony
during blast furnace smelting of lead-containing materials. JOM. 2017; V
6. P. 381-387. 69:381--7. DOI:10.1007/s11837-016-2152-2

10. Novikov D.O. Fiziko-himicheskoe obosnovanie utilizacii
mysh\textquotesingle jakovistyh kekov medno-cinkovogo proizvodstva.
Avtoref. dis. kand. tehnicheskih nauk. 2022. Ekaterinburg. 25 s. {[}in
Russian{]}.

11. Mihalina E.O. Issledovanie povedenija mysh\textquotesingle jaka,
soderzhashhegosja v tehnogennom i prirodnom syr\textquotesingle e
chernoj metallurgii s cel\textquotesingle ju ocenki vozdejstvija na
okruzhajushhuju sredu. Avtoref. dis. kand. tehnicheskih nauk. 2023.
Moskva. 31 s. {[}in Russian{]}

12. Isabaev S.M. Nauchnye osnovy utilizacii
mysh\textquotesingle jakovistyh tehnogennyh othodov / S.M. Isabaev, H.
Kuzgibekova, T.A. Zikanova, E.V. Zhinova. // Trudy mezhdunarodnogo
kongressa «Fundamental\textquotesingle nye osnovy tehnologij pererabotki
i utilizacii tehnogennyh othodov». 2012. -- Ekaterinburg: OOO «UIPC».
2012 - S. 72-76. ISBN 978-5-4430-0004-6 {[}in Russian{]}

13. Isabaev S.M., Kuzgibekova H. i drugie. Kompleksnaja
gidrometallurgicheskaja pererabotka svincovyh
mysh\textquotesingle jaksoderzhashhih pylej mednogo proizvodstva.
Cvetnye metally. 2017. № 8 (896). S. 33-38. {[}in Russian{]}

14. Isabaev S.M. i drugie. Issledovanie povedenija soedinenij
mysh\textquotesingle jaka v razlichnyh sredah. // \\Mezhdunarodnyj
nauchno-issledovatel\textquotesingle skij zhurnal. 2013. № 8 (15). S.
22-24. URL:

https://research-journal.org/archive/8-15-2013-august/issledovanie-povedeniya-soedinenij-myshyaka\\-v-razlichnyx-sredax
{[}in Russian{]}

15. Isabaev S.M., Pashinkin A.S., Mil\textquotesingle ke Je.G.,
Zhambekov M.I. Fiziko-himicheskie osnovy
sul\textquotesingle fidirovanija mysh\textquotesingle jaksoderzhashhih
soedinenij. Alma-Ata: Nauka. 1986. 183 s. {[}in Russian{]}

16. Isabaev S.M. Sul\textquotesingle fidirovanie
mysh\textquotesingle jaksoderzhashhih soedinenij i razrabotka sposobov
vyvoda mysh\textquotesingle jaka iz koncentratov i promproduktov cvetnoj
metallurgii: Avtoref. dis. doktora tehnicheskih nauk. Irkutsk: ITI.
1991. 39 s. {[}in Russian{]}

17. Sedel\textquotesingle nikova G. V., Savari E. E. i drugie.
Izvlechenie zolota iz upornyh vysokosul\textquotesingle fidnyh
koncentratov s primeneniem biogidrometallurgii. Cvetnye metally. 2012. №
4. S. 31-36. {[}in Russian{]}

18. Kopylov N.I., Kaminskij Ju.D. Mysh\textquotesingle jak. Novosibirsk:
Sib. univer. izd-vo, 2004. 387s. ISBN 5-94087-155-0 {[}in Russian{]}

19. Rusalev R.Je. Gidrometallurgicheskaja tehnologija pererabotki Au-Sb
sul\textquotesingle fidnyh koncentratov \\Olimpiadinskogo mestorozhdenija.
Dissert. kand. tehnicheskih nauk. 2021. Ekaterinburg. 105 s. {[}in
Russian{]}

20. Rahmanov O.B., Aksenov A.V. i drugie. Pererabotka upornogo
zolotosoderzhashhego mysh\textquotesingle jakovistogo koncentrata
mestorozhdenija «Ikkizhelon» s ispol\textquotesingle zovaniem
avtoklavnogo okislenija. Vestnik Irkutskogo gosudarstvennogo
tehnicheskogo universiteta. 2018. T.22. № 8.S. 162-172. {[}in Russian{]}

21. Choong S.Y.,~Chuah T.G.,~Robiah Y., Gregory Koay
F.L.,~Azni~I.~Arsenic toxicity, health hazards and removal techniques
from water: an overview. Desalination. Volume 217, Issues 1--3.

https://doi.org/10.1016/j.desal.2007.01.015

22. Leist M.,~Casey R.J.,~Caridi D. The management of arsenic wastes:
problems and prospects. Journal of Hazardous Materials ,~28 August 2000,
Pages 125-138. https://doi.org/10.1016/S0304-3894(00)00188-6

23. Omarov Kh.B. Mysh\textquotesingle jaksoderzhashhie othody: analiz,
reshenie problem i perspektivy prakticheskogo
ispol\textquotesingle zovanija. Vestnik KarGU. Serija Himija. 2011.
№1(61). S.69-63. {[}in Russian{]}

24. Zhurinov M., Zhumashev K. Netradicionnye metody vyvoda sery i
mysh\textquotesingle jaka iz metallurgicheskogo syr\textquotesingle ja.
Saarbrucken, Deutschland: LAP LAMBERT Academic Publishing, 2015. 110 s.
ISBN 978-3-659-69481-3
\end{noparindent}

\emph{{\bfseries Information about authors}}

\begin{noparindent}
Kopylov N.I. - Doctor of Technical Sciences, Professor, Leading
Researcher, Institute of Solid State Chemistry and Mechanochemistry,
Siberian Branch of the Russian Academy of Sciences, Russia, Novosibirsk,
e-mail: kopylov@narod.ru;

Omarov K.B. - Doctor of Technical Sciences, Professor, Kazakh University
of Technology and Business named after K.Kulazhanov, Republic of
Kazakhstan, Astana, e-mail: homarov1963@mail.ru
\end{noparindent}

\emph{{\bfseries Сведения об авторах}}

\begin{noparindent}
Копылов Н.И.-доктор технических наук, профессор, ведущий научный
сотрудник Института химии твердого тела и механохимии Сибирского
отделения Российской Академии наук, Россия, г. Новосибирск, e-mail:
kopylov@narod.ru;

Омаров Х.Б. - доктор технических наук, профессор, Казахского
университета технологии и бизнеса имени К.Кулажанова, Республика
Казахстан, Астана, e-mail: homarov1963@mail.ru
\end{noparindent}
