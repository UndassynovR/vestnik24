\newpage
IRSTI 87.15.02
\hfill {\bfseries \href{https://doi.org/10.58805/kazutb.v.3.24-450}{https://doi.org/10.58805/kazutb.v.3.24-450}}

\sectionwithauthors{K.A. Kurtibay, Ye.Ye. Zhatkanbayev, A. Kappassuly, A.A. Ussenova Zh.K. Zhatkanbayeva, N.B. Moldagulova, E.B. Moldagulova}{STUDY OF WASTEWATER TREATMENT EFFICIENCY USING A COMBINED METHOD
OF FORWARD AND REVERSE OSMOSIS INTEGRATED WITH ACTIVATED CARBON
TREATMENT}

\begin{center}
{\bfseries \textsuperscript{1}K.A. Kurtibay\envelope, \textsuperscript{2}Ye.Ye. Zhatkanbayev, \textsuperscript{1}A. Kappassuly, \textsuperscript{1}A.A. Ussenova \textsuperscript{3}Zh.K. Zhatkanbayeva, \textsuperscript{1}N.B. Moldagulova, \textsuperscript{1}E.B. Moldagulova}

\textsuperscript{1}«Scientific and Production Center of Ecological and
Industrial Biotechnology» LLP,\\
Astana, Kazakhstan,

\textsuperscript{2}Kazakh University of Technology and Business named
after K. Kulazhanov,\\
Astana, Kazakhstan,

\textsuperscript{3}L.N. Gumilyev Eurasian National University, Astana,
Kazakhstan
\end{center}
\envelope Correspondent-author: kurtibayqb@gmail.com


In Kazakhstan, the issue of providing the population with quality
drinking water remains one of the topical and important issues. Despite
significant efforts in the field of infrastructure and water treatment
technologies, many regions of Kazakhstan still face problems of
pollution and insufficient water resources. In this regard, the
development of effective water treatment methods is of particular
importance. This study considers the issue of wastewater treatment by
the combined method of forward and reverse osmosis, which is a key
direction in modern water treatment. The relevance of the work is the
use of integration of forward osmosis methods with the process of
adsorption by powdered activated carbon in wastewater treatment. The use
of powdered activated carbon in the process of wastewater treatment
leads to a reduction in the content of organic matter, which are water
pollutants. As a result of the experiment with pretreatment of
wastewater with powdered activated carbon, a decrease in the level of
chemical oxygen demand (COD) in wastewater from 538 mg O/dm3 to 256 mg
O/dm3 was observed, as well as a relative decrease in the concentration
of other indicators. The draw solution in the form of 1.5 M NaCl
gradually drawing water molecules from the feed solution pretreated with
activated carbon on the tenth day had a concentration of 0.425 M
absorbing 2.529 liters of water entering through the membrane that is
27.6\% more than the control version of the experiment. The
effectiveness of integrated methods lies in the successful combination
of different technologies to improve water treatment processes and
ensure the availability of clean water not only in Kazakhstan, but also
in different parts of the world.

{\bfseries Key words:} wastewater treatment, forward osmosis, wastewater,
powdered activated carbon, chemical oxygen demand (COD), organic
pollutants, reverse osmosis.

\sectionheading{БЕЛСЕНДІРІЛГЕН КӨМІРМЕН ӨҢДЕУМЕН ИНТЕГРАЦИЯЛАНҒАН ТІКЕЛЕЙ ЖӘНЕ КЕРІ ОСМОС ӘДІСІН ҚОЛДАНА ОТЫРЫП, АҒЫНДЫ СУЛАРДЫ ТАЗАРТУ ТИІМДІЛІГІН ЗЕРТТЕУ}

\begin{center}
{\bfseries \textsuperscript{1}Қ.А. Куртибай\envelope, \textsuperscript{2}Е.Е.
Жатканбаев, \textsuperscript{1}Ә. Қаппасұлы,
\textsuperscript{1}А.Ә.Үсенова, \textsuperscript{3}Ж.К. Жатканбаева, \textsuperscript{1}Н.Б.
Молдагулова, \textsuperscript{1}Э.Б. Молдагулова}

\textsuperscript{1}«Экологиялық және өнеркәсіптік биотехнологияның
ғылыми-өндірістік орталығы» ЖШС,

Астана, Қазақстан,

\textsuperscript{2}Қ. Құлажанов атындағы Қазақ технология және бизнес
университеті,

Астана, Қазақстан,

\textsuperscript{3}Л.Н. Гумилев атындағы Еуразия Ұлттық Университеті,
Астана, Қазақстан,

e-mail: kurtibayqb@gmail.com
\end{center}

Қазақстанда халықты сапалы ауыз сумен қамтамасыз ету мәселесі өзекті
және маңызды мәселелердің бірі болып қала береді. Инфрақұрылым және су
дайындау технологиялары саласындағы елеулі күш-жігерге қарамастан,
Қазақстанның көптеген өңірлері әлі де судың ластануы мен жеткіліксіздігі
проблемаларына тап болып отыр. Осыған байланысты суды тазартудың тиімді
әдістерін жасау ерекше маңызға ие. Бұл зерттеу ағынды суларды тікелей
және кері осмостың аралас әдісімен тазарту мәселесін қарастырады, бұл
қазіргі заманғы суды дайындаудың негізгі бағыты болып табылады. Жұмыстың
өзектілігі ағынды суларды тазарту кезінде ұнтақталған белсендірілген
көмірмен адсорбциялау процесімен тікелей осмос әдістерін интеграциялау
арқылы қолдану болып табылады. Ағынды суларды тазарту процесінде
ұнтақталған белсендірілген көмірді пайдалану суды ластайтын органикалық
заттардың азаюына әкеледі. Ағынды суларды ұнтақталған белсендірілген
көмірмен алдын ала өңдеу экспериментінің нәтижесінде ағынды сулардағы
оттегінің химиялық тұтынылу (ОХТ) деңгейінің 538
мгО/дм\textsuperscript{3}-тен 256 мгО/дм\textsuperscript{3}-ке дейін
төмендеуі, сондай-ақ басқа көрсеткіштер концентрациясының салыстырмалы
төмендеуі байқалады. 1,5 М NaCl түріндегі тартқыш ерітіндісі су
молекулаларын алдын ала белсендірілген көмірмен өңделген бастапқы
ерітіндіден оныншы күні біртіндеп тартып, 0,425 М концентрациясына ие
болып, мембрана арқылы өткен су 2,529 литрді құрады. Бұл эксперименттің
бақылау нұсқасына қарағанда 27,6\% - ға көп. Интеграцияланған әдістердің
тиімділігі суды тазарту процестерін жақсарту және таза судың тек
Қазақстанда ғана емес, әлемнің әр түкпірінде де қолжетімділігін
қамтамасыз ету мақсатында түрлі технологияларды сәтті комбинациялаудан
құралады.

{\bfseries Түйін сөздер:} ағынды суларды тазарту, тікелей осмос, ағынды
сулар, ұнтақталған белсендірілген көмір, оттегінің химиялық тұтынылуы
(ОХТ), органикалық ластаушы заттар, кері осмос.

\sectionheading{ИССЛЕДОВАНИЕ ЭФФЕКТИВНОСТИ ОЧИСТКИ СТОЧНЫХ ВОД С ИСПОЛЬЗОВАНИЕМ
КОМБИНИРОВАННОГО МЕТОДА ПРЯМОГО И ОБРАТНОГО ОСМОСА, ИНТЕГРИРОВАННОГО С
ОБРАБОТКОЙ АКТИВИРОВАННЫМ УГЛЕМ}

\begin{center}
{\bfseries \textsuperscript{1}Қ.А. Куртибай\envelope, \textsuperscript{2}Е.Е.
Жатканбаев, \textsuperscript{1}Ә. Қаппасұлы, \textsuperscript{1}А.Ә.
Үсенова, \textsuperscript{3}Ж.К. Жатканбаева, \textsuperscript{1}Н.Б.
Молдагулова, \textsuperscript{1}Э.Б. Молдагулова}

\textsuperscript{1} ТОО «Научно-производственный центр экологической и
промышленной биотехнологии»,

Астана, Казахстан,

\textsuperscript{2} Казахский университет технологии и бизнеса имени К.
Кулажанова, Астана, Казахстан,

\textsuperscript{3} Евразийский Национальный Университет имени Л.Н.
Гумилёва, Астана, Казахстан,

e-mail: kurtibayqb@gmail.com
\end{center}

В Казахстане вопрос обеспечения населения качественной питьевой водой
остается одним из актуальных и важных. Несмотря на значительные усилия в
области инфраструктуры и технологий водоподготовки, многие регионы
Казахстана по-прежнему сталкиваются с проблемами загрязнения и
недостаточности водных ресурсов. В связи с этим особое значение
приобретает разработка эффективных методов очистки воды. В данном
исследовании рассматривается вопрос очистки сточных вод комбинированным
методом прямого и обратного осмоса, который является ключевым
направлением в современной водоподготовке. Актуальностью работы является
использование интеграции методов прямого осмоса с процессом адсорбции
порошкообразным активированным углем при очистке сточных вод.
Использование порошкообразного активированного угля в процессе очистки
сточных вод приводит к уменьшению содержания органических веществ,
которые являются загрязняющими воду. В результате эксперимента с
предварительной обработкой сточных вод порошкообразным активированным
углем наблюдается снижение уровня химического потребления кислорода
(ХПК) в сточной воде с 538 мг О/дм3 до 256 мг О/дм3, а также
относительное уменьшение концентрации других показателей. Вытяжной
раствор в виде 1,5 М NaCl постепенно вытягивая молекул воды с исходного
раствора предварительно обработанного активированным углем на десятый
день имел концентрацию 0,425 М поглощая 2,529 литров поступающей через
мембрану воды что на 27,6\% больше, чем на контрольном варианте
эксперимента. Эффективность интегрированных методов заключается в
успешном сочетании различных технологий с целью улучшения процессов
очистки воды и обеспечения доступности чистой воды не только в
Казахстане, но и в различных уголках мира.

{\bfseries Ключевые слова:} очистка сточных вод, прямой осмос, сточные
воды, порошкообразный активированный уголь, химическое потребление
кислорода (ХПК), органические загрязнители, обратный осмос.

\begin{multicols}{2}
{\bfseries Introduction.} In the modern world, the problem of access to
clean drinking water remains one of the most urgent, affecting the
health and well-being of mankind. In Kazakhstan, the issue of providing
the population with quality drinking water remains one of the urgent and
important.

The main pollutants in both natural and polluted water are
microorganisms, as well as organic and inorganic substances. Wastewaters
are solid and liquid substances in water that enter the sewer system as
a by-product of society\textquotesingle s activities. They include
dissolved and suspended organic solids that undergo decomposition or
biodegradation {[}1{]}. Over time, domestic and industrial wastewater
increasingly contains non-biodegradable organic compounds such as humic
substances, which reduce the effectiveness of low-cost wastewater
treatment methods {[}2{]}.

Forward osmosis is used to treat municipal wastewater that is treated in
municipal wastewater treatment plants {[}3{]}. Forward osmosis membrane
systems can remove large ions and concentrate wastewater 10-15 times
{[}4{]}. Forward osmosis can be integrated with reverse osmosis in a
hybrid system. In this case, forward osmosis is used to pre-treat
wastewater to produce high quality water, which is then used to dilute
seawater before the reverse osmosis step. Another study {[}5{]} used a
hybrid process combining direct osmosis and membrane distillation to
remove tetracycline from wastewater. This method provided a purification
rate of 99.9\%, and the water recovery efficiency was 15-22\%. In
addition to water recovery, forward osmosis can be used to treat
wastewater to extract nutrients and generate energy {[}6{]}. Examples of
such recovery include biogas production and recovery of valuable
components such as phosphates, ammonia, and potassium. Moreover, forward
osmosis can help to improve the environmental sustainability of
treatment processes by reducing energy costs and decreasing the burden
on the environment {[}6{]}.

As a result, numerous physicochemical treatment methods such as
coagulation, flocculation, adsorption and accelerated oxidation have
been proposed and investigated for effective wastewater treatment
{[}7{]}. However, physicochemical treatment methods alone are not
sufficient to completely remove various organic pollutants from
wastewater. To address these limitations, the use of membrane filtration
processes in combination with these methods has been proposed. Membrane
filtration methods such as nanofiltration (NF) and reverse osmosis (RO)
have been successfully used for wastewater treatment. However, operation
at high transmembrane pressure increases operating costs and leads to
significant organic fouling of the membrane {[}8,9{]}.

From an economic point of view, the forward osmosis (FO) process is
superior to nanofiltration (NF) and reverse osmosis (RO) due to several
advantages. These advantages include efficient removal of organic and
inorganic contaminants, no need for external hydraulic pressure, and
less membrane fouling with better reversibility {[}10{]}. As previously
stated, the application of membrane filtration techniques in integration
with pretreatment by physicochemical methods such as, coagulation,
flocculation, magnetic ion exchange resins and powdered activated carbon
to avoid membrane fouling shows high efficiency {[}11{]}.

Pretreatment of wastewater to remove organic pollutants with powdered
activated carbon has attracted attention due to its advantages in
conjunction with membrane filtration processes. Powdered activated
carbon provides improved membrane surface cleaning, effective reduction
of irreversible fouling and removal of organic matter.

Currently, biochar stands out as a promising alternative with additional
advantages such as environmental sustainability, low production cost,
soil fertility enhancement and carbon sequestration. Biochar also
possesses additional cationic functional groups, which makes it easy to
modify its surface properties to improve functionality {[}12{]}.

The use of adsorption pretreatment for partial removal of organic
pollutants provides reduction of chemical oxygen demand (COD) of
wastewater.

{\bfseries Methods and Materials.} As an object of study for wastewater
treatment by direct osmosis method, incoming wastewater was taken. The
wastewater sample was collected in the volume of 10 liters at the sewage
treatment plant (STP) located in the vicinity of Astana, Kazakhstan.

To determine the efficiency integrated using powdered activated carbon,
experiments were performed without adsorption step and with pretreatment
with powdered activated carbon for adsorption to investigate a larger
flow of clean water. 1.5 M NaCl was used as a draw solution with higher
osmotic pressure than the stock solution. The concentration of the stock
and draw solutions was measured every 24 hours for 10 days.

\emph{Adsorption of wastewater using powdered activated carbon}

For adsorption, untreated wastewater is filtered through filter paper to
remove coarse suspended solids and the filtrate in a total volume of 500
cm\textsuperscript{3} is placed in Erlenmeyer flasks, then 3.5 g/L
powdered activated carbon is added to the filtrate. The flasks were then
placed on a horizontal shaker for intensive-continuous stirring at 150
r/min for 24 hours at room temperature (25-28°C). During the experiment,
the pH of the solution was maintained at 7 by adding 5M HCl or 5M NaOH
every 4 hours. After adsorption, the solution was filtered through a
paper filter with a pore diameter of 0.45 μm and the resulting filtrate
was stored for the following analyses {[}13{]}. After treatment of the
initial sample by adsorption with powdered activated carbon, a general
physicochemical analysis was performed.

\emph{Physico-chemical methods of research}

\emph{Refractometry.} To determine the concentration of sodium chloride
and sucrose solutions, the refractometry method was used using a
refractometer of Abbemat 350/550 Performance Plus series (Anton Paar,
Austria). All analyses were performed in triplicate.

\emph{Determination of chemical oxygen demand (COD).} COD was measured
according to GOST 31859-2012 ``Water. Method for determination of
chemical oxygen demand'' on the Expert 003 device (spectrophotometer
with thermoreactor). Calibration of the device was carried out with
solutions of the standard sample of chemical oxygen consumption GSO
7552-99. The device has a function of built-in construction of
calibration curve and automatic determination of COD value.

\emph{Determination of chloride ions.} Determination of chloride content
was carried out according to GOST 26425-85 ``Method for determination of
chloride in water extract'', the method of determination - titration by
silver nitrate solution with potassium bichromate indicator.

\emph{Determination of ammonium nitrogen.} Determination of ammonium and
nitrate nitrogen according to GOST 15476-2013 ``Fertilizers.
Determination of nitrate and ammonium nitrogen by the Devard method''.

\emph{Determination of nitrates and nitrites.} Determination of nitrates
and nitrites in water was carried out according to GOST 33045-2014
``Water. Methods of determination of nitrogen-containing substances",
the method of determination by photocalorimetry.

\emph{Determination of phosphate ions.} Determination of phosphate
content in water according to GOST 18309-2014 ``Water. Methods of
determination of phosphorus-containing substances''.

\emph{Determination of suspended solids.} Suspended solids were
determined according to PND F 14.1:2:4.254-09 ``Methods of measuring
mass concentrations of suspended and calcined suspended solids in
drinking, natural and waste water samples by gravimetric method''.

\emph{Processing of the obtained results}

The osmotic pressure of solutions with known concentration was
determined by the osmotic pressure (π) Vant-Goff equation:

\begin{equation}
\pi = C(x)RT
\end{equation}

The osmotic pressure gradient of the initial and extraction solutions
was determined by the equation:

\begin{equation}
\mathrm{\Delta}\pi = \ \pi_{DS} - \pi_{FS}
\end{equation}

Water flux is determined by water transport across the semipermeable
membrane due to osmotic pressure difference by equation {[}14{]}:

\begin{equation}
J_{w} = A(\pi_{DS} - \pi_{FS})
\end{equation}

The analysis and processing of the obtained data was carried out through
calculations and equations, and visualizations in the form of graphs and
charts were constructed using Microsoft Excel software (Office 16)

{\bfseries Results and discussion.} Before the beginning of all experiments
the initial general physico-chemical analysis of the selected sample of
incoming wastewater of the sewage treatment plant of Astana city was
carried out. All chemical analyses were carried out in accordance with
the State standards. The results of the general physico-chemical
analysis are given in Table 1.
\end{multicols}

\begin{table}[H]
\caption*{Table 1 - Indicators of physico-chemical analysis of incoming wastewater of the sewage treatment plant of Astana city}
\centering
\begin{tabular}{|l|l|}
\hline
Name of physico-chemical parameters & Results \\ \hline
pH                                  & 7,83    \\ \hline
Temperature                         & 22      \\ \hline
COD, mg O/dm3                       & 538     \\ \hline
Suspended solids, mg/ dm3           & 2584    \\ \hline
Phosphorus (РO4), mg/ dm3           & 3,69    \\ \hline
Ammonium nitrogen, mg/ dm3          & 18,34   \\ \hline
Nitrite nitrogen, mg/ dm3           & 1,51    \\ \hline
Nitrate nitrogen, mg/ dm3           & 8,26    \\ \hline
Chlorides, mg/ dm3                  & 13,2    \\ \hline
Ash content, \%                     & 128     \\ \hline
\end{tabular}
\end{table}

\begin{multicols}{2}
According to the results of the general physico-chemical analysis we can
notice a high level of water pollution by organic substances, which is
estimated by the value of chemical oxygen demand (COD) of the object
under study. In this case, the incoming wastewater of the sewage
treatment plant of Astana city has COD - 538 mg O/dm3. Also relatively
high content of nitrate, nitrite, ammonium, phosphate and chloride ions,
exceeding the maximum permissible concentration (MPC). The hydrogen
index of this wastewater is 7.83, which characterizes the alkalinity of
this sample.

\emph{Adsorption of wastewater using powdered activated carbon}

For adsorption, the untreated wastewater is filtered through filter
paper to remove coarse suspended solids and the filtrate in a total
volume of 500 cm3 is placed in Erlenmeyer flasks, then 3.5 g/L powdered
activated carbon is added to the filtrate. The flasks were then placed
on a horizontal shaker for intensive-continuous stirring at 150 r/min
for 24 hours at room temperature (25-28°C). During the experiment, the
pH of the solution was maintained at 7 by adding 5M HCl or 5M NaOH every
4 hours. After adsorption, the solution was filtered through a paper
filter with a pore diameter of 0.45 μm and the resulting filtrate was
stored for the following analyses {[}13{]}. After treatment of the
initial sample by adsorption with powdered activated carbon, a general
physicochemical analysis was carried out.

The results obtained after adsorption are shown in Table 2.
\end{multicols}

\begin{table}[H]
\caption*{Table 2 - Indicators of general physicochemical analysis after adsorption}
\centering
\begin{tabular}{|l|l|}
\hline
Name of physico-chemical parameters & Results \\ \hline
pH                                  & 7,71    \\ \hline
Temperature                         & 25      \\ \hline
COD, mg O/dm3                       & 256     \\ \hline
Suspended solids, mg/ dm3           & 367     \\ \hline
Phosphorus (РO4), mg/ dm3           & 3,25    \\ \hline
Ammonium nitrogen, mg/ dm3          & 17,67   \\ \hline
Nitrite nitrogen, mg/ dm3           & 1,42    \\ \hline
Nitrate nitrogen, mg/ dm3           & 8,11    \\ \hline
Chlorides, mg/ dm3                  & 12,84   \\ \hline
Ash content, \%                     & 84      \\ \hline
\end{tabular}
\end{table}

\begin{multicols}{2}
According to the results obtained after the analysis with pretreatment
with powdered activated carbon we can observe a decrease in COD level of
wastewater from 538 mg O/dm3 to 256 mg O/dm3. And also, relative
decrease of concentration of other indicators.

\emph{Integration of the adsorption process with a forward osmosis
system}

The forward osmosis system used was a plant that was designed for
desalination of sea salt water. The process of concentrating wastewater
to produce clean secondary water was carried out in a similar way to the
forward osmosis desalination method, since the basic principle of this
method is the same for all types of treatment, recovery and
concentration. The methods differ only in that, depending on the
purpose, chemical composition, concentration and osmotic pressure of the
feed solution, different draw solutions are used, respectively higher
osmotic pressure than the feed solution for greater water flux through
the semipermeable membrane. And also, different membranes of different
nature are applied relative to the purpose of the solution.

To determine the efficiency integrated using powdered activated carbon,
experiments were performed without adsorption step and with pretreatment
with powdered activated carbon for adsorption to investigate the higher
flow of pure water. 1.5 M NaCl was used as a draw solution with higher
osmotic pressure than the feed solution. Concentration measurements of
feed and draw solutions were made every 24 hours for 10 days. The water
flux was measured by evaluating the change in concentration and osmotic
pressure of sodium chloride in the feed and draw solutions, as well as
the change in the volume of solution in the draw solution compartment.
The concentration of sodium chloride was measured by refractometry.

The results of experiments on water treatment by forward osmosis method
without pretreatment and by complex method are given in Tables 3-4.
\end{multicols}

\begin{table}[H]
\caption*{Table 3 - Wastewater treatment by forward osmosis without pretreatment with powdered activated carbon}
\centering
\begin{tabular}{|p{0.2\textwidth}|l|l|l|l|l|l|}
\hline
Time, days                & 0     & 2     & 4     & 6     & 8     & 10    \\ \hline
COD, mg O/dm3             & 538   & 1216  & 1932  & 2367  & 2527  & 2714  \\ \hline
NO33-, mg O/dm3           & 8,26  & 18,66 & 29,96 & 36,7  & 39,18 & 42,08 \\ \hline
NH44+, mg O/dm3           & 18,34 & 41,45 & 66,52 & 81,48 & 86,99 & 93,43 \\ \hline
PO43-, mg O/dm3           & 3,69  & 8,34  & 13,38 & 16,39 & 17,5  & 18,8  \\ \hline
NO22-, mg O/dm3           & 1,51  & 3,41  & 5,47  & 6,7   & 7,16  & 7,69  \\ \hline
Cl-, mg O/dm3             & 13,2  & 29,81 & 47,82 & 58,57 & 62,59 & 67,22 \\ \hline
Concentration of draw solution, mol/L   & 1,5 & 1,252     & 1,128     & 0,985     & 0,912     & 0,884     \\ \hline
Osmotic pressure of draw solution π, Pa & –   & 3,102*103 & 2,795*103 & 2,440*103 & 2,260*103 & 2,143*103 \\ \hline
Gradient of osmotic pressure Δπ, Pa     & –   & 2,054*103 & 1,850*103 & 1,616*103 & 1,497*103 & 1,450*103 \\ \hline
Water flux Jw, L*m-2*h-1  & –     & 20,54 & 18,5  & 16,15 & 14,97 & 14,5  \\ \hline
Volume of clean water, ml & –     & 0,198 & 0,131 & 0,198 & 0,122 & 0,054 \\ \hline
\end{tabular}
\end{table}

\begin{table}[H]
\caption*{Table 4 - Wastewater treatment by integrated forward osmosis method with pretreatment with powdered activated carbon}
\centering
\begin{tabular}{|p{0.2\textwidth}|l|l|l|l|l|l|}
\hline
Time, days                & 0     & 2     & 4      & 6      & 8      & 10     \\ \hline
COD, mg O/dm3             & 256   & 1024  & 1792   & 2612   & 2965   & 3288   \\ \hline
NO33-, mg O/dm3           & 8,11  & 32,44 & 56,77  & 82,75  & 93,9   & 104,16 \\ \hline
NH44+, mg O/dm3           & 17,67 & 70,68 & 123,69 & 180,29 & 204,58 & 226,94 \\ \hline
PO43-, mg O/dm3           & 3,25  & 13    & 22,75  & 33,16  & 37,63  & 41,74  \\ \hline
NO22-, mg O/dm3           & 1,42  & 5,68  & 9,94   & 14,48  & 16,43  & 18,23  \\ \hline
Cl-, mg O/dm3             & 12,84 & 51,36 & 89,88  & 130,93 & 148,56 & 167,84 \\ \hline
Concentration of draw solution, mol/L   & 1,5 & 1,056     & 0,884     & 0,693     & 0,512     & 0,425     \\ \hline
Osmotic pressure of draw solution π, Pa & –   & 2,616*103 & 2,190*103 & 1,717*103 & 1,269*103 & 1,053*103 \\ \hline
Gradient of osmotic pressure Δπ, Pa     & –   & 2,386*103 & 1,998*103 & 1,566*103 & 1,157*103 & 0,961*103 \\ \hline
Water flux Jw, L*m-2*h-1  & –     & 23,86 & 19,98  & 15,66  & 11,57  & 9,61   \\ \hline
Volume of clean water, ml & –     & 0,42  & 0,249  & 0,498  & 0,763  & 0,599  \\ \hline
\end{tabular}
\end{table}

\begin{multicols}{2}
A forward osmosis system without adsorbent pretreatment of the feed
solution was taken into consideration as a control for this experiment,
which allowed us to compare the water flux through the membrane and
membrane fouling potential with the integrated forward osmosis method
with pre-adsorption by powdered activated carbon.

The integrated method with pre-adsorption with powdered activated carbon
resulted in a significant increase in the maximum water flux, which is
observed after 2 days from the beginning of the experiment compared to
the control by 16.17\%. In addition, judging by the results obtained, it
can be seen that after pretreatment of wastewater with activated carbon,
the membrane fouling potential decreases, at the expense of this
wastewater in the compartment for feed water is more quickly
concentrated losing water molecules, and the draw solution is diluted
reducing its osmotic pressure to the onset of osmotic pressure
equilibrium in the two compartments. The draw solution in the form of
1.5 M NaCl gradually drawing water molecules from the feed solution on
day 10 had a concentration of 0.425 M absorbing 2.529 liters of water
entering through the membrane, which is 27.6\% more than the control
version of the experiment.

Concentrated wastewater with a significantly higher content of organic
matter can serve as an alternative substrate for anaerobic digestion
{[}15{]}, most often carried out for biogas production. The high content
of organic matter in concentrated wastewater is evidenced by COD values,
which reached 2714 mg O/dm3 in the control, and in the experiment with
pretreatment due to high water flux concentrated to 3288 mg O/dm3
despite the fact that during adsorption part of the organic matter
remained on the sorbent.

\emph{Pure water recovery by reverse osmosis method}

To obtain clean secondary water for domestic and agricultural use, a
number of methods are used today. One of these methods is the reverse
osmosis method. In comparison with forward osmosis, reverse osmosis has
a number of disadvantages such as high energy consumption, high degree
of membrane fouling, scaling, etc. Therefore, modern researchers have
proposed a hybrid method ``FO-RO'', a method of forward and reverse
osmosis {[}16{]}.

After completion of wastewater treatment by forward osmosis, a dilute
draw solution remains from which pure water must be extracted.
Extraction of pure water by thermal distillation requires a lot of
electrical energy and takes more time. Thus, it was decided to recover
pure water from dilute solutions using «Atoll» reverse osmosis systems
with a pure water flux capacity of 7.5 liters per hour.

{\bfseries Conclusions.} Application of adsorption method by powdered
activated carbon in wastewater treatment allows to reduce the
concentration of organic substances that pollute water. According to the
results of the experiment with pretreatment of wastewater with powdered
activated carbon we can observe a decrease in the level of COD of
wastewater from 538 mgO/dm3 to 256 mgO/dm3. As well as a relative
decrease in the concentration of other indicators.

The use of the integrated method of pre-adsorption of wastewater by
powdered activated carbon led to a significant increase in the maximum
water flux, which is observed after 2 days from the beginning of the
experiment compared to the control by 16.17\%. In addition, judging by
the results obtained, it can be seen that after pretreatment of
wastewater with activated carbon, the membrane fouling potential
decreases, at the expense of this wastewater in the compartment for feed
water is more quickly concentrated losing water molecules, and the draw
solution is diluted reducing its osmotic pressure to the onset of
osmotic pressure equilibrium in the two compartments. The draw solution
in the form of 1.5 M NaCl gradually drawing water molecules from the
feed solution on day 10 had a concentration of 0.425 M absorbing 2.529
liters of water entering through the membrane, which is 27.6\% more than
the control version of the experiment. After completion of wastewater
treatment by forward osmosis method, there remains a dilute draw
solution from which pure water should be extracted. Extraction of pure
water by thermal distillation requires a lot of electricity and takes
more time. Therefore, it was decided to recover pure water from diluted
solutions by reverse osmosis method using reverse osmosis systems
«Atoll», the capacity of pure water flux of 7.5 l/h. Which proves the
effectiveness of the combined method «FO-RO».

Thus, the results obtained at the end of the experiments indicate that
wastewater treatment using the forward osmosis method, as well as the
integration of this process with the activated carbon adsorption method,
represent a cost-effective and efficient approach to obtaining clean
water for reuse in various industries and agriculture. These integrated
methods can successfully combine different technologies to improve water
treatment processes and increase the availability of clean water not
only in Kazakhstan, but also in other regions of the world.
\end{multicols}

\begin{center}
{\bfseries References}
\end{center}


\begin{noparindent}
1.
  Sonune A., Ghate R. Developments in wastewater treatment methods
  //Desalination. -- 2004. -Vol. 167. -P. 55-63. DOI
  10.1016/j.desal.2004.06.113

2.
  Da Costa F. M. et al. Evaluation of the biodegradability and toxicity
  of landfill leachates after pretreatment using advanced oxidative
  processes //Waste management.-2018. -Vol. 76.- P. 606-613. DOI

  10.1016/j.wasman.2018.02.030

3.
  Hey, T., Bajraktari, N., Davidsson, Å., Vogel, J., Madsen, H. T.,
  Hélix-Nielsen, C., ... \& Jönsson, K. Evaluation of direct membrane
  filtration and direct forward osmosis as concepts for compact and
  energy-positive municipal wastewater treatment //Environmental
  technology.- 2018. -Vol. 39(3).- P. 264-276. DOI
  10.1080/09593330.2017.1298677

4.
  Gao, Y., Fang, Z., Liang, P., \& Huang, X. Direct concentration of
  municipal sewage by forward osmosis and membrane fouling behavior
  //Bioresource technology. -2018. -Vol. 247.-P. 730-735. DOI

  10.1016/j.biortech.2017.09.145

5.
  Pan, S. F., Dong, Y., Zheng, Y. M., Zhong, L. B., \& Yuan, Z. H.
  Self-sustained hydrophilic nanofiber thin film composite forward
  osmosis membranes: Preparation, characterization and application for
  simulated antibiotic wastewater treatment //Journal of Membrane
  Science.- 2017. -Vol. 523.- P. 205-215.

6.
  Ansari, A. J., Hai, F. I., Price, W. E., Drewes, J. E., \& Nghiem, L.
  D. Forward osmosis as a platform for resource recovery from municipal
  wastewater-A critical assessment of the literature //Journal of
  membrane science.-2017. -Vol. 529.-P. 195-206. DOI
  10.1016/j.memsci.2017.01.054

7.
  Liu Z. P. et al. Characterization of dissolved organic matter in
  landfill leachate during the combined treatment process of air
  stripping, Fenton, SBR and coagulation //Waste Management. - 2015.
  -Vol. 41. -P. 111-118. DOI 10.1016/j.wasman.2015.03.044

8.
  Oloibiri V. et al. Characterisation of landfill leachate by
  EEM-PARAFAC-SOM during physical-chemical treatment by
  coagulation-flocculation, activated carbon adsorption and ion exchange
  //Chemosphere.-2017. -Vol.186.-P. 873-883. DOI
  10.1016/j.chemosphere.2017.08.035

Šír M. et al. The effect of humic acids on the reverse osmosis treatment
of hazardous landfill leachate //Journal of Hazardous
Materials.-2012.-Vol.207.- P.86-90.

DOI 10.1016/j.jhazmat.2011.08.079

9. Iskander S. M. et al. Energy consumption by forward osmosis treatment
  of landfill leachate for water recovery //Waste Management.-2017.
  -Vol.63.- P. 284-291.

DOI 10.1016/j.wasman.2017.03.026

10.
  Dolar D., Košutić K., Strmecky T. Hybrid processes for treatment of
  landfill leachate: coagulation/UF/NF-RO and adsorption/UF/NF-RO
  //Separation and purification technology. -- 2016.-Vol.168.- P. 39-46.
  DOI 10.1016/j.seppur.2016.05.016

11.
  Rajapaksha A. U. et al. Engineered/designer biochar for contaminant
  removal/immobilization from soil and water: potential and implication
  of biochar modification //Chemosphere. -2016. -Vol.148.-P. 276-291.
  DOI 10.1016/j.chemosphere.2016.01.043

12.
  Aftab B. et al. Targeted removal of organic foulants in landfill
  leachate in forward osmosis system integrated with biochar/activated
  carbon treatment //Water research. -2019. -Vol.160.-P. 217-227. DOI
  10.1016/j.watres.2019.05.076

13.
  Darwish M. A. et al. The forward osmosis and desalination
  //Desalination and Water Treatment. - 2016. -Vol. 57(10).-P.
  4269-4295. DOI 10.1080/19443994.2014.995140

14.
  Chen L. et al. Performance of a submerged anaerobic membrane
  bioreactor with forward osmosis membrane for low-strength wastewater
  treatment //Water research.- 2014.-Vol.50. -P. 114 -123. DOI
  10.1016/j.watres.2013.12.009

15.
  Bamaga O. A. et al. Hybrid FO/RO desalination system: Preliminary
  assessment of osmotic energy recovery and designs of new FO membrane
  module configurations //Desalination. - 2011. -Vol. 268(1-3).- P.
  163-169. DOI 10.1016/j.desal.2010.10.013
\end{noparindent}

\emph{{\bfseries Information about the authors}}

\begin{noparindent}
Kurtibay K.A. - Master\textquotesingle s student, Research Associate of
«Scientific and Production Center of Ecological and Industrial
Biotechnology» LLP, Astana, Kazakhstan, e-mail: kurtibayqb@gmail.com;

Zhatkanbayev Ye.Ye. - Doctor of Technical Sciences, Associate Professor,
Kazakh University of Technology and Business named after K. Kulazhanov,
Astana, Kazakhstan, e-mail: erlan.ntp@mail.ru;

Kappassuly A. - Master of Engineering and Technology, Research
Associate, «Scientific and Production Center of Ecological and
Industrial Biotechnology» LLP, Astana, Kazakhstan, e-mail:
kappasuly@mail.ru;

Ussenova A.A. - Master of Natural Sciences, General Director,
«Scientific and Production Center of Ecological and Industrial
Biotechnology» LLP, Astana, Kazakhstan, e-mail: ussenovaaall@gmail.com;

Zhatkanbayeva Zh.K. - Candidate of Chemical Sciences, Associate
Professor, L.N. Gumilev Eurasian National University, Astana,
Kazakhstan, e-mail: zhanna01011973@mail.ru;

Moldagulova N.B. - Candidate of Veterinary Sciences, Leading Researcher,
«Scientific and Production Center of Ecological and Industrial
Biotechnology» LLP, Astana, Kazakhstan, e-mail: m\_nazira1967@mail.ru;

Moldagulova E.B. - junior researcher, «Scientific and Production Center
of Ecological and Industrial Biotechnology» LLP, Astana, Kazakhstan,
e-mail: molgagulova\_elmira1968@mail.ru.
\end{noparindent}

\emph{{\bfseries Сведения об авторах}}

\begin{noparindent}
Куртибай Қ.А.- магистрант, научный сотрудник ТОО
«Научно-производственный центр экологической и промышленной
биотехнологии», Астана, Казахстан, e-mail: kurtibayqb@gmail.com;

Жатканбаев Е.Е. - д.т.н., ассоциированный профессор, Казахский
университет технологии и бизнеса имени К. Кулажанова, Астана, Казахстан,
e-mail: erlan.ntp@mail.ru;

Қаппасұлы Ә. -магистр техники и технологии, научный сотрудник ТОО
«Научно-производственный центр экологической и промышленной
биотехнологии», Астана, Казахстан, e-mail: kappasuly@mail.ru;

Үсенова А.Ә. - магистр естествознания, генеральный директор ТОО
«Научно-производственный центр экологической и промышленной
биотехнологии», Астана, Казахстан, e-mail: ussenovaaall@gmail.com;

Жатканбаева Ж.К. - к.х.н., доцент, Евразийский Национальный Университет
имени Л.Н. Гумилева, Астана, Казахстан, e-mail: zhanna01011973@mail.ru;

Молдагулова Н.Б. - к.в.н, ведущий научный сотрудник ТОО
«Научно-производственный центр экологической и промышленной
биотехнологии», Астана, Казахстан, e-mail: m\_nazira1967@mail.ru;

Молдагулова Э.Б.-младший научный сотрудник ТОО «Научно-производственный
центр экологической и промышленной биотехнологии», Астана, Казахстан,
e-mail: molgagulova\_elmira1968@mail.ru.
\end{noparindent}
