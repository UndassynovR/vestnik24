\newpage
IRSTI 53.37.13
\hfill {\bfseries \href{https://doi.org/10.58805/kazutb.v.3.24-461}{https://doi.org/10.58805/kazutb.v.3.24-461}}

\sectionwithauthors{Kh.B. Omarov, Zh.T. Nurtai, N.U. Nurgaliev, A.Kh Takirova}{STUDY OF THE BEHAVIOR OF ARSENIC IN COPPER ELECTROLYTE}

\begin{center}
{\bfseries Kh.B. Omarov\textsuperscript{🖂}, Zh.T. Nurtai, N.U.
Nurgaliev\textsuperscript{🖂}, A.Kh Takirova}

K.Kulazhanov named Kazakh University of Technology and Business,

Astana, Kazakhstan,
\end{center}

{\bfseries \textsuperscript{🖂}}Correspondent-author: homarov1963@mail.ru,
nurgaliev\_nao@mail.ru

In industrial wastewater, arsenic is most often present in the trivalent
state. An analysis of existing purification methods shows that in all
cases, with a few exceptions, the most complete removal of arsenic is
observed from solutions in which it is in the pentavalent state. This is
explained by the significantly lower solubility of arsenates of heavy
and alkaline earth metals compared to the corresponding arsenites. The
chemical oxidation of As(III) in the following systems was studied:
As(III)-H\textsubscript{2}SO\textsubscript{4}-H\textsubscript{2}O,
As(III)-MeSO\textsubscript{4}-H\textsubscript{2}O,
As(III)--H\textsubscript{2}SO\textsubscript{4}--MeSO\textsubscript{4}--H\textsubscript{2}O
(where Me is Cu, Ni, Co, Mn, Zn). The influence of temperature, the
amount of transition metal ions, the duration of the experiment, and the
concentration of sulfuric acid on the degree of transition of As(III) to
As(V) was studied. The effect of catalytic oxidation of As(III) in
sulfuric acid solutions was discovered. The maximum degree of oxidation
of trivalent arsenic (55-60\%) is observed when transition metal salts
are introduced into an arsenic solution in the presence of a platinum
plate. It has been established that the nature of the studied transition
metal salts (Cu\textsuperscript{2+}, Ni\textsuperscript{2+},
Co\textsuperscript{2+}, Mn\textsuperscript{2+}, Zn\textsuperscript{2+}),
process duration and temperature do not have a significant effect on the
degree of As(III) conversion. The results obtained largely explain the
predominance of pentavalent forms of arsenic in production solutions of
non-ferrous metallurgy.

{\bfseries Keywords:} copper electrolyte, arsenic, transition metals,
catalytic role, instant oxidation effect, activated oxygen.

\sectionheading{МЫС ЭЛЕКТРОЛИТІНДЕГІ МЫШЬЯКТІҢ КҮЙІН ЗЕРТТЕУ}

\begin{center}
{\bfseries Х.Б. Омаров\textsuperscript{🖂}, Ж.Т. Нұртай, Н.У.
Нургалиев\textsuperscript{🖂}, А.Х. Такирова}

Қ.Құлажанов атындағы Қазақ технология және бизнес университеті, Астана,
Қазақстан

e-mail: homarov1963@mail.ru, nurgaliev\_nao@mail.ru.
\end{center}

Өнеркәсіптік ағынды суларда мышьяк көбінесе үш валентті күйде болады.
Қолданыстағы тазарту әдістерін талдау барлық жағдайларда, бірнеше
ерекшеліктерді қоспағанда, мышьяктың ең толық жойылуы оның бес валентті
күйде болатын ерітінділерден байқалатынын көрсетеді. Бұл ауыр және
сілтілі жер металдарының арсенаттарының сәйкес арсениттермен
салыстырғанда айтарлықтай төмен ерігіштігімен түсіндіріледі. Жүйелердегі
As(III) химиялық тотығуы:
As(III)-H\textsubscript{2}SO\textsubscript{4}-H\textsubscript{2}O,
As(III)-MeSO\textsubscript{4}-H\textsubscript{2}O,
As(III)-H\textsubscript{2}SO\textsubscript{4}-MeSO\textsubscript{4}-H\textsubscript{2}O
(мұндағы Me - Cu, Ni, Co, Mn, Zn). As(III)-тің As(V)-ке өту дәрежесіне
температураның, өтпелі металл иондарының мөлшерінің, тәжірибенің
ұзақтығының және күкірт қышқылының концентрациясының әсері зерттелді.
Күкірт қышқылы ерітінділеріндегі As(III) каталитикалық тотығуының әсері
ашылды. Үш валентті мышьяктың максималды тотығу дәрежесі (55-60\%)
өтпелі металл тұздарын платина пластинасының қатысуымен мышьяк
ерітіндісіне енгізгенде байқалады. Зерттелетін өтпелі металл тұздарының
табиғаты (Cu\textsuperscript{2+}, Ni\textsuperscript{2+},
Co\textsuperscript{2+}, Mn\textsuperscript{2+}, Zn\textsuperscript{2+}),
процестің ұзақтығы мен температурасы As(III) түрлену дәрежесіне
айтарлықтай әсер етпейтіні анықталды. Алынған нәтижелер түсті
металлургияның өндірістік ерітінділерінде мышьяктың бес валентті
түрлерінің басым болуын көптеп түсіндіреді.

{\bfseries Түйін сөздер:} мыс электролиті, мышьяк, өтпелі металдар,
каталитикалық рөл, лезде тотығу эффектісі, белсендірілген оттегі.

\sectionheading{ИССЛЕДОВАНИЕ ПОВЕДЕНИЯ МЫШЬЯКА В МЕДНОМ ЭЛЕКТРОЛИТЕ}

\begin{center}
{\bfseries Х.Б. Омаров\textsuperscript{🖂}, Ж.Т. Нұртай, Н.У.
Нургалиев\textsuperscript{🖂}, А.Х. Такирова}

\textsuperscript{1}Казахский университет технологии и бизнеса имени
К.Кулажанова, Астана, Казахстан

e-mail: homarov1963@mail.ru, nurgaliev\_nao@mail.ru.
\end{center}

В промышленных сточных водах мышьяк присутствует чаще всего в
трехвалентном состоянии. Анализ существующих способов очистки
показывает, что во всех случаях за небольшим исключением, наиболее
полное удаление мышьяка наблюдается из растворов, в которых он находится
в пятивалентном состоянии. Объясняется это значительно меньшей
растворимостью арсенатов тяжелых и щелочноземельных металлов по
сравнению с соответствующими арсенитами. Исследовано химическое
окисление As(III) в системах:
As(III)-H\textsubscript{2}SO\textsubscript{4}-H\textsubscript{2}O,
As(III)-MeSO\textsubscript{4}-H\textsubscript{2}O,
As(III)--H\textsubscript{2}SO\textsubscript{4}--MeSO\textsubscript{4}--H\textsubscript{2}O
(где Me - Cu, Ni, Co, Mn, Zn). Изучено влияние температуры, количества
ионов переходных металлов, продолжительности опыта, концентрации серной
кислоты на степень перехода As(III) в As(V). Обнаружен эффект
каталитического окисления As(III) в сернокислых растворах. Максимальная
степень окисления трехвалентного мышьяка (55-60\%) наблюдается при
введении в мышьяковый раствор солей переходных металлов в присутствии
платиновой пластинки. Установлено, что природа изученных солей
переходных металлов (Cu\textsuperscript{2+}, Ni\textsuperscript{2+},
Co\textsuperscript{2+}, Mn\textsuperscript{2+}, Zn\textsuperscript{2+}),
продолжительность процесса и температура не оказывают существенного
влияния на степень превращения As(III). Полученные результаты во многом
объясняют факт преобладания пятивалентных форм мышьяка в
производственных растворах цветной металлургии.

{\bfseries Ключевые слова:} медный электролит, мышьяк, переходные металлы,
каталитическая роль, эффект мгновенного окисления, активированный
кислород.

\begin{multicols}{2}
{\bfseries Introduction.} In hydrometallurgical methods for producing
non-ferrous metals, the electrorefining process is characterized by the
accumulation of impurities in the electrolyte, one of which is arsenic.
Depending on the type of production, arsenic in aqueous solutions can be
in various forms. In the presence of free sulfide ions in water, arsenic
is present in the form of anions of thiosalts
AsS$^-_2$,
AsS$^{3-}_3$ and
AsS$^{3-}_4$, in other cases - in the form of
oxygen-containing molecules and anions. The form of existence of arsenic
in aqueous solutions depends on its valency {[}1-2{]}.

According to literature data {[}3, 4{]}, during electrorefining of
copper, 60-80\% of arsenic passes from the anode into the electrolyte,
the rest goes into sludge. The distribution of arsenic between the
electrolyte and sludge, as well as the form of its presence in the
solution, depends on the composition of the electrolyte and the
electrolysis mode. For example, the presence of As and Sb of different
valencies in the electrolyte leads to the precipitation of
antimony-arsenic (Sb\textsuperscript{3+}-As\textsuperscript{5+}) and
antimony-arsenic (Sb\textsuperscript{5+}-As\textsuperscript{3+})
precipitation. An increase in the total content of impurities in the
electrolyte leads to an increase in the electrical resistance of the
solution and its viscosity, an increase in electricity consumption, a
decrease in current efficiency and an increase in the content of harmful
impurities in the cathode metal. Therefore, blister copper, as the main
raw material of the copper electrorefining process, determines the
composition of the electrolyte during electrolysis. An increase in the
content of impurities such as nickel, arsenic and antimony leads to the
production of defective copper or copper of low grades.

During the processing of technogenic arsenic-containing materials by
hydrometallurgical methods, a large proportion of arsenic goes into
solution in trivalent form. Practice and research show that arsenic in
copper electrolyte is mainly pentavalent. For effective purification and
precipitation of sparingly soluble iron arsenates, it is necessary to
oxidize arsenic (III) ions to the pentavalent state {[}5{]}. If we take
into account that during electrochemical oxidation, metallic arsenic
forms arsenic trioxide {[}6, 7{]}, then elucidating the reasons for this
fact is an interesting problem.

The relevance of these studies is due to the need to remove arsenic from
the technological process of non-ferrous metal production in an
environmentally safe form, taking into account their subsequent storage
or disposal. Such forms are arsenates, where arsenic is pentavalent.

{\bfseries Materials and methods.} To obtain information about the behavior
of arsenic in a copper electrolyte, the chemical oxidation of As(III)
was studied in the following systems: As
(III)-H\textsubscript{2}SO\textsubscript{4}-H\textsubscript{2}O, As
(III)-MeSO\textsubscript{4}-H\textsubscript{2}O,
As(III)--H\textsubscript{2}SO\textsubscript{4}--MeSO\textsubscript{4}--H\textsubscript{2}O
(where Me - Cu, Ni, Co, Mn, Zn). Trivalent arsenic was introduced into
the studied systems in the form of sodium arsenite (0,011 gEq/l) and
solid trioxide. The influence of temperature, the amount of transition
metal ions, the duration of the experiment, and the concentration of
sulfuric acid on the degree of transition of As(III) to As(V) was
studied. Experiments were carried out with air blowing., Oxidation of
As(III) in solutions, where dissolved oxygen was previously removed by
blowing with an inert gas (argon), was also studied.

The oxidation process was carried out as follows: solutions of sulfuric
acid and transition metal sulfates were introduced into a vessel with a
NaAsO\textsubscript{2} solution, and the contents were stirred with a
magnetic stirrer for a certain time. Air and argon were supplied into
the solution under pressure while stirring.

To avoid the presence of atmospheric oxygen in the reaction zone, the
experiments were carried out under sealed conditions with constant argon
purging. Reagents for this purpose were preliminarily prepared in a box
using transition metal salts that were twice recrystallized and dried in
an argon atmosphere.

Quantitative determination of As(III) was carried out by amperometric
titration.

 We discovered the fact of instantaneous
oxidation of As(III) in small quantities (up to 10\%) when various
amounts of sulfuric acid are added to the working solution. In this
case, the content of sulfuric acid does not change in all experiments,
i.e. the oxidation of arsenic (III) occurs with the participation of
dissolved oxygen, and the fact of the instantaneous transformation of
part of As(III) into As(V) is due to the catalytic nature of the
process. The ability of H\textsubscript{2}SO\textsubscript{4} to
increase the rate of ``spontaneous'' oxidation of arsenic in solutions
is indicated by the author of the work {[}8{]}.

The effect of catalytic oxidation of As(III) is observed in all
experiments when introducing transition metal salts into an arsenic
solution. Moreover, in As(III)-MeSO\textsubscript{4}-H\textsubscript{2}O
systems the degree of conversion reaches 25\%.

It should be noted that in the systems
As(III)-H\textsubscript{2}SO\textsubscript{4}-MeSO\textsubscript{4}-H\textsubscript{2}O
(where Me - Cu, Ni, Co, Mn, Zn), the amount of instantly oxidized
arsenic increases, and no significant dependence on the nature of the
salt (among those studied) is observed (Table 1). The degree of
transition of As(III) to As(V) depends on the concentration of metal
ions and reaches its maximum at a molar ratio of Me:As(III) above
(6$\div$9):1.
Based on data {[}9{]} on the solubility of oxygen in aqueous solutions,
it was determined that about 60-80\% of dissolved oxygen is involved in
the process of catalytic oxidation of trivalent arsenic.

The duration of the process and temperature do not play a significant
role. This is evidenced by the following experimental data: within 6
hours after instant oxidation, the transition of As(III) to As(V) in the
As(III)-H\textsubscript{2}SO\textsubscript{4}-H\textsubscript{2}O system
was about 6-7\%, and in a solution, for example, containing 0,1 gEq/l of
copper sulfate, the degree of conversion at temperatures of
22\textsuperscript{0}C, 40\textsuperscript{0}C and
60\textsuperscript{0}C was 34.5\%, 34.8\% and 36\%, respectively. The
results of experiments on the oxidation of As(III) in the presence of
ions of other transition metals under study are relatively close to
these data.
\end{multicols}

\begin{table}[H]
\caption*{Table 1 - The degree of conversion of As(III) (\%) in a sulfuric
acid medium in the presence of transition metal ions and an adsorbing
surface (Pt plate) when blown with air}
\caption*{($C$\textsuperscript{\begin{tabular}{@{}l@{}}$ucx$ \\ $NaAsO_2$\end{tabular}}
= 0,011 gEq/l, C$^{H_2SO_4}$ = 0,12 gEq/l, t =
60\textsuperscript{0}C, τ = 3 hours)}
\centering
\begin{tabular}{|l|p{0.2\textwidth}|p{0.2\textwidth}|p{0.3\textwidth}|}
\hline
Ме &
  Concentration $C^{Me^{2+}}\cdot 10^{-2}$ gEq/l &
  Effect of catalytic oxidation, \% &
  Oxidation state of As(III) in the presence of the ion $Me^{2+}$, \% \\ \hline
-                   & -   & 4,21 & 29   \\ \hline
\multirow{3}{*}{Cu} & 4,8 & 21,4 & 51,9 \\ \cline{2-4} 
                    & 7,8 & 33,5 & 60,2 \\ \cline{2-4} 
                    & 9,8 & 33,8 & 60,1 \\ \hline
\multirow{3}{*}{Ni} & 4,8 & 32,9 & 49,4 \\ \cline{2-4} 
                    & 7,8 & 34,9 & 58,2 \\ \cline{2-4} 
                    & 9,8 & 34,9 & 57,7 \\ \hline
\multirow{3}{*}{Co} & 4,8 & 33,6 & 54,1 \\ \cline{2-4} 
                    & 7,8 & 31,7 & 58,6 \\ \cline{2-4} 
                    & 9,8 & 32,1 & 58,7 \\ \hline
\multirow{3}{*}{Mn} & 4,8 & 30   & 52,5 \\ \cline{2-4} 
                    & 7,8 & 30,8 & 55,3 \\ \cline{2-4} 
                    & 9,8 & 30   & 55   \\ \hline
\multirow{3}{*}{Zn} & 4,8 & 33,9 & 50,8 \\ \cline{2-4} 
                    & 7,8 & 34,8 & 55,2 \\ \cline{2-4} 
                    & 9,8 & 30,5 & 54,9 \\ \hline
\end{tabular}
\end{table}

\begin{multicols}{2}
Our experiments have shown that blowing air through solutions has such a
weak effect on the oxidation process in the systems under study that
even for 3 or more hours, air bubbling both in the
As(III)-H\textsubscript{2}SO\textsubscript{4}-H\textsubscript{2}O system
and in the system
As(III)-H\textsubscript{2}SO\textsubscript{4}-MeSO\textsubscript{4}-H\textsubscript{2}O
did not lead to significant oxidation of As(III). Varying the
temperature in the range of 20-60\textsuperscript{0}C did not have a
significant effect on the progress of the process.

In experiments with air blowing, the oxidation process of As(III) can be
intensified at the liquid-solid interface due to the sorption of oxygen
on it. In addition, the interface is an integral element of the copper
electrorefining process, which to some extent brings the experimental
conditions closer to production ones. We tested the effect of a platinum
plate immersed in the reaction zone on the oxidation of As(III). This
metal has the ability to adsorb oxygen on its surface and oxidation on
such a surface occurs with the participation of oxygen from platinum
oxides {[}10{]}, according to the scheme:

\begin{equation*}
    AsO_3^{3-}+PtO[O]\rightarrow AsO_4^{3-}+PtO
\end{equation*}

Initially, the
As(III)-H\textsubscript{2}SO\textsubscript{4}-H\textsubscript{2}O system
was studied with air blowing in the presence of a Pt plate. At the same
time, at the initial moment, the oxidation of As(III) is a fairly
intense process, which slows down over time. Thus, at a temperature of
21\textsuperscript{0}C in 2 hours the degree of conversion was 26,7\%,
and in the next 1,5 hours it increased by only 2\%. Increasing the
temperature of the working solution to 60\textsuperscript{0}C allowed us
to speed up the process. In this case, the same amount of As(III) -
26.7\% transforms into As(V) in 0,5 hours, and subsequent oxidation
takes a long time.

Another factor that had a significant impact on the oxidation of As(III)
in the presence of a Pt plate was the introduction of transition metal
ions (Cu, Ni, Co, Mn, Zn) into the working solution, the ability of
which to reversibly bind and activate oxygen is well known {[}11{]}. The
highest degree of transition of As(III) to As(V) was recorded at a
temperature of 60\textsuperscript{0}C. Within 3 hours, the percentage of
conversion of trivalent arsenic reaches 55-60\%, where the effect of
catalytic oxidation is about 30-35\%. These results (Table 1) are
further evidence of the participation of atmospheric oxygen in the
process of catalytic oxidation of As(III) in the presence of some
transition metal ions. The importance of iron, copper and zinc ions in
the oxidation of arsenic was also confirmed in works {[}12-14{]}.

Therefore, if the presence of oxygen in the reaction zone is eliminated,
the oxidation of As(III) should be practically absent. For this purpose,
we conducted experiments with preliminary blowing of argon through a
NaAsO\textsubscript{2} solution and a sulfuric acid solution of
MeSO\textsubscript{4} (where Me is Cu, Ni, Co, Mn, Zn) before mixing
them. The experiments were carried out at a temperature of
60\textsuperscript{0}C and constant bubbling with argon with stirring.
Within 3 hours, we observed a slight decrease in the concentration of
As(III), which was about 5\%. In our opinion, the observed phenomenon is
explained by the fact that blowing argon does not contribute to the
complete removal of oxygen bound in the active complex. In our opinion,
the activation of oxygen by transition metal ions occurs in the process
of hydrolysis of these ions and complex formation. Thus, in an aqueous
solution of CuSO\textsubscript{4}, hydrated ions
{[}Cu(H\textsubscript{2}O)\textsubscript{4}{]}\textsuperscript{2+} are
able to attach an oxygen molecule O\textsubscript{2} as a weak ligand,
promoting its activation. Based on this, solutions of sulfuric acid and
transition metal salts were prepared in an inert environment-box. Purge
of trivalent arsenic solution with argon began 0,5 hour before mixing
the solutions and was carried out throughout the experiment. Under the
conditions of this experiment, the percentage of oxidation was 0,5\%
within 3 hours. Those. Despite a slight decrease in the degree of
As(III) conversion under these conditions, we were not able to
completely eliminate the presence of oxygen in the system.

The catalytic role of transition metal ions in the oxidation of arsenic
is also evidenced by the results of our studies of the behavior of solid
arsenic trioxide in sulfuric acid solutions in the presence of
Cu\textsuperscript{2+}, Ni\textsuperscript{2+}, Co\textsuperscript{2+},
Mn\textsuperscript{2+}, Zn\textsuperscript{2+} ions (Table 2). The
dissolution of As\textsubscript{2}O\textsubscript{3} in sulfuric acid
solutions is a slow process with simultaneous oxidation of
As\textsuperscript{3+} to As\textsuperscript{5+} in the liquid phase. As
can be seen from Table 2, the influence of transition metal ions on the
oxidation of solid arsenic trioxide is very significant. The degree of
transition of As\textsuperscript{3+} to As\textsuperscript{5+} increases
with increasing temperature of the reaction mixture.
\end{multicols}

\begin{table}[H]
\caption*{Table 2 - Degree of conversion of
As\textsubscript{2}O\textsubscript{3} (\%) in a sulfuric acid medium
in the presence of transition metal ions (Me\textsuperscript{2+})
($m^{As_2O_3}$ = 0,30 g, $C^{H_2SO_4}$ = 0,12 gEq/l,
C\emph{\textsubscript{Me2+}} = 0,10 gEq/l, V = 50 ml, τ = 3 hours)
depending on the temperature when blowing air}
\centering
\begin{tabular}{|l|llllll|}
\hline
Temperature, $^o$С & \multicolumn{6}{l|}{Degree of As$_2$O$_3$ conversion in the presence of Me$^{2+}$, \%} \\ \hline
   & \multicolumn{1}{l|}{-}     & \multicolumn{1}{l|}{Сu$^{2+}$} & \multicolumn{1}{l|}{Ni$^{2+}$} & \multicolumn{1}{l|}{Co$^{2+}$} & \multicolumn{1}{l|}{Mn$^{2+}$} & Zn$^{2+}$ \\ \hline
20 & \multicolumn{1}{l|}{0,04}  & \multicolumn{1}{l|}{14,9} & \multicolumn{1}{l|}{13,6} & \multicolumn{1}{l|}{14,1} & \multicolumn{1}{l|}{13,5} & 12,8 \\ \hline
40 & \multicolumn{1}{l|}{0,084} & \multicolumn{1}{l|}{26,4} & \multicolumn{1}{l|}{23}   & \multicolumn{1}{l|}{24,9} & \multicolumn{1}{l|}{24,8} & 21,1 \\ \hline
60 & \multicolumn{1}{l|}{0,12}  & \multicolumn{1}{l|}{58,2} & \multicolumn{1}{l|}{54,9} & \multicolumn{1}{l|}{56,4} & \multicolumn{1}{l|}{56}   & 52,1 \\ \hline
\end{tabular}
\end{table}

\begin{multicols}{2}
{\bfseries Conclusions.} Thus, the stage of the transition of arsenic from
the trivalent to the pentavalent state occurs with the participation of
atmospheric oxygen and occupies a central place in the process of anodic
oxidation of elemental arsenic in sulfuric acid solutions in the
presence of Cu\textsuperscript{2+}, Ni\textsuperscript{2+},
Co\textsuperscript{2+}, Mn\textsuperscript{2+}, Zn\textsuperscript{2+}
ions, and the nature of the studied transition metal salts does not have
a significant effect on the degree of As(III) conversion. This property
of some transition metal ions to catalytically oxidize As(III) was used
by us later in the development of new methods for processing copper
electrolyte with transition metal compounds with the removal of arsenic
in the form of arsenates. In practice, preliminary introduction of
transition metals into arsenic-containing solutions will allow the
arsenic to be maximally converted into the pentavalent state, which will
ensure their effective purification with the removal of arsenic in an
environmentally safe form.
\end{multicols}

\begin{center}
{\bfseries References}
\end{center}

\begin{noparindent}
1. Tolstikov V. P. Vzaimozavisimost\textquotesingle{}
okislitel\textquotesingle no-vosstanovitel\textquotesingle nyh processov
i pH reakcionnoj sredy // Zhurnal obshhej himii. - 1969. -Vyp. 39, -№ 2.
-S. 240-247. {[}in Russian{]}

2. Levin A.I., Nomberg M.I. Jelektroliticheskoe rafinirovanie medi. --
M.: Metallurgizdat, 1963. -219 s. {[}in Russian{]}

3. Bajmakov Ju. V., Zhurin A. I. Jelektroliz v gidrometallurgii. -- M.:
Metallurgija, 1977. -336 s. {[}in Russian{]}

4. Kuznecova T. A., Fedorov V.A. Jelektroliticheskoe rafinirovanie medi
s povyshennym soderzhaniem Sb i As i vyvod ih iz jelektrolita //
Metallurgija cvetnyh metallov. -1974.-№ 4. -S. 174-179. {[}in Russian{]}

5. Tret\textquotesingle jak M.A., Karimov K.A., Nabojchenko S.S.
Avtoklavnoe okislenie ionov mysh\textquotesingle jaka (III) ionami
zheleza (II), (III) // Sovremennye tehnologii proizvodstva cvetnyh
metallov : materialy Mezhdunarodnoj nauchnoj konferencii, posvjashhennoj
80-letiju S. S. Nabojchenko, Ekaterinburg, 24--25 marta 2022 g. -
Ekaterinburg: Izdatel\textquotesingle stvo Ural\textquotesingle skogo
universiteta, 2022. - S. 68-72. http://elar.urfu.ru/handle/10995/110239.
{[}in Russian{]}.

6. Efimov E. A., Erusalimchik I. G. Jelektrohimicheskie processy na
mysh\textquotesingle jakovom jelektrode // \\Jelektrohimija. -1965. -Vyp.
1, -№ 9. -S. 1133- 1137. {[}in Russian{]}

7. Tomilov A. P., Osadchenko I. M., Homutov E. M. Jelektrohimija
mysh\textquotesingle jaka i ego soedinenij. Itogi nauki i tehniki VINITI
AN SSSR // Jelektrohimija. -1979. -№ 14. -S. 168-207. {[}in Russian{]}

8. Uil\textquotesingle jams U.Dzh. Opredelenie anionov. -M.:
Himija,Spravochnik. Per. s angl. --- M.: Himija, 1982 --- 624 s. {[}in
Russian{]}

9. Rabinovich V.A., Havin Z.Ja. Kratkij himicheskij spravochnik.
-Leningrad: Himija, 1978. -392 s. ISBN: 5-7245-0703-X. {[}in Russian{]}

10. Kasenov B.K., Aldabergenov M.K., Pashinkin A.S. Termodinamicheskie
metody v himii i metallurgii. Almaty: Rauan, 1994. -126 s. ISBN
5-625-02445-6. {[}in Russian{]}

11. Basolo F., Pirson R. Mehanizmy neorganicheskih reakcij. M.: Mir,
1971. -592 s. {[}in Russian{]}

12. Perelomov L.V., Perelomova I.V., Levkin N.D. i dr. Adsorbcija i
okislenie soedinenij mysh\textquotesingle jaka mineralami zheleza i v
bio-mineral\textquotesingle nyh sistemah // Izvestija
Tul\textquotesingle skogo gosudarstvennogo universiteta. Estestvennye
nauki. -2012. -Vyp. 3. -S. 231-241. {[}in Russian{]}

13. K.Z. Song, P. C Ke, Z. Y. Liu et. аl. Co-oxidation of arsenic (III)
and iron (II) ions by pressurized oxygen in acidic solutions / // Int.
J. Miner. Metall. Mater. - 2020. - № 27. - Р. 181-189.

https://dx.doi.org/10.1007/s12613-019-1786-9

14. P. Zhang, C. Li, C. Wei et. аl. Effects of zinc and copper ions on
ferric arsenate precipitation in hydrothermal scorodite // J. Cent.
South Univ. Sci. Technol. - 2019. - № 50. - Р. 2645--2655.
\end{noparindent}

\emph{{\bfseries Information about authors}}

\begin{noparindent}
Omarov K.B. - Doctor of Technical Sciences, Professor, Kazakh University
of Technology and Business named after K.Kulazhanov, Astana, Kazakhstan,
e-mail: homarov1963@mail.ru;

Nurtai Zh.T. - PhD, Associate Professor, Kazakh University of Technology
and Business named after K.Kulazhanov, Astana, Kazakhstan, e-mail:
zhadira\_nurtai@mail.ru;

Nurgaliyev N.U. - Candidate of Chemical Sciences, Associate Professor,
Kazakh University of Technology and Business, Astana,, Kazakhstan named
after K.Kulazhanov, e-mail: nurgaliev\_nao@mail.ru;

Takirova A.K. - master\textquotesingle s degree, senior lecturer, Kazakh
University of Technology and Business named after K. Kulazhanov, Astana,
Kazakhstan, e-mail: adem\_1996@mail.ru
\end{noparindent}

\emph{{\bfseries Сведения об авторах}}

\begin{noparindent}
Омаров Х.Б. - доктор технических наук, профессор, профессор, Казахский
университетт технологии и бизнеса имени К.Кулажанова, Астана, Казахстан,
e-mail: homarov1963@mail.ru;

Нұртай Ж.Т. - доктор PhD, ассоциированный профессор, Казахский
университет технологии и бизнеса имени К.Кулажанова, Астана Казахстан, ,
e-mail: zhadira\_nurtai@mail.ru;

Нургалиев Н.У. - кандидат химических наук, доцент, Казахский университет
технологии и бизнеса имени К.Кулажанова, Республика Казахстан, Астана,
e-mail: nurgaliev\_nao@mail.ru;

Такирова A.Х. - магистр, старший преподаватель, Казахский университет
технологии и бизнеса имени К.Кулажанова, Республика Казахстан, Астана,
e-mail: аdem\_1996@mail.ru
\end{noparindent}
