\emph{{\bfseries Пищевая технология}}
\newpage
{\bfseries ҒТАМР 65.63.03}

\sectionwithauthors{Хастаева А.Ж.}{СҮТТІҢ МАЙ ҚЫШҚЫЛДЫ ҚҰРАМЫНА CSN3 ГЕНОТИПІНІҢ ӘСЕРІ}

\begin{center}

{\bfseries Хастаева А.Ж.}

Қ.Құлажанов атындағы қазақ технология және бизнес университеті, Астана қ., Қазақстан,
\end{center}

{\bfseries \textsuperscript{🖂}} Корреспондент-автор: gera\_or@mail.ru

Сүтті сиырларда май қышқылдарының типтік құрамы әртүрлі құрамдағы
қышқылдармен ұсыныл-ған, олардың шамамен 70\% - ы қаныққан май қышқылдары
(SFA), 25\% - ы моноқанықпаған май қышқылдары (MUFA) және 5\% - ы
полиқанықпаған май қышқылдары (PUFA), бұл адам денсаулығы үшін май
қышқылдарының идеалды профилінен айтарлықтай ерекшеленеді. Май
қышқылдарының метил эфирлерін талдау Shimadzu GC 2010 Plus газ
хроматографын жалын-иондау детекторымен (ПИД), сондай-ақ «CPSil 88 for
FAME» (Agilent Technologies) ұзындығы 100 м, ішкі диаметрі 0.25 мм,
жылжымалы емес фазалық пленка қалыңдығы 0,20 мкм капиллярлық бағанымен
пайдалана отырып жүргізілді. Каппа-казеин генінің полиморфизмін анықтау
және каппа-казеин генотиптері әртүрлі жануарлардың экономикалық пайдалы
қасиеттерін бағалау үшін барлығы 60 сиыр таңдалды, оның ішінде 20
голштин сиыры, 20 алатау сиыры және 20 қара-ала сиыр. Каппа-казеин
гендерінің полиморфизмі ПТР талдауы арқылы бағаланды. Әрі қарай зерттеу
үшін аналогтар принципі бойын-ша каппа-казеин генін генотиптеу нәтижелері
бойынша әр топта сиырлардың 3 кіші тобы құрылды. Бірінші топқа АА
каппа-казеин генотипі бар сиырлар, екіншісі-АВ генотипі, үшіншісі-ВВ
генотипі кірді.

{\bfseries Түйін сөздер:} каппа-казеин, полимеразды тізбекті реакция, сүт,
май қышқылдары, сиыр тұқымы.

\sectionheading{ВЛИЯНИЕ ГЕНОТИПА CSN3 НА СОДЕРЖАНИЕ ЖИРНЫХ КИСЛОТ В МОЛОКЕ}
\begin{center}

{\bfseries Хастаева А.Ж.}

Казахский университет технологии и бизнеса имени К.Кулажанова, г.Астана,
Казахстан,

e-mail: gera\_or@mail.ru
\end{center}

У молочных коров типичный состав жирных кислот представлен кислотами
разного состава, около 70\% из них составляют насыщенные жирные кислоты
(SFA), 25\% мононенасыщенные жирные кисло-ты (MUFA) и 5\%
полиненасыщенные жирные кислоты (PUFA), что значительно отличается от
идеаль-ного профиля жирных кислот для здоровья человека. Анализ метиловых
эфиров жирных кислот проводили с использованием газового хроматографа
Shimadzu GC 2010 Plus с пламенно-ионизацион-ным детектором (ПИД), также с
капиллярной колонкой «CPSil 88 for FAME» (Agilent Technologies) длиной
100 м, внутренним диаметром 0.25 мм, толщиной пленки не подвижной фазы
0.20 мкм. Для определения полиморфизма генов каппа-казеина и оценки
хозяйственно-полезных признаков у животных с разными генотипами
каппа-казеина, было отобрано всего 60 коров, из них 20 коров
голштинской, 20 коров алатауской и 20 коров черно-пестрой породы. Оценку
полиморфизма генов каппа-казеина проводили методом ПЦР анализа. Для
дальнейшего исследования согласно принципу аналогов по результатам
генотипирования по гену каппа-казеина были сформированы в каждой группе
по 3 подгруппы коров. В первую группу были включены коровы с генотипом
каппа-казеина АА, во вторую -- генотипом АВ, в третью -- генотипом ВВ.

{\bfseries Ключевые слова:} каппа-казеин, полимеразная цепная реакция,
молоко, жирные кислоты, порода.

\sectionheading{THE EFFECT OF THE CSN3 GENOTYPE ON THE CONTENT OF FATTY ACIDS IN
MILK}
\begin{center}

{\bfseries Khastayeva A.Zh.}

K.Kulazhanov Kazakh University of Technology and Business, Astana,
Kazakhstan,

e-mail: gera\_or@mail.ru
\end{center}

In dairy cows, acids of different compositions represent the typical
composition of fatty acids, about 70\% of them are saturated fatty acids
(SFA), 25\% monounsaturated fatty acids (MUFA) and 5\% polyunsatu-rated
fatty acids (PUFA), which significantly differs from the ideal profile
of fatty acids for human health. The analysis of methyl esters of fatty
acids was carried out using a Shimadzu GC 2010 Plus gas chromatograph
with a flame ionization detector (PID), also with a capillary column
"CPSil 88 for FAME" (Agilent Technologies) with a length of 100 m, an
inner diameter of 0.25 mm, and a film thickness of 0.20 microns of
non-mobile phase.To determine the polymorphism of kappa-casein genes and
evaluate economically useful traits in animals with different genotypes
of kappa-casein, only 60 cows were selected, including 20 Holstein cows,
20 Alatau cows and 20 black-and-white cows. The polymorphism of
kappa-casein genes was evaluated by PCR analysis. For further research,
3 subgroups of cows were formed in each group according to the principle
of analogues, the results of genotyping and kappa-casein gene. The first
group included cows with the AA kappa-casein genotype, the second -- the
AB genotype, and the third -- the BB genotype.

{\bfseries Key words:} kappa-casein, polymerase chain reaction, milk, fatty
acids, breed.

{\bfseries Кіріспе.} Сүт майының май қышқылдық құрамы сүттің тағамдық және
биологиялық құндылығына және технологиялық қасиеттеріне айтарлықтай әсер
етеді {[}1{]}. Сүтте май қышқылдары бірдей дерлік екі көзден түзіледі --
азықтық және сиыр қарынындағы микробтық белсенділік {[}2 - 7{]}. Май
фазасына сонымен қатар фосфолипидтер (фосфатидхолин, сфингомиелин және
т.б.), гликолипидтер (цериброзидтер), стеролдар және олардың күрделі
эфирлері жатады {[}8{]}. Сүт майының құрамы қандағы холестерин
деңгейінің жоғарылау қаупімен, жүрек ауруының дамуымен, салмақтың
жоғарылауымен және семіздікпен байланысты SFA-ның жоғары болуына
байланысты жиі сынға ұшырайды {[}9 - 14{]}. Керісінше, MUFA-лар
холестеринді төмендететін қасиеттеріне байланысты адам денсаулығына
пайдалы әсер етеді деп саналады {[}15{]}. Профилактикалық және емдік
қасиеттері бойынша, әсіресе жүрек-қан тамырлары ауруларының алдын алу
үшін ὠ-3 және ὠ-6 май қышқылдары құнды болып саналады {[}16 -- 19{]}.
Қаныққан майдың мөлшері өте маңызды көрсеткіш болып табылады, өйткені
теңгерімсіз тұтыну жүрек-қан тамырлары ауруларының қаупінің
жоғарылауымен байланысты {[}20{]}. Липидті компоненттің қаныққан төмен
молекулалы ұшпа май қышқылдары C4:0-ден C8:0-ге дейін тек сүт майында
болады. Олар сүт пен сүт өнімінің дәмі мен иісін қамтамасыз етеді.
Колонокарцинома ингибиторы болып табылатын май қышқылы үлкен маңызға ие
{[}21{]}. Май қышқылдарының құрамының өзгергіштігінің көп бөлігі
генетикалық жолмен анықталады {[}22{]}. Мал шаруашылығына ДНҚ
технологиясын енгізу жануарлардағы экономикалық пайдалы белгілерді
бақылауға және болжауға мүмкіндік береді, бұл әрбір жануардың қосымша
пайдаланылуын анықтауда өте маңызды {[}23{]}.

Каппа казеин гені (CSN3) сүт ақуыздарымен және оның ұю қасиеттерімен
байланысты. В аллелі ақуыздар мен майлардың жоғары пайызымен, сондай-ақ
ірімшік өндірудің оңтайлы сипаттамаларымен байланысты ең құнды аллельдер
ретінде

аталған. А аллелі негізінен сүт, май және ақуыз өндірісіне оң әсер етеді
{[}24, 25{]}. Авторлар әр түрлі сиыр тұқымдарын зерттеуде А аллелінің В
аллеліне үстемдігін сипаттаған {[}26, 27{]}.

{\bfseries Материалдар мен зерттеу әдістері.} Полиморфизмді Каппа-казеин
генімен анықтау Қазақстан--Жапон инновациялық Орталығының «Жасыл
биотехнология және жасушалық инженерия» зертханасының қызметкерлерімен
бірлесіп, CSN3 генімен генотиптеу үшін ПТР әдісін қолдана отырып
жүргізілді. Каппа казеин генін зерттеу және бағалау үшін сыналған
жануарлардан қан үлгілері мен ДНҚ үлгілері алынды. ДНҚ экстракциясы
өндірушінің нұсқауларына сәйкес Pure Link Genomic DNA Mini Kit
(Invitrogen by Thermo Fisher Scientific, АҚШ) арқылы жүзеге асырылды.
Күшейту Master Cycler Nexus анықтау циклінің көмегімен жүзеге асырылды
(Eppendorf AG, Германия). ПТР қоспасы келесі құрамда пайдаланылды:
зерттелетін геннің аймағын күшейту үшін праймер жұбы,
нуклеозидтрифосфаттардың қоспасы (2/5 мМ), магний хлориді (25 мМ), ПТР
үшін 10 еселік буфер, Так полимераза {[}28, 29{]}. CSN3 генінің
фрагменттерін күшейту үшін {[}30{]} олигонуклеотидтердің келесі жұптары
пайдаланылды:

-- алға праймер
5\textquotesingle-ATAGCCAAATATATCCCAATTCAGT-3\textquotesingle;

-- кері праймер 5\textquotesingle-TTTATTAATAAGTCCATGAATCTTG-3.

Осы праймерлермен амплификаттау келесі бағдарлама бойынша жүргізілді:
CSN3 генінің фрагменті үшін бірінші цикл 95 °C, 5 мин; кейінгі 35 цикл:
денатурация -- 95 °C температурада 30 с, күйдіру -- 63 °C температурада
50 с, синтез -- 72 °C температурада 30 с; ұзарту -- 72 °C температурада
5 мин. Алынған ампликондар өндірушінің ұсыныстарына сәйкес рестриктаза
ферменттеріне -- Hinf I (CSN3 гені) шектеу ферменттеріне ұшырады.
Рестрикциядан кейін ампликон фрагменттері 2,5\% агарозды гельде көлденең
электрофорезге ұшырады. Фрагменттерді бояу және визуализациялау үшін
электрофорезден кейін агарозды гельдер 0,005\% этидий бромиді
ерітіндісінде 15 минут бойы ұсталды және GelDoc жүйесі (Bio-Rad, АҚШ)
арқылы түсірілді. Фрагменттердің молекулалық салмақтары ампликон
фрагменттерімен параллель орындалатын молекулалық салмақ стандарттарының
«баспалдақтары» арқылы анықталды.

Харди-Вайнберг заңы {[}31{]} бойынша зерттелетін популяциядағы генотип
жиіліктерінің күтілетін нәтижелері есептелді.

Май қышқылдарының метил эфирлерін талдау Қазақстан--Жапон инновациялық
орталығында Shimadzu GC 2010 plus газ хроматографын жалын-иондау
детекторымен (ПИД), сондай-ақ «CPSil 88 for FAME» (Agilent Technologies)
ұзындығы 100 м, ішкі диаметрі 0.25 мм, жылжымалы емес фазалық пленка
қалыңдығы 0,20 мкм капиллярлық бағанымен пайдалана отырып жүргізілді.
Сүт майының май қышқылының құрамын анықтау ішкі қалыпқа келтіру әдісін
қолдануға негізделген -- қоспаның құрамдас бөліктерінің құрамын анықтау
әдісі, онда кез-келген параметрлердің қосындысы, барлық шыңдардың
аудандарының қосындысы 100\% деп қабылданады, содан кейін жеке шың
ауданының көбейту аудандарының қосындысына қатынасы және 100-ге қоспаның
құрамдас бөліктерінің массалық үлесін (\%) сипаттайды. Бұл әдіс
талданатын компоненттердің аудандарының олардың концентрациясына
әдеттегі градуирлеу тәуелділігін құруды қажет етпейді. Алайда,
хроматографиялық жүйені бітіру шыңдарды одан әрі дұрыс анықтау үшін май
қышқылдарының метил эфирлерінің уақытын бағалау үшін қажет.
Хроматография буландырғыштың температурасы 250°С, детектордың
температурасы 260°С болған кезде жүргізілді. Тасымалдаушы газ (жылжымалы
фаза)-азот, тұтыну 95.5 мл/мин. Хроматографқа ағынның бөлінуі 1:40
болатын 1 мкл сынама енгізілді. Май қышқылдарының метил эфирлерін
толығымен бөлу үшін температураны бағдарламалаумен арнайы бөлу режимі
таңдалды (талдаудың жалпы уақыты -- 68.5 мин):

- бағанның бастапқы температурасы 100°С - 5 минут ішінде;

- 27,5 мин ішінде 4°С/мин жылдамдықпен температураның 210°С дейін
градиенттік өсуі;

- 8 минут ішінде 210°С температурада изотермиялық бөлігі.

- 3 минут ішінде 10°С/мин жылдамдықпен температураның 240 °С дейін
градиентті жоғарылауы;

- 25 минут ішінде 240°С температурада изотермиялық учаске.

Калибрлеу май қышқылдарының 37 метил эфирлері қоспасының стандартты
үлгісін қолдана отырып жүргізілді {[}32{]}. Май қышқылының құрамын
сынама дайындау және анықтау МЕМСТ 32915-2014 «Сүт және сүт өнімдері.
Газ хроматографиясы арқылы май фазасының май қышқылының құрамын анықтау»
сәйкес жүзеге асырылды.

Сүт липидтерінің биологиялық тиімділік коэффициенті полиқанықпаған май
қышқылдарының жалпы санының қаныққан май қышқылдарының жалпы санына
қатынасы ретінде анықталды {[}33{]}:

\begin{equation}
  БЭ = \frac{\sum PUFA}{\sum NFA}
\end{equation}

\(БЭ = \frac{\sum_{}^{}{PUFA}}{\sum_{}^{}{NFA}}\), (1)

мұндағы: БЭ -- биологиялық тиімділік коэффициенті, Σ PUFA --
липидтердегі полиқанықпаған май қышқылдарының жалпы мөлшері,\%, Σ NFA --
қаныққан май қышқылдарының жалпы мөлшері,\%.

Тұтынылатын тағамның май қышқылдық құрамын сипаттайтын көрсеткіштерге
атерогендік, тромбогендік және денсаулық индекстері жатқызылуы мүмкін.
Атерогендік индексі (ИА) -- қаныққан және қанықпаған май қышқылдарының
қосындысы арасындағы қатынасты көрсететін көрсеткіш {[}34{]}. Бұл индекс
қаныққан және қанықпаған май қышқылдарының арақатынасынан есептеледі.
Тромбогендік индексі (ИТ) -- протромбогендік (қаныққан май қышқылдары)
және антитромбогендік (моно және полиқанықпаған май қышқылдары)
арақатынасымен анықталатын қан тамырларындағы тромбоз тенденциясын
сипаттайтын көрсеткіш. Сонымен, атерогендік және тромбогендік индексі
жоғары қаныққан майлар мен өнімдерді шамадан тыс тұтыну атеросклероздың
даму қаупін едәуір арттырады, нәтижесінде инфаркт пен инсульт, сондай-ақ
жүректің ишемиялық ауруы туындайды {[}35, 36{]}.

Атерогендік индексі Уилбрихт пен Саутгейт формуласы бойынша есептелді
{[}37{]}.

\(ИА = \frac{С12:0 + 4 \times С14:0 + С16:0}{\sum_{}^{}{МНЖК + \sum_{}^{}{ПНЖК}}}\)
(2)

Тромбогендік индексі Уилбрихт пен Саутгейт формуласымен есептелді

\(ИТ = \frac{С14:0 + С16:0 + С18:0}{0.5 \times МНЖК + 0.5 \times ПНЖК - \omega 6 + 0.5ПНЖК - \omega 3 + ПНЖК - \frac{\omega 3}{ПНЖК} - \omega 6}\)
(3)

Денсаулық индексі келесі формула бойынша есептелді:

\(ИЗ = \frac{\sum_{}^{}{МНЖК + \sum_{}^{}{ПНЖК}}}{С12:0 + 4 \times С14:0 + С16:0}\)
(4)

Денсаулық индексі (ИЗ) - полиқанықпаған және моноқанықпаған май
қышқылдарының қосындысының қаныққан май қышқылдарына қатынасын көрсетеді
{[}38{]}. Әдетте, соя және зәйтүн сияқты өсімдік майлары денсаулықтың ең
үлкен индексіне ие-7-ден жоғары, ал PUFA және MUFA төмен құрамымен
сипатталатын жануарлар майларының индексі 2-ден аз. CSN3 генотипіне
байланысты сүттің сапалық көрсеткіштері мен биологиялық құндылығын
зерттеу бойынша ғылыми зерттеу жүргізу үшін сиырлардың әр тобында үш
кіші топ құрылды.

{\bfseries Нәтижелер және талқылау.} 1-кестеге сәйкес зерттелген улгілерде
каппа казеинінің AA, AB және BB үш генотипінің болуын көрсетті. Голштин
тұқымдас сиырларда А аллелінің пайда болу жиілігі 0,65, ал В аллелі 0,35
болды. АА генотипі үшін пайда болу жиілігі 45\%, АВ үшін -- 40\% және BB
-- 15\% құрады. Ал қара-ала тұқымды жануарларда А аллелінің пайда болу
жиілігі 0,78, ал В аллелі 0,22 болды. АА генотипі үшін пайда болу
жиілігі 60\%, АВ үшін -- 35\% және BB -- 5\% құрады. Ал Алатау тұқымды
сиырларда А аллелінің кездесу жиілігі 0,75, ал В аллелі 0,25 болды. АА
генотипі үшін пайда болу жиілігі 55\%, АВ үшін -- 40\% және BB -- 5\%
құрады. Барлық үш тұқымның сиырларында каппа-казеин генінің AA, AB және
BB генотиптері мен аллельдерінің кездесу жиілігі 1-кестеде барынша анық
көрсетілген {[}39{]}.

{\bfseries 1-кесте -- Зерттелген сиырлардағы каппа-казеин генінің
генотиптері мен аллельдерінің}

{\bfseries кездесу жиілігі}

\begin{longtable}[]{@{}
  >{\raggedright\arraybackslash}p{(\columnwidth - 20\tabcolsep) * \real{0.2987}}
  >{\raggedright\arraybackslash}p{(\columnwidth - 20\tabcolsep) * \real{0.0580}}
  >{\raggedright\arraybackslash}p{(\columnwidth - 20\tabcolsep) * \real{0.0720}}
  >{\raggedright\arraybackslash}p{(\columnwidth - 20\tabcolsep) * \real{0.0688}}
  >{\raggedright\arraybackslash}p{(\columnwidth - 20\tabcolsep) * \real{0.0758}}
  >{\raggedright\arraybackslash}p{(\columnwidth - 20\tabcolsep) * \real{0.0682}}
  >{\raggedright\arraybackslash}p{(\columnwidth - 20\tabcolsep) * \real{0.0755}}
  >{\raggedright\arraybackslash}p{(\columnwidth - 20\tabcolsep) * \real{0.0606}}
  >{\raggedright\arraybackslash}p{(\columnwidth - 20\tabcolsep) * \real{0.0562}}
  >{\raggedright\arraybackslash}p{(\columnwidth - 20\tabcolsep) * \real{0.0833}}
  >{\raggedright\arraybackslash}p{(\columnwidth - 20\tabcolsep) * \real{0.0830}}@{}}
\toprule\noalign{}
\multicolumn{2}{@{}>{\raggedright\arraybackslash}p{(\columnwidth - 20\tabcolsep) * \real{0.3567} + 2\tabcolsep}}{%
\multirow{3}{=}{\begin{minipage}[b]{\linewidth}\raggedright
Сиыр тұқымы
\end{minipage}}} &
\multirow{3}{=}{\begin{minipage}[b]{\linewidth}\raggedright
n
\end{minipage}} &
\multicolumn{6}{>{\raggedright\arraybackslash}p{(\columnwidth - 20\tabcolsep) * \real{0.4050} + 10\tabcolsep}}{%
\begin{minipage}[b]{\linewidth}\raggedright
Генотип жиілігі
\end{minipage}} &
\multicolumn{2}{>{\raggedright\arraybackslash}p{(\columnwidth - 20\tabcolsep) * \real{0.1663} + 2\tabcolsep}@{}}{%
\begin{minipage}[b]{\linewidth}\raggedright
Аллель жиілігі
\end{minipage}} \\
& & &
\multicolumn{2}{>{\raggedright\arraybackslash}p{(\columnwidth - 20\tabcolsep) * \real{0.1446} + 2\tabcolsep}}{%
\begin{minipage}[b]{\linewidth}\raggedright
АА
\end{minipage}} &
\multicolumn{2}{>{\raggedright\arraybackslash}p{(\columnwidth - 20\tabcolsep) * \real{0.1437} + 2\tabcolsep}}{%
\begin{minipage}[b]{\linewidth}\raggedright
АВ
\end{minipage}} &
\multicolumn{2}{>{\raggedright\arraybackslash}p{(\columnwidth - 20\tabcolsep) * \real{0.1167} + 2\tabcolsep}}{%
\begin{minipage}[b]{\linewidth}\raggedright
ВВ
\end{minipage}} &
\multirow{2}{=}{\begin{minipage}[b]{\linewidth}\raggedright
А
\end{minipage}} &
\multirow{2}{=}{\begin{minipage}[b]{\linewidth}\raggedright
В
\end{minipage}} \\
& & & \begin{minipage}[b]{\linewidth}\raggedright
n
\end{minipage} & \begin{minipage}[b]{\linewidth}\raggedright
\%
\end{minipage} & \begin{minipage}[b]{\linewidth}\raggedright
n
\end{minipage} & \begin{minipage}[b]{\linewidth}\raggedright
\%
\end{minipage} & \begin{minipage}[b]{\linewidth}\raggedright
n
\end{minipage} & \begin{minipage}[b]{\linewidth}\raggedright
\%
\end{minipage} \\
\midrule\noalign{}
\endhead
\bottomrule\noalign{}
\endlastfoot
\multirow{2}{=}{Голштин} & Н & \multirow{2}{=}{20} & 9 & 45 & 8 & 40 & 3
& 15 & \multirow{2}{=}{0.65} & \multirow{2}{=}{0.35} \\
& О & & 9 & 45 & 9 & 45 & 2 & 10 \\
\multirow{2}{=}{Қара-ала} & Н & \multirow{2}{=}{20} & 12 & 60 & 7 & 35 &
1 & 5 & \multirow{2}{=}{0.78} & \multirow{2}{=}{0.22} \\
& О & & 12 & 60 & 7 & 35 & 1 & 5 \\
\multirow{2}{=}{Алатау} & Н & \multirow{2}{=}{20} & 11 & 55 & 8 & 40 & 1
& 5 & \multirow{2}{=}{0.75} & \multirow{2}{=}{0.25} \\
& О & & 11 & 55 & 8 & 40 & 1 & 5 \\
\end{longtable}

Жалпы алғанда, голштин сиырларында генотиптердің күтілетін жиілігі ВВ
аллельдерінің генотипі бақыланатын шамаларға қарағанда 5\% - ға төмен,
АВ генотипі бақыланатын шамаларға қарағанда жоғары және АА генотиптері
бірдей. Қара-ала және Алатау тұқымдарының сиырларында күтілетін және
байқалатын АА, АВ және ВВ генотиптерінің кездесу жиілігі бірдей болды .

Каппа-казеин генінің локустары бойынша зерттелетін тұқымды сиырлар
сүтінің май қышқылдық құрамы зерттелді. 2-кестеге сәйкес талдау
нәтижелері бойынша - АА генотиптері бар зерттеліп отырған сиырлардың сүт
майындағы май қышқылының құрамы бойынша PUFA көрсеткіші төмен екенін
және Алатау тұқымды сиырларда ең төменгі көрсеткіштер болғанын көрсетті,
яғни -- 3,77\%-ды құрады, біз білетініміздей, шикізат құрамында маңызды
май қышқылдарының болуы адам ағзасына жағымды әсер етеді. Май қышқылының
құрамы бойынша ВВ генотипі бар қара-ала тұқымды сиырлар көш бастап тұр,
яғни ол 3,80\%-ды құрады, содан кейін АВ және ВВ аллельдері бар голштин
сиырларында 3,62\% және 3,70\%-ды құрады, біз білетіндей, бұл қышқыл
адам ағзасына негізінен тек сүтпен енеді (2-кесте).

{\bfseries Кесте 2 -- Каппа-казеин генотипіне байланысты зерттелетін
тұқымды сиырлар сүтінің май қышқылдық құрамы}

\begin{longtable}[]{@{}
  >{\raggedright\arraybackslash}p{(\columnwidth - 20\tabcolsep) * \real{0.1426}}
  >{\raggedright\arraybackslash}p{(\columnwidth - 20\tabcolsep) * \real{0.2500}}
  >{\raggedright\arraybackslash}p{(\columnwidth - 20\tabcolsep) * \real{0.0674}}
  >{\raggedright\arraybackslash}p{(\columnwidth - 20\tabcolsep) * \real{0.0676}}
  >{\raggedright\arraybackslash}p{(\columnwidth - 20\tabcolsep) * \real{0.0676}}
  >{\raggedright\arraybackslash}p{(\columnwidth - 20\tabcolsep) * \real{0.0676}}
  >{\raggedright\arraybackslash}p{(\columnwidth - 20\tabcolsep) * \real{0.0676}}
  >{\raggedright\arraybackslash}p{(\columnwidth - 20\tabcolsep) * \real{0.0676}}
  >{\raggedright\arraybackslash}p{(\columnwidth - 20\tabcolsep) * \real{0.0676}}
  >{\raggedright\arraybackslash}p{(\columnwidth - 20\tabcolsep) * \real{0.0676}}
  >{\raggedright\arraybackslash}p{(\columnwidth - 20\tabcolsep) * \real{0.0666}}@{}}
\toprule\noalign{}
\multirow{2}{=}{\begin{minipage}[b]{\linewidth}\raggedright
Май қышқылының коды
\end{minipage}} &
\multirow{2}{=}{\begin{minipage}[b]{\linewidth}\raggedright
Май қышқылының атауы
\end{minipage}} &
\multicolumn{3}{>{\raggedright\arraybackslash}p{(\columnwidth - 20\tabcolsep) * \real{0.2027} + 4\tabcolsep}}{%
\begin{minipage}[b]{\linewidth}\raggedright
Голштин
\end{minipage}} &
\multicolumn{3}{>{\raggedright\arraybackslash}p{(\columnwidth - 20\tabcolsep) * \real{0.2029} + 4\tabcolsep}}{%
\begin{minipage}[b]{\linewidth}\raggedright
Қара-ала
\end{minipage}} &
\multicolumn{3}{>{\raggedright\arraybackslash}p{(\columnwidth - 20\tabcolsep) * \real{0.2018} + 4\tabcolsep}@{}}{%
\begin{minipage}[b]{\linewidth}\raggedright
Алатау
\end{minipage}} \\
& & \begin{minipage}[b]{\linewidth}\raggedright
АА
\end{minipage} & \begin{minipage}[b]{\linewidth}\raggedright
АВ
\end{minipage} & \begin{minipage}[b]{\linewidth}\raggedright
ВВ
\end{minipage} & \begin{minipage}[b]{\linewidth}\raggedright
АА
\end{minipage} & \begin{minipage}[b]{\linewidth}\raggedright
АВ
\end{minipage} & \begin{minipage}[b]{\linewidth}\raggedright
ВВ
\end{minipage} & \begin{minipage}[b]{\linewidth}\raggedright
АА
\end{minipage} & \begin{minipage}[b]{\linewidth}\raggedright
АВ
\end{minipage} & \begin{minipage}[b]{\linewidth}\raggedright
ВВ
\end{minipage} \\
\midrule\noalign{}
\endhead
\bottomrule\noalign{}
\endlastfoot
C4:0 & май & 3.06 & 3.62 & 3.70 & 3.50 & 3.49 & 3.80 & 3.50 & 2.14 &
2.18 \\
C6:0 & капрон & 1.59 & 2.24 & 1.52 & 1.68 & 3.00 & 2.13 & 1.81 & 1.55 &
1.58 \\
C8:0 & каприл & 1.32 & 1.57 & 1.40 & 1.11 & 1.66 & 1.20 & 1.56 & 0.97 &
0.99 \\
C10:0 & каприн & 2.11 & 3.50 & 2.41 & 2.15 & 3.53 & 2.12 & 2.14 & 2.24 &
2.37 \\
C12:0 & лаурин & 2.78 & 3.25 & 2.97 & 3.01 & 3.67 & 3.12 & 2.33 & 2.62 &
2.81 \\
C14:0 & миристин & 10.01 & 12.96 & 9.63 & 11.36 & 11.57 & 11.50 & 10.60
& 10.16 & 10.05 \\
C16:0 & пальмитин & 27.59 & 28.44 & 27.69 & 28.10 & 26.04 & 28.40 &
27.63 & 28.17 & 27.71 \\
C18:0 & стеарин & 9.18 & 9.50 & 13.14 & 10.25 & 10.71 & 10.36 & 11.30 &
12.26 & 12.31 \\
C20:0 & арахин & 0.14 & 0.10 & 0.20 & 0.25 & 0.09 & 0.18 & 0.20 & 0.15 &
0.17 \\
C22:0 & беген & 0.07 & 0.08 & 0.11 & 0.08 & 0.04 & 0.07 & 0.08 & 0.06 &
0.06 \\
\multicolumn{2}{@{}>{\raggedright\arraybackslash}p{(\columnwidth - 20\tabcolsep) * \real{0.3926} + 2\tabcolsep}}{%
қаныққан май қышқылдары} & 57.85 & 65.28 & 62.77 & 61.49 & 63.80 & 62.88
& 61.14 & 60.32 & 60.24 \\
C10:1 & децен & 0.26 & 0.40 & 0.28 & 0.26 & 0.21 & 0.30 & 0.33 & 0.21 &
0.20 \\
C14:1* & миристинолеин & 1.44 & 1.48 & 0.87 & 1.15 & 1.21 & 1.33 & 1.15
& 1.12 & 1.02 \\
C16:1* & пальмитолеин & 2.33 & 2.37 & 1.60 & 2.03 & 2.37 & 2.14 & 2.05 &
2.13 & 1.96 \\
C18:1* & олеин (омега 9) & 29.46 & 21.64 & 25.58 & 25.80 & 23.49 & 23.89
& 26.90 & 26.80 & 27.12 \\
\multicolumn{2}{@{}>{\raggedright\arraybackslash}p{(\columnwidth - 20\tabcolsep) * \real{0.3926} + 2\tabcolsep}}{%
моноқанықпаған май қышқылдары} & 33.49 & 25.89 & 28.34 & 29.24 & 27.29 &
27.66 & 30.43 & 30.26 & 30.30 \\
C18:2* & линол (омега 6) & 3.52 & 3.79 & 3.89 & 3.01 & 3.82 & 4.01 &
2.63 & 3.94 & 4.10 \\
C18:3* & a\_ линолен (омега 3) & 0.99 & 0.86 & 0.94 & 1.40 & 1.06 & 1.20
& 1.14 & 1.06 & 1.13 \\
\multicolumn{2}{@{}>{\raggedright\arraybackslash}p{(\columnwidth - 20\tabcolsep) * \real{0.3926} + 2\tabcolsep}}{%
полиқанықпаған май қышқылдары} & 4.51 & 4.65 & 4.83 & 4.41 & 4.88 & 5.21
& 3.77 & 5.00 & 5.23 \\
\multicolumn{2}{@{}>{\raggedright\arraybackslash}p{(\columnwidth - 20\tabcolsep) * \real{0.3926} + 2\tabcolsep}}{%
қанықпаған май қышқылдары} & 38.0 & 30.54 & 33.17 & 33.65 & 32.17 &
32.87 & 34.20 & 35.26 & 35.53 \\
\multicolumn{2}{@{}>{\raggedright\arraybackslash}p{(\columnwidth - 20\tabcolsep) * \real{0.3926} + 2\tabcolsep}}{%
басқа май қышқылдары} & 4.15 & 4.19 & 4.06 & 4.86 & 4.03 & 4.25 & 4.65 &
4.42 & 4.22 \\
\end{longtable}

ω-6 /ω-3 қатынасы маңызды көрсеткіш болып табылады, АВ және ВВ
генотиптері бар голштин сиырларында оңтайлы шамаларға жақын болды, яғни
ол 3-кестеге сәйкес 4,41\% және 4,14\%-ды құрайды. Полиқанықпаған май
қышқылдары жоғары биологиялық белсенділікпен сипатталады - олар жасуша
алмасуына қатысады және антисклеротикалық әсерге ие {[}40, 41{]}. Май
қышқылы құрамының жалпы санынан PUFA ең жоғары көрсеткіштері ВВ және АВ
генотиптері бар жануарларда (5.23\%; 5\%) болды . Барлық сүт топтарында
сәйкесінше және C18: 2 басым болды (3-кесте).

{\bfseries 3-кесте -- Каппа-казеин генотипіне байланысты зерттелетін
тұқымды сиырлардағы негізгі өмірлік маңызды май қышқылдарының құрамы мен
қатынасы}

\begin{longtable}[]{@{}
  >{\raggedright\arraybackslash}p{(\columnwidth - 18\tabcolsep) * \real{0.1951}}
  >{\raggedright\arraybackslash}p{(\columnwidth - 18\tabcolsep) * \real{0.0899}}
  >{\raggedright\arraybackslash}p{(\columnwidth - 18\tabcolsep) * \real{0.0896}}
  >{\raggedright\arraybackslash}p{(\columnwidth - 18\tabcolsep) * \real{0.0898}}
  >{\raggedright\arraybackslash}p{(\columnwidth - 18\tabcolsep) * \real{0.0892}}
  >{\raggedright\arraybackslash}p{(\columnwidth - 18\tabcolsep) * \real{0.0892}}
  >{\raggedright\arraybackslash}p{(\columnwidth - 18\tabcolsep) * \real{0.0894}}
  >{\raggedright\arraybackslash}p{(\columnwidth - 18\tabcolsep) * \real{0.0892}}
  >{\raggedright\arraybackslash}p{(\columnwidth - 18\tabcolsep) * \real{0.0892}}
  >{\raggedright\arraybackslash}p{(\columnwidth - 18\tabcolsep) * \real{0.0894}}@{}}
\toprule\noalign{}
\multirow{2}{=}{\begin{minipage}[b]{\linewidth}\raggedright
Май қышқылының мөлшері, \%
\end{minipage}} &
\multicolumn{3}{>{\raggedright\arraybackslash}p{(\columnwidth - 18\tabcolsep) * \real{0.2692} + 4\tabcolsep}}{%
\begin{minipage}[b]{\linewidth}\raggedright
Голштин
\end{minipage}} &
\multicolumn{3}{>{\raggedright\arraybackslash}p{(\columnwidth - 18\tabcolsep) * \real{0.2678} + 4\tabcolsep}}{%
\begin{minipage}[b]{\linewidth}\raggedright
Қара-ала
\end{minipage}} &
\multicolumn{3}{>{\raggedright\arraybackslash}p{(\columnwidth - 18\tabcolsep) * \real{0.2678} + 4\tabcolsep}@{}}{%
\begin{minipage}[b]{\linewidth}\raggedright
Алатау
\end{minipage}} \\
& \begin{minipage}[b]{\linewidth}\raggedright
АА
\end{minipage} & \begin{minipage}[b]{\linewidth}\raggedright
АВ
\end{minipage} & \begin{minipage}[b]{\linewidth}\raggedright
ВВ
\end{minipage} & \begin{minipage}[b]{\linewidth}\raggedright
АА
\end{minipage} & \begin{minipage}[b]{\linewidth}\raggedright
АВ
\end{minipage} & \begin{minipage}[b]{\linewidth}\raggedright
ВВ
\end{minipage} & \begin{minipage}[b]{\linewidth}\raggedright
АА
\end{minipage} & \begin{minipage}[b]{\linewidth}\raggedright
АВ
\end{minipage} & \begin{minipage}[b]{\linewidth}\raggedright
ВВ
\end{minipage} \\
\midrule\noalign{}
\endhead
\bottomrule\noalign{}
\endlastfoot
ΣННЖК & 38.0 & 30.54 & 33.17 & 33.65 & 32.17 & 32.87 & 34.20 & 35.26 &
35.53 \\
ΣНЖК & 57.85 & 65.28 & 62.77 & 61.49 & 63.80 & 62.88 & 61.14 & 60.32 &
60.24 \\
ΣС12:0-С16:0 & 40.38 & 44.65 & 40.29 & 42.47 & 41.28 & 43.02 & 40.56 &
40.95 & 40.57 \\
ΣПНЖК (С18:2-С18:3) & 4.51 & 4.65 & 4.83 & 4.41 & 4.88 & 5.21 & 3.77 &
5.00 & 5.23 \\
Олеин қышқылы (С18:1) & 29.46 & 21.64 & 25.58 & 25.8 & 23.49 & 23.89 &
26.90 & 26.80 & 27.12 \\
Май қышқылы (С4:0) & 3.06 & 3.62 & 3.70 & 3.5 & 3.49 & 3.8 & 3.50 & 2.14
& 2.18 \\
ΣС8:0-С12:0 & 6.21 & 8.32 & 6.78 & 6.27 & 8.86 & 6.44 & 6.03 & 5.83 &
6.17 \\
ΣС4:0-С12:0+ ΣПНЖК & 15.37 & 18.83 & 16.83 & 15.86 & 20.23 & 17.58 &
15.11 & 14.52 & 15.16 \\
С18:2 : С18:3 & 3.56 & 4.41 & 4.14 & 2.15 & 3.60 & 3.34 & 2.31 & 3.72 &
3.63 \\
ΣНЖК/ ΣННЖК & 1.52 & 2.14 & 1.89 & 1.83 & 1.98 & 1.91 & 1.79 & 1.71 &
1.70 \\
ИА & 1.85 & 2.74 & 2.09 & 2.27 & 2.36 & 2.36 & 2.12 & 2.03 & 1.99 \\
ИТ & 2.67 & 3.74 & 3.38 & 3.17 & 3.34 & 3.40 & 3.09 & 3.17 & 3.12 \\
ИЗ & 0.54 & 0.37 & 0.48 & 0.44 & 0.42 & 0.42 & 0.47 & 0.49 & 0.50 \\
\end{longtable}

{\bfseries Қорытынды.}ҒЗЖ шеңберінде каппа-казеин генотипінің зерттелетін
тұқымды сиырлардағы негізгі өмірлік маңызды май қышқылдарының құрамы мен
арақатынасына әсері туралы зерттеулер жүргізілді. АВ және ВВ аллельдерін
тасымалдайтын қара-ала тұқымды сиырлардың сүтінің құрамында ω-6 және ω-3
саны -- 4,88\% және 5,21\% құрады, ал Алатау тұқымды сиырларда АВ және
ВВ генотиптері бар бұл көрсеткіш ең жоғары болды -- 5,00\% және 5,23\%,
бұл адам ағзасына анағұрлым қолайлы. Алынған нәтижелер арқылы сүттің
құрылымын егжей-тегжейлі зерттеуге көмектесетін қосымша зерттеулер
жүргізуге болады.

{\bfseries Литература}

1. Юдахина М. А., Табаков Н. А. Влияние скармливания плющеного ячменя
дойным коровам но молочную продуктивность и качество продуктов
переработки молока // Вестник красноярского государственного аграрного
университета. -- Красноярск, 2011.- № 8(59). - С. 172 - 175.

2. Fenelon M. A., and Guinee T. P. The effect of milk fat on Cheddar
cheese yield and its prediction, using modifications of the van Slyke
cheese yield formula // J. Dairy Sci. -- 1999. -- № 82(11).- P. 2287 -
2299. DOI.10.3168/jds.S0022-0302(99)75477-9

3. Esposito G., Masucci F., Napolitano F., Braghieri A., Romano R.,
Manzo N., and Di Francia A. Fatty acid and sensory profiles of
Caciocavallo cheese as affected by management system // J. Dairy Sci. -
2014.-№ 97(4). -P. 1918-1928. ~DOI~10.3168/jds.2013-7292

4. Martini M., Salari F., and Altomonte I. The macrostructure of milk
lipids: The fat globules // Crit. Rev. Food Sci. Nutr.- 2016. -№
56(7).-P.1209-1221. DOI 10.1080/10408398.2012.758626

5.Иванов В. А., Таджиев К. П. Состав и технологические свойства молока
симментальских и симментал-голштинских помесных коров//Аграрное
образование и наука.- Екатеринбург.- 2014.- № 4.- С. 1 -7.

6. Lucey J. A. Raw milk consumption. Risks and benefits // Nutr. Today.
2015.-Vol. 50(4).- P. 189---193. ~DOI~10.1097/NT.0000000000000108

7. Буянова И. В., Дьяченко С. А. Требования к сырью и готовой продукции
в сыроделии алтайского края // Техника и технология пищевых производств.
--Кемерово.- 2013.- № 4. - С. 3 - 8.

8. Bilal G., Cue R. I., Mustafa A. F., and Hayes J. F. Short
communication: Genetic parameters of individual fatty acids in milk of
Canadian Holsteins // J. DairySci.- 2014.-Vol.97(2).- P.1150-1156.
DOI~10.3168/jds.2012-6508

9. Loginova T. P. and Vorobyeva N. V. Fat phase of milk -- raw material
of cows of different breeding in LLC «Plrmzavod named after Lenin» //
Vestnik of Ulanovsk state agricultural academy.-2016.-№ 2.- P. 145-150.

9. Логинова Т.П., Воробьева Н.П.Фаза молока-сырья у коров различной
селекции в ООО «Племзавод им.Ленина»// Вестник Ульяновской
государственной сельскохозяйственной академии.-2016.-№ 2.- С.145-150.
DOI 10.18286/1816-4501-2016-2-145-150

10. Parodi P. Milk fat in human nutrition // Australian J. Dairy.
Technol. -2004.-Vol. 59.- P.3-59.

11. Топникова Е. В., Горшкова Е. И., Меркулова М. И. Исследование
состава жирных кислот молочного масла // Сыроделие и маслоделие.-
2013.-№ 3.- С. 47 - 49.

12. Shingfield K. J., Bonnet M., and Scollan N. D. Recent developments
in altering the fatty acid composition of ruminant-derived foods //
Animal (Suppl)-2013.- Vol. 7(1).-P. 132-162.

DOI 10.1017/S1751731112001681

13. Nullo E., Frigo E., Rossoni A., Finocchiaro R., Serra M., Rizzi N.,
Samore A. B., Canavesi F., Strillacci M. G., Prinsen R. T. M. M.,
Bagnato A. Genetic parameters of fatty acids in Italian Brown Swiss and
Holstein cows // Ital. J. Anim. Sci.-2014.- Vol.13(3).- P. 397-403.

DOI 10.4081/ijas.2014.3208

14. Pulina G., Francesconi A. H. D., Stefanon B., Sevi A., Calamari L.,
Lacetera N., Dell'Orto V., Pilla F., Marsan P. A., Mele M., Rossi F.,
Bertoni G., Crovetto G. M., and Ronchi B. Sustainable ruminant
production to help feed the planet // Ital. J. Anim. Sci.- 2017.-
Vol.16(1).-P. 140-171.

DOI 10.1080/1828051X.2016.1260500

15. Schwingshackl L., and Hoffmann G. Monounsaturated fatty acids and
risk of cardiovascular disease: Synopsis of the evidence available from
systematic reviews and meta-analyses // Nutrients -2012.- Vol. 4(12)- P.
1989--2007. DOI 10.3390/nu4121989

16. Food and Agriculture Organization. (2010). Fats and fatty acids in
human nutrition. Report of an expert consultation. FAO Food Nutr. -166
p. ISBN 978-92-5-106733-8

17. Li C., Sun D., Zhang S., Wang S., Wu X., Zhang Q., Liu L., Li Y.,
and Qiao L. Genome wide association study identifies 20 novel promising
genes associated with milk fatty acid traits in Chinese Holstein // PLoS
One -- 2014- Vol. 9 (5). - P. 96-186. DOI~10.1371/journal.pone.0096186

18. Zhang, W., Zhang, J., Cui, L., Ma, J., Chen, C., Ai, H., Xie, X.,
Li, L., Xiao, S., Huang, L., Ren, J., Yang, B. Genetic architecture of
fatty acid compositions in the longissimus dorsi muscle revealed by
genome-wide association studies on diverse pig populations // Genetics
Selection Evolution.-2016.-№ 48 (5).- Р. 1-10. DOI
10.1186/s12711-016-0184-2

19. Добрян Е. И., Юрова Е. А., Жижин Н. А. Функциональные молочные
продукты, обогощенные полиненасыщенными жирными кислотами семейства
омега-3 и омега-6 // Молочная промышленность.- 2013.-№ 11.- С. 45 - 46.

20. Перова Н. В., Метельская В. А., Соколов Е. И., Щукина Г. Н., Фомина
В. М. Диетические жирные кислоты. Влияние на риск сердечно-сосудистых
заболеваний // Рациональная аптекарь.-2011.-№ 7(5).- С. 620 - 627.

21. Хромова Л. Г., А. В. Востроилов, Н. В. Байлова. Молочное дело:
Учебник -- СПБ.: Издательство «Лань», 2022. - 332 с. ISBN
978-5-507-44239-3

22. Stoop, W.M., van Arendonk, J.A.M., Heck, J.M.L., Valenberg, H.J.F.,
Bovenhuis, H. Genetic parameters for major milk fatty acids and milk
production traits of Dutch Holstein Friesians // Journal of Dairy
Science.-2008.- Vol. 91(1).-P. 385 - 394. DOI 10.3168/jds.2007-0181

23. Narukami, T., Sasazaki, S., Oyama, K., Nogi, T., Taniguchi, M.,
Mannen, H. Effect of DNA polymorphisms related to fatty acid composition
in adipose tissue of Holstein cattle // Animal Science Journal.
-2011.-Vol. 82(3).- P. 406- 411. DOI 10.1111/j.1740-0929.2010.00855.x

~

24. Калашникова Л. А., Труфанов В. Г. Влияние генотипа каппа-казеина на
молочную продуктивность и технологические свойства молока коров
холмогроской породы // Доклады Российской академии сельскохозяйственных
наук.-2006.- № 4.- С. 43 - 44.

25. Тюлькин С. В., Ахметов Т. М., Загидуллин Л. Р., Рачкова Е. Н.,
Шайдуллин С. Ф., Гильманов Х. Х. Полиморфизм гена каппа-казеина в стадах
крупно рогатого скота Республики Татарстан // Ученые записки КГАВМ им.
Н. Э. Баумана. -2016. -- Т. 255, № 1. -- С. 148--151.

26. Павлова Н. И., Филиппова Н. П., Куртанов Х. А., Корякина Л. П.
Оценка аллельного и генотипического разнообразия крупного рогатого скота
Якутии по генам молочности // Наука и образование.- 2016.-№ 3.- С.
122-127.

27. Khaizaran, Z., Al-Razem, F. Analysis of selected milk traits in
Palestinian Holstein-Friesian cattle in relation to genetic polymorphism
// Journal of Cell and Animal Biology.-2014. -Vol. 8(5).- P. 74 - 85.
DOI 10.5897/JCAB2014.0409

28. Овсяников А. И. Основы опытного дела в животноводстве. -- М.: Колос,
1976.- 303 с.

29. Дунин И. М., Переверзев Д. Б., Козанков А. Г. Проведение научных
исследований в скотоводстве: Методические рекомендации.- М., 2000.-80
с.ISBN 5-87958-126-8

30. Калашникова Л. А., Дунин И. М., Глазко В. И., Рыжова Н. В., Голубина
Е. П. ДНК-технологии оценки сельскохозяйственных животных.-Лесные
Поляны: ВНИИплем, 1999. - 148 с.

31. Петухов В. Л., Жигачев А. И., Назарова Г. А. Ветеринарная генетика с
основами вариационной статистики.- М.: Агропромиздат, 1985.-369 с.

32. Нургалиева М. Т., Тойшиманов М. Р., Сериков М. С., Мырзабаева Н. Е.,
Хастаева А. Ж. Калибровка газохроматографического прибора для
определения жирнокислотного состава пищевых продуктов // «Ізденістер,
нәтижелер. Исследования, результаты».-2019.-№ 1(18) -- С. 79--85.

33. Нечаев А. П. Пищевая химия / А. П. Нечаев и др.: под ред. А. П.
Нечаева. -- СПб.: ГИОРД, 2015. -- 672 с. ISBN 978-5-98879-196-6.

34. Fehily, A.M.. Dietary indices of atherogenicity and thrombogenicity
and ischemic heart disease risk: the Caerphilly Prospective Study //
British Journal of Nutrition.-1994.-Vol.71(2)- P. 249--258. DOI
10.1079/bjn19940131

35. ВОЗ Здоровое питание Информационный бюллетень № 394 сентябрь 2015 г.

36. Перова Н. В., Метельская В. А., Соколов Е. И., Щукина Г. Н., Фомина
В. М. Диетические жирные кислоты. Влияние на риск сердечно-сосудистых
заболеваний // Рациональная аптекарь.- 2011.-№ 7(5). -- С. 620--627.

37. Ulbritch, T.L.V., Southgate, D.A.T. Coronary heart disease: seven
dietary factors // Lancet. -- 1991.-Vol. 338(8773).- P. 985 - 992.
DOI:~10.1016/0140-6736(91)91846-m

38. Tait, J.R., Reecy, Investigating opportunities available in genetic
selection for healthy beef. 2007. https://slideplayer.com/slide/4750876/

39. Хастаева А.Ж., Альжаксина Н.Е., Сағымбаева Д.Е. Сүттің тағамдық
құндылығына каппа-казеин генотипінің әсері // Шәкәрім университетінің
хабаршысы. Техникалық ғылымдар сериясы. № 1(13). -- 2024. - С. 273 --
280. DOI 10.53360/2788-7995-2024-1(13)-35

40. Емец З. В., Мирошникова О. С., Хруцкий С. С., Баско С. О.
Воздействие факторов породы и отец на жиронкислотный состав молока коров
// Serenity -- Group. -- 2017. -- № 3-2(43). -- С. 37--40.

41. Андрианов Ю. П., Вышемирский Ф. А., Качераускис Д. В. Производства
сливочного масла: Справочник; под ред. д.т.н. Ф. А. Вышемирского. -- М.:
Агропромиздат, 1988. -- 303 с.

{\bfseries References}

1. Judahina M. A., Tabakov N. A. Vlijanie skarmlivanija pljushhenogo
jachmenja dojnym korovam no molochnuju produktivnost\textquotesingle{} i
kachestvo produktov pererabotki moloka // Vestnik krasnojarskogo
gosudarstvennogo agrarnogo universiteta. -- Krasnojarsk, 2011.- № 8(59).
- S. 172 - 175.{[}in Russ.{]}

2. Fenelon M. A., and Guinee T. P. The effect of milk fat on Cheddar
cheese yield and its prediction, using modifications of the van Slyke
cheese yield formula // J. Dairy Sci. -- 1999. -- № 82(11).- P. 2287 -
2299. DOI.10.3168/jds.S0022-0302(99)75477-9

3. Esposito G., Masucci F., Napolitano F., Braghieri A., Romano R.,
Manzo N., and Di Francia A. Fatty acid and sensory profiles of
Caciocavallo cheese as affected by management system // J. Dairy Sci. -
2014.-№ 97(4). -P. 1918-1928. ~DOI~10.3168/jds.2013-7292

4. Martini M., Salari F., and Altomonte I. The macrostructure of milk
lipids: The fat globules // Crit. Rev. Food Sci. Nutr.- 2016. -№
56(7).-P.1209-1221. DOI 10.1080/10408398.2012.758626

5.Ivanov V. A., Tadzhiev K. P. Sostav i tehnologicheskie svojstva moloka
simmental\textquotesingle skih i simmental-golshtinskih pomesnyh
korov//Agrarnoe obrazovanie i nauka.- Ekaterinburg.- 2014.- № 4.- S. 1
-7.{[}in Russ.{]}

6. Lucey J. A. Raw milk consumption. Risks and benefits // Nutr. Today.
2015.-Vol. 50(4).- P. 189 - 193. ~DOI~10.1097/NT.0000000000000108

7. Bujanova I. V., D\textquotesingle jachenko S. A. Trebovanija k
syr\textquotesingle ju i gotovoj produkcii v syrodelii altajskogo kraja
// Tehnika i tehnologija pishhevyh proizvodstv. --Kemerovo.- 2013.- № 4.
- S. 3 - 8.{[}in Russ.{]}

8. Bilal G., Cue R. I., Mustafa A. F., and Hayes J. F. Short
communication: Genetic parameters of individual fatty acids in milk of
Canadian Holsteins // J. DairySci.- 2014.-Vol.97(2).- P.1150-1156.
DOI~10.3168/jds.2012-6508

9. Loginova T.P., Vorob\textquotesingle eva N.P.Faza
moloka-syr\textquotesingle ja u korov razlichnoj selekcii v OOO
«Plemzavod im.Lenina»// Vestnik Ul\textquotesingle janovskoj
gosudarstvennoj sel\textquotesingle skohozjajstvennoj akademii.-2016.-№
2.- S.145-150. {[}in Russ.{]}DOI 10.18286/1816-4501-2016-2-145-150

10. Parodi P. Milk fat in human nutrition // Australian J. Dairy.
Technol. -2004.-Vol. 59.- P.3-59.

11. Topnikova E. V., Gorshkova E. I., Merkulova M. I. Issledovanie
sostava zhirnyh kislot molochnogo masla // Syrodelie i maslodelie.-
2013.-№ 3.- S. 47 - 49. {[}in Russ.{]}

12. Shingfield K. J., Bonnet M., and Scollan N. D. Recent developments
in altering the fatty acid composition of ruminant-derived foods //
Animal (Suppl)-2013.- Vol. 7(1).-P. 132-162.

DOI 10.1017/S1751731112001681

13. Nullo E., Frigo E., Rossoni A., Finocchiaro R., Serra M., Rizzi N.,
Samore A. B., Canavesi F., Strillacci M. G., Prinsen R. T. M. M.,
Bagnato A. Genetic parameters of fatty acids in Italian Brown Swiss and
Holstein cows // Ital. J. Anim. Sci.-2014.- Vol.13(3).- P. 397-403.

DOI 10.4081/ijas.2014.3208

14. Pulina G., Francesconi A. H. D., Stefanon B., Sevi A., Calamari L.,
Lacetera N., Dell'Orto V., Pilla F., Marsan P. A., Mele M., Rossi F.,
Bertoni G., Crovetto G. M., and Ronchi B. Sustainable ruminant
production to help feed the planet // Ital. J. Anim. Sci.- 2017.-
Vol.16(1).-P. 140-171.

DOI 10.1080/1828051X.2016.1260500

15. Schwingshackl L., and Hoffmann G. Monounsaturated fatty acids and
risk of cardiovascular disease: Synopsis of the evidence available from
systematic reviews and meta-analyses // Nutrients -2012.- Vol. 4(12)- P.
1989--2007. DOI 10.3390/nu4121989

16. Food and Agriculture Organization. (2010). Fats and fatty acids in
human nutrition. Report of an expert consultation. FAO Food Nutr. -166
p. ISBN 978-92-5-106733-8

17. Li C., Sun D., Zhang S., Wang S., Wu X., Zhang Q., Liu L., Li Y.,
and Qiao L. Genome wide association study identifies 20 novel promising
genes associated with milk fatty acid traits in Chinese Holstein // PLoS
One -- 2014- Vol. 9 (5). - P. 96-186. DOI~10.1371/journal.pone.0096186

18. Zhang, W., Zhang, J., Cui, L., Ma, J., Chen, C., Ai, H., Xie, X.,
Li, L., Xiao, S., Huang, L., Ren, J., Yang, B. Genetic architecture of
fatty acid compositions in the longissimus dorsi muscle revealed by
genome-wide association studies on diverse pig populations // Genetics
Selection Evolution.-2016.-№ 48 (5).- Р. 1-10. DOI
10.1186/s12711-016-0184-2

19. Dobrjan E. I., Jurova E. A., Zhizhin N. A.
Funkcional\textquotesingle nye molochnye produkty, obogoshhennye
polinenasyshhennymi zhirnymi kislotami semejstva omega-3 i omega-6 //
Molochnaja promyshlennost\textquotesingle.- 2013.-№ 11.- S. 45 - 46.
{[}in Russ.{]}

20. Perova N. V., Metel\textquotesingle skaja V. A., Sokolov E. I.,
Shhukina G. N., Fomina V. M. Dieticheskie zhirnye kisloty. Vlijanie na
risk serdechno-sosudistyh zabolevanij // Racional\textquotesingle naja
aptekar\textquotesingle.-2011.-№ 7(5).- S. 620 - 627. {[}in Russ.{]}

21. Hromova L. G., A. V. Vostroilov, N. V. Bajlova. Molochnoe delo:
Uchebnik -- SPB.: Izdatel\textquotesingle stvo «Lan\textquotesingle»,
2022. - 332 s. ISBN 978-5-507-44239-3{[}in Russ.{]}

22. Stoop, W.M., van Arendonk, J.A.M., Heck, J.M.L., Valenberg, H.J.F.,
Bovenhuis, H. Genetic parameters for major milk fatty acids and milk
production traits of Dutch Holstein Friesians // Journal of Dairy
Science.-2008.- Vol. 91(1).-P. 385 - 394. DOI 10.3168/jds.2007-0181

23. Narukami, T., Sasazaki, S., Oyama, K., Nogi, T., Taniguchi, M.,
Mannen, H. Effect of DNA polymorphisms related to fatty acid composition
in adipose tissue of Holstein cattle // Animal

Science Journal. -2011.-Vol. 82(3).- P. 406- 411. DOI
10.1111/j.1740-0929.2010.00855.x

24. Kalashnikova L. A., Trufanov V. G. Vlijanie genotipa kappa-kazeina
na molochnuju produktivnost\textquotesingle{} i tehnologicheskie
svojstva moloka korov holmogroskoj porody // Doklady Rossijskoj akademii
sel\textquotesingle skohozjajstvennyh nauk.-2006.- № 4.- S. 43 - 44.
{[}in Russ.{]}

25. Tjul\textquotesingle kin S. V., Ahmetov T. M., Zagidullin L. R.,
Rachkova E. N., Shajdullin S. F., Gil\textquotesingle manov H. H.
Polimorfizm gena kappa-kazeina v stadah krupno rogatogo skota Respubliki
Tatarstan // Uchenye zapiski KGAVM im. N. Je. Baumana. -2016. -- T. 255,
№ 1. -- S. 148--151. {[}in Russ.{]}

26. Pavlova N. I., Filippova N. P., Kurtanov H. A., Korjakina L. P.
Ocenka allel\textquotesingle nogo i genotipicheskogo raznoobrazija
krupnogo rogatogo skota Jakutii po genam molochnosti // Nauka i
obrazovanie.- 2016.-№ 3.- S. 122-127. {[}in Russ.{]}

27. Khaizaran, Z., Al-Razem, F. Analysis of selected milk traits in
Palestinian Holstein-Friesian cattle in relation to genetic polymorphism
// Journal of Cell and Animal Biology.-2014. -Vol. 8(5).- P. 74 - 85.
DOI 10.5897/JCAB2014.0409

28. Ovsjanikov A. I. Osnovy opytnogo dela v zhivotnovodstve. -- M.:
Kolos, 1976.- 303 s. {[}in Russ.{]}

29. Dunin I. M., Pereverzev D. B., Kozankov A. G. Provedenie nauchnyh
issledovanij v skotovodstve: Metodicheskie rekomendacii.- M., 2000.-80
s.ISBN 5-87958-126-8.{[}in Russ.{]}

30. Kalashnikova L. A., Dunin I. M., Glazko V. I., Ryzhova N. V.,
Golubina E. P. DNK-tehnologii ocenki
sel\textquotesingle skohozjajstvennyh zhivotnyh.-Lesnye Poljany:
VNIIplem, 1999. - 148 s. {[}in Russ.{]}

31. Petuhov V. L., Zhigachev A. I., Nazarova G. A. Veterinarnaja
genetika s osnovami variacionnoj statistiki.- M.: Agropromizdat,
1985.-369 s. {[}in Russ.{]}

32. Nurgalieva M. T., Tojshimanov M. R., Serikov M. S., Myrzabaeva N.
E., Hastaeva A. Zh. Kalibrovka gazohromatograficheskogo pribora dlja
opredelenija zhirnokislotnogo sostava pishhevyh produktov //
«Іzdenіster, nәtizheler. Issledovanija,
rezul\textquotesingle taty».-2019.-№ 1(18) -- S. 79--85. {[}in Russ.{]}

33. Nechaev A. P. Pishhevaja himija / A. P. Nechaev i dr.: pod red. A.
P. Nechaeva. -- SPb.: GIORD, 2015. -- 672 s. ISBN 978-5-98879-196-6.
{[}in Russ.{]}

34. Fehily, A.M.. Dietary indices of atherogenicity and thrombogenicity
and ischemic heart disease risk: the Caerphilly Prospective Study //
British Journal of Nutrition.-1994.-Vol.71(2)- P. 249--258. DOI
10.1079/bjn19940131

35. VOZ Zdorovoe pitanie Informacionnyj bjulleten\textquotesingle{} №
394 sentjabr\textquotesingle{} 2015 g. {[}in Russ.{]}

36. Perova N. V., Metel\textquotesingle skaja V. A., Sokolov E. I.,
Shhukina G. N., Fomina V. M. Dieticheskie zhirnye kisloty. Vlijanie na
risk serdechno-sosudistyh zabolevanij // Racional\textquotesingle naja
aptekar\textquotesingle.- 2011.-№ 7(5). - S. 620-627. {[}in Russ.{]}

38. Tait, J.R., Reecy, Investigating opportunities available in genetic
selection for healthy beef. 2007. https://slideplayer.com/slide/4750876/

39. Hastaeva A.Zh., Al\textquotesingle zhaksina N.E., Saғymbaeva D.E.
Sүttің taғamdyқ құndylyғyna kappa-kazein genotipіnің әserі // Shәkәrіm
universitetіnің habarshysy. Tehnikalyқ ғylymdar serijasy. № 1(13). --
2024. - S. 273 -280. DOI 10.53360/2788-7995-2024-1(13)-35 {[} in
Kazakh.{]}

40. Emec Z. V., Miroshnikova O. S., Hruckij S. S., Basko S. O.
Vozdejstvie faktorov porody i otec na zhironkislotnyj sostav moloka
korov // Serenity -Group.- 2017.-№ 3-2(43).- S. 37-40. {[}in Russ.{]}

41. Andrianov Ju. P., Vyshemirskij F. A., Kacherauskis D. V.
Proizvodstva slivochnogo masla: Spravochnik; pod red. d.t.n. F. A.
Vyshemirskogo.-M.: Agropromizdat, 1988. -303 s. {[}in Russ.{]}

\emph{{\bfseries Автор туралы мәліметтер}}

Хастаева А.Ж.-- Phd, Қ.Құлажанов атындағы қазақ технология және бизнес
университеті, Астана, Қазақстан, e-mail: gera\_or@mail.ru.

\emph{{\bfseries Information about the author}}

Khastayeva А. {\bfseries --} Phd, K.Kulazhanov Kazakh University of
Technology and Business, Astana, Kazakhstan, e-mail: gera\_or@mail.ru.





