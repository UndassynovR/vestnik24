\newpage
{\bfseries IRSTI 65.51.03}

\sectionwithauthors{B.Khamitova, F.Dikhanbayeva , G.Koshtayeva}{STUDY OF PHYSICAL AND CHEMICAL PROPERTIES OF SOFT DRINKS
OBTAINED WITH ADDITION OF GOOSEBERRY EXTRACT}
\begin{center}

{\bfseries \textsuperscript{1}B.Khamitova}\textsuperscript{🖂}{\bfseries ,
\textsuperscript{2}F.Dikhanbayeva , \textsuperscript{1}G.Koshtayeva}

\textsuperscript{1}M.Auezov South-Kazakhstan University, Shymkent,
Kazakhstan,

\textsuperscript{2}Almaty Technological University, Almaty, Kazakhstan
\end{center}

\textsuperscript{🖂}Correspondent-author: barno-007@mail.ru

The manufacturing of juice-containing soft drinks is experiencing
significant expansion in both interna-tional and domestic markets,
including Kazakhstan. This tendency can be attributed to the aspiration
of the working-age population in developed countries to adopt a
health-conscious lifestyle. The manufacturing of beverages using natural
fruit and berry ingredients is highly intriguing in this context.
Enriching soft drinks with a diverse array of physiologically active
chemicals derived from plant materials is highly significant. This study
aims to investigate the physicochemical characteristics of soft drinks
and assess the potential application of gooseberry extract.~

In order to create a non-alcoholic beverage, the valuable wild
gooseberries were utilised. These gooseber-ries are rich in biologically
active elements, including vitamins, vitamin-like compounds, flavonoids,
\\minerals, and other chemicals.~

The acquired results are derived from a substantial volume of empirical
study and are founded upon an examination of literary data pertaining to
the chemical makeup of gooseberries. The study investigated the chemical
makeup of gooseberries using contemporary techniques of chemical
analysis. A recipe for a non-alcoholic beverage made using gooseberry
extract, free from any artificial ingredients, has been devised. The
organoleptic and physico-chemical features of it have been determined.~

{\bfseries Keywords:} soft drinks, gooseberriy, plant extracts, fruit and
berry raw materials, functional purpose, ultrasonic extraction method

\sectionheading{ИССЛЕДОВАНИЕ ФИЗИКО-ХИМИЧЕСКИХ СВОЙСТВ БЕЗАЛКОГОЛЬНЫХ НАПИТКОВ,
ПОЛУЧЕННЫХ С ДОБАВЛЕНИЕМ ЭКСТРАКТА КРЫЖОВНИКА}
\begin{center}

{\bfseries \textsuperscript{1} Б.М. Хамитова}\textsuperscript{🖂}{\bfseries ,
\textsuperscript{2} Ф.Т. Диханбаева, \textsuperscript{1}Г.Е. Коштаева}

\textsuperscript{1}Южно-Казахстанский университет им. М.Ауэзова,
Казахстан, Шымкент, Казахстан,

\textsuperscript{2}Алматинский технологический университет, Казахстан,
Алматы, Казахстан,

e-mail: barno-007@mail.ru
\end{center}

В последнее время наблюдается значительный рост производства
безалкогольных напитков, содер-жащих сок, как на внутреннем, так и на
международном рынках, включая Казахстан. Эту тенденцию можно связать с
желанием трудоспособного населения в промышленно развитых странах вести
здоро-вый образ жизни. В этом контексте производство напитков на основе
натуральных фруктово-ягодных компонентов вызывает большой интерес.
Актуальным является и добавление в безалкогольные на-питки широкого
спектра биологически активных веществ, полученных из растительного
сырья. Цель данного исследования - изучение физико-химических свойств
безалкогольных напитков и оценка возможности применения экстракта
крыжовника.

Использование дикого крыжовника позволило получить безалкологольный
напиток, содержащий значительное количество биологически активных
компонентов, таких как витамины, витаминоподоб-ные соединения,
флавоноиды, минералы и другие вещества.

Полученные результаты основаны на анализе литературы, посвященной
химическому составу кры-жовника, а также на значительном объеме
проведенных экспериментальных исследований. В ходе исследования
использовались современные методы химического анализа для изучения
химического состава крыжовника. С использованием экстракта крыжовника
была разработана рецептура напитка, не содержащего искусственных
компонентов и спирта. Также помимо физико-химических свойств, были
выявлены и органолептические характеристики напитка.

{\bfseries Ключевые слова:} безалкогольные напитки, крыжовник, растительные
экстракты, фруктово-\\ягодное сырье, функционального назначения,
ультразвуковой способ экстрагирования

\sectionheading{ҚАРЛЫҒАН СЫҒЫНДЫСЫ ҚОСЫЛҒАН АЛКОГОЛЬСІЗ СУСЫНДАРДЫҢ
ФИЗИКАЛЫҚ-ХИМИЯЛЫҚ ҚАСИЕТТЕРІН ЗЕРТТЕУ}
\begin{center}

{\bfseries \textsuperscript{1}Б.М. Хамитова}\textsuperscript{🖂}{\bfseries ,
\textsuperscript{2} Ф.Т. Диханбаева, \textsuperscript{1}Г.Е. Коштаева}

\textsuperscript{1}М. Әузов атындағы Оңтүстік-Қазақстан университеті,
Шымкент, Қазақстан,

\textsuperscript{2}Алматы технологиялық университеті, Алматы, Қазақстан,

e-mail: barno-007@mail.ru
\end{center}

Құрамында шырыны бар алкогольсіз сусындар өндірісінің ассортименті
шетелде де, Қазақстанда да қарқынды өсуде. Бұл үрдіс дамыған елдердегі
тұрғындардың белсенді бөлігінің салауатты өмір салтына ұмтылысына
байланысты. Осыған орай, табиғи жеміс-жидек шикізаты негізіндегі
сусындар өндірісі үлкен қызығушылық тудыруда. Алкогольсіз сусындарды
өсімдік тектес биологиялық белсен-ді заттардың кең спектрімен байыту өте
маңызды. Бұл жұмыстың мақсаты алкогольсіз сусындардың физика-химиялық
қасиеттерін зерттеу, сонымен қатар қарлыған сығындысын пайдалану
мүмкіндігін зерттеу.

Алкогольсіз сусын алу үшін дәрумендер мен витаминге ұқсас қосылыстар,
флавоноидтар, минерал-дар және басқа заттар сияқты биологиялық белсенді
заттардың құнды көзі болып табылатын жабайы қарлыған пайдаланылды.

Алынған нәтижелер тәжірибелік зерттеулердің айтарлықтай көлеміне және
қарлығанның химия-лық құрамы туралы әдеби деректерді талдауға
негізделген. Жұмыста қазіргі заманғы химиялық талдау әдістерін қолдану
арқылы қарлығанның химиялық құрамы зерттелді. Құрамында синтетика-лық
компоненттері жоқ қарлыған сығындысы бар алкогольсіз сусынның рецепті
әзірленді. Оның органо-лептикалық және физика-химиялық көрсеткіштері
анықталды.

{\bfseries Түйін сөздер:} алкогольсіз сусындар, қарлыған, өсімдік
сығындылары, жеміс-жидек шикізаты, функционалдық мақсаты, ультрадыбыстық
экстрагирлеу тәсілі

{\bfseries Introduction.} The soft drink business is experiencing
significant growth and is one of the fastest-growing sub-industries in
the global food sector. Both the production and consumption of soft
drinks are exhibiting a consistent upward trend, both currently and in
the foreseeable future. Furthermore, alongside the proactive adoption of
novel packaging formats, a media-based advertising campaign is being
executed, effectively capturing the interest of a growing customer base
{[}1{]}.

The priority direction of that area is considered to be the
diversification of soft drinks, including low-calorie specialized drinks
with various functional orientations. Development of non-alcoholic
industry was to be carried out in two main directions: increasing the
production of drinks on fruit and berry and malt raw materials;
increasing the production of tonic and fortified drinks, as well as
"protection" drinks having a special purpose {[}2{]}.

At present, the challenges of the logical and efficient use of commonly
available plant materials as a valuable source of functional components
and the creation of healthy soft beverages are highly essential.

It is widely known that food has a significant impact on human health.
Antioxidants can help reduce environmental oxidative stress caused by
free radicals, which can damage the body\textquotesingle s cellular
system. Free radicals can be generated intracellularly due to the impact
of detrimental factors such radiation, UV radiation, and chemical
processes involving polycyclic aromatic hydrocarbons. Free radicals are
accountable for the partial or complete degradation of lipids and
proteins in the human body. The degradation mentioned causes cellular
and genetic mutations, as well as interactions with polyunsaturated
fatty acids, DNA, and proteins. These interactions ultimately contribute
to the development of various illnesses.

Conversely, antioxidants hinder the oxidation of lipids by interacting
with free radicals. Plant-based commodities, such as fruits and berries,
serve as the main sources of antioxidants. The reason for this is that
only plant-based products have the ability to generate bioflavonoids and
other polyphenolic compounds. The exploitation of indigenous plant
resources, which provide the greatest health benefits to those living in
the same region, is a particularly promising strategy {[}3,4{]}.

In recent years, scientists have focused their research on developing
innovative formulations and technologies for soft drinks that not only
quench thirst and provide refreshment, but also have physiological or
preventative effects. Considerable emphasis is placed on enhancing the
longevity of beverages throughout the storing process. There are novel
varieties of soft drinks that distinguish themselves from conventional
ones in terms of both ingredients and production methods, as well as in
terms of taste and their effects on the human body {[}5{]}.~

An efficient approach to addressing nutritional deficiencies caused by
vitamin deficiencies is the advancement of novel formulations and
technology for juice-based products with functional properties.
Therefore, it is necessary to develop novel plant-based products
utilising indigenous raw ingredients.~

The utilisation of plant-derived raw materials for the development of
novel food items offers several benefits owing to the elevated
bioactivity and bioavailability of the active food constituents present
in them. Fruits and berries have a limited duration before they spoil,
which necessitates the development of processing techniques to ensure a
continuous supply of these items to the population throughout the year.
The plants contain biologically active compounds that determine the
specific attributes of the resulting product and provide essential
technological characteristics. This eliminates the need for the addition
of flavours, colours, and preservatives. One method for maintaining the
advantageous qualities of fruits and berries, such as their antioxidant
capabilities, all year round is by creating fruit and berry extracts and
incorporating them into food {[}6{]}.~

The efficiency of the process of extracting biologically active
chemicals from plants is influenced by key technological elements such
as temperature, extraction time, degree of raw material grinding, type
of extractant, hydromodule, and others. Every variety of plant raw
material possesses certain parameters, modes, and conditions that have
been determined through experimental research {[}7{]}. To introduce
natural flavors, including essential oils, into drink recipes,
surfactants (surfactants) are needed to distribute them evenly
throughout the volume of the drink. Highly effective surfactants include
triterpene plant saponins, which have a wide range of pharmacological
effects (hypercholesterolemic, anticarcinogenic, hepatoprotective
effects; antioxidant, immunological effects, and so on) {[}8,9{]}.

Here are a few instances of biologically active supplements that have
been proposed: the ginseng biomass infusion, known as "BAD-GS," consists
of potassium, sodium ions, and 12 trace elements. The preparation called
"MIGI-K-LP" is derived from mussel meat and possesses radioprotective
and anti-inflammatory effects. The preparation called "Zosterin" is
obtained from seaweed and contains a substantial quantity of
polygalacturonic acid. In addition, the therapy process include
infusions of medicinal plants such as Chinese lemongrass, levzei
safflower, and eleuterococcus. This foundation has been employed in the
development of several beverages that possess both preventive and
therapeutic properties: {[}1,10{]}.

Drinks on flavors occupy a significant segment of the market, as they
are the most popular due to the presence of a large variety of flavoring
components, high organoleptic indicators and relatively low cost. For
flavouring beverages, artificial and identical natural flavors are
mainly used, and water of various degrees of carbonation is used as a
base {[}11,12{]}.

It is crucial to incorporate plant extracts in the formulation of
flavoured beverages to enhance the presence of their functional elements
and biologically active substances (BAS) with antioxidant properties.
This is due to the fact that contemporary clients possess tastes that
diverge from those held by prior generations. Plant raw materials
contain a substantial amount of phenolic compounds, alkaloids,
glycosides, polysaccharides, organic acids, essential oils, vitamins,
minerals, and other components. These molecules exert a favourable
influence on the physiological functioning of several systems inside the
human body, encompassing the digestive, urinary, cardiovascular,
immunological, and other systems {[}13{]}.~

Specialised beverages tailored for athletes are currently being
formulated, which include energy drinks infused with juices, extracts,
caffeine, ginseng preparations, and other natural adaptogens. A diverse
assortment of powdered drink combinations incorporating medicinal and
preventative characteristics derived from vegetable raw materials has
been created {[}14,15{]}.

For completion of losses of liquid during trainings and competitions use
specialized sports drinks generally on the basis of a carbohydrate
chloridno - sodium composition. But at the same time it is necessary
that sports drinks not only recovered losses of liquid, but also had
functional focus that is reached by enrichment of a compounding with
biologically active agents. A specific place is held by the substances
possessing adaptogenny action, in particular, extracts of plants, for
example, of an echinacea, a ginseng, ginger and a St.
John\textquotesingle s wort {[}16,17{]}.

Enriching soft drinks with polycomponent systems of plant extractives in
the form of concentrates and bases is a new approach to promoting
health, improving productivity, and supporting the
body\textquotesingle s natural healing processes.

{\bfseries Materials and methods}\emph{. Methods (methodology) of the
experiment}

Apple juice and gooseberry extracts were the primary components utilised
in the manufacturing of soft drinks.~

Due to the distinctive composition of gooseberry (Red Large variety),
which contains significant amounts of vitamins A and C, as well as
vitamins E, PP, B groups, and various minerals like potassium, calcium,
iron, zinc, and others, gooseberry was selected as the main
ingredient.~\\
The chemical makeup of gooseberries is influenced by various elements
such as the variety, age, soil conditions, and other environmental
factors. Consequently, the data regarding the chemical composition of
gooseberries from different sources are more prone to variation compared
to the data for other garden crops {[}1{]}.~

The health advantages of berries are attributed to a combination of
beneficial compounds and vital vitamins. The product\textquotesingle s
pulp is distinguished by the presence of pectins, minerals, and metals.
Table 1 shows the amount of beneficial components and vitamins found in
100 grammes of gooseberries.

{\bfseries Table 1 - Useful components and vitamins in composition of
gooseberries}

\begin{longtable}[]{@{}
  >{\raggedright\arraybackslash}p{(\columnwidth - 10\tabcolsep) * \real{0.1633}}
  >{\raggedright\arraybackslash}p{(\columnwidth - 10\tabcolsep) * \real{0.1668}}
  >{\raggedright\arraybackslash}p{(\columnwidth - 10\tabcolsep) * \real{0.1668}}
  >{\raggedright\arraybackslash}p{(\columnwidth - 10\tabcolsep) * \real{0.1669}}
  >{\raggedright\arraybackslash}p{(\columnwidth - 10\tabcolsep) * \real{0.1516}}
  >{\raggedright\arraybackslash}p{(\columnwidth - 10\tabcolsep) * \real{0.1845}}@{}}
\toprule\noalign{}
\begin{minipage}[b]{\linewidth}\raggedright
{\bfseries Vitamins}
\end{minipage} & \begin{minipage}[b]{\linewidth}\raggedright
{\bfseries Quantity, mg}
\end{minipage} & \begin{minipage}[b]{\linewidth}\raggedright
{\bfseries \% from 100 g norm}
\end{minipage} & \begin{minipage}[b]{\linewidth}\raggedright
{\bfseries Minerals}
\end{minipage} & \begin{minipage}[b]{\linewidth}\raggedright
{\bfseries Quantity, mg}
\end{minipage} & \begin{minipage}[b]{\linewidth}\raggedright
{\bfseries \% from 100 g norm}
\end{minipage} \\
\midrule\noalign{}
\endhead
\bottomrule\noalign{}
\endlastfoot
А & 0.033 & 3.6 & potassium & 260 & 10.4 \\
В\textsubscript{1} & 0.01 & 0.7 & calcium & 22 & 2.2 \\
В\textsubscript{2} & 0.02 & 1.1 & magnesium & 9.0 & 2.3 \\
В\textsubscript{4} & 42.1 & 5.11 & sodium & 23 & 1.8 \\
В\textsubscript{5} & 0.286 & 5.0 & sulfur & 18 & 1.8 \\
В\textsubscript{6} & 0.03 & 1.5 & phosphorus & 28 & 4.0 \\
В\textsubscript{9} & 5.0 mkg & 1.3 & chlorine & 1.0 & 3.5 \\
С & 30.0 & 33.3 & iron & 0.8 & 4.4 \\
Е & 0.5 & 3.3 & iodine & 1.0 & 0.7 \\
K & 7.8 mkg & 0.7 & manganese & 0.45 & 22.5 \\
РР & 0.4 & 2.0 & copper & 130 & 13 \\
Niacin B\textsubscript{3} & 0.3 & 3.1 & molybdenum & 12 & 17.1 \\
Antioxidants & 0.389±0.005 & 0.413±0.006 & fluorine & 12 & 0.3 \\
& & & chromous & 1.0 & 2.0 \\
& & & zink & 0.09 & 0.8 \\
\multicolumn{6}{@{}>{\raggedright\arraybackslash}p{(\columnwidth - 10\tabcolsep) * \real{1.0000} + 10\tabcolsep}@{}}{%
Note: compiled based on source {[}1{]}} \\
\end{longtable}

Gooseberries are popular in diets due to their low calorie level, high
liquid content, presence of fibre, and pectin content.~

Based on our analysis of the literature and existing recipes for
producing beverages, we have determined that incorporating gooseberries
into the formulation will enhance its composition, while also imparting
a more delicate hue and flavour to the drink.

\emph{Experimental part}

Extracts are highly concentrated juices that are free of pectin and can
be produced using sulfitated materials. Consequently, the extraction of
aromatic compounds does not occur throughout the manufacturing process.
The extracts are utilised in the production of carbonated beverages.~

Ultrasound is a highly promising technique for enhancing the extraction
of plant resources. Utilising the ultrasonic extraction method can
effectively decrease the time required for the procedure and result in a
more thorough extraction of compounds {[}18 in Russian{]}.~

The extraction of nutrients from a mixture depends not only on the
composition of the raw materials, but also on the specific type of
extractant used. In order to ascertain the most effective extractant and
the optimal percentage of raw materials, we generated numerous samples
of gooseberry extracts using the technique outlined below. To attain a
particle size of 1-2 millimetres, we measured and pulverised the
unprocessed components. Subsequently, we mixed the raw materials with
distilled water and an aqueous solution of ethyl alcohol, which had
concentrations of 10\%, 15\%, and 20\%. This was done using the standard
method and the ethyl alcohol solution had a volume of 40\% at room
temperature. The mixture was left for a duration of one hundred twenty
minutes {[}1{]}.~

The low-frequency ultrasonic device was utilised to perform ultrasonic
processing brand PLS-FSJ-300 made in China. The container containing a
sample of raw materials is positioned into an isothermal bath that has
been pre-heated to a temperature range of 38-40 degrees Celsius. The
reverse refrigerator initiates operation upon turning on the water pipe
valve. An electrically powered hoover pump is activated. Once the
residual pressure in the system has been measured and the length of
ultrasound treatment for the raw materials has been set to 15 minutes,
the low-frequency ultrasonic device is activated. After completing the
ultrasound processing of the raw materials, the vacuum pump is turned
off and the vacuum flow valve is opened to remove the container
containing the extract. Subsequently, the extract is strained using a
sieve, and the residual substance is then subjected to compression. The
obtained extract is forwarded for more investigation.~

{\bfseries Results and discussion.} The study employed physicochemical
research methodologies, adhering to the technical regulations and
standards specified for this particular product {[}19{]}. The sensory
parameters of the juice-containing beverage were determined using
established procedures {[}20{]}.~

Gooseberry extract was utilized at every stage of the soft drink
manufacturing process. The recipes for soft drinks containing gooseberry
extract are provided in table 2 {[}1{]}.

{\bfseries Table 2 - Formulations of soft carbonated drinks with goosberry
juice per 100 dal of finished product}

\begin{longtable}[]{@{}
  >{\raggedright\arraybackslash}p{(\columnwidth - 12\tabcolsep) * \real{0.1598}}
  >{\raggedright\arraybackslash}p{(\columnwidth - 12\tabcolsep) * \real{0.1407}}
  >{\raggedright\arraybackslash}p{(\columnwidth - 12\tabcolsep) * \real{0.1461}}
  >{\raggedright\arraybackslash}p{(\columnwidth - 12\tabcolsep) * \real{0.1369}}
  >{\raggedright\arraybackslash}p{(\columnwidth - 12\tabcolsep) * \real{0.1461}}
  >{\raggedright\arraybackslash}p{(\columnwidth - 12\tabcolsep) * \real{0.1339}}
  >{\raggedright\arraybackslash}p{(\columnwidth - 12\tabcolsep) * \real{0.1367}}@{}}
\toprule\noalign{}
\multirow{3}{=}{\begin{minipage}[b]{\linewidth}\raggedright
Raw material
\end{minipage}} &
\multicolumn{2}{>{\raggedright\arraybackslash}p{(\columnwidth - 12\tabcolsep) * \real{0.2868} + 2\tabcolsep}}{%
\begin{minipage}[b]{\linewidth}\raggedright
formulation 1
\end{minipage}} &
\multicolumn{2}{>{\raggedright\arraybackslash}p{(\columnwidth - 12\tabcolsep) * \real{0.2829} + 2\tabcolsep}}{%
\begin{minipage}[b]{\linewidth}\raggedright
formulation 2
\end{minipage}} &
\multicolumn{2}{>{\raggedright\arraybackslash}p{(\columnwidth - 12\tabcolsep) * \real{0.2705} + 2\tabcolsep}@{}}{%
\begin{minipage}[b]{\linewidth}\raggedright
formulation 3
\end{minipage}} \\
&
\multicolumn{6}{>{\raggedright\arraybackslash}p{(\columnwidth - 12\tabcolsep) * \real{0.8402} + 10\tabcolsep}@{}}{%
\begin{minipage}[b]{\linewidth}\raggedright
Content of raw material in juice
\end{minipage}} \\
& \begin{minipage}[b]{\linewidth}\raggedright
measuring unit
\end{minipage} & \begin{minipage}[b]{\linewidth}\raggedright
quantity
\end{minipage} & \begin{minipage}[b]{\linewidth}\raggedright
measuring unit
\end{minipage} & \begin{minipage}[b]{\linewidth}\raggedright
quantity
\end{minipage} & \begin{minipage}[b]{\linewidth}\raggedright
measuring unit
\end{minipage} & \begin{minipage}[b]{\linewidth}\raggedright
quantity
\end{minipage} \\
\midrule\noalign{}
\endhead
\bottomrule\noalign{}
\endlastfoot
sugar & kg & 75.16 & kg & 65.90 & kg & 29.26 \\
apple juice & \emph{l} & 95.5 & \emph{l} & 95.5 & \emph{l} & 95.5 \\
raspberry juice & \emph{l} & 23.46 & \emph{l} & 26.3 & \emph{l} &
24.7 \\
gooseberry extract & \emph{l} & 0.35 & \emph{l} & 1.43 & \emph{l} &
1.408 \\
citric acid & kg & 2.46 & kg & 2.32 & kg & 2.12 \\
essential oil & \emph{l} & 0.002 & \emph{l} & - & \emph{l} & 0.004 \\
color & kg & 0.35 & kg & - & kg & - \\
\multicolumn{7}{@{}>{\raggedright\arraybackslash}p{(\columnwidth - 12\tabcolsep) * \real{1.0000} + 12\tabcolsep}@{}}{%
Note: compiled based on source {[}1{]}} \\
\end{longtable}

Organoleptic indicators are assessed through visual observation and
taste evaluation to evaluate quality factors such as appearance, colour,
taste, scent, and transparency of the drink.~

The table 3 displays the sensory properties of the soft drink containing
gooseberry extract.~

{\bfseries Table 3 - Organoleptic characteristics of soft drink with
gooseberry extract}

\begin{longtable}[]{@{}
  >{\raggedright\arraybackslash}p{(\columnwidth - 6\tabcolsep) * \real{0.2500}}
  >{\raggedright\arraybackslash}p{(\columnwidth - 6\tabcolsep) * \real{0.2500}}
  >{\raggedright\arraybackslash}p{(\columnwidth - 6\tabcolsep) * \real{0.2500}}
  >{\raggedright\arraybackslash}p{(\columnwidth - 6\tabcolsep) * \real{0.2501}}@{}}
\toprule\noalign{}
\begin{minipage}[b]{\linewidth}\raggedright
Indicator
\end{minipage} & \begin{minipage}[b]{\linewidth}\raggedright
formulation 1
\end{minipage} & \begin{minipage}[b]{\linewidth}\raggedright
formulation 2
\end{minipage} & \begin{minipage}[b]{\linewidth}\raggedright
formulation 3
\end{minipage} \\
\midrule\noalign{}
\endhead
\bottomrule\noalign{}
\endlastfoot
Appearance & Nontransparent liquid, without seeds and impurities &
Nontransparent liquid, without seeds and impurities & Nontransparent
liquid, without seeds and impurities \\
Color & ruby & ruby & Saturated ruby \\
Taste, aroma & Taste is peculiar to gooseberry, pleasant aroma & Taste
is peculiar to gooseberry, pleasant aroma & Taste is peculiar to
gooseberry, pleasant aroma \\
\multicolumn{4}{@{}>{\raggedright\arraybackslash}p{(\columnwidth - 6\tabcolsep) * \real{1.0000} + 6\tabcolsep}@{}}{%
Note: compiled based on source {[}1{]}} \\
\end{longtable}

The table 4 displays the physical and chemical characteristics of the
soft drink containing gooseberry extract.

{\bfseries Table 4 - Physical and chemical parameters of soft drink with
gooseberry extract}

\begin{longtable}[]{@{}
  >{\raggedright\arraybackslash}p{(\columnwidth - 6\tabcolsep) * \real{0.2500}}
  >{\raggedright\arraybackslash}p{(\columnwidth - 6\tabcolsep) * \real{0.2500}}
  >{\raggedright\arraybackslash}p{(\columnwidth - 6\tabcolsep) * \real{0.2500}}
  >{\raggedright\arraybackslash}p{(\columnwidth - 6\tabcolsep) * \real{0.2501}}@{}}
\toprule\noalign{}
\begin{minipage}[b]{\linewidth}\raggedright
Indicator
\end{minipage} & \begin{minipage}[b]{\linewidth}\raggedright
formulation 1
\end{minipage} & \begin{minipage}[b]{\linewidth}\raggedright
formulation 2
\end{minipage} & \begin{minipage}[b]{\linewidth}\raggedright
formulation 3
\end{minipage} \\
\midrule\noalign{}
\endhead
\bottomrule\noalign{}
\endlastfoot
Mass share of dry substances, \% & 7.3 & 8.1 & 8.7 \\
Acidity, ml of 1 М solution NaОН for 100 ml of drink & 2.5 & 2.9 &
3.7 \\
Mass share of vitamin С, \%~ & 3.2 & 3.3 & 4.3 \\
Vitamin P, mg\% & 3.5 & 3.7 & 4.2 \\
Pectin substances, \% & 2.7-3.0 & 1.6-2.2 & 2.1-2.7 \\
рН & 4.4 & 4.6 & 4.4±0.2 \\
\multicolumn{4}{@{}>{\raggedright\arraybackslash}p{(\columnwidth - 6\tabcolsep) * \real{1.0000} + 6\tabcolsep}@{}}{%
Note: compiled based on source {[}1{]}} \\
\end{longtable}

Tables 3 and 4 demonstrate that the sensory, physical and chemical
characteristics of the soft drink align with the established criteria
for soft drinks. The preparation of the drink enables the following:
diversifying the range of options, enhancing the presence of
biologically active compounds, improving the sensory characteristics of
the product, and imparting functional properties to the drink {[}1{]}.~

The study findings reveal that the produced soft drink includes
significant amounts of biologically active components, including 3.2-4.3
mg\% of ascorbic acid and 3.5-4.2 mg\% of vitamin P.

{\bfseries Conclusions.} Based on the experimental results, it can be
concluded that gooseberry extract can be used as a supplement for soft
drinks. The utilisation of gooseberry extract in the manufacturing of
soft drinks serves as proof of its ability to enhance the sensory
characteristics of the beverage and enable the creation of a functional
beverage that offers both therapeutic and preventative benefits.



\begin{center}
  {\bfseries References}
  \end{center}
  
  \begin{noparindent}

1. B.M.Khamitova, B.T. Abdizhapparova, E.E. Tazhenov.~ Study of physical
and chemical properties of soft drinks obtained with addition of
gooseberry extract // Industrial Technologies and Engineering - 2019.
Vol. I. - P. 306-310.

2.Pomozova V.A. Proizvodstvo kvasa i bezalkogol\textquotesingle nyh
napitkov: Uchebnoe posobie. - SPb.: GIORD, 2006. - 192 s. ISBN
5-98879-029-1 {[}in Russian{]}

3. Donchenko G.V., Krichkovskaya L.V., CHernyshov S.I., Nikitchenko
YU.V. i dr. Prirodnye antioksidanty (biotekhnologicheskie,
biologicheskie i medicinskie aspekty): monografiya.
Har\textquotesingle kov: «Model\textquotesingle{} Vselennoj». 2011. -
376 s. {[}in Russian{]}

4. Belokurova E.V., Solohin S.A., Rodionov A.A. Razrabotka tehnologii
bulochnyh izdelij s vneseniem probioticheskogo bakkoncentrata
«Immunolakt»// Tekhnologii pishchevoj i pererabatyvayushchej
\\promyshlennosti APK -- produkty zdorovogo pitaniya -- 2016. -- № 3. --
S. 51-55. {[}in Russian{]}

5. SHumann, G. Bezalkogol\textquotesingle nye napitki:
syr\textquotesingle e, tekhnologii, normativy / G. SHumann ; perevod s
nemeckogo pod obshch. red. A. V. Oreshchenko i L. N. Benevolenskoj. -
Sankt-Peterburg : Professiya, 2004 (GP Tekhn. kn.). - 278 s. ISBN
5-93913-063-1

6. Filonova, G.L. Prjano-aromaticheskoe
syr\textquotesingle\textquotesingle e dlja sozdanija pozitivnoj
bezalkogol\textquotesingle\textquotesingle noj produkcii/G.L. Filonova,
I.L. Kovaleva, N.A. Komrakova, E.V. Nikiforova/ / Pivo i napitki. -- №5.
- 2015. - S. 58-61 {[}in Russian{]}

7. Skripnikov YU. G. Proizvodstvo plodovo-yagodnyh vin i sokov. -- M.:
Kolos, 1983. -- 256 s. {[}in Russian{]}

8. Gulcu-Ustundag, O. Saponins: properties, applications and processing.
/ O. Guclu-Ustundag, O. Mazza// Crit. Rev. Food Sci. Nutr. - 2007. -
Vol. 47. - P. 231-258 DOI: 10.1080/10408390600698197

9. Man S. Chemical study and medical application of saponins as
anti-cancer agents / S. Man, W. Gao, Y. Zhang // Fitoterapia. - 2010. -
Vol. 81. - P. 703-714 DOI: 10.1016/j.fitote.2010.06.004

10. SHlykova, A. P. Primenenie ekstrakta citronelly v tekhnologii
bezalkogol\textquotesingle nyh napitkov / A. P. SHlykova, E. O. Ivanova,
A. A. Kolobaeva, O. A. Kotik // Sovremennye naukoemkie tekhnologii. -- №
5. - 2014. - S. 192-196. {[}in Russian{]}

11. Sarafanova L. A. Primenenie pishchevyh dobavok v proizvodstve
napitkov / L.A. Sarafanova. - SPb.: Professiya, 2007. - 239 s. ISBN
5-93913-125-5 {[}in Russian{]}

12. Gazirovannye bezalkogol\textquotesingle nye napitki : receptury i
proizvodstvo / pod red. Djevida P. Stina i Filipa R. Jeshhersta ; per. s
angl. T. O. Zverevich. - Sankt-Peterburg : Professija, 2008. - 415 s.
ISBN 978-5-93913-160-5 (V per.) {[}in Russian{]}

13. Palagina, M. V. Resursy pishchevogo syr\textquotesingle ya
Dal\textquotesingle nevostochnogo regiona: ucheb. posobie / M. V.
Palagina, YA. V. Dubnyak, V. I. Golov. - Vladivostok: Izdat. dom
Dal\textquotesingle nevost. feder. un-ta, 2012. - 153 s. ISBN
978-5-7444-2728-3 {[}in Russian{]}

14. Gavrilova N.B., Petrova E.I.. Tekhnologiya produktov dlya
sportivnogo pitaniya // Molochnaya \\promyshlennost\textquotesingle. -
2013. - № 9. -S. 82-83 {[}in Russian{]}

15. Gavrilova N.B., SHCHetinin M.P., Moliboga E.A. Sovremennoe
sostoyanie i perspektivy razvitiya proizvodstva specializirovannyh
produktov dlya pitaniya sportsmenov // Voprosy pitaniya. 2017. - № 2. -
S. 100-106 {[}in Russian{]}

16. Atkins R. Biodobavki: prirodnaja al\textquotesingle ternativa
lekarstvam / per. s angl. G.I. Levitana. Minsk : Popurri, 2012. -800 s.
ISBN: 978-985-15-1243-6 {[}in Russian{]}

17. SHerman S.V., Kachak V.V., SHerman B.K. Nauchnye osnovy
formirovaniya sostava i potrebitel\textquotesingle skih harakteristik
gejnerov kak produktov intensivnogo sportivnogo pitaniya // Pishchevaya
promyshlennost\textquotesingle. 2012. - № 6. - S. 55-58. ISSN 0235-2486
{[}in Russian{]}

18. RodionovaN.S., Manukovskaya M.V., Nebol\textquotesingle sin A.E.,
Serchenya M.V. Primenenie metoda \\ul\textquotesingle trazvukovogo
jekstragirovanija v prigotovlenii napitka napravlennogo dejstvija iz
jagod chjornoj \\smorodiny// vestnik Voronezhskogo gosudarstvennogo
universiteta inzhenernyh tehnologij. -2016. -№(2). --S. 162-169№
DOI:10.20914/2310-1202-2016-2-162-169 {[}in Russian{]}

19. Aret V.A. Fiziko-himicheskie svojstva syr\textquotesingle ya i
gotovoj produkcii: uchebnoe posobie / V. A. Aret, B. L. Nikolaev, L. K.
Nikolaev. -- SPb. 2009. - 442 s. ISBN 978-5-98879-066-2 {[}in Russian{]}

20. GOST 6687.5-86. Produkciya bezalkogol\textquotesingle noj
promyshlennosti. Metody opredeleniya organolepticheskih pokazatelej i
ob"ema proizvodstva. --M, 1986. {[}in Russian{]}.

\end{noparindent}

\emph{{\bfseries Information about authors}}

\begin{noparindent}

Khamitova B.{\bfseries M}. -Candidate of Technical Sciences, Associate
Professor, M.Auezov South-Kazakhstan University, Shymkent, Kazakhstan,
e-mail: barno-007@mail.ru;

Dikhanbayeva F.- Doctor of Technology Professor Almaty Technological
University, Almaty, Kazakhstan, e-mail: fatima6363@mail.ru;

Koshtayeva G.E.- teacher M.Auezov South-Kazakhstan University, Shymkent,
Kazakhstan, \\e-mail: gulk1979@mail.ru

\end{noparindent}

\emph{{\bfseries Сведения об авторах}}

\begin{noparindent}

Хамитова Б.М. - кандидат технических наук, ассоциированный профессор,
Южно-Казахстанский университет им. М.Ауэзова, Шымкент, Казахстан,
e-mail: barno-007@mail.ru;

Диханбаева Ф.Т. - доктор технических наук, профессор, Алматинский
технологический университет, Алматы, Казахстан, e-mail:
fatima6363@mail.ru;

Коштаева Г.Е. {\bfseries -} преподаватель Южно-Казахстанский университет
им. М.Ауэзова, Шымкент,\\ Казахстан, e-mail: gulk1979@mail.ru
\end{noparindent}
