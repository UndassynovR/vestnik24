\newpage
МРНТИ 65.59.23, 62.13.02
\hfill {\bfseries \href{https://doi.org/10.58805/kazutb.v.3.24-534}{https://doi.org/10.58805/kazutb.v.3.24-534}}

\sectionwithauthors{С.Л. Гаптар, С.Б. Байтукенова}{ПРИМЕНЕНИЕ БИОТЕХНОЛОГИЧЕСКИХ МЕТОДОВ УЛУЧШЕНИЯ КАЧЕСТВА КОНИНЫ
ПРИ ПЕРЕРАБОТКЕ}
\begin{center}

{\bfseries \textsuperscript{1}С.Л. Гаптар, \textsuperscript{2}С.Б.
Байтукенова\envelope}

\textsuperscript{1}ФГБОУ ВО «Новосибирский государственный аграрный
университет», Новосибирск, Россия,

\textsuperscript{2}АО «Казахский университет технологии и бизнеса имени
К.Кулажанова», Астана, Казахстан
\end{center}

\envelope Корреспондент-автор: saule7272@mail.ru\vspace{0.5cm}

В данной работе рассматриваются современные биотехнологические методы,
направленные на улучшение качества конины. Рассматриваются принципы их
действия, влияние на структуру, вкус и питательные свойства мяса.

Для оптимизации производственного цикла и улучшения качества соленых
мясопродуктов, осо-бенно из конины с высокой жесткостью, рекомендуется
использовать биотехнологические и физичес-кие методы обработки. Одним из
эффективных подходов является использование парного мяса, которое
обладает высокой влагосвязывающей способностью и выраженными
бактериостатическими свойствами, что замедляет рост микробов. Для
ускорения процессов посола и созревания применяют методы, такие как
электростимуляция, шприцевание и механическая обработка. Результаты
показы-вают, что опытные образцы соленой конины содержат больше влаги и
имеют лучшую влагосвязываю-щую способность, что улучшает выход и сочность
продукта. При этом прочностные характеристики у них ниже на 32\% по
сравнению с контрольными образцами. Микробиологические показатели
соответствуют нормам, хотя на начальной стадии механической обработки
наблюдается небольшой рост микроорганизмов. Тепловая обработка и
использование парного сырья способствуют качеству продукта, а применение
13\% посола и 6 часов циклической механической обработки ускоряет
про-цесс посола и улучшает физико-химические свойства мяса.

{\bfseries Ключевые слова:} конина, биотехнологические методы, посол мяса,
физико-химические показа-тели, структурно-механические свойства.

\sectionheading{ӨҢДЕУ КЕЗІНДЕ ЖЫЛҚЫ ЕТІНІҢ САПАСЫН ЖАҚСАРТУДЫҢ БИОТЕХНОЛОГИЯЛЫҚ
ӘДІСТЕРІН ҚОЛДАНУ}
\begin{center}

{\bfseries \textsuperscript{1}С.Л. Гаптар, \textsuperscript{2}С.Б.
Байтукенова\envelope}

\textsuperscript{1}Новосибирск мемлекеттік аграрлық университеті,
Новосибирск, Ресей,

\textsuperscript{2} Қ.Құлажанов атындағы Қазақ технология және бизнес
университеті, Астана, Қазақстан,

e-mail: saule7272@mail.ru
\end{center}

Бұл мақалада жылқы етінің сапасын жақсартуға бағытталған заманауи
биотехнологиялық әдістер қарастырылады. Олардың әрекет ету принциптері
ет құрылымына, дәмі мен тағамдық қасиеттеріне әсері қарастырылады.

Өндірістік циклды оңтайландыру және тұзды ет өнімдерінің сапасын
жақсарту үшін, әсіресе қатты-лығы жоғары жылқы етінен биотехнологиялық
және физикалық өңдеу әдістерін қолдану ұсынылады. Тиімді тәсілдердің
бірі - ылғалмен байланысу қабілеті жоғары және микробтардың өсуін
бәсеңдететін айқын бактериостатикалық қасиеттері бар балғын етті
қолдану. Тұздау және жетілу процестерін жеделдету үшін
электрстимуляциялау, шприцтеу және өңдеу сияқты әдістер қолданылады.
Зерттеу нәтижелері көрсеткендей, тәжірибелік үлгіде тұздалған жылқы
етінің ылғалы көп және ылғалмен байланысу қабілеті жоғары, яғни өнімнің
өнімділігі мен шырындылығын жақсартады. Бұл жағдайда олардың беріктік
сипаттамалары бақылау үлгілерімен салыстырғанда 32\%-ға төмен.
Микробиология-лық көрсеткіштер нормаларға сәйкес келеді, дегенмен
өңдеудің бастапқы кезеңінде микроорганизм-дердің аз өсуі байқалады.
Термиялық өңдеу және балғын ет шикізатын пайдалану өнімнің сапасына
ықпал етеді, ал 13\% тұзды және 6 сағаттық циклдік өңдеуді қолдану
тұздау процесін тездетеді және физика-химиялық қасиеттерін жақсартады.

{\bfseries Түйін сөздер:} жылқы еті, биотехнологиялық әдістер, етті тұздау,
физика-химиялық көрсеткіштер, құрылымдық-механикалық қасиеттер.

\sectionheading{THE USE OF BIOTECHNOLOGICAL METHODS TO IMPROVE THE QUALITY OF HORSE MEAT DURING PROCESSING}
\begin{center}

{\bfseries \textsuperscript{1}S.L. Gaptar, \textsuperscript{2}}
{\bfseries S.B. Baitukenova\envelope}

\textsuperscript{1}Novosibirsk State Agrarian University, Novosibirsk,
Russia,

\textsuperscript{2} Kazakh University of Technology and Business named
after K. Kulazhanov, Astana, Kazakhstan,

e-mail: saule7272@mail.ru
\end{center}

This paper considers modern biotechnological methods aimed at improving
the quality of horse meat. The principles of their action, influence on
the structure, flavor and nutritional properties of meat are considered.

To optimize the production cycle and improve the quality of salted meat
products, especially from horsemeat with high hardness, it is
recommended to use biotechnological and physical processing methods. One
of the effective approaches is the use of steam meat, which has a high
moisture-binding capacity and pronounced bacteriostatic properties,
which slows down the growth of microbes. Methods such as electrical
stimulation, syringing and mechanical treatment are used to accelerate
salting and ripening processes. The results show that experimental
samples of salted horsemeat contain more moisture and have better
moisture-binding capacity, which improves the yield and juiciness of the
product. At the same time, their strength characteristics are lower by
32\% compared to control samples. Microbiological parameters correspond
to the norms, although at the initial stage of mechanical processing a
small growth of microorga-nisms is observed. Heat treatment and the use
of paired raw materials contribute to the quality of the product, and
the use of 13\% salting and 6 hours of cyclic mechanical processing
accelerates the salting process and improves the physical and chemical
properties of meat.

{\bfseries Keywords:} horse meat, biotechnological methods, meat
processing, physico-chemical parameters,\\ structural and mechanical
properties.
\begin{multicols}{2}

{\bfseries Введение.} В условиях растущего спроса на устойчивое и
экологически безопасное производство продуктов питания,
биотехнологи-ческие методы становятся все более востребован-ными для
переработки различных видов мяса, включая конину. Конина, несмотря на
свою питательную ценность и уникальные органолептические свойства,
остается недооце-ненным продуктом в ряде регионов мира.
Биотехнологические подходы, такие как ферментация, использование
микроорганизмов и энзимов, позволяют значительно улучшить качество
конины, продлить срок ее хранения, а также разработать новые виды
продуктов на ее основе. Введение этих методов открывает новые
перспективы для рационального использования ресурсов и улучшения
продовольственной безопасности, делая переработку конины более
эффективной и экономически целесообразной.

Зарубежные ученые в настоящее время рассматривают современные
биотехнологические методы, применяемые в процессе переработки конины.
Исследование охватывает инновацион-ные технологии, включая использование
ферментов, микробиологических культур и генетически модифицированных
организмов для улучшения качества и безопасности продукции {[}1{]}.

Известны исследования совместного воздейс-твия фермента бромелайн и
бактериальных культур на мясо лошадей. Бромелайн, протеолитический
фермент, использован для улучшения нежности мяса, а бактериальные
культуры применяются для повышения окислительной стабильности мяса.
Исследование показало улучшение физико-химических свойств, нежности и
окислительной стабильности, предлагая новые способы увеличения срока
хранения и качества конского мяса {[}2{]}.

Применение микробиологических культур в процессе переработки конины
рассматривают разнообразие микроорганизмов, используемых для улучшения
качества продукта, включая повышение его пищевой ценности и
безопасности. Также перспективы использования данных технологий в мясной
промышленности {[}3, 4{]}.

Известны различные биотехнологические подходы, включая использование
ферментов и микробиологических культур, для улучшения текстуры, вкуса и
безопасности продукции, где больше акцентирует внимание на важности
соблюдения стандартов качества при применении данных методов {[}5, 6,
7{]}.

Проводятся также исследования эффективнос-ти различных методов, включая
использование пробиотических культур и ферментов, для улучшения
органолептических характеристик и повышения пищевой ценности продукта.
Работа подчеркивает перспективы внедрения данных методов в
мясоперерабатывающее производство {[}8, 9{]}.

Механическая обработка мяса конины, такая как измельчение и размягчение,
может подготовить мясо к процессу ферментации. В этом процессе
микроорганизмы, такие как молочнокислые бактерии, используются для
улучшения текстуры, вкуса и срока хранения продукта. Механическая
обработка помогает разрушить соединительные ткани и клеточные структуры
мяса, что облегчает последующий гидролиз белков. Использование
ферментов, таких как протеазы, позволяет получить пептиды и
аминокислоты, которые могут улучшить питательную ценность и вкус
продукта. При помощи метода экстракции можно выделить белки из мяса
конины для последующего использования в различных биотехнологических
процессах. Например, экстрагированные белки могут быть использованы для
создания функциональных продуктов.

Модификация структуры мяса, такая как эмульгирование и гомогенизация,
может изменить текстуру мяса конины, улучшая его свойство в производстве
мясных изделий. Это позволяет создавать продукты с желаемыми текстурными
характеристиками и улучшенной консистенцией. Использование механических
методов, таких как взбивание и перемешивание, может способствовать более
равномерному распределению биотехнологических агентов (например,
ферментов или микроорганизмов) в мясе, что улучшает эффективность
процесса переработки и консистенцию конечного продукта.

{\bfseries Материалы и методы.} Физико-химические показатели: ГОСТ
34567-2019 «Мясо и мясные продукты. Метод определения влаги, жира,
белка, хлористого натрия и золы с применением спектроскопии в ближней
инфракрасной области».

Определение pH в мясных продуктах производится в соответствии с ГОСТ
2878-82 «Мясо и мясные продукты. Метод определения pH».

Структурно-механические показатели: ГОСТ Р 50814-95 «Мясопродукты.
Методы определения пенетрации конусом и игольчатым индентором»; ГОСТ
33609-2015~Мясо и мясные продукты. Органолептический анализ.

{\bfseries Результаты и обсуждение.} В настоящее время получены и
используются различные биологические активные комплексы на основе
продуктов убоя животных и растительного сырья. В качестве компонентов
применяются различные ферменты, микробиологические культуры, красители,
усилители вкуса и аромата, белковые препараты животного и растительного
происхождения.

Внесение при обработке вышеуказанных комплексов в состав мясопродуктов с
разрушенной структурой, например фаршевая композиция, в процессе
перемешивания и массирования дает достаточно равномерное их
распределение. Важную роль здесь играет степень измельчения частиц,
влагосодеражание, растворимость и структурообразующие свойства добавок.
Определенную трудность представляет введение многокомпонентных систем в
мясопродукты с неразрушенной структурой, например, при посоле и
обработке мяса, субпродуктов. Определенную трудность представляет
введение многокомпонентных систем в мясопродукты с неразрушенной
структурой, например, при посоле мяса. Для обеспечения равномерного
распределения соли рекомендуются массирование и интенсивный метод
обработки мяса.

Для сокращения производственного цикла, трудовых затрат и улучшения
качественных показателей соленых мясопродуктов необходимо использовать
биотехнологические и физические методы обработки мясного сырья. Эта
проблема особенно актуальна для производства мясопродуктов из конины,
т.к. так как они обладают достаточно высоким содержанием межмышечной
соединительной ткани и, следовательно, и повышенной жесткостью.

Одним из направлений улучшения качества и интенсификации производства
соленых изделий является использование мяса в парном состоянии. Основным
достоинством его является высокая влагосвязывающая способность (ВСС),
которая зависит от активной реакции среды. Способность мяса удерживать
влагу зависит от растворимости и эмульгирующих действий белков. В парном
мясе она максимальная. Парное мясо обладает хорошо выраженными
бактериостатическими свойствами по отношению ко многим видам бактерий,
поэтому размножение микробов в нем замедляется. В зависимости от
температуры бактериостатическая фаза удерживается от 3 до 24 ч.

Парное мясо обладает высокой влагосвязываю-щей способности и при рН
активности 5,9 поглощает в среднем 86\% воды (охлажденное мясо только
33\%). Преимущество парного мяса проявляется также при изучении свойств
белков соединительной ткани {[}10{]}.

Использование парного мяса для производства соленых изделий
предусматривает применение специальных методов обработки
(электростиму-ляция, шприцевание, механическая обработка) с целью
ускорения гликолиза или процесса посола и созревания. Контрольный
образец шприцевали традиционным методом посола в количестве 13\% к массе
сырья и выдерживали при температуре 0-4 \textsuperscript{0}С в течение 6
ч. В таблицах 1 и 2 приведены результаты исследования физико-химические
и структурно-механические показатели до и после обработки соленой конины
в течение 6 часов:
\end{multicols}

{\bfseries Таблица 1 - Физико-химические показатели до и после обработки
соленой конины}
\begin{longtable}[]{|@{}
  >{\raggedright\arraybackslash}p{(\columnwidth - 4\tabcolsep) * \real{0.4174}}|
  >{\raggedright\arraybackslash}p{(\columnwidth - 4\tabcolsep) * \real{0.2913}}|
  >{\raggedright\arraybackslash}p{(\columnwidth - 4\tabcolsep) * \real{0.2913}}@{}|}
\hline
\begin{minipage}[b]{\linewidth}\raggedright
Показатель
\end{minipage} & \begin{minipage}[b]{\linewidth}\raggedright
до массирования
\end{minipage} & \begin{minipage}[b]{\linewidth}\raggedright
после массирования
\end{minipage} \\
\endhead
\hline
массовая доля влаги, \% & 65,2±0,3 & 70,4±0,2 \\
\hline
массовая доля белка, \% & 20,0±0,4 & 22,0±0,3 \\
\hline
массовая доля жира, \% & 10,1±0,2 & 8,2±0,2 \\
\hline
массовая доля соли, \% & 3,0±0,2 & 2,8±0,3 \\
\hline
pH & 5,8±0,3 & 6,0±0,3 \\
\hline
Плотность, г/см³ & 1,1±0,2 & 1,0±0,3 \\
\hline
Жесткость, н/см² & 12,0±0,3 & 9,0±0,2 \\
\hline
\end{longtable}




\begin{longtable}[]{|@{}
  >{\raggedright\arraybackslash}p{(\columnwidth - 4\tabcolsep) * \real{0.3391}}|
  >{\raggedright\arraybackslash}p{(\columnwidth - 4\tabcolsep) * \real{0.3304}}|
  >{\raggedright\arraybackslash}p{(\columnwidth - 4\tabcolsep) * \real{0.3304}}@{}|}
\caption*{Таблица 2 - Структурно-механические показатели до и после
обработки соленой конины}\\
\hline
\begin{minipage}[b]{\linewidth}\raggedright
Показатель
\end{minipage} & \begin{minipage}[b]{\linewidth}\raggedright
до массирования
\end{minipage} & \begin{minipage}[b]{\linewidth}\raggedright
после массирования
\end{minipage} \\
\endhead
\hline
Напряжение среза, кПа & 32±0,2 & 26±0,3 \\
\hline
Влагосвязывающая способность, \% & 55,3±0,2 & 65,1±0,3 \\
\hline
Влагоудерживающая способность, \% & 60,1±0,2 & 75,3±0,2 \\
\hline
\end{longtable}

\begin{multicols}{2}

После массирования отмечается снижение жесткости и силы сдвига, что
указывает на улучшение текстуры продукта. Повышение влагосодержания и
удержания влаги свидетель-ствует о более сочной и мягкой конине. Снижение
содержания соли и повышение pH улучшают вкусовые качества. Сокращение
времени готовности продукта делает его более удобным для потребителя.
Общая оценка продукта после массирования выше, что говорит о
положительном влиянии процесса на вкусовые и органолептические
характеристики.

Эти результаты показывают, что циклическое массирование соленой конины в
течение 6 часов значительно улучшает физико-химические и
структурно-механические свойства продукта, делая его более
привлекательным для потребителей.

Результаты исследований растворимости саркоплазматических белков соленой
конины, обработанной белковым комплексом показали, что растворимость
белков этой фракции при интенсивной обработке возрастает за счет
взаимодействия их с ионами хлорида натрия. Наиболее существенным
изменениям при посоле конины подвержены белки миозиновой фракции. По
мере проникновения хлорида натрия в мышечную ткань конины наблюдается
повышение растворимости миофибриллярных белков.

Высокая растворимость миофибриллярных белков мяса обусловлена низкой
концентрацией водородных ионов, что обеспечивает им высокую
стабильность.

Установлено, что извлекаемость водораствори-мых белков конины находится в
весьма специфичной зависимости от концентрации соли и продолжительности
интенсивной обработки при посоле. В процессе посола извлекаемость
водорастворимых белков уменьшается в среднем на 10-15\% в начале
процесса, затем постепенно повышается.

Микроструктурные исследования показали, что в парной конине мышечные
волокна расположены прямолинейно и проявляются их саркомеры, а после
посола и механической обработкой мышечные волокна принимают
волнообразный, складчатый характер. В местах S-образных изгибов чаще
встречаются разрывы и разрушения миофибрилл. Разрыхление и волнообразные
изгибы мышечных волокон увеличивают их диаметр на 20-25\%, которые
выявлены на поперечном срезе образцов при гистометрическом анализе
мышечных волокон. Отмечено значительное количество микротрещин по ходу
мышечных волокон, без заметных нарушений сарколеммы и структуры волокон.

Совокупность деструктивных изменений в конине ускоряет фильтрационное
микрораспреде-ление посолочных веществ и образование липкого
поверхностного слоя из солерастворимых белков. Механическая обработка
также способствует выходу тканевых ферментов из мышечных волокон и
интенсификации вкусоароматообразования.

Сравнительные исследования влияния условий посола на изменения
структурно-механических свойств конины указывают на прямую зависимость
между гидратацией мышечных белков и нежностью мяса, приобретаемой в
процессе посола с применением интенсивных методов обработки. Важное
значение в улучшении консистенции мяса при посоле, несомненно, имеет
изменение микроструктуры тканей.

Исследование образцов соленой конины после циклической механической
обработки показали, что происходит разрыхление миофибриллярной
структуры, деструкция и разрыв протофибрилл в области S-линий, смещение
структурных элементов соседних миофибрилл по отношению друг к другу.
Наблюдается дальнейшие повреждения целостности сарколеммы.
Миофибриллярные структуры - растянутые и набухшие. В местах разрушения
миофибрилл и образовавшихся пространств наблюдается скопление
мелкозернистой белковой и жировой массы.

Применение биофизических методов для производства соленых изделий из
конины продемонстрировало значительные преимущества по сравнению с
традиционными способами переработки. В основе этих методов лежат
современные физические воздействия, такие как электрофизическая
обработка, использование ультразвука, электромагнитного поля и других
технологий, способных изменять структуру мясных тканей и ускорять
процессы посола.

Одним из главных преимуществ новой технологии является использование
парного сырья, что позволяет сохранить природные вкусовые и питательные
свойства мяса. Продукты, произведенные по этим методам, отличаются более
высоким выходом готовой продукции благодаря сокращению потерь влаги и
улучшению проникновения соли в ткань мяса. Это обеспечивает улучшенные
органолептические качества, такие как вкус, цвет и аромат, что делает
продукт более привлекательным для потребителей.

Кроме того, биофизические методы оказывают положительное влияние на
структурно-механические характеристики мяса, делая его более мягким и
сочным. Благодаря интенсивным методам обработки, процесс засолки
ускоряется, что позволяет значительно сократить длительность
производственного цикла. Это не только уменьшает затраты времени и
ресурсов, но и повышает экономическую эффективность производства.

Образцы парной соленой конины подвергались тепловой обработке при
температуре 85 \textsuperscript{0}C до тех пор, пока температура в
центре продукта не достигла 70-72 \textsuperscript{0}C. Далее готовые
изделия охлаждали до температуры +4...+8 \textsuperscript{0}C, после
чего определяли их качество. Исследование показало следующие результаты:
опытные образцы имели повышенное содержание влаги по сравнению с
контрольными образцами. Это способствовало улучшению выходных
показателей продукта; опытные образцы продемонстрировали высокую
способность связывать влагу, что положительно сказалось на сочности
конечного продукта. Вследствие улучшенной влагосвязывающей способности,
выход продукта увеличился, а сочность значительно повысилась. Несмотря
на улучшенные показатели сочности и выхода, прочностные характеристики
опытных образцов снизились. Измерения напряжения среза показали снижение
прочности на 32\% по сравнению с контрольными образцами, что
свидетельствует о более мягкой текстуре готового продукта. Таким
образом, улучшение влагосвязывающих характеристик конины позитивно
влияет на сочность и выход, однако снижает её механическую прочность.

Микробиологические показатели, как соленого полуфабриката, так и готовой
продукции соответствуют санитарно-гигиеничес-ким требованиям. Отмечается
небольшой рост общего числа микроорганизмов на начальной стадии
механической обработки.

Тепловая обработка соленого полуфабриката положительно сказалась на
качестве продукта. Использование парного сырья для производства соленых
изделий особенно эффективно для малых предприятий, где нет возможностей
для холодильного хранения мяса. При наличии компактных установок для
механической обработки можно завершить процесс производ-ства соленых
изделий в течение 8-10 часов.

{\bfseries Выводы.} Применение посола в количестве 13 \% от массы сырья в
сочетании с циклической механической обработкой в течение 6 часов
значительно ускоряет процесс посола. Это также оказывает положительное
влияние на физико-химические и структурно-механические свойства, как
соленой конины, так и готового продукта. В результате обработки
улучшаются характеристики соленого мяса, что способствует повышению его
качества и стабильности в процессе хранения и дальнейшей переработки.
\end{multicols}

\begin{center}
  {\bfseries References}
  \end{center}

\begin{noparindent}

\begin{enumerate}
\def\labelenumi{\arabic{enumi}.}
\item
  Lorea R.~Beldarrain,~Enrique~Sentandreu,~Noelia~Aldai,~Miguel A.
  Sentandreu //~ Horse meat tenderization in relation
  to~post-mortem~evolution of the myofibrillar sub-proteome. Meat
  Science.-2022. --Vol.
  188.\\~https://doi.org/10.1016/j.meatsci.2022.108804
\item
  Orynbekov D., Amirkhanov K., Kalibekkyzy Zh., Smolnikova F., Assenova
  B., Nurgazezova A., \\Nurymkhan G., Kassenov A., Baytukenova Sh.,
  Yessimbekov Zh. Study on the combined effects of bromelain enzyme
  treatment and bacteria cultures on the physicochemical properties and
  oxidative stability of horse meat. -Processes\emph{~}2024. --Vol.
  12(8). https://doi.org/10.3390/pr12081766
\item
  Jazila El Malti,~Hamid Amarouch. Microbial and physicochemical
  characterization of the horse meat in fermented sausage // Food
  Biotechnology. -2008. --Vol. 22(3). --P. 276-296, \\DOI:10.1080/08905430802262830
\item
  Marta Laranjo, Maria Eduarda Potes, Miguel Elias Role of Starter
  Cultures on the Safety of Fermented Meat Products // Food
  Microbiology. -- 2019. -Vol. 10. https://doi.org/10.3389/fmicb.2019.00853
\item
  Renata Stanisławczyk, Mariusz Rudy, Stanisław Rudy. The quality of
  horsemeat and selected methods of improving the properties of this raw
  material // Processes. --~2021. --Vol.~9(9).\\ ~https://doi.org/10.3390/pr9091672
\item
  Lorenzo J.M., Maggiolino A., Sarriés M.V.,~Polidori P., Franco D.,
  Lanza M., De Palo~P. // Horsemeat: Increasing Quality and Nutritional
  Value. Springer, Cham. -2019. -P. 31-67.
  https://doi.org/10.1007/978-3-030-05484-7\_3
\item
  Il\textquotesingle ina N.M., Kucova A.E., Bujlenko Ju.S., Fomina T.Ju.
  Primenenie metodov biotehnologii v mjasnoj promyshlennosti // Vestnik
  Juzhno-Ural\textquotesingle skogo gosudarstvennogo universiteta.
  -2017. -Tom 5. -№ 3. -S. 21-28. doı:10.14529/food170303 {[}in
  Russian{]}
\item
  Ryspaeva U.A., Bajtukenova Sh.B., Bajtukenova S.B. Vlijanie
  propionovokislyh mikroorganizmov na kachestvennye pokazateli
  polukopchenoj kolbasy // Vestnik Almatinskogo tehnologicheskogo
  universiteta. -2023. -№4. --S. 83-90.
  https://doi.org/10.48184/2304-568X-2023-4-83-90 {[}in Russian{]}
\item
  Dah-Sol Kim,~Nami Joo (2020). Texture Characteristics of Horse Meat
  for the Elderly Based on the Enzyme Treatment // Food Sci Anim Resour.
  --~2020. --Vol.~40~(1). --P.~74-86. DOI:~https://doi.org/10.5851\\/kosfa.2019.e86
\item
  Амирханов К.Ж. Биотехнологические методы обработки парной конины //
  Все о мясе. -2009. - № 5. - С. 26-28.
\end{enumerate}
\end{noparindent}

\emph{{\bfseries Сведения об авторах}}
\begin{noparindent}

Гаптар С.Л. - заведующая кафедрой технологии пищевых производств и
индустрии питания, к.т.н., доцент, ФГБОУ ВО «Новосибирский
государственный аграрный университет», Новосибирск, Россия, e-mail:
466485@mail.ru;

Байтукенова C.Б.- к.т.н., ассоциированный профессор кафедры технологии и
стандартизации, \\Казахский университет технологии и бизнеса имени
К.Кулажанова, Астана, Казахстан, \\e-mail: saule7272@mail.ru
\end{noparindent}

\emph{{\bfseries Information about the authors}}
\begin{noparindent}

Gaptar S.L. -- Head of the Department of Food Production Technology and
the Food Industry, Candidate of Technical Sciences, Associate Professor,
Novosibirsk State Agrarian University, Novosibirsk, Russia, e-mail:
466485@mail.ru;

Baitukenova S.B. -- Head of Department ``Technology and
Standardization'', Candidate of Technical Sciences, Associate Professor,
Kazakh University of Technology and Business named after K. Kulazhanov,
Astana,\\Kazakhstan, e-mail: saule7272@mail.ru
\end{noparindent}

